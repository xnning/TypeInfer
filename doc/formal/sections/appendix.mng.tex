\newcommand{\tpmapsto}[3]{#2\mapsto#3}
\newcommand{\castupe}{\ensuremath{\mathsf{cast}^{\uparrow}\ }}
\newcommand{\castdowne}{\ensuremath{\mathsf{cast}^{\downarrow}\ }}
\newcommand{\judge}{\Gamma\vdash}
\newcommand{\forallvars}[1]{\forall \overbar{#1}}
\newcommand{\olpolymorphic}[2]{\vdash^{ol} #1 \sqsubseteq #2}

\gram{\otte\ottinterrule}

\gram{\ottR\ottinterrule}

\gram{\ottt\ottinterrule}

\subsection{algorithmic Typing}

\newcommand{\checktype}{\Gamma\vdash_\Downarrow}
\newcommand{\infertype}{\Gamma\vdash_\Uparrow}
\newcommand{\infercheck}{\Gamma\vdash_\delta}

\newcommand{\checktypeno}{\vdash_\Downarrow}
\newcommand{\infertypeno}{\vdash_\Uparrow}
\newcommand{\infercheckno}{\vdash_\delta}

\newcommand{\instinfer}{\vdash^{inst}_\Uparrow}
\newcommand{\instcheck}{\vdash^{inst}_\Downarrow}
\newcommand{\instinfercheck}{\vdash^{inst}_\delta}

\newcommand{\polyinfer}{\Gamma\vdash^{poly}_\Uparrow}
\newcommand{\polycheck}{\Gamma\vdash^{poly}_\Downarrow}
\newcommand{\polycheckno}{\vdash^{poly}_\Downarrow}
\newcommand{\polyinfercheck}{\vdash^{poly}_\delta}

\newcommand{\dsk}{\vdash^{dsk}}
\newcommand{\unify}{\vdash^{unify}}

\newcommand{\genvar}{\widehat}
\newcommand{\genA}{\genvar{\alpha}}
\newcommand{\genB}{\genvar{\beta}}


$\Gamma, \Theta, \Delta  = \cdot~|~\Gamma, x~|~\Gamma, x:\sigma~|~\Gamma, \genA:\tau~|~\Gamma, \genA:\tau = \tau_2~|~\Gamma, \blacktriangleright_{\genA}$

\framebox{$ \infercheck e : \rho \dashv \Theta $ } infer $\Uparrow$ check $\Downarrow$ $\delta = \Uparrow \mid \Downarrow$

\[
\inferrule*[right=AX]
{ }
{\infertype \star : \rho \dashv \Theta}
\]

\[
\inferrule*[right=Var]
{x:\sigma \in \Gamma \\ \Gamma \instinfer \sigma \sqsubseteq \rho \dashv \Theta }
{\infertype x : \rho \dashv \Theta}
\]

\[
\inferrule*[right=Lam-Infer]
{\Gamma, \genA:\star, \genB:\star, x: \genA \checktypeno e : \genB \dashv \Theta, x:\genA , \Delta}
{\infertype (\lambda x.\, e) : (\Pi x: \genA. \genB) \dashv \Theta}
\]

\[
\inferrule*[right=Lam-Check]
{\Gamma,x: \sigma_1 \polycheckno e : \sigma_2 \dashv \Theta, x: \sigma_1, \Delta}
{\checktype (\lambda x.\, e) : (\Pi x: \sigma_1. \sigma_2) \dashv \Theta}
\]

\newcommand{\app}[1]{\colorbox{Cyan}{#1}}

\[
\inferrule*[right=App]
{\infertype e_1 : \rho_1 \dashv \Theta_1 \\
\Theta_1 \infertypeno^{app} [\Theta_1]\rho_1\ \cdot e_2 :\rho \dashv \Theta
}
{\infertype e_1\,e_2 : \rho \dashv \Theta}
\]

\[
\app{$
\inferrule*[right=App-pi]
{
\polycheck e_2 : \sigma_1 \dashv \Theta_1 \\
\Theta_1 \instinfer \sigma_2 [\tpmapsto {\sigma_1} x {e_2}] \sqsubseteq \rho \dashv \Theta}
{\infertype^{app} (\Pi x:\sigma_1. \sigma_2) \cdot e_2 : \rho \dashv \Theta}
$}
\]

\[
\app{$
\inferrule*[right=App-tvar]
{
\Gamma[\genA:\tau_\alpha] \instcheck \tau_\alpha \sqsubseteq \star \dashv \Theta_1[\genA:\tau_\alpha] \\
\Theta_1[\genA_1:\star, \genA_2:\star, \genA = \Pi x:\genA_1. \genA_2] \checktypeno e_2 : \genA_1 \dashv \Theta
}
{\Gamma[\genA:\tau_\alpha] \infertypeno^{app} \genA \cdot e_2 : \genA_2 \dashv \Theta}
$}
\]

\[
\inferrule*[right=Ann]
{
\checktype \sigma : \star \dashv \Theta_1\\
\Theta_1 \polycheckno (e:\sigma) \dashv \Theta_2\\
\Theta_2 \instinfer \sigma : \rho \dashv \Theta}
{\infertype (e : \sigma) : \rho \dashv \Theta}
\]

\[
\inferrule*[right=CastUp-Check]
{[\Gamma]\rho \longrightarrow \sigma \\
\polycheck e : \sigma \dashv \Theta}
{\checktype (\castupe e) : \rho \dashv \Theta}
\]

\[
\inferrule*[right=CastDown]
{\infertype e : \rho_1 \dashv \Theta_1 \\
[\Theta_1]\rho_1 \longrightarrow \sigma \\
\Theta_1 \instinfer \sigma \sqsubseteq \rho_2 \dashv \Theta}
{\infertype (\castdowne e) : \rho_2 \dashv \Theta}
\]

\[
\inferrule*[right=Let]
{\Gamma \infertypeno e_1 : \sigma \dashv \Theta_1 \\
\infercheck e_2[\tpmapsto {\sigma} x {e_1}]: \rho \dashv \Theta}
{\infercheck ( let\ x = e_1\ in\ e_2 ) : \rho \dashv \Theta}
\]

\[
\inferrule*[right=ImplicitPi]
{\checktype \tau : \star \dashv \Theta_1 \\
\Theta_1, x:\tau \checktypeno \rho_1 : \star \dashv \Theta, x:\tau, \Delta
}
{\infertype (\forall x : \tau. \rho_1) : \star \dashv \Theta}
\]

\[
\inferrule*[right=ExplicitPi]
{\checktype \sigma_1 : \star \dashv \Theta_1 \\
\Theta_1, x:\sigma_1 \checktypeno \sigma_2 : \star \dashv \Theta, x:\sigma_1, \Delta
}
{\infertype (\Pi x : \sigma_1. \sigma_2) : \star \dashv \Theta}
\]

\[
\inferrule*[right=Sub]
{\infertype e : \rho_1 \dashv \Theta_1\\
\Theta_1 \instcheck \rho_1 \sqsubseteq \rho \dashv \Theta
}
{\checktype e: \rho \dashv \Theta}
\]


\framebox{$ \Gamma \polyinfercheck e : \sigma \dashv \Theta$ }

\[
\inferrule*[right=Gen-Infer]
{\infertype e :\rho_1 \dashv \Theta\\
\rho = [\Theta]\rho_1 \\
\overbar{\genA:\tau}=\Gamma \cap (ftv(\rho)-ftv(\Gamma))
}
{\polyinfer e :(\forallvars{x:\tau}. \rho[\overbar{\genA} \mapsto \overbar{x}]) \dashv \Theta}
\]

\[
\inferrule*[right=Gen-Check]
{
[\Gamma]\sigma = \forallvars{\alpha:\tau}. \rho \\
\Gamma, \overbar{\alpha:\tau} \checktypeno e :\rho \dashv \Theta
}
{\polycheck e : \sigma \dashv \Theta}
\]

\framebox{$ \Gamma \instinfercheck \sigma \sqsubseteq \rho \dashv \Theta$ }

\[
\inferrule*[right=Inst-Infer]
{[\Gamma] \sigma = \forallvars{x:\tau}. \rho
}
{\Gamma \instinfer \sigma \sqsubseteq \rho[\overbar{\tpmapsto {\tau} x {\genA}}] \dashv \Gamma, \overbar{\genA:\tau}}
\]

\[
\inferrule*[right=Inst-Check]
{\Gamma \dsk [\Gamma]\sigma \sqsubseteq [\Gamma]\rho \dashv \Theta} { \Gamma \instcheck \sigma \sqsubseteq \rho \dashv \Theta}
\]

\newpage

\newcommand{\dskunify}[0]{\vdash^{\chi}}
\framebox{$ \Gamma \dskunify \sigma_1 \sqsubseteq \sigma_2 \dashv \Theta$} $\chi = dsk~|~unify$

dsk is normal subtyping relationship in higher-rank.

unify is used in cases like when we want to dsk two app $\tau_1 \sigma_1$ and $\tau_2 \sigma_2$, we request $\sigma_1$ and $\sigma_2$ are exactly the same.

\[
\inferrule*[right=ImplicitPi-dsk-R]
{
\Gamma, \overbar{x:\tau} \dsk \sigma_1 \sqsubseteq \rho \dashv \Theta, \overbar{x:\tau}, \Delta}
{\Gamma \dsk \sigma_1 \sqsubseteq \forallvars{x:\tau}. \rho \dashv \Theta}
\]

\[
\inferrule*[right=ImplicitPi-dsk-L]
{\Gamma , \blacktriangleright_{\genA}, \overbar{\genA : \tau} \dsk \rho_1[\overbar{\tpmapsto \tau x {\genA}}] \sqsubseteq \rho_2 \dashv \Theta, \blacktriangleright_{\genA}, \Delta}
{\Gamma \dsk \forallvars{x:\tau}.\rho_1 \sqsubseteq \rho_2 \dashv \Theta}
\]

\[
\inferrule*[right=ImplicitPi-unify]
{\Gamma \unify  \sigma_1 \sqsubseteq \sigma_3 \dashv \Theta_1 \\
\Theta_1, x:\sigma_1 \unify [\Theta_1]\sigma_2 \sqsubseteq [\Theta_1]\sigma_4 \dashv \Theta, x:\sigma_1, \Delta}
{ \Gamma \unify \forall x:\sigma_1. \sigma_2 \sqsubseteq \forall x:\sigma_3. \sigma_4 \dashv \Theta}
\]

\[
\inferrule*[right=ExplicitPi]
{\Gamma \dskunify  \sigma_3 \sqsubseteq \sigma_1 \dashv \Theta_1 \\
\Theta_1, x:\sigma_1 \dskunify [\Theta_1]\sigma_2 \sqsubseteq [\Theta_1]\sigma_4 \dashv \Theta, x:\sigma_1, \Delta}
{ \Gamma \dskunify \Pi x:\sigma_1. \sigma_2 \sqsubseteq \Pi x:\sigma_3. \sigma_4 \dashv \Theta}
\]

\[
\inferrule*[right=App]
{
\Gamma \unify \sigma_1 \sqsubseteq \sigma_2 \dashv \Theta_1 \\
\Theta_1 \dskunify [\Theta_1]\tau_1 \sqsubseteq [\Theta_1]\tau_2 \dashv \Theta
}
{\Gamma \dskunify \tau_1\, \sigma_1 \sqsubseteq \tau_2\, \sigma_2 \dashv \Theta}
\]

\[
\inferrule*[right=Lam]
{
\Gamma, x \dskunify \sigma_1 \sqsubseteq \sigma_2 \dashv \Theta, x, \Delta}
{\Gamma \dskunify \lambda x. \sigma_1 \sqsubseteq \lambda x. \sigma_2 \dashv \Theta }
\]

\[
\inferrule*[right=Ann]
{
\Gamma \unify \sigma_2 \sqsubseteq \sigma_4 \dashv \Theta_1 \\
\Theta_1 \dskunify  [\Theta_1]\sigma_1 \sqsubseteq [\Theta_1]\sigma_3 \dashv \Theta }
{\Gamma \dskunify \sigma_1:\sigma_2 \sqsubseteq \sigma_3:\sigma_4 \dashv \Theta}
\]

\[
\inferrule*[right=Castup]
{
\Gamma \dskunify  \sigma_1 \sqsubseteq \sigma_2 \dashv \Theta}
{\Gamma \dskunify (\castupe \sigma_1) \sqsubseteq  (\castupe \sigma_2) \dashv \Theta}
\]

\[
\inferrule*[right=Castdown]
{
\Gamma \dskunify  \sigma_1 \sqsubseteq \sigma_2 \dashv \Theta}
{\Gamma \dskunify  (\castdowne \sigma_1) \sqsubseteq  (\castdowne \sigma_2) \dashv \Theta}
\]

\[
\inferrule*[right=Let]
{
\Gamma \unify \sigma_1 \sqsubseteq \sigma_3 \dashv \Theta_1 \\
\Theta_1 \dskunify  [\Theta_1]\sigma_2 \sqsubseteq [\Theta_1]\sigma_4 \dashv \Theta }
{\Gamma \dskunify  (let\ x = \sigma_1\ in\ \sigma_2) \sqsubseteq  (let\ x = \sigma_3\ in\ \sigma_4) }
\]

\[
\inferrule*[right=Var]
{ }
{\Gamma[x\{:\sigma\}] \dskunify x \sqsubseteq x \dashv \Gamma[x\{:\sigma\}] }
\]


\newpage

\framebox{$\Gamma[\alpha:\tau] \dskunify \genA \sqsubseteq \rho \dashv \Theta, \alpha \notin ftv(\rho)$}

\newcommand{\co}[0]{\vdash^{co}}
\newcommand{\contra}[0]{\vdash^{contra}}
\newcommand{\cocontra}[0]{\vdash^{\phi}}
\newcommand{\bemono}[0]{\leadsto}

\[
\inferrule*[right=EVar]
{ }
{\Gamma[\genA:\tau] \dskunify \genA \sqsubseteq \genA \dashv \Gamma[\genA:\tau] }
\]

\[
\inferrule*[right=EVar-Uni]
{
\sigma =  get\_type(\tau_1)\\
\Gamma_{\genA} \dsk[\Gamma_{\genA}] \sigma \sqsubseteq [ \Gamma_{\genA} ]\tau \dashv \Theta_1\\
\Theta_1; \emptyset \co \tau_1 \bemono \tau_2 \dashv \Theta[\genA:\tau]
}
{\Gamma[\genA:\tau] \unify \genA \sqsubseteq \tau_1 \dashv \Theta[\genA:\tau=\tau_1]}
\]

\[
\inferrule*[right=EVar-Dsk]
{
\sigma_1 =  get\_type(\sigma)\\
\Gamma_{\genA} \dsk[\Gamma_{\genA}] \sigma_1 \sqsubseteq [ \Gamma_{\genA} ]\tau \dashv \Theta_1\\
\Theta_1; \emptyset \co \sigma \bemono \tau_1 \dashv \Theta[\genA:\tau]
}
{ \Gamma[\genA:\tau] \dsk \genA \sqsubseteq \sigma \dashv \Theta[\genA:\tau = \tau_1] }
\]

question: there will be no alpha in $\sigma_1$?

\framebox{$\Gamma[\alpha:\tau]; \Sigma \cocontra \sigma \bemono \tau \dashv \Theta$} $\phi = co~|~contra$

\[
\inferrule*[right=Mono-Var]
{ }
{\Gamma[x][\genA]; \Sigma \cocontra x \bemono x \dashv \Gamma[x][\genA]}
\]

\[
\inferrule*[right=Mono-Var]
{ }
{\Gamma; \Sigma[x] \cocontra x \bemono x \dashv \Gamma}
\]

\[
\inferrule*[right=Mono-Star]
{ }
{\Gamma; \Sigma \cocontra \star \bemono \star \dashv \Gamma }
\]

\[
\inferrule*[right=Mono-Lam]
{ }
{\Gamma; \Sigma \cocontra \lambda x.\ e \bemono \lambda x.\ e \dashv \Gamma}
\]

\[
\inferrule*[right=Mono-EVar-left]
{ }
{\Gamma[\genB:\tau_1][\genA:\tau]; \Sigma \cocontra \genB \bemono \genB \dashv \Gamma[\genB:\tau_1][\genA:\tau]}
\]

\[
\inferrule*[right=Mono-EVar-right]
{ }
{\Gamma[\genA:\tau][\genB:\tau_1]; \Sigma \cocontra \genB \bemono \genB_1 \dashv \Gamma[\genB_1:\tau_1, \genA:\tau][\genB :\tau_1= \genB_1]}
\]

\[
\inferrule*[right=Mono-Pi-co]
{\Gamma; \Sigma \contra \sigma_1 \bemono \tau_1 \dashv \Theta_1\\
\Theta_1; \Sigma, x \co \sigma_2 \bemono \tau_2 \dashv \Theta_2
}
{\Gamma; \Sigma \co \Pi x: \sigma_1. \sigma_2 \bemono \Pi x:\tau_1. \tau_2 \dashv \Theta_2}
\]

\[
\inferrule*[right=Mono-Pi-contra]
{
\Gamma; \Sigma \co \sigma_1 \bemono \tau_1 \dashv \Theta_1\\
\Theta_1; \Sigma, x \contra \sigma_2 \bemono \tau_2 \dashv \Theta\_
}
{\Gamma; \Sigma \contra \Pi x: \sigma_1. \sigma_2 \bemono \Pi x:\tau_1. \tau_2 \dashv \Theta}
\]

\[
\inferrule*[right=Mono-Forall]
{\Gamma[\genA_1 : \tau,\genA]; \Sigma \contra \sigma[ x \mapsto \genA_1] \bemono \tau_2 \dashv \Theta}
{\Gamma[\genA]; \Sigma \contra \forall x : \tau. \sigma \bemono \tau_2 \dashv \Theta}
\]

\[
\inferrule*[right=Mono-Other]
{ \Gamma; \Sigma \cocontra \sigma_0 \bemono \tau_0 \dashv \Theta_1\\
\Theta_i; \Sigma \cocontra \sigma_i \bemono \tau_i \dashv \Theta_{i+1}
}
{ \Gamma,\Sigma  \cocontra T~\overbar{\sigma}_n \bemono T~\overbar{\tau}_n \dashv \Theta_n}
\]

\clearpage


\iffalse
\subsection{Translation}

\newcommand{\transto}[1]{\leadsto#1}
\newcommand{\translated}[1]{|#1|}
\newcommand{\cyancolorbox}[1]{\colorbox{cyan!30}{$#1$}}
%\newcommand{\invariant}[2]{\leadsto #1 :#2}
\newcommand{\invariant}[2]{}

\subsubsection{Target Language}

\gram{\ottE\ottinterrule}

\subsubsection{Translation Rules}

\framebox{$ \infercheck e : \rho \invariant{E}{|\rho|}$ } infer $\Uparrow$ check $\Downarrow$ $\delta = \Uparrow \mid \Downarrow$

\[
\inferrule*[right=AX]
{} {\infercheck \star : \star \transto{\star}}
\]

\[
\inferrule*[right=Var]
{x:\sigma \in \Gamma \\ \instinfercheck \sigma \sqsubseteq \rho \transto{f} } {\infercheck x : \rho \transto{f\ x}}
\]

\[
\inferrule*[right=Lam-Infer]
{\Gamma,x: \tau \infertypeno e : \rho \transto{E} }
{\infertype (\lambda x.\, e) : (\Pi x: \tau. \rho) \transto {\lambda(x:\translated{\tau}). E}}
\]

\[
\inferrule*[right=Lam-Check]
{\Gamma,x: \sigma_1 \polycheckno e : \sigma_2 \transto{E}
} {\checktype (\lambda x.\, e) : (\Pi x: \sigma_1. \sigma_2) \transto{\lambda(x:\translated{\sigma_1}).E}}
\]

\[
\inferrule*[right=LamAnn-Infer]
{
\checktype \sigma:\star \\
\Gamma,x: \sigma \infertypeno e : \rho \transto{E}
} {\infertype (\lambda x:\sigma.\, e) : (\Pi x: \sigma. \rho) \transto{\lambda(x:\translated{\sigma}). E}}
\]

\[
\inferrule*[right=LamAnn-Check]
{
\checktype \sigma:\star \\
\translated{\sigma_1}, \translated{\sigma} \dsk \sigma_1 \sqsubseteq \sigma \transto{f} \\
\Gamma,x: \sigma \polycheckno e : \sigma_2 \transto{E} \\
}
{\checktype (\lambda x:\sigma.\, e) : (\Pi x: \sigma_1. \sigma_2) \transto{ \lambda(x:\translated{\sigma_1}). E[\tpmapsto \sigma x {f\ x}] }}
\]

\[
\inferrule*[right=App]
{\infertype e_1 : (\Pi x:\sigma_1. \sigma_2) \transto {E_1} \\
\polycheck e_2 : \sigma_1 \transto{E_2}\\
\instinfercheck \sigma_2 [\tpmapsto {\sigma_1} x e_2] \sqsubseteq \rho \transto{f}}
{\infercheck e_1\,e_2 : \rho \transto{f\ (E_1\ E_2)}}
\]

\[
\inferrule*[right=ExplicitPi]
{\checktype \tau_1 : \star \\ \Gamma, x:\tau_1 \checktypeno \tau_2 : \star }
{\infercheck (\Pi x:\tau_1. \tau_2) : \star \transto{(\Pi x:\translated{\tau_1}. \translated{\tau_2})}}
\]

\[
\inferrule*[right=Ann]
{
\checktype \sigma : \star \\
\polycheck (e:\sigma) \transto{E}\\
\instinfercheck \sigma : \rho \transto{f}}
{\infercheck (e : \sigma) : \rho \transto{f\ E}}
\]

\[
\inferrule*[right=CastUp-Check]
{ \rho \longrightarrow \sigma \\ \polycheck e : \sigma \transto{E} }
{\checktype (\castupe e) : \rho \transto{\castupe [\translated{\rho}]\ E}}
\]

\[
\inferrule*[right=CastDown]
{\infertype e : \rho_1 \transto{E}\\
\rho_1 \longrightarrow \sigma \\
\instinfercheck \sigma \sqsubseteq \rho_2\transto {f}}
{\infercheck (\castdowne e) : \rho_2 \transto {f\ (\castdowne E)}}
\]

\[
\inferrule*[right=Let]
{\polyinfer e_1 : \sigma \transto{E_1}\\
\infercheck e_2[\tpmapsto \sigma x {e_1}]: \rho \transto{E_2}
}
{\infercheck ( let\ x = e_1\ in\ e_2 ) : \rho \transto{(\lambda x:\translated \sigma. E_2)\ E_1}}
\]

\framebox{$ \infercheck \sigma : \star \invariant{E}{*}$ }

\[
\inferrule*[right=ImplicitPi]
{\checktype \tau : \star \\ \Gamma, x:\tau \checktypeno \rho : \star}
{\infercheck (\forall x : \tau. \rho) : \star \transto {\Pi (x:\translated \tau). \translated \rho}}
\]

\framebox{$ \infercheck \rho : \star \invariant{E}{*}$ }

\[
\inferrule*[right=FunPoly]
{\checktype \sigma_1 : \star \\
\Gamma, x:\sigma_1 \checktypeno \sigma_2 : \star }
{\infercheck (\Pi x : \sigma_1. \sigma_2) : \star \transto{\Pi (x:\translated {\sigma_1}). \translated {\sigma_2}}}
\]

\framebox{$ \Gamma \polyinfercheck e : \sigma \invariant{E}{|\sigma|}$ }

\[
\inferrule*[right=Gen-Infer]
{\infertype e :\rho \transto{E} \\ \overbar{x:\tau}=ftv(\rho)-ftv(\Gamma) \\
\checktype (\forallvars{x:\tau}. \rho):\star } {\polyinfer e :(\forallvars{x:\tau}. \rho) \transto{\lambda (\overbar{x:\translated \tau}). E}}
\]

\[
\inferrule*[right=Gen-Check]
{
pr(\sigma) = \forallvars{x:\tau}. \rho \transto{f} \\
\overbar{x:\tau} \notin ftv(\Gamma) \\
\checktype e :\rho \transto{E}
} {\polycheck e : \sigma \transto{f\ (\lambda (\overbar{x:\translated \tau}). E)}}
\]

\framebox{$ \instinfercheck \sigma \sqsubseteq \rho \invariant{f}{|\sigma|\to|\rho|}$ }

\[
\inferrule*[right=Inst-Infer]
{\overbar{y : \tau}
}
{\instinfer \forallvars{x:\tau} . \rho \sqsubseteq \rho[\overbar{\tpmapsto \tau x y}] \transto
{\lambda (a: \translated{\forallvars{x:\tau} . \rho} ). (a\ \translated{\overbar{\tau_{\beta}}})}}
\]

\[
\inferrule*[right=Inst-Check]
{\translated{\sigma}, \translated{\rho}\dsk \sigma \sqsubseteq \rho \transto {f} \\
}
{\instcheck \sigma \sqsubseteq \rho \transto {f}}
\]

\framebox{$pr(\sigma)=\forallvars{x:\tau}.\rho \invariant{f}{|\forallvars{x:\tau}.\rho|\to|\sigma|}$}

\[
\inferrule*[right=Pr-poly]
{pr(\rho_1) = \forallvars{b:\tau_2}. \rho_2 \transto{f}\\ \overbar{a} \cap \overbar{b} = \emptyset
}
{pr(\forallvars{a:\tau_1}. \rho_1) = \forallvars{a:\tau_1. b:\tau_2}. \rho_2 \\
\transto{\lambda (x: \translated{\forallvars{a:\tau_1. b:\tau_2}. \rho_2  }).\lambda (\overbar{a:\translated{{\tau_1}}}). f\ (x\ \overbar{a})}}
\]

\[
\inferrule*[right=Pr-fun]
{pr(\sigma_2) = \forallvars{a:\tau}. \rho_2 \transto{f} \\ \overbar{a} \cap ftv(\sigma_1) = \emptyset
}
{pr(\Pi x:\sigma_1. \sigma_2) = \forallvars{a:\tau}. \Pi x:\sigma_1. \rho_2 \\
\transto{\lambda (x:\translated{ \forallvars{a:\tau}. \Pi x:\sigma_1. \rho_2 }). \lambda (y:\translated{\sigma_1}). f\ (\lambda (\overbar{a:\translated{\tau}}). x\ a\ y) }}
\]

\[
\inferrule*[right=Pr-other-case]
{  } {pr(\tau)=\tau \transto {\lambda (x:\translated{\tau}). x}}
\]

\framebox{$\sigma \dsk \sigma_1 \sqsubseteq \sigma_2 \invariant{f}{|\sigma_1|\to|\sigma_2|}$}

\newcommand{\sigmal}{\sigma_{l}}
\newcommand{\sigmar}{\sigma_{r}}

\[
\inferrule*[right=Deep-skol]
{pr(\sigma_2)=\forallvars{a:\tau}. \rho \transto{f_1}\\ a \notin ftv(\sigma_1) \\
\sigmar = \Pi a:\translated{\tau}. \rho_r \\
\sigmal, \rho_r \dsk \sigma_1 \sqsubseteq \rho \transto{f_2}
}
{ \sigmal, \sigmar\dsk \sigma_1 \sqsubseteq \sigma_2 \transto{\lambda x:\sigma_l. f_1\ (\lambda(\overbar{a:\translated{\tau}}). f_2\ x) }}
\]

\framebox{$\sigma \dsk \sigma \sqsubseteq \rho \invariant{f}{|\sigma| \to|\rho|}$}

\[
\inferrule*[right=Alpha-Equal]
{  \alpha-equal(\sigma \sigma)  }
{  \sigmal, \sigmar \dsk \sigma \sqsubseteq \rho \transto{\lambda x: \sigmal. x}}
\]

\[
\inferrule*[right=Spec]
{\overbar{b: \tau} \\
\translated{\rho_1[\overbar{\tpmapsto \tau a b}]}, \translated{\rho_2} \dsk \rho_1[\overbar{\tpmapsto \tau a b}] \sqsubseteq \rho_2 \transto{f} \\
}
{\sigmal, \sigmar \dsk \forallvars{a:\tau}.\rho_1 \sqsubseteq \rho_2 \\
\transto {\lambda x:\sigmal. f\ (x\ \translated{\overbar{b}}) }}
\]

\[
\inferrule*[right=Fun]
{\translated{\sigma_3}, \translated{\sigma_1}\dsk  \sigma_3 \sqsubseteq \sigma_1 \transto{f_1} \\
\translated{\sigma_2[x \mapsto g_1\ y]}, \translated{\rho_4}\dsk  \sigma_2[x \mapsto g_1\ y] \sqsubseteq \rho_4 \transto{f_2} \\
 }
{\sigmal, \sigmar \dsk \Pi x:\sigma_1. \sigma_2 \sqsubseteq \Pi x:\sigma_3. \rho_4 \\
\transto {\lambda x:\sigmal. \lambda y:\translated{\sigma_3} . f_2\ (x\ (f_1\ y))}
}
\]

\[
\inferrule*[right=App]
{\sigmal \longrightarrow \sigmal' \\
\sigmar \longrightarrow \sigmar' \\
\sigmal', \sigmar' \dsk \tau_1 \sqsubseteq \tau_2 \transto {f}\\
}
{\sigmal, \sigmar \dsk \tau_1\, \sigma_1 \sqsubseteq \tau_2\, \sigma_1 \\
\transto {\lambda x:\sigmal. \castupe [\sigmar] (f\ (\castdowne x))}}
\]

\[
\inferrule*[right=Lam]
{
\sigmal, \sigmar \dsk \sigma_1 \sqsubseteq \sigma_2 \transto {f}}
{ \sigmal, \sigmar \dsk \lambda x. \sigma_1 \sqsubseteq \lambda x. \sigma_2 \transto {f}}
\]


\[
\inferrule*[right=Lamann]
{
\sigmal, \sigmar \dsk \sigma_2 \sqsubseteq \sigma_4 \transto{f}}
{\sigmal, \sigmar \dsk \lambda x:\sigma_1. \sigma_2 \sqsubseteq \lambda x:\sigma_1. \sigma_4 \\
\transto {f}}
\]


\[
\inferrule*[right=Ann]
{
\sigmal, \sigmar \dsk  \sigma_1 \sqsubseteq \sigma_3 \transto{f}}
{\sigmal, \sigmar \dsk \sigma_1:\sigma_2 \sqsubseteq \sigma_3:\sigma_2 \transto {f}}
\]

\[
\inferrule*[right=Castup]
{
\sigmal, \sigmar \dsk  \sigma_1 \sqsubseteq \sigma_2 \transto {f}}
{\sigmal, \sigmar \dsk  (\castupe \sigma_1) \sqsubseteq  (\castupe \sigma_2) \transto {f}}
\]


\[
\inferrule*[right=Castdown]
{\sigmal \longrightarrow \sigmal' \\
\sigmar \longrightarrow \sigmar' \\
\sigmal', \sigmar' \dsk  \sigma_1 \sqsubseteq \sigma_2 \transto {f}
}
{\sigmal, \sigmar \dsk  (\castdowne \sigma_1) \sqsubseteq  (\castdowne \sigma_2)
\transto{ \lambda x:\sigmal. \castupe [\sigmar] (f\ (\castdowne x))}}
\]

\[
\inferrule*[right=Let]
{\sigmal \longrightarrow \sigmal'\\
\sigmar \longrightarrow \sigmar' \\
\sigmal', \sigmar' \dsk  \sigma_1 \sqsubseteq \sigma_2 \transto {f}
}
{\sigmal, \sigmar \dsk  (let\ x = e_1\ in\ \sigma_1) \sqsubseteq  (let\ x = e_1\ in\ \sigma_2) \\
\transto{ \lambda x:\sigmal. \castupe [\sigmar] (f\ (\castdowne x))}}
\]

\clearpage

%\begin{landscape}

\begin{figure*}

\[
\inferrule* [right=CastDown]
    {
        \inferrule* [right=CastUp]
        {
            \inferrule* [right=Deep-Skol]
            { %pr(\forall a. a \to a) = \forall a. a \to a \\
                \inferrule* [right=Spec]
                {
                    \inferrule*[right=alpha-equal]
                    {  }
                    {(\Pi a:\star. a \to a), (a \to a)  \dsk (a \to a) \sqsubseteq (a \to a) \\
                     \transto{\lambda x:(a \to a). x}
                    }
                }
                {(\Pi a:\star. \Pi b:\star. a \to b), ( a \to a)
                 \dsk (\forall a. \forall b. a \to b) \sqsubseteq (a \to a) \\
                 \transto{\lambda x:(\Pi a:\star. \Pi b:\star. a \to b).
                    (\lambda x:(a \to a). x)\ (x\ a\ a) }
                }
            }
            {(\Pi a:\star. \Pi b:\star. a \to b), (\Pi a:\star. a \to a)
             \dsk (\forall a. \forall b. a \to b) \sqsubseteq (\forall a. a \to a) \\
             \transto{\lambda x:(\Pi a:\star. \Pi b:\star. a \to b). f_1 (\lambda a:\star.
                 (\lambda x:(\Pi a:\star. \Pi b:\star. a \to b).  (\lambda x:(a \to a). x)\ (x\ a\ a) ) \ x) }
            }
        }
        {(\Pi a:\star. \Pi b:\star. a \to b), (\Pi a:\star. a \to a)
          \dsk \castupe (\forall a. \forall b. a \to b) \sqsubseteq \castupe (\forall a. a \to a) \\
         \transto{\lambda x:(\Pi a:\star. \Pi b:\star. a \to b). f_1 (\lambda a:\star.
                 (\lambda x:(\Pi a:\star. \Pi b:\star. a \to b).  (\lambda x:(a \to a). x)\ (x\ a\ a) ) \ x) }
        }
    }
    {\castdowne \castupe (\Pi a:\star. \Pi b:\star. a \to b), \castdowne \castupe (\Pi a:\star. a \to a) \dsk \castdowne \castupe (\forall a. \forall b. a \to b) \sqsubseteq \castdowne \castupe (\forall a. a \to a) \\
        \transto{ \lambda x: \castdowne \castupe (\Pi a:\star. \Pi b:\star. a \to b). \castupe [\castdowne \castupe (\Pi a:\star. a \to a)]} \\
        \transto{ ((\lambda x:(\Pi a:\star. \Pi b:\star. a \to b). f_1 (\lambda a:\star.  (\lambda x:(\Pi a:\star. \Pi b:\star. a \to b).  (\lambda x:(a \to a). x)\ (x\ a\ a) ) \ x) )\ (\castdowne x))}
    }
\]
\end{figure*}
%\end{landscape}

\fi
