\section{Language Overview}
\label{sec:language}

Below shows the syntax of the system.

\begin{tabular}{lrcl}
  Type & $\sigma, \tau$ & \syndef & $\genA \mid e$ \\
  Expr & $e$ & \syndef & $x \mid \star \mid e_1~e_2 \mid \blam x \sigma e$ \\
             && \synor & $\bpi x {\sigma_1} \sigma_2$ \\
             && \synor & $\erlam x e \equiv \blam x \genA e$ \\
  Contexts &
             $\tctx, \ctxl, \ctxr$ & \syndef & $\ctxinit \mid \tctx,x:\sigma
                                               \mid \tctx, \genA
                                               \mid \tctx, \genA = \tau $ \\
\end{tabular}

\begin{figure}[t]
    \headercapm{\tctx\wc}

    \[\TWCEmpty \quad \TWCTVar\]
    \[\TWCTypedVar \quad \TWCEVar\]
    \[\TWCSolvedEVar\]

    \headercapm{\tctx \bywf \sigma}

    \[\TWFOther \]

    \headercapm{\tctx \bywt \sigma}
    \[\TWTVar \quad \TWTEVar \quad \TWTStar\]
    \[\TWTApp \quad \TWTLam \quad \TWTPi \]

    \caption{Well formedness}
    \label{fig:wellform}
\end{figure}

\paragraph{Expression. }
Expressions $e$ are variables x, a single sort $\star$ to represent the type of
types,
application $e_1~e_2$,
function $\blam x \sigma e$,
and Pi type
$\bpi x {\tau_1} {\tau_2}$.
The unannotated function $\erlam x e$ is equivalent to the same function with
the bound variable annotated by a fresh unification variable $\genA$.

Notice here we merge types and kinds together by an impredicative axiom $\star:\star$.
This choice is
ok for our
purposes since
we only intend to use the calculus as an example for unification.

\paragraph{Types.} Types $\sigma, \tau$ contain all expression forms, with an
additional $\genA$ to denote unification variables.

\paragraph{Context.} A context is an ordered list of variables, and all
unification variables.
Unification variables can either be unsolved
(with form $\genA$) or solved by a type $\tau$ (with form $\genA = \tau$).

It is important for a context to be ordered to solve the dependency between
variables.
For example, the expression

\begin{lstlisting}
  \z. \x. \y:x. y == z
\end{lstlisting}

\noindent cannot type check because $z$ cannot be of type $x$ since $x$
appears after $z$ in the context.

Figure \ref{fig:wellform} gives the definition of well formedness of a context
($\tctx \wc$),
and the well formedness of
a type under certain context ($\tctx \bywf \sigma$).
Both definitions rely on the typing process,
which due to the limitation of space is put in the appendix.
For a type to be well formed, the typing should not solve existing unification
variables.
The last judgment, the
well-scopedness
$\tctx \bywt \sigma$ means under context $\tctx$, $\sigma$ is a syntactically
legal term with all variables bounded.

\paragraph{Hole notation.} We use hole notations like $\tctx[x]$ to
denote that the variable $x$ appears in the context, sometimes it is also
written as $\tctx_1, x, \tctx_2$.

Multiple holes also keep the order. For example, $\tctx[x][\genA]$ not only
require the existance of both variable $x$ and $\genA$, but also require that
$x$ appears before $\genA$.

Hole notation is also used for replacement and modification. For example,
$\tctx[\genA = \star]$ means the context keeps unchanged except $\genA$
now is solved by $\star$.

\paragraph{Applying Context.} Since the context records all the solutions of
solved unification variables, it can be used as a substitution on types. Figure
\ref{fig:applyctx} defines the substitution process, where all solved
unification variables are substituted by their solution.

\begin{figure}[t]
    \headercapm{[\tctx]\sigma_1 = \sigma_2}{}

    \begin{mathpar}
    \begin{tabular}{r c l l}
        $[\ctxinit]e$   & = & $e$       \\
        $[\tctx,x:\tau]e$ & = & $[\tctx]e$ \\
        $[\tctx, \genA]e$ & = & $[\tctx]e$ \\
        $[\tctx, \genA=\tau]e$ & = & $[\tctx](e[\genA \mapsto \tau])$
    \end{tabular}
    \end{mathpar}
    \caption{Applying a context with existential variables.}
    \label{fig:applyctx}
\end{figure}
