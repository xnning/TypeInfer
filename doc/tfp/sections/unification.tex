\section{Unification}
\label{sec:unification}

We first formalize the unification problem as:

\begin{lstlisting}
$\tpreuni \tau_1 \uni \tau_2 \toctx$
\end{lstlisting}

The input of the unification is the current context $\tctx$, and two types
$\tau_1$ and $\tau_2$ that are being unified. The output of the unification
is a new context $\ctxl$ which extends the original context with probably more
new unification variables or more existing
unification variables solved. For example,

\begin{lstlisting}
$\genA \byuni \genA = \Int \dashv \genA = \Int$
\end{lstlisting}

For a valid unification problem, it must have the invariant: $[\tctx] \tau_1 =
\tau_1$, and $[\tctx] \tau_2 = \tau_2$. Namely, the input types must be
fully applied under the input context. So the following is not a valid
unification problem input:

\begin{lstlisting}
$\genA = \String \byuni \genA = \Int$
\end{lstlisting}

We assume this invariant is maintained through the whole
formalization.

\subsection{Type Sanitization}

As we mentioned before, our unification is based on alpha-equality. So in most
cases, the unification rules are intuitively structural. The most difficult
one which is also the most essential one, is how to unify a unification variable
with another type. We discuss those cases first.

\paragraph{Variable Orders matter.}

While unifying existential variable $\genA$ and type $\tau$, we tends to
directly derive that $\genA = \tau$. But $\tau$ may not be a valid type as the
solution for $\genA$.
Consider a unification example:

\begin{lstlisting}
$\tctx, \genA, x \byuni \genA \uni x$
\end{lstlisting}

\noindent which has no feasible solution. Because $\genA$ cannot be solved by
$x$ since $x$ is not in the scope of $\genA$.

From this observation, it seems whenever we have the unification problem as

\begin{lstlisting}
$\tctx, \genA, \ctxr \byuni \genA \uni \tau$
\end{lstlisting}

\noindent we need to check if $\tau$ is in the scope of $\tctx$. Only if it
satisfies the scope constraint, we can derive $\genA = \tau$.

\paragraph{Unification variable orders do not matter.}

However, there are unification problems that even if the type does not satisfies the
scope constraint, it still has feasible solutions. Consider:

\begin{lstlisting}
$\tctx, \genA, \genB \byuni \genA \uni \bpi x \genB x$
\end{lstlisting}

Here because $\genB$ appears after $\genA$, we cannot directly derive $\genA =
\bpi x \genB x$ which is ill typed.
But this unification does have a solution context if we could sanitize the
appearance of $\genB$ by solving it by a fresh unification variable $\genA_1$
which is put in the scope of $\genA$, and then
get an equivalent unification problem:

\begin{lstlisting}
$\tctx, \genA_1, \genA, \genB = \genA_1 \byuni \genA \lt \bpi x {\genA_1} x$
\end{lstlisting}

\noindent Then we can derive the solution of this equivalent unification problem:

\begin{lstlisting}
$\tctx, \genA_1, \genA = \bpi x {\genA_1} x, \genB = \genA_1$.
\end{lstlisting}

From this example, we can see that the orders of unification variables do not
matter because we can always solve it by a fresh unification variable that
satisfies the scope constraint. So for a unification problem

\begin{lstlisting}
$\tctx, \genA, \ctxr \byuni \genA \uni \tau$
\end{lstlisting}

\noindent we need to sanitize the unification variables in $\tau$
before we check the scope constraint. We call this process \textit{type
  sanitization}, which is given in Figure \ref{fig:algo-resolve}. The
interesting cases are only \rul{I-EVar} and \rul{I-EVar2}. In \rul{I-EVar},
because $\genB$ appears after $\genA$, so we create a fresh unification variable
$\genA_1$, which is put before $\genA$, and solve $\genB$ by $\genA_1$. In
\rul{I-EVar2}, because $\genB$ is in the scope of $\genA$, so we leave it
unchanged.

\begin{figure*}[t]
     \headercapm{\tctx[\genA] \bycg \tau_1 \cgto \tau_2 \toctx}{}
     \[\IEVarB \quad \IEVarA \]
     \[\IVar \quad \IStar \]
     \[\IApp\]
     \[\ILam \]
     \[\IPi\]
    \caption{Type Sanitization}
    \label{fig:algo-resolve}
\end{figure*}

\subsection{Unification}

\begin{figure*}[t]
\headercapm{\tpreuni \tau_1 \uni \tau_2 \toctx}{Unification of Types}
    \[\UAEq \quad \UEVarTy\]
    \[\UTyEVar\]
    \[\UApp\]
    \[\ULamAnn\]
    \[\UPi\]
    \caption{Unification rules}
    \label{fig:algo-unification}
\end{figure*}

Based on type sanitization, Figure \ref{fig:algo-unification} gives the
unification rules.

Rule \rul{U-AEq} corresponds to the case when two types are already
alpha-equivalent. Most of the rest rules are structural.
Two most subtle ones are rule \rul{U-EVarTy} and \rul{U-TyEVar}, which
corresponding respectively to when the unification variable is on the left and on the
right. We go through the first one. There are three preconditions.
First is the occurs check, which is to make sure $\genA$ does not appear in the
free variables of $\tau_1$.
Then we use type sanitization to make sure all the
unification types in $\tau_1$ that are out of scope of $\genA$ are turned into
fresh ones that are in the scope of $\genA$. This process gives us the output
type $\tau_2$, and output context $\ctxl_1, \genA, \ctxl_2$. In this step, we
eliminate the orders that do not matter. Finally, $\tau_2$ could also contain
variables of which the
orders matter, so
we use $\ctxl_1 \bywt \tau_2$ to make sure $\tau_2$ is well scoped.
Unlike well-formedness $\tctx \bywf \sigma$, well-termedness $\tctx \bywt
\sigma$ is a
relation that does not depend on typing. So there is no coupling between
unification and typing.

\paragraph{Example.} Below shows the process for the unification problem

\begin{lstlisting}
$\tctx, \genA, \genB \byuni \genA \uni \bpi x \genB x$
\end{lstlisting}

For clarity, we denote $\ctxl = \tctx,\genA_1,\genA,\genB =\genA_1$. And it is
easy to verify  $\genA \notin FV(\bpi x \genB x)$.

\[
   \ExUni
\]


\paragraph{Comparison.}

In [], they do not need to do type sanitization because if one side is
a unification variable, and the other side is some type constructors, they will
decompose the unification problem into several sub-unifications, until it cannot
be decomposed. Then they check about the well-scopedness of the type. If both
sides are unification variables, they will take two cases into consideration:
the unification on the left appears first in the context, or the one on the
right appears first in the context. For example, a decomposing rule for
unification where the left hand side is a function type is:


\[
  \inferrule{
     \tctx [\genA_2, \genA_2, \genA = \genA_1 \to \genA_2] \byuni \genA_1 \uni \tau_1
     \toctx
  \\ \ctxl \byuni [\ctxl]\tau_2 \uni \genA_2 \dashv \ctxr
  }{
     \tctx [\genA] \byuni \tau_1 \to \tau_2 \uni \genA \dashv \ctxr
  }
\]

Here the original unification variable $\genA$ is solved by a function type
consisting of two fresh
unification variables $\genA_1$ and $\genA_2$, which are then unified with
$\tau_1$ and $\tau_2$ sequentially.

However, this cannot hold in dependent types. For example consider we apply this rule to
this specific unification:

$\tpreuni \genA \uni \bpi x \star x $

\noindent It is obviously $\genA_2$ should be unified with $x$. However, $x$ is
not in the context, so it will be ill typed. One choice is to put $x$
into the context, and it must be put before $\genA_2$ to make $\genA_2 = x$ well
formed. However, this means $x$ will remains in the context, and it will make
$x$ available for any unification variable after $\genA_2$ even though they
should not have access to $x$.
Or more subtly, we can make a local context for each
unification variable. Then we can put $x$ into the local context of $\genA_2$.
In this way, there will be a local context attached with each unification
variable, which complicate the system.

Type sanitization solves thin problem in another way. We do not destruct Pi
types but sanitize the unification variables that are out of scope.

\subsection{Meta-theory}

We do not give formal proofs for the following conjectures, but we believe most of
them are quite intuitive:

\begin{conjecture}
  if $\tctx[\genA] \bycg \sigma_1 \cgto \sigma_2 \toctx$,
  then $[\ctxl] \sigma_1 = [\ctxl] \sigma_2$.
\end{conjecture}
\begin{hproof}
  By induction on type sanitization rules.
\end{hproof}

\begin{conjecture}
  if $\tctx \byuni \sigma_1 \uni \sigma_2 \toctx$,
  then $[\ctxl] \sigma_1 = [\ctxl] \sigma_2$.
\end{conjecture}
\begin{hproof}
  By induction on unification rules.
\end{hproof}

\begin{conjecture}
  \label{conj:correct_sani_2}
  if $\tctx[\genA] \bycg \sigma_1 \cgto \sigma_2 \toctx$,
  and $\tctx[\genA] \bywf \sigma_1$,
  then $\ctxl \bywf \sigma_2$.
\end{conjecture}
\begin{hproof}
  Type sanitization does nothing but replaces some unification variables with
  fresh unification variables. And all fresh unification variables are in the
  scope of the output context. So the output type preserves the well formedness
  of the input type.
\end{hproof}

\begin{conjecture}
  if $\tctx \byuni \sigma_1 \uni \sigma_2 \toctx$,
  and $\tctx \bywf \sigma_1$,
  and $\tctx \bywf \sigma_2$,
  then $\ctxl \bywf \sigma_1$,
  and $\ctxl \bywf \sigma_2$.
\end{conjecture}
\begin{hproof}
  Unification extends the input context with more fresh unification variables
  and newly solved existing unification variables. Those updates will not change
  the well formedness of the type. However, the key point here is to prove the rule
  \rul{U-EVarTy} and \rul{E-TyEVar} give well-formed output contexts. This is
  based on Conjecture \ref{conj:correct_sani_2}, from where we know $\ctxl_1,
  \genA, \ctxl_2 \bywf \tau_2$. From another precondition $\ctxl_1 \bywt
  \tau_2$, we know $\tau_2$ is well scoped in $\ctxl_1$. This means $\genA, \ctxl_2$
  plays no role for the well formedness of $\tau_2$. So $\ctxl_1 \bywf \tau_2$.
\end{hproof}