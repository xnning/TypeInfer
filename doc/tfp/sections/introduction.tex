\section{Introduction}

Dependently typed languages are currently increasingly adopted in many language
designs due to its expressiveness[...]. However, there are little literatures on
how to do type inference or unification on those languages. This is because
more power a type system has, more sophisticated the type system
becomes. The dependency between expressions and types bring lots of complexities
for jobs like type inference or unification.

Existing literatures [...] that tries to give specification for type inference
or unification of
a dependent language is complex in the sense it usually couples typing
and unification, which makes proof of meta-theory difficult.
Also the algorithm becomes non-intuitive or even unpredictable once it involves
so many constructs or processes.

In this paper, we presents an easy strategy to do unification based on
alpha-equality for simple dependent
types. Our notation is inspired by [Dunfield], in which the authors give a complete
and easy algorithm for higher-ranked type inference. We come up with a new
process called \textit{type sanitization} that helps resolve the dependency
between expressions and types. Later on, the type sanitization process is
extended to deal with restricted polymorphic types.
Based on type sanitization, our algorithm
decouples unification and subtyping from typing, which simplifies the
meta-theory. Though there are no formal proofs for those meta-theory yet (which
is still in progress),
we give many conjectures that we believe are intuitive.

We expect our algorithm works as an attempt
for a simple and predicative unification/subtyping algorithm for dependent
types. This is also to
fill the gap between
delicate unification algorithms for simple types
and
complicated unification algorithm for dependent types.
More precisely, our main contributions are:

\begin{itemize}
  \item A specification of an easy algorithm for unification algorithms based on
    alpha equality for
    simple dependent types.
  \item A heuristic strategy \textit{type sanitization} that resolves the
    dependency between expressions and types.
  \item A specification of subtyping based on extended type sanitization that
    decouples
      subtyping from typing in a language including restricted polymorphic types.
\end{itemize}

In Section \ref{sec:language}, we present a overview of the language with simple
dependent types. In Section \ref{sec:unification}, we formalize the unification
problem and introduce the process of type sanitization and unification. In
Section \ref{sec:extension}, we extend the language with a restricted
polymorphic type, and then present the process of extended type sanitization
along with subtyping rules. Finally Section \ref{sec:conclusion} concludes the
paper.