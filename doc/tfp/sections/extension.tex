\section{Extended with Polymorphism}
\label{sec:extension}

In this section, we discuss how to extends the type sanitization to deal with
restricted polymorphic types. Our system has two simplifications: type variables
in polymorphic types only have type $\star$, and subtyping only concerns Pi
types and polymorphic types.

\subsection{Language}

The new definition of syntax extends the original one:

\begin{tabular}{lrcl}
  Type & $\sigma $ & \syndef & $\genA \mid e$ \\
  Expr & $e$ & \syndef & $x \mid \star \mid e_1~e_2 \mid \blam x \sigma e \mid \bpi x {\sigma_1} \sigma_2$ \\
       && \synor & $\erlam x e \equiv \blam x \genA e$ \\
       && \synor & $\forall x: \star. \sigma$ \\
  Monotype & $\tau$ & \syndef & $ \{ \sigma' \in \sigma, \forall \notin \sigma'\} $ \\
  Contexts &
             $\tctx, \ctxl, \ctxr$ & \syndef & $\ctxinit \mid \tctx,x:\sigma
                                               \mid \tctx, \genA
                                               \mid \tctx, \genA = \tau $ \\
\end{tabular}

\paragraph{Expression.} Expressions $e$ are extended with the polymorphic type
$\forall x : \star . \sigma$. Note our definition of polymorphic type is not
the general version
$\forall x:\sigma. \sigma$.
The restriction on type variable $x$ is that it can only have type $\star$.

\paragraph{Types.} The definition of types $\sigma$ is the same.
Monotype $\tau$ is a special kind of $\sigma$ that
contains no universal quantifiers. Only monotypes can be solutions of
unification variables.
The well-scopedness of polymorphic type is
in Figure \ref{fig:poly-wellform}.

\begin{figure}[t]
    \headercapm{\tctx \bywt \sigma}
    \[ \TWTPoly \]

    \caption{Well scopedness (extends Figure \ref{fig:wellform})}
    \label{fig:poly-wellform}
\end{figure}

\subsection{Subtyping}

\subsubsection{Declarative Subtyping}

\begin{figure*}[t]
    \headercapm{\tpresub \sigma_1 \dsub \sigma_2}
    \[\DSPolyR \quad \DSPolyL\]
    \[\DSPi \quad \DSAEq \]
    \\
    \caption{Declarative Subtyping}
    \label{fig:decl-subtyp}
\end{figure*}

Unification is only for monotypes. So the definition of unification for the new
polymorphic language is the same as Figure \ref{fig:algo-unification}. However,
for a polymorphic language, we need to define the subtyping relationship.

The subtyping relationship for polymorphic language is quite standard. The
version extended with dependent type is shown
in Figure \ref{fig:decl-subtyp}. This is a simplified version of subtyping,
where we do not consider the polymorphic relationship between constructs other
than Pi types and polymorphic types. For
example, it is also possible to have subtyping
relationship between those two types:

\begin{lstlisting}
  $(\blam x \Int {\forall a. a \to a}) ~ 3$
  $(\blam x \Int {\Int \to \Int}) ~ 3$
\end{lstlisting}

This simplification is acceptable since our purpose is to deal with unification
variables in algorithmic subtyping. Also we expect extending the subtyping
with other
constructs like application will not cause any problem.

Rule \rul{DS-PolyR} first put $x:\star$ to the context and continue subtyping.
Rule \rul{DS-PolyL} guesses a monotype $\tau$ to instantiate $x$.
Rule \rul{DS-Pi} is contra-variant on argument, and co-variant on
return type. Type variable $x$ is assigned the less general type $\sigma_1$. For
all other cases, rule \rul{DS-AEq} requests two types to be alpha-equivalent.

\subsubsection{Algorithmic Subtyping}

\begin{figure*}[t]
    \headercapm{\tctx[\genA] \byrf^s \sigma \rfto \tau \toctx}{$s = + | -$}
    \[\IPoly \quad \IMono\]
    \[\IPiA \]
    \caption{Polymorphic Type Sanitization}
    \label{fig:algo-poly-saniti}
\end{figure*}

\begin{figure*}[t]
    \headercapm{\tpresub \sigma_1 \lt \sigma_2 \toctx}{}
    \[\SPolyR \quad \SPolyL\]
    \[\SPi \]
    \[\SEVarL \quad \SEVarR\]
    \[\SUnify\]
    \caption{Algorithmic Subtyping}
    \label{fig:subtyping}
\end{figure*}

For the algorithmic system, we need to add consideration for unification
variables in subtyping. Usually, in the case when a unification variable on one
side in subtyping process, we will do unification:

\[
  \inferrule{\tpreuni \genA \uni \sigma \toctx}{\tpresub \genA \lt \sigma \toctx}
\]

However, unification only accept monotypes as input types. But we cannot
restrict the type in the above rule to be monotype, because there are cases
when the type on the other side is a polymorphic type, and the subtyping still
holds. For example, in this subtyping

\[
\tpresub \genA \lt \bpi x {(\forall y .y \to y)} \Int
\]

\noindent one possible solution for $\genA$ is $\bpi x {(\Int \to Int)} \Int$.

One observation from this example is that, when the unification variable is on the left,
even though there can be polymorphic components on the right hand side, those
polymorphic components must appear contra-variantly. In the above example,
because the type $\forall y. y \to y$ is contra-variant, so it has feasible
solutions.

Similar, when the unification variable is on the right, the polymorphic
components on the left must appear co-variantly.

Also, if such a solution exists, there could be unlimited number of solutions.
For the above example, another solution of $\genA$ can be $(\bpi x {(\String \to String)}
\Int)$, or $(\bpi x {((\Int \to \Int) \to \Int \to \Int)} \Int)$. Namely, any type
of form
$\genA_1 \to \genA_1$, where $\genA_1$ is a fresh unification variable. This
means, we can eliminate the universal quantifier in the type $(\bpi x {(\forall y
  .y \to y)} \Int)$ by replacing $y$ with a fresh unification variable $\genA_1$,
and then unify $\genA$ with $(\bpi x {(\genA_1 \to \genA_1)} \Int)$.

Based on those observations, we give another type sanitization process for
polymorphic types in Figure \ref{fig:algo-poly-saniti}. The polymorphic
sanitization process will sanitize a probably polymorphic type $\sigma$ to a
monotype $\tau$. This process has two different modes:
contra-variant($s=+$), and co-variant($s = -$). Rule \rul{I-Poly} is when a
polymorphic type appears contra-variantly, we eliminate the universal
quantifier by replacing $x$ with a fresh unification variable $\genA_1$. Rule
\rul{I-Mono} is when the type is already a monotype so it remains unchanged.
Rule \rul{I-Pi} switches the mode in this argument type, and preserves the mode in
thi return type. Notice there is no rule for a polymorphic type under co-variant
mode. For example,

\[
\tpresub \genA \lt \forall y .y \to y
\]

\noindent will try to sanitize the type $\forall y. y \to y$ under co-variant
mode, then it fails, which corresponds to the
fact that no solution is feasible for this subtyping.

Having polymorphic type sanitization, we are ready to define algorithmic subtyping
relation, which is shown in Figure \ref{fig:subtyping}. Rule \rul{S-PolyR} is
intuitive. The trailing context $\ctxr$ is discarded because the variables are
already out of scope. The original
non-determinism in rule \rul{DS-PolyL} is replaced by a new unification variable
$\genA$ in rule \rul{S-PolyL}. Rule \rul{S-Pi} also discards useless trailing
context.
Rule \rul{S-EVarL} deals with the case a unification variable is on
the left, where the type $\sigma$ is first sanitized co-variantly.
Meanwhile, in rule \rul{S-EVarR}, the unification variable is on the right,
and the type $\sigma$ is sanitized contra-variantly.

\subsection{Meta-theory}

\begin{conjecture}
  if $\tctx[\genA] \byrf^s \sigma \cgto \tau \toctx$,
  and $\tctx[\genA] \bywf \sigma$,
  then $\ctxl \bywf \tau$,
  and if $s=+$, then $\ctxl \bysub \sigma \lt \tau \toctx_2$,
  else $\ctxl \bysub \tau \lt \sigma \toctx_3$.
\end{conjecture}
\begin{hproof}
  By induction on polymorphic type sanitization rules.
\end{hproof}

\begin{conjecture}
  if $\tpresub \sigma_1 \lt \sigma_2 \toctx$,
  and $\ctxr$ is an extension of $\ctxl$ with all unsolved unification
  variables solved,
  then $\ctxr \bysub [\ctxr] \sigma_1 \lt [\ctxr] \sigma_2$.
\end{conjecture}
\begin{hproof}
  By induction on subtyping rules.
\end{hproof}

\begin{conjecture}
  if $\ctxl$ extends $\tctx$ with all unification variables solved,
  and $[\ctxl]\tctx \bysub [\ctxl] \sigma_1 \lt [\ctxl]\sigma_2$,
  then there exists a context $\ctxl_2$, and $\ctxr'$,
  that $\tctx \bysub [\tctx] \sigma_1 \lt [\tctx]\sigma_2 \toctx_2$,
  and both $\ctxl$ and $\ctxl_2$ can be extended to $\ctxr'$.
\end{conjecture}
\begin{hproof}
  The idea is by induction on
  declarative subtyping rules, and analyze the form of $\sigma_1$, and
  $\sigma_2$. One key observation is that since $[\ctxl]\tctx \bysub [\ctxl]
  \sigma_1 \lt [\ctxl]\sigma_2$ holds, $\ctxl$ is itself a feasible solution for
  the algorithmic subtyping problem $\tctx
  \bysub [\tctx] \sigma_1 \lt [\tctx]\sigma_2 \dashv ?$. However the algorithmic
  system could have a slightly different solution $\ctxl_2$, where some
  unification variables could remain unsolved.Then both $\ctxl$ and
  $\ctxl_2$ can extend to a same context $\ctxr$.
\end{hproof}