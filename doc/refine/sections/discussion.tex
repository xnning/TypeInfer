\section{Discussion}

\subsection{Type Inference Algorithm For Dependent Types}

In Section~\ref{sec:dependent}, we use unification algorithm for dependent types
to show that type sanitization is applicable to advanced features. However,
notice that the typing rules presented in Section~\ref{subsec:typing} works
specifically for well-formedness of types. In other words, it is \textit{not} an
algorithmic type system containing type inference. Therefore, readers may wonder:
how can we design a type inference algorithm for practical usage based on the
unification algorithm, and what is the relation between the type inference
algorithm and the well-formedness typing rules? In this section, we discuss
these two questions.

\paragraph{Unification and Type Inference.} In a program, programmers may omit
some type annotations, and the type system will try to infer them. A typical
example is unannotated lambdas. For example

$\erlam x {x + 1} $

There is no annotation for the binder $x$, but the type system is able to
recover that $x:\Int$ from the expression $x + 1$. This is done by generating a
fresh existential variable $\genA$ for $x$, and then using the unification
algorithm to unify $\genA$ with $\Int$. In summary, type inference generates
existential variables that are waiting to be solved, while unification is in
charge of finding the solutions according to the type constraints.

Therefore, we can talk about unification separately from type inference. And
type inference for dependent types is also a challenging topic, where the
unification algorithm would definitely help with. Of course, in a type system
with unannotated lambdas, we also need to extend the unification to deal with
the new lambdas, which is not so complicated.

\paragraph{Type Inference and Well-formedness.} Even if we are given an
algorithmic type system containing type inference, the well-formedness system
still works correctly as long as the definition of well-formedness not changed.
Namely, any input type to the unification algorithm is already fully
type-checked, therefore it will introduce no more new constraints. However,
in our unification algorithm, well-formedness is only used for propositions,
so we leave it non-deterministic for simplicity. If it is used in the
algorithm, we would need to change it to an deterministic and decidable
relation.

\subsection{Future Work}

One possible future work is to apply the strategy of type sanitization to type
systems with more advanced features. For example, the polymorphic type
sanitization presented in Section~\ref{sec:higherrank} works specifically for
subtyping between polymorphic types. We can try to extend type sanitization to
other kinds of subtyping, for instance, nominal subtyping and/or intersection
types~\cite{}.

Also, since type sanitization resolves unification problem between dependent
types, another possible future work is to come up with a complete type inference
algorithm for dependent types based on alpha-equality and first-order
constraints. Extending the type sanitization to see whether it could play
any roles in extended setting, such as beta-equality or higher-order constraints
is also a feasible future work.
