\section{Overview}
\label{sec:overview}

This section provides background and gives an overview of our work. We discuss
the background of type inference in context~\citep{gundry2010type}, and also a
variant of this approach in a higher rank polymorphic type
system~\citep{dunfield2013complete}. While presenting the key ideas of the work,
we also talk about the challenges of each approach, which motivates our work.
Then we discuss the key ideas of type sanitization and polymorphic type
sanitization as a way to solve those problems.

\subsection{Background and Motivation}

\subsubsection{Type Inference In Context.}
\citet{gundry2010type} models unification and type inference from a general
perspective of information increase. The problem context is a ML-style
polymorphic system, based on the invariant that types can only depend on
bindings appearing earlier in the context.

Specifically, information and constraints about variables are stored in the
context, which is a list maintaining information order. For example\footnote{We
  adopt our notations and terminologies for the examples}:

$\tctx_1 = \genA, \genB, x :\genB$

Here $\genA, \genB$ are existential variables waiting to be solved. Their
meanings can be given by solutions, such as:

$\tctx_2 = \genA, \genB = \Int, x :\genB$

Then the unification problem becomes finding a more informative context that
contains solutions for the existential variables so that two expressions are
equivalent up to substitution of the solutions. For
example, $\tctx_2$ can be the solution context for unifying
$\Int$ and $\genB$ under $\tctx_1$.

Besides contexts being ordered, a key insight of the approach lies in how to
unify existential variables with other types. In this case, unification needs to
resolve the dependency between existential variables. Consider unifying $\genA$
with $\genB \to \genB$ under context $\genA, \genB, x : \genB$. Here $\genB$ is
out of the scope of $\genA$. The way Gundry et al. solve this problem is to examine variables in
the context from the tail to the head, \textit{moving segments of context to the
  left if necessary}. The process finishes when the existential variable being unified is found.
This design is implemented by an additional context that records the context
needed to be moved. This can be interpreted from the judgment $\tctx \ctxsplit
\Xi \byuni \genA \equiv \tau \toctx$, which is read as: given input context
$\tctx$, $\Xi$, solve $\genA$ with $\tau$ succeeds and produces an output
context $\ctxl$.

The essential rules involving in this process is given in
Figure~\ref{fig:inference-context}. If a variable is useless for the
unification, it is ignored (\rul{Ignore}). Otherwise, if the variable is needed,
but it is out of the scope of $\genA$, then it is moved to the additional
context (\rul{Depend}). Finally we arrive at the variable we want to unify,
which is $\genA$, we then insert $\Xi$ before $\genA$, and solve $\genA = \tau$
(\rul{Define}). Therefore, the output context for the above unification problem
is $\genB, \genA = \genB \to \genB, x: \genB$. Note that $\genB$ is now placed
in the front of $\genA$.

\begin{figure*}[t]
\[\Ignore \]
\[\Depend \quad \Define\]
  \caption{Unification between an existential variable and a type (incomplete).}
  \label{fig:inference-context}
\end{figure*}

\paragraph{Challenges.}

While moving type variables around is a feasible way to resolve the dependency
between existential variables, the unpredictable context movements make
the information increase hard to formalize and reason about. In this sense, the
information increase of contexts is defined in a much \textit{semantic} way:
$\ctxl$ is more informative than $\tctx$, if there exists a substitution $S$
that for every $v \in \tctx$, we have $\ctxl \bywf S(v)$.

This semantic definition makes it hard to prove meta-theory formally, especially
when advanced features are involved. For example, in dependent type systems,
typing and types/contexts well-formedness are usually coupled, which brings even
more complication to the proofs.

\subsubsection{Instantiation in Higher Rank Type Systems.}

\begin{figure*}[t]
    % \framebox{$\tctx \bybuni \sigma_1 \uni \sigma_2 \toctx$} \\
    \[\InstLSolve \quad \InstLReach\]
    \[\InstLArr  \]
    \[\InstRReach\]
  \caption{Instantiation between an existential variable and a type (incomplete).}
  \label{fig:instantiation}
\end{figure*}

\citet{dunfield2013complete} also use ordered contexts as input and output for
type inference for a higher rank polymorphic type system. However, they do it in
a more \textit{syntactic} way.

Instead of moving variables to the left in the context, in their subtyping
between an existential variable and a type, which they call instantiation, they
choose to decompose type constructs so that unification between existential
variables can only happen between single variables.
Then there are only two cases to discuss: whether the left
existential variable appears first, or the right one appears first.
Specifically, Figure~\ref{fig:instantiation} shows the key idea of instantiation
rules between an existential variable and a type. For space reasons, we only
present the rules when the left hand side is an existential variable, but the
other case is quite symmetric. We save the formal explanations for notation
to Section~\ref{sec:dependent}. But we can still get the rough idea
there. Notice that in \rul{InstLArr}, the existential variable $\genA$ is solved by a
function type consisting of two fresh existential variables, and then the
function is decomposed to do instantiation successively. Rule \rul{InstLReach}
deals with the case that $\genA$ appears first, and \rul{InstRReach} deals with the
other case.

In this way, the information increase of contexts, which is named context
extension, is formalized in an intuitive and straightforward \textit{syntactic}
way, which enables them to prove the meta-theory thoroughly and formally.
The complete definition of context extension can be found in the
appendix. Our
definition presented in Section~\ref{sec:context-extension} also mimics their
definition.

\paragraph{Challenges.} While destructing type constructors makes perfect sense
in their setting, it cannot deal with dependent types correctly. For example,
given the context $\genA, \genB$, suppose that we want to unify $\genA$ with a
dependent type $\bpi x \genB x$. Here because $\genB$ appears after $\genA$, we
cannot directly derive $\genA = \bpi x \genB x$ which is ill typed. However, if
we try to decompose this Pi type, then according to rule \rul{InstLArr}, it is
obvious that $\genA_2$ should be solved by $x$. In order to make the solution
well typed, we need to put $x$ before $\genA_2$ into the context. However, this
means $x$ will remain in the context, and it is available for any later
existential variable that should not have access to $x$. In essence the strategy
for simple function types presented by ~\citet{dunfield2013complete} does not
work for dependent function types.

Another drawback of decomposition is that it produces duplication. The rules in
Figure~\ref{fig:instantiation} is repeated for the cases when the existential
variable is on the right. For example, there will be a symmetric \rul{InstRArr}
corresponding to \rul{InstLArr}. Worse, this kind of ``duplication'' would
scales up with the type constructs in the system.

\subsection{Our Approach: Type Sanitization}
\label{subsec:sanitization}

Type sanitization provides another way to resolve the dependency between existential
variables. It combines the advantages of the previous two approaches. First, it is a
simple and quite predictable process, so that information increase can still be
modeled as \textit{syntactic} context extension. Also, it will not decompose
types, nor cause any duplication.

To understand how type sanitization works, we revisit the unification
problem: given context

$\genA, \genB, x: \genB$

\noindent we want to unify $\genA$ with $\genB \to \genB$. The problem here is
that $\genB$ is out of the scope of $\genA$. Therefore, we first ``sanitize''
the type $\genB \to \genB$. The process of type sanitization will sanitize the
existential variables in right hand side that are out of the scope of $\genA$ by
solving them with fresh existential variables added to the front of $\genA$.
Specifically, we will solve
$\genB$ with a fresh variable $\genA_1$, which results in an output context

$\genA_1, \genA, \genB = \genA_1, x: \genB$

Notice that $\genA_1$ is inserted right before $\genA$. Now the unification
problem becomes unifying $\genA$ with $\genA_1 \to \genA_1$, and $\genA_1 \to
\genA_1$ is a valid solution for $\genA$. Therefore, we get a final solution
context:

$\genA_1, \genA = \genA_1 \to \genA_1, \genB = \genA_1, x : \genB$.


\paragraph{Interpretation of Type Sanitization.}
Moving existential variables around in the approach of type inference in context
\citep{gundry2010type}, the symmetric rules \rul{InstLReach} and
\rul{InstRReach} \citep{dunfield2013complete}, and the approach of type
sanitization all enjoy the same philosophy: \textit{for an unsolved existential
  variable $\genA$, the relative order between $\genA$ and another unsolved
  existential variable $\genB$ does not matter}.

This seems to go against the design principle that the
contexts are ordered lists. However, keeping order is still important for
variables \textit{whose order matter}. For instance, for polymorphic types, the
order between existential variables ($\genA$) and type variables ($\varA$) is
important, so we cannot unify $\genA$ with $\varA$ under the context $(\genA,
\varA)$ since $\varA$ is not in the scope of $\genA$. A similar reason exists in
dependent type systems: we cannot unify $\genA$ with $x$ if $x$ appears behind
$\genA$ in the context.

The task of type sanitization is to ``move'' existential variables to suitable
positions in a roundabout way: solving the existential variables with a fresh existential variables
and making sure that these new fresh variables appear in a suitable position.

\subsection{Unification for Dependent Types}

As a first illustration of the utility of the type sanitization, we present a
unification algorithm for dependent types with alpha-equality based
first-order constrains.

\paragraph{Explicit Casts.}

Type systems with explicit type casts on type-level computation have been
adopted in several dependent type calculi \citep{van2013explicit,
  kimmell2012equational, sjoberg2012irrelevance, sjoberg2015programming,
  stump2009verified, sulzmann2007system, yang2016unified}. In order to have
type-level computations, explicit casts are needed to trigger the computation.
In this work we adopt the approach by ~\citet{yang2016unified}, where type casts
can be triggered by two language constructs: $\castdn e$ that does one-step beta
reduction on the type of $e$, and $\castup e$ that does one-step beta expansion
on the type of $e$. For example, given

$e: (\blam x \Int \star) ~ 3$

\noindent then we have

$\castdn e : \star$

$\castup (\castdn e) : (\blam x \Int \star) ~ 3$

In type systems with such type casts, type comparison is naturally based on alpha-equality. This
simplifies the unification algorithm in the sense that unification can be mostly
structural. However, we still need to deal with the dependency introduced by
dependent types carefully, which is mainly reflected in the unification problems
between existential variables.

\paragraph{Type Sanitization In Dependent Type System.}
Type sanitization is applicable to dependently type systems. Consider the previous
example, where we unify $\genA$ with $\bpi x \genB x$. By a similar process
described in Section~\ref{subsec:sanitization}, we can sanitize the type to be
$\bpi x {\genA_1} x$ without destructing the Pi type, and solve $\genA$ with
$\bpi x {\genA_1} x$.

% The key rules in the type sanitization are:

% \begin{mathpar}
%   \IEVarAfter \and \IEVarBefore
% \end{mathpar}

% In \rul{I-EVarAfter}, we encounter an existential variable $\genB$ that is out
% of the scope of the existential variable $\genA$, so we sanitize it by a fresh
% variable $\genA_1$. In \rul{I-EVarBefore}, the existential variable $\genB$ is
% contained in the scope of $\genA$, therefore it remains unchanged.

Based on type sanitization, we come up with a unification algorithm, which is
later proved to be sound and complete.

\subsection{Polymorphic Type Sanitization For Subtyping}

Instead of unification, the instantiation relation in
\citet{dunfield2013complete} is actually aiming at dealing with the polymorphic
subtyping relation between existential variables and other types. Here we say
that type $A$ is a subtype of $B$ if and only if $A$ is more polymorphic than $B$.
The difficulty of subtyping is that it needs to take unification
into account at the same time. For example, given that $\genA$ is a subtype of
$\Int$, then the only possible solution is $\genA = \Int$. And if given $\forall
a. a \to a$ is a subtype $\genA$, then a feasible solution can be $\genA = \Int
\to \Int$.

The type sanitization we described above only works for unification. If given
the context

$\genA$

\noindent and that $\forall \varA. \varA \to \varA$ is a subtype of $\genA$, how
can we sanitize the polymorphic type into a valid solution for an existential
variable while solutions can only be monotypes? One observation is that, the most
general solution for this subtyping problem is $\genA = \genB \to \genB$ for
fresh $\genB$. Namely, we remove the universal quantifier and replace the
variable $a$ with a fresh existential variable that should be in the scope of
$\genA$, which results in the solution context:

$\genB, \genA = \genB \to \genB$

From this observation, we extend type sanitization to polymorphic type
sanitization, which is able to resolve the polymorphic subtyping relation for
existential variables. Function types are contra-variant in argument
types, and co-variant in return types. For example, we have that:

$(\Int \to \Int) \to \Int \tsub (\forall \varA. \varA \to \varA) \to \Int $

$\forall \varA. \varA \to \varA \tsub \Int \to \Int$

\noindent According to the position the universal
quantifier appears, polymorphic type sanitization has two modes, which we call
contra-variant mode and co-variant mode respectively.
In contra-variant mode, the universal quantifier is replaced by a fresh
existential variable, while in co-variant mode, it is put in the context as a
common type variable.

We show that the original instantiation relationship in
\citet{dunfield2013complete} can be replaced by our polymorphic type
sanitization process, while subtyping remains sound and complete.

%%% Local Variables:
%%% mode: latex
%%% TeX-master: "../main"
%%% org-ref-default-bibliography: "citation.bib"
%%% End:
