\clearpage

\section{Appendix Overview}

In this section, we give an overview of the all the appendixes.

\subsection{Appendix \ref{sec:appendix-complete-definition}}

Due to the space limitation, in the paper we only give some incomplete
definitions. In this appendix, we present the full version of them.

\subsection{Appendix \ref{sec:appendix-dependent}}

This section gives the proofs for the dependent type system.
Here we list some important lemmas.

\subsubsection{Type Sanitization}
\begin{itemize}
\item Lemma~\ref{lemma:\TypeSanitizationExtensionName}.
  \TypeSanitizationExtensionName.\\
  \TypeSanitizationExtensionBody
\item Lemma~\ref{lemma:\TypeSanitizationEquivalenceName}.
  \TypeSanitizationEquivalenceName.\\
  \TypeSanitizationEquivalenceBody
\item Lemma~\ref{lemma:\TypeSanitizationWellFormednessName}.
  \TypeSanitizationWellFormednessName.\\
  \TypeSanitizationWellFormednessBody
\item Lemma~\ref{lemma:\TypeSanitizationCompletenessName}.
  \TypeSanitizationCompletenessName.\\
  \TypeSanitizationCompletenessBody
\item Corollary~\ref{lemma:\TypeSanitizationCompletenessPrettyName}.
  \TypeSanitizationCompletenessPrettyName.\\
  \TypeSanitizationCompletenessPrettyBody
\item Lemma~\ref{lemma:\TypeSanitizationCompletenessUnificationName}.
  \TypeSanitizationCompletenessUnificationName.\\
  \TypeSanitizationCompletenessUnificationBody
\end{itemize}

\subsubsection{Unification}
\begin{itemize}
\item Lemma~\ref{lemma:\UnificationExtensionName}.
  \UnificationExtensionName.\\
  \UnificationExtensionBody
\item Lemma~\ref{lemma:\UnificationEquivalenceName}.
  \UnificationEquivalenceName.\\
  \UnificationEquivalenceBody
\item Lemma~\ref{lemma:\UnificationCompletenessName}.
  \UnificationCompletenessName.\\
  \UnificationCompletenessBody
\end{itemize}

\subsection{Appendix \ref{sec:appendix-dependent}}

This section gives the proofs for the higher rank type system.
Here we list some important lemmas.

\subsubsection{Polymorphic Type Sanitization}

\begin{itemize}
\item Lemma~\ref{lemma:\PolymorphicTypeSanitizationExtensionName}.
  \PolymorphicTypeSanitizationExtensionName.\\
  \PolymorphicTypeSanitizationExtensionBody
\item Lemma~\ref{lemma:\PolymorphicTypeSanitizationSoundnessName}.
  \PolymorphicTypeSanitizationSoundnessName.\\
  \PolymorphicTypeSanitizationSoundnessBody
\item Lemma~\ref{lemma:\PolymorphicTypeSanitizationCompletenessName}.
  \PolymorphicTypeSanitizationCompletenessName.\\
  \PolymorphicTypeSanitizationCompletenessBody
\item Corollary~\ref{lemma:\PolymorphicTypeSanitizationCompletenessSubtypingName}.
  \PolymorphicTypeSanitizationCompletenessSubtypingName.\\
  \PolymorphicTypeSanitizationCompletenessSubtypingBody
\end{itemize}

\section{Complete Definitions}
\label{sec:appendix-complete-definition}

In this section, we give the full definitions from \citet{dunfield2013complete} that are omitted in the paper.

\subsubsection{Subtyping}

\begin{figure*}[t]
  \begin{mathpar}
    \framebox{$\tctx \bysub A \tsub B \toctx$}
     \\
    \ADunVar \and \ADunUnit \and
    \ADunExvar \and \ADunArrow \and
    \ADunForallL \and \ADunForallR \and
    \ADunInstL \and \ADunInstR
  \end{mathpar}
  \caption{Algorithmic Subtyping (Original).}
  \label{fig:subtyping-complete}
\end{figure*}

The original definition of algorithmic subtyping is given in
Figure~\ref{fig:subtyping-complete} \citep{dunfield2013complete}. This is the complete definition for
Figure~\ref{fig:subtyping}.
The judgment $\tctx \bysub A \tsub B \toctx$ is
interpreted as: given the context $\tctx$, $A$ is a subtype of $B$ under the
output context $\ctxl$.
The subtyping relation is reflexive (\rul{Var}, \rul{Exvar} and
\rul{Unit}). The function type is contra-variant on the argument type,
and co-variant on the return type (\rul{$\to$}). Here the intermediate context
$\ctxl$ is applied to $A_2$ and $B_2$ to make sure the input types are fully
substituted under the input context. In \rul{$\forall$L}, new existential
variable $\genA$ is created to represent the universal quantifier. There is a
marker before $\genA$ so that we can throw away all tailing contexts that are
out of scope after the subtyping. Similarly, rule \rul{$\forall$R} puts a
new type variable in the context and throws away all the contexts after the type variable.

When the left hand side (\rul{InstL}) or the right hand side (\rul{InstR}) is an
existential variable, the subtyping leaves all the work to the instantiation
judgment ($\ist$), which is given at the bottom of Figure~\ref{fig:inst-complete}.

\subsubsection{Instantiation}

\begin{figure*}[t]
  \begin{mathpar}
    \framebox{$\tctx \bysub \genA \ist A \toctx$}
     \\
     \InstLSolve \and \InstLReach \and \InstLArr \and
     \InstLAllR
     \\
    \framebox{$\tctx \bysub A \ist \genA \toctx$}
     \\
     \InstRSolve \and \InstRReach \and \InstRArr \and
     \InstRAllL
  \end{mathpar}
  \caption{Instantiation (Original).}
  \label{fig:inst-complete}
\end{figure*}

The original definition of instantiation is given in
Figure~\ref{fig:inst-complete} \citep{dunfield2013complete}. This is the complete definition for
Figure~\ref{fig:instantiation}. There are two judgments corresponding to whether
the existential variable is on the left or right.

The judgment $\tctx \bysub \genA \ist A \toctx$ is
interpreted as: given the context $\tctx$, instantiate $\genA$ so that it is a
subtype of $A$. If $A$ is already a monotype and it is well-formed under context
$\tctx_1$, then we directly solve $\genA = \tau$ (\rul{InstLSolve}). If $A$ is
another existential variable that appears after $\genA$, then we solve $\genB =
\genA$ (\rul{InstLReach}). If $A$ is a function type, it means $\genA$ must be a
function. Therefore we generate two fresh existential variables $\genA_1,
\genA_2$, and continue to instantiate $A_1$ with $\genA_1$, and $\genA_2$ with
$\applye \ctxl {A_2}$ (\rul{InstLArr}). If $A$ is a polymorphic type, and we put
the type variable $\varB$ into the context and continue to instantiate $\genA$
with $B$ (\rul{InstLAllR}). Notice that since $\varB$ is not in the scope of $\genA$
the context, if $\varB$ appears in $B$, then the instantiation will fail.

The other judgment $\tctx \bysub A \ist \genA \toctx$ plays a symmetric role.
The only difference is in \rul{InstRAllL}, where we put a fresh existential
variable $\genB$ in the context. For example $\genA \bysub \forall \varB. \varB \to \varB
\ist \genA$ has solution $\genB \to \genB$.

\subsubsection{Context Extension}

\begin{figure*}[t]
  \begin{mathpar}
    \framebox{$\tctx \exto \ctxr$}\\
    \DCEEmpty \and \DCEVar \and
    \DCETVar \and \DCEEVar \and \DCESolvedEVar \and
    \DCESolve \and \DCEAdd \and \DCEAddSolved \and
    \DCEMarker
  \end{mathpar}
  \caption{Context Extension.}
  \label{fig:context-extension-original}
\end{figure*}

We give the definition of context extension in \citet{dunfield2013complete} in
Figure~\ref{fig:context-extension-original}. Our definition in
Figure~\ref{fig:context-extension} mimics the original definition. Therefore we
save the explanation here.

%%% Local Variables:
%%% mode: latex
%%% TeX-master: "../main"
%%% org-ref-default-bibliography: "citation.bib"
%%% End: