\section{Related Work}

\subsection{Type Inference in Context}

We have discussed the work by \citet{gundry2010type} and
\citet{dunfield2013complete} in Section~\ref{sec:overview}, which in some sense
can both be understood in terms of the proofs systems of
\citet{miller1992unification}. They adopt different strategies to resolve the
order problem of existential variables, with different emphasizes.
\citet{gundry2010type} supports ML-style polymorphism, and as they mention the
longer-term goal is to elaborate high-level dependently typed programs into
fully explicit calculi. However they do not present the algorithm nor prove it.
\citet{dunfield2013complete} use the strategy specifically for higher rank
polymorphism, while as we have seen their strategy cannot be applied to dependent types.

\subsection{Unification for Dependent Types}

Unification for dependent types has been a challenging topic for years. While
our unification only solves first-order constraints based on alpha-equality,
existing literatures takes some advanced features into account, with the price
that the algorithm is usually complicated to understand, and also the
meta-theory is hard to prove. \citet{elliott1989higher} develop a higher order
unification algorithm for dependent function types in the spirit of
\citet{huet1975unification}, though the problem is undecidable.
\citet{reed2009higher} comes up with a constraint simplification algorithm that
works on dynamic pattern fragment of higher-order unification in a
dependent\bruno{finish sentence}.
Later, \citet{abel2011higher} propose a constraint-based unification algorithm
that solves a richer class of patterns.
\citet{ziliani2015unification} formalize the unification algorithm used in the
language Coq~\citep{coqsite}, which is quite sophisticated and the correctness
proof is lacking.

\subsection{Type Inference for Higher Rank Type Systems}

Type inference involving higher ranks has been well studied in recent years.
\citet{jones2007practical} develop an approach using traditional bi-directional
type checking, which is built upon \citet{odersky1996putting}. In this system,
subtyping and unification are separated, and unification is only between
monotypes. \citet{dunfield2013complete} build a simple and concise algorithm for
higher ranked polymorphism, which is also based on traditional bidirectional
type checking. The instantiation in their system is introduced in
Section~\ref{sec:overview}. These two systems are predicative, in the sense that
universal quantifiers can only be instantiated with monotypes. There are also
impredictive systems, including $ML^F$
\citep{le2014mlf,remy2008graphic,le2009recasting}, the HML system
\citep{leijen2009flexible} and the FPH system \citep{vytiniotis2008fph}.
%%% Local Variables:
%%% mode: latex
%%% TeX-master: "../main"
%%% org-ref-default-bibliography: "citation.bib"
%%% End:
