\section{Unification and Type Sanitization for Dependent Types}
\label{sec:dependent}

This section introduces a simple dependently typed calculus with 
alpha-equality and type casts. The main novelties are the unification and type
sanitization mechanisms.

\subsection{Language Overview}
\label{subsec:language}

The syntax of the calculus is shown below:\\

\begin{tabular}{lrcl}
  Expressions & $e$ & \syndef & $x \mid \star
                         \mid e_1~e_2 \mid \blam x \sigma e
                         \mid \bpi x {\sigma_1} \sigma_2$ \\
       && \synor & $\castup e \mid \castdn e$ \\
  Types & $\tau, \sigma$ & \syndef & $e \mid \genA$ \\
  Contexts & $\tctx, \ctxl, \ctxr$ & \syndef & $\ctxinit \mid \tctx,x:\sigma
             \mid \tctx, \genA
             \mid \tctx, \genA = \tau $ \\
  Complete Contexts & $\cctx$ & \syndef & $\ctxinit \mid \cctx,x:\sigma
             \mid \cctx, \genA = \tau $ \\
\end{tabular}

\paragraph{Expressions.}
Expressions $e$ include variables x,
a single sort $\star$ to represent the type of
types,
applications $e_1~e_2$,
functions $\blam x \sigma e$,
Pi types
$\bpi x {\sigma_1} {\sigma_2}$,
$\castup e$ that does beta expansion,
and $\castdn e$ that does beta reduction.

\paragraph{Types.}
Types $\tau, \sigma$ are the same as expressions, except that types contain
existential variables $\genA$.
The categories of expressions and types are stratified to make sure that
existential variables only appears in positions where types are expected.

\paragraph{Contexts.}
Contexts $\tctx$ are an ordered list of variables and
existential variables, which
can either be unsolved
($\genA$) or solved by a type $\tau$ ($\genA = \tau$).
It is important for a context to be ordered to solve the dependency between
variables.
Complete contexts $\cctx$ only contain variables and solved existential variables.

\paragraph{Hole notation.}
We use hole notations like $\tctx[x]$~\citep{dunfield2013complete} to
denote that the variable $x$ appears in the context. Sometimes such a
hole notation is also written as $\tctx_1, x, \tctx_2$.
Multiple holes also keep the order. For example, $\tctx[x][\genA]$ not only
requires the existence of both variables $x$ and $\genA$, but also requires that
$x$ appears before $\genA$.
The hole notation is also used for replacement and modification. For example,
$\tctx[\genA = \star]$ means the context is mostly unchanged except
that $\genA$ now is solved by $\star$.

\paragraph{Applying Contexts.} Since the context records all the solutions of
solved existential variables, it can be used as a substitution. Figure
\ref{fig:context-application} defines the substitution process, where all solved
existential variables are substituted by their solutions.

\begin{figure*}[t]
  \centering
  \begin{tabular}{rlp{2cm}rll}
    $\applye {\emptyset} \sigma$ & = & $\sigma$ &
    $\applye {\tctx, x: \tau} \sigma$ & = & $\applye \tctx \sigma$ \\
    $\applye {\tctx, \genA} \sigma$ & = & $\applye \tctx \sigma$ &
    $\applye {\tctx, \genA = \tau} \sigma$ & = & $\applye \tctx {\sigma \subst \genA \tau}$\\
  \end{tabular}
    \caption{Context application.}
    \label{fig:context-application}
\end{figure*}

\subsection{Typing in Detail}
\label{subsec:typing}

\begin{figure*}[t]
    \[\framebox{$\tctx \byinf \sigma_1 \infto \sigma_2$}\]
    \[\AAx \quad \AVar \quad \AEVar \]
    \[\ASolvedEVar \quad \ALamAnn\]
    \[\APi                          \]
    \[\AApp\]
    \[\ACastDn \quad \ACastUp       \]

    \[\framebox{$\sigma_1 \redto \sigma_2$} \]
    \[\RApp \quad \RBeta \]
    \[\RCastDown \quad \RCastDownUp\]
    \caption{Typing and semantics.}
    \label{fig:typing}
\end{figure*}

In order to show that our unification algorithm works correctly, we need to make
sure that inputs to the algorithm are well-formed.
In a dependent type system, the well-formedness of types and contexts are relying
on typing judgments.
Therefore, to introduce the well-formedness of type and contexts,
we first introduce the typing rules.

Before we give the typing rules, we need to consider: what does it
mean for an input type to the unification algorithm to be well-formed?
For example, Given a context $(\genA, x : \genA)$,
is the type $((\blam y \Int \star) ~ x)$ well formed?
Here, the type requests solving $\genA = \Int$.
We do not regard this type well formed, because we keep the
invariant: \textit{the
constraints contained in the input type have already been solved.}
Namely, the unification process only accepts inputs that are already type
checked in current context.
In this example, this type is well-formed under context
$(\genA = \Int, x : \genA)$.
This can also help prevent ill-formed contexts that contains conflicting
constraints, for example, the context:

$\genA, x: \genA, y : ((\blam x \Int \star)~x), z : ((\blam x \Bool \star)~x)$

\noindent contains two constraints that request $\genA$ to be solved by $\Int$
and $\Bool$ respectively, which cannot be satisfied at the same time.

\paragraph{Type System}
Given this interpretation of well-formedness, the typing rules that serve
specifically for well-formedness is shown at the top of Figure~\ref{fig:typing}.
The judgment $\tctx \byinf \sigma_1 \infto \sigma_2$ is read as: under the
context $\tctx$, the type $\sigma_1$ has type $\sigma_2$.

Rule \rul{A-Ax} states that $\star$ always has type $\star$.
Rule \rul{A-Var} acquires the type of the variable from the typing context, and
applies the context to the type.
This reveals another invariant that we keep:
\textit{the typing output is always fully substituted under current context.}
Rules \rul{A-EVar} and \rul{A-SolvedEVar} ensures that existential variables
always have type $\star$.
Rule \rul{A-LamAnn} first infers the type $\star$ for the annotation, then puts $x:
\sigma_1$ into the typing context to infer the body. To make sure output type is
fully substituted, we apply the context to $\sigma_1$ in the output Pi type.
Rule \rul{A-Pi} infers the type $\star$ for the argument type $\sigma_1$, then puts
$x: \sigma_1$ into the typing context to infer $\sigma_2$, whose type is also a
$\star$. And the result type for a Pi type is $\star$.
Rule \rul{A-App} first infers a function type for $e_1$, and then infers $e_2$ to
have the argument type. Again, to maintain the invariant, we apply the context
to $e_1$ before substituting $x$ with $e_1$.
Rule \rul{A-CastDn} infers the type $\sigma_1$ of $e$, that reduce the type
$\sigma_1$ to $\sigma_2$, while rule \rul{A-CastUp} finds a fully substituted
type $\applye \tctx
{\sigma_1}$ that reduces to $\sigma_2$ as the output type. The call by name
reduction is defined at the bottom of Figure~\ref{fig:typing}.
Due to the design of the rule \rul{A-CastUp}, the typing rules are
non-deterministic, which does not matter for our purpose: the typing is only
used in propositions (such as lemmas, theories) but it never appears in the
unification and type sanitization algorithms.

\paragraph{Context well-formedness.}

The first four typing rules have a common precondition $\tctx \wc$,
which requests the context is well-formed.
The judgment is defined at the top of Figure~\ref{fig:type-well}.
Rule \rul{AC-Empty} states that an empty context is always well formed.
Rule \rul{AC-Var} requires $x$ fresh, and the type annotation is typed with
$\star$. Rule \rul{AC-EVar} and \rul{AC-SolvedEVar} are defined in a similar
way.

\paragraph{Type well-formedness.}

We denote a well-formed type $\sigma$ as $\tctx \byinf \sigma \infto \star$.
We will also sometimes write it as $\tctx \bywf \sigma$.
A weaker version of type well-formedness, which is type well-scopedness, written
as $\tctx \bywt \sigma$, is defined at the bottom of Figure~\ref{fig:type-well}.
Well-scopedness of types only requires all the variables involved in a type are
bound in the typing contexts.

\begin{figure*}[t]
    \[\framebox{$\tctx \wc$} \]
    \[\ACEmpty \quad \ACVar \quad\]
    \[\ACEVar \quad \ACSolvedEVar\]

    \[\framebox{$\tctx \bywt \sigma$}     \]
    \[\WSVar \quad \WSEVar \quad \WSSolvedEVar \]
    \[\quad \WSPi \quad \WSLamAnn \]
    \[\WSApp \quad \WSCastDn \quad \WSCastUp          \]
    \caption{Context well-formedness and type well-scopedness.}
    \label{fig:type-well}
\end{figure*}

\begin{figure*}[t]
    \[\framebox{$\tctx[\genA] \bysa \tau_1 \sa \tau_2 \toctx$} \]
    \[\IEVarAfter \]
    \[\IEVarBefore \quad \IVar \]
    \[\IStar \quad \IApp \]
    \[\ILamAnn \]
    \[\IPi \]
    \[\ICastDn \quad \ICastUp\]
  \caption{Type sanitization.}
  \label{fig:sanitization}
\end{figure*}

\begin{figure*}[t]
   \[\framebox{$\tctx \bybuni \sigma_1 \uni \sigma_2 \toctx$} \]
   \[\UAEq \]
   \[\UEVarTy \]
   \[\UTyEVar \]
   \[\UApp \]
   \[\ULamAnn \]
   \[\UPi \]
   \[\UCastDn \quad \UCastUp \]
  \caption{Unification.}
  \label{fig:unification}
\end{figure*}


\subsection{Unification}
\label{subsec:unification}

As we mentioned before, our unification is based on alpha-equality. So in most
cases, the unification rules are intuitively structural. The most difficult one
which is also the most essential one, is how to unify an existential variable
with another type. In this section, we first present the judgment of
unification, then we discuss those cases before we present the unification
process.

\paragraph{Judgment of unification.}

Due to our design choice, there are two modes in the
unification: expressions, and types. The expression mode ($\delta = e$) does
unification between expressions, while the type mode ($\delta = \sigma$) does
unification between types.
The judgment of unification problem is formalized as:

\begin{lstlisting}
$\tctx \bybuni \sigma_1 \uni \sigma_2 \toctx$
\end{lstlisting}

The input of the unification is a context $\tctx$, and two types
($\delta = \sigma$) or two expressions ($\delta = e$).
The output of the unification
is a new context $\ctxl$ which extends the original context with probably more
new existential variables or more existing
existential variables solved.
The formal definition of context extension is discussed in
Section~\ref{sec:context-extension}.
Following is an example of a unification problem:

\begin{lstlisting}
$\genA \bysuni \genA \uni \Int \dashv \genA = \Int$
\end{lstlisting}

\noindent where we want to unify $\genA$ with $\Int$ under the input context
$\genA$, which results in the output context $\genA = \Int$ that solves $\genA$
with $\Int$.

For a valid unification problem, it must have the invariant: $[\tctx] \tau_1 =
\tau_1$, and $[\tctx] \tau_2 = \tau_2$. Namely,
\textit{the input types must be
fully applied under the input context}.
 So the following is not a valid
unification problem input:

\begin{lstlisting}
$\genA = \Bool \bysuni \genA \uni \Int$
\end{lstlisting}

We assume this invariant is given with the inputs at the beginning,
and the unification process would maintain it through the whole
formalization.

\paragraph{Process of type sanitization.}

As we discuss in Section~\ref{sec:overview}, before unifying an existential
variable with a type, we first sanitize the type so that the existential
variables in the type that are out of the scope are solved by fresh existential
variables within the scope. We call this process \textit{type sanitization},
which is formally defined in Figure \ref{fig:sanitization}. The judgment
$\tctx[\genA] \bysa \tau_1 \sa \tau_2 \toctx$ is interpreted as: under the
context $\tctx$, which contains an existential variable $\genA$, we sanitize all
the existential variables in the type $\tau_1$ that appears before $\genA$,
which results in a sanitized type $\tau_2$ and an output context $\ctxl$. Computationally, there are three
inputs $\tctx$, $\genA$ and $\tau_1$, with two outputs $\tau_2$ and $\ctxl$.
We sometimes omit $\genA$ if it is clear which existential variable is being referred.

The most interesting cases are \rul{I-EVarAfter} and \rul{I-EVarBefore}. In
\rul{I-EVarAfter}, because $\genB$ appears after $\genA$, we create a fresh
existential variable $\genA_1$, which is put before $\genA$, and solve $\genB$
by $\genA_1$. In \rul{I-EVarBefore}, because $\genB$ is in the scope of $\genA$,
we leave the existential variable unchanged.
The remaining rules are structural.
Similarly to unification, we always apply intermediate output
contexts to the input types to maintain the invariant that the types are fully
substituted under current contexts.

The sanitization process is remarkably simple, while it solves exactly what we
want: resolve the order of existential variables so that we can focus on the
order that really matters.

\paragraph{Process of unification.}

Based on type sanitization, Figure \ref{fig:unification} presents the
unification rules.
Rule \rul{U-AEq} corresponds to the case when two types are already
alpha-equivalent. Most of the rest rules are structural. The two most subtle cases
are rules \rul{U-EVarTy} and \rul{U-TyEVar}, which correspond respectively to
when the existential variable is on the left and on the right. We go through the
first rule. There are three preconditions. First is the occurs check, which is to
make sure $\genA$ does not appear in the free variables of $\tau_1$. Then we use
type sanitization to make sure all the existential variables in $\tau_1$ that
are out of scope of $\genA$ are turned into fresh ones that are in the scope of
$\genA$. This process gives us the output type $\tau_2$, and output context
$\ctxl_1, \genA, \ctxl_2$. Finally, $\tau_2$ could also contain variables whose
order matters, so we use $\ctxl_1 \bywt \tau_2$ to make sure $\tau_2$ is well
scoped. Rule \rul{U-TyEVar} is symmetric to \rul{U-EVarTy}. Using
well-scopedness instead of well-formedness gets us rid of the dependency on
typing.

\paragraph{Example.}

The derivation of the unification problem $\tctx, \genA, \genB \bysuni \genA
\uni \bpi x \genB x$ is given below. For clarity, we denote $\ctxl =
\tctx,\genA_1,\genA,\genB =\genA_1$. And it is easy to verify $\genA \notin
FV(\bpi x \genB x)$.

\[
  \small
   \ExUni
\]

\subsection{Context Extension}
\label{sec:context-extension}

\begin{figure*}[t]
   \[\framebox{$\tctx \exto \ctxl$} \]
   \[\CEEmpty \quad \CEVar \quad \CEEVar \] \[\CESolvedEVar\]
   \[\CESolve \quad \CEAdd \]
   \[\CEAddSolved \]
  \caption{Context extension.}
  \label{fig:context-extension}
\end{figure*}

We mentioned that the algorithmic output context extends the input context with
new existential variables or more existential variables solved. To accurately
capture this kind of information increase, we present the definition of context
extension in Figure~\ref{fig:context-extension}. This definition is
similar to the one by \citet{dunfield2013complete}.

The empty context is an extension of itself (\rul{CE-Empty}). 
Variables or existential variables are preserved during the extension
(\rul{CE-Var}, \rul{CE-EVar}), while the solutions for existential variables can
be different only if they are equivalent under context application
(\rul{CE-SolvedEVar}). The definition in \citet{{dunfield2013complete}} further
allows the type annotation to change (in \rul{CE-Var}), which is not necessary for
our algorithm. The extension can also add solutions to unsolved existential
variables (\rul{CE-Solve}), or add new existential variables (\rul{CE-Add},
\rul{CE-AddSolved}).

\paragraph{Context application on contexts.}

Complete contexts $\cctx$ are contexts with all existential variables solved, as
defined in Section~\ref{subsec:language}. Applying a complete context to a
well-formed type $\sigma$ yields a type without existential variables $\applye
\cctx \sigma$. Similarly, given $\tctx \to \cctx$, we can get a context without
existential variables $\applye \cctx \tctx$.
The formal
definition of context application is in
Figure~\ref{fig:context-application-on-context}.

\begin{figure*}[t]
  \centering
  \begin{tabular}{rlp{2cm}rll}
    $\applye {\emptyset} \emptyset$ & = & $\emptyset$ &
    $\applye {\cctx, x: \tau} {\tctx, x: \tau}$ & = & $\applye \cctx \tctx, x : \applye \cctx {\tau} $
    \\
    $\applye {\cctx, \genA = \tau} {\tctx, \genA}$ & = & $\applye \cctx \tctx$ &
    $\applye {\cctx, \genA = \tau_1} {\tctx, \genA = \tau_2}$ & = & $\applye \cctx \tctx$
                                    \quad If $\applye \cctx {\tau_1} = \applye \cctx {\tau_2}$ \\
    $\applye {\cctx, \genA = \tau_1} {\tctx}$ & = & $\applye \cctx \tctx$ \\
  \end{tabular}
    \caption{Context application.}
    \label{fig:context-application-on-context}
\end{figure*}

\subsection{Soundness}

Our overall framework for proofs is quite similar as
\citet{dunfield2013complete}. However, unlike their work which always implicitly assumes
that contexts and types are well-formed, we put extra efforts on
dealing with the well-formedness and the typing explicitly since we
have to resolve the dependency carefully.

We proved that our type sanitization strategy and the unification algorithm
are sound.
First, we show that except for resolving the order problem, type sanitization will not change the
type. Namely, the input type and the output type are equivalent after
substitution by the output context:

\begin{lemma}[\TypeSanitizationEquivalenceName]
  \TypeSanitizationEquivalenceBody
\end{lemma}

Moreover, if the input type is well-formed under the input context, then the
output type is still well-formed under the output context:

\begin{lemma}[\TypeSanitizationWellFormednessName]
  \TypeSanitizationWellFormednessBody
\end{lemma}

Having those lemmas related to type sanitization,
we can prove that two input types are really unified by the
unification algorithm:

\begin{lemma}[\UnificationEquivalenceName]\leavevmode
  \UnificationEquivalenceBody
\end{lemma}

\subsection{Completeness}

We use the notation $\tctx \bywf \tau_1 = \tau_2$ to mean that
$\tctx \byinf \tau_1 \infto \sigma$, $\tctx \byinf \tau_2 \infto \sigma$ for
some $\sigma$,
and $\tau_1 = \tau_2$.

The completeness of type sanitization is proved by a more general lemma, which
can be found in the appendix. Here we present a more readable version:

\begin{corollary}[\TypeSanitizationCompletenessPrettyName]
  \TypeSanitizationCompletenessPrettyBody
\end{corollary}

Having the completeness of type sanitization, we are ready to prove that our
unification algorithm is complete:

\begin{lemma}[\UnificationCompletenessName]
    \UnificationCompletenessBody
\end{lemma}

%%% Local Variables:
%%% mode: latex
%%% TeX-master: "../main"
%%% org-ref-default-bibliography: "citation.bib"
%%% End: