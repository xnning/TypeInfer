\section{Higher Rank Polymorphic Type System}
\label{sec:higherrank}

In this section, we adopt the type sanitization strategy to a higher rank
polymorphic type system from \citet{dunfield2013complete}. We show that type
sanitization can be further extended to polymorphic type sanitization to deal
with subtyping, which can be used to replace the instantiation relation in
original system while preserving the completeness and subtyping.

\subsection{Language}

The syntax of \citet{dunfield2013complete} is given below. Notice that
notations from Section~\ref{sec:dependent} are reused here. We will always
make it clear from the context which system is
being referred. \\

\begin{tabular}{lrcl}
  Type & $A, B$ & \syndef & $\Unit \mid \varA \mid \genA \mid \forall \varA. A \mid A \to B $ \\
  Monotype & $\sigma, \tau$ & \syndef & $\Unit \mid \varA \mid \genA \mid \sigma \to \tau $ \\
  Contexts & $\tctx, \ctxl, \ctxr$ & \syndef & $\ctxinit \mid \tctx, \varA
                                               \mid \tctx, x: A
                                               \mid \tctx, \genA
                                               \mid \tctx, \genA = \tau
                                               \mid \tctx, \marker \genA $\\
  Complete Contexts & $\cctx$ & \syndef & $\ctxinit \mid \cctx, \varA
                                          \mid \cctx, x: A
                                          \mid \cctx, \genA = \tau
                                          \mid \cctx, \marker \genA $\\
\end{tabular}

\paragraph{Types.}
Types $A, B$ include unit type $\Unit$, type variables $\varA$, existential
variables $\genA$, polymorphic types $\forall \varA. A$ and function types $A
\to B$.

Monotypes $\sigma, \tau$ are a special kind of types without universal
quantifiers.

\paragraph{Contexts.}
Contexts $\tctx$ are an ordered list of type variables $\varA$, variables $x:
A$, existential variables $\genA$ and $\genA = \tau$, and special markers
$\marker \genA$ for scoping reasons.
All existential variables in complete contexts $\cctx$ are solved.


\subsection{Subtyping}

\begin{figure*}[t]
    \[\framebox{$\tctx \bysub A \tsub B \toctx$}\]
    \[\ADunInstL \quad \ADunInstR\]
  \caption{Algorithmic Subtyping (Original, incomplete).}
  \label{fig:subtyping}
\end{figure*}

The original definition of algorithmic subtyping is given in
Figure~\ref{fig:subtyping}. The judgment $\tctx \bysub A \tsub B \toctx$ is
interpreted as: given the context $\tctx$, $A$ is a subtype of $B$ under the
output context $\ctxl$. Due to the space limitation, we only present when
the left hand side (\rul{InstL}) or the right hand side (\rul{InstR}) is an
existential variable. In those cases, the subtyping leaves all the work to the instantiation
judgment ($\ist$), some of which we have seen in Section~\ref{sec:overview}.
The complete definition can be found in the appendix.

\subsection{Polymorphic Type Sanitization}

\begin{figure*}[t]
    \[\framebox{$\tctx \bysub A \tsub B \toctx$} \]
    \[\ADunSaL                                     \]
    \[\ADunSaR                                     \]

    \[\framebox{$\tctx[\genA] \bybsa A \sa \sigma \toctx$}\]
    \[\IAllPlus \quad \IAllMinus    \]
    \[\IPiPoly                \]
    \[\IUnit \quad \IEVarAfterPoly \]
    \[\ITVar \quad \IEVarBeforePoly     \]
  \caption{New subtyping rules, polymorphic type sanitization.}
  \label{fig:polymorphic-sanitization}
\end{figure*}

The goal of polymorphic type sanitization is to replace the instantiation
relationship with even simpler rules.

Therefore, we replace the rules \rul{InstL} and \rul{InstR} with new rules
\rul{SaL} and \rul{SaR} shown at the top of
Figure~\ref{fig:polymorphic-sanitization}. Namely, the subtyping leaves the job
to polymorphic type sanitization to resolve the order problem with existential
variables and also sanitize the input type to a monotype. A final check
ensures that the sanatized type is well-formed under the context before $\genA$.
If the type being sanitized appears on the right (as \rul{SaL}), we say it
appears co-variantly, and we say it appears contra-variantly if it appears on
the left (as \rul{SaR}). This corresponds to the fact that a monotype
can be a subtype of a polymorphic type only if all the universal quantifiers
that appear in the body are contra-variant in the type.

The rules for polymorphic type sanitization are shown at the bottom of
Figure~\ref{fig:polymorphic-sanitization}. According to whether the type appears
co-variantly ($s = -$) or contra-variantly ($s = +$), we have two modes. The
judgment $\tctx[\genA] \bybsa A \sa \sigma \toctx$ is interpreted as: under
typing context $\tctx$ which contains $\genA$, sanitize a possibly polymorphic
type $A$ to a monotype
$\sigma$ according to the mode $s$, with output context $\ctxl$. Computationally, there are four inputs
($\tctx, s, \genA$ and $A$), and two outputs ($\sigma$ and $\ctxl$).

The only difference between these two modes is how to sanitize polymorphic
types. If a polymorphic type appears contra-variantly (\rul{I-All-Plus}), it
means a monotype would make the final type more polymorphic. Therefore, we
replace the universal binder $\varA$ with a fresh existential variable $\genB$
and put it before $\genA$. Otherwise, in rule \rul{I-All-Minus}, we put $\varA$
in the context and sanitize $A$. Notice that the result $\sigma$ might not be
well-formed under the output context $\ctxl$, since $\varA$ is discarded in the
output context. Rule \rul{I-Pi-Poly} is where the mode is flipped.

Polymorphic type sanitization does nothing if it is a unit type (\rul{I-Unit})
or a type variable (\rul{I-TVar}). Rule \rul{I-EVarAfter-Poly} and
\rul{I-EVarBefore-Poly} deals with existential variables, and creates fresh
existential variables if the input existential variable appears after $\genA$,
as we have seen in type sanitization in Section~\ref{subsec:unification}.

\paragraph{Example}

The derivation of the subtyping problem $\genA \bysub \genA \tsub (\forall
\varA. \varA \to \varA) \to \Unit$ is given below. For clarity, we omit some
detailed process, and denote $\ctxl = \genB, \genA$.

\[
  \small
\ExSub
\]

\subsection{Meta-theory}

The soundness and completeness of subtyping relies on the soundness and
completeness of instantiation in \citet{dunfield2013complete}. To prove
polymorphic type sanitization works correctly, there is no need to re-prove the
lemmas about subtyping. Instead, we prove that polymorphic type sanitization
leads to exactly the same result as instantiation.

For soundness, we prove that
under contra-variant mode ($s = +$), the input type is a
declarative subtype of the output type after substituted by a complete context,
while under co-variant mode ($s = -$), the input type is a declarative supertype:

\begin{lemma}[\PolymorphicTypeSanitizationSoundnessName]\leavevmode
  \label{lemma:\PolymorphicTypeSanitizationSoundnessName}
  \PolymorphicTypeSanitizationSoundnessBody
\end{lemma}

For completeness, we first prove that for a possibly polymorphic type $\applye
\cctx A$, if there is a monotype $\tau$ that is more polymorphic than it, there
is a polymorphic type sanitization result $\sigma$ of $A$, and it is the most
general subtype in the sense that we can recover the $\tau$ from applying a
complete context to $\sigma$.

\begin{lemma}[\PolymorphicTypeSanitizationCompletenessName]
  \label{lemma:\PolymorphicTypeSanitizationCompletenessName}
  \PolymorphicTypeSanitizationCompletenessBody
\end{lemma}

Based on the completeness of polymorphic type sanitization, we can prove exactly
the same completeness lemma as the instantiation to show that subtyping for
existential variables is complete:

\begin{corollary}[\PolymorphicTypeSanitizationCompletenessSubtypingName]
  \label{lemma:\PolymorphicTypeSanitizationCompletenessSubtypingName}
  \PolymorphicTypeSanitizationCompletenessSubtypingBody
\end{corollary}

%%% Local Variables:
%%% mode: latex
%%% TeX-master: "../main"
%%% org-ref-default-bibliography: "citation.bib"
%%% End: