\newpage
\section{Dependent Type System}

\subsection{Properties of Context Application}

\begin{lemma}[\ContextApplicationIsIdempotentName]
  \label{lemma:\ContextApplicationIsIdempotentName}
  \ContextApplicationIsIdempotentBody
\end{lemma}
\proof
By induction on the well formedness of context. Case \rul{AC-Empty},
\rul{AC-Var}, and \rul{AC-EVar} are trivial. We discuss \rul{AC-SolvedEVar} below.

\begin{itemize}
  \item Case \[\ACSolvedEVar\]
    \begin{longtable}[l]{lll}
      & $ \applye {\tctx, \genA = \tau} {\applye {\tctx, \genA = \tau} \sigma} =
      \applye {\tctx, \genA = \tau} {\applye \tctx {\sigma \subst \genA
          \tau}} $ & By definition \\
      & $\tctx \byinf \tau \infto \star$ & Given \\
      & $\genA \notin \tctx$ & Given \\
      & $\genA \notin FV(\tau)$ & By above propositions \\
      & $\genA \notin FV(\sigma \subst \genA \tau)$ & Follows directly \\
      & $\genA \notin FV(\applye \tctx {\sigma \subst \genA \tau})$ & Follows
      directly \\
      & $\applye {\tctx, \genA = \tau} {\applye \tctx {\sigma \subst \genA
          \tau}} = \applye {\tctx} {\applye \tctx {\sigma \subst \genA \tau}}$ &
      Substitute fresh variable \\
      & $ = \applye \tctx {\sigma \subst \genA \tau}$ & By induction hypothesis
      \\
      & $ = \applye {\tctx, \genA = \tau} \sigma$ & By definition of context application
    \end{longtable}
\end{itemize}
\qed

\begin{lemma}[\ReductionPreservesFullySubstitutionName]
  \label{lemma:\ReductionPreservesFullySubstitutionName}
  \ReductionPreservesFullySubstitutionBody
\end{lemma}
\proof
Follows directly from the definition of context substitution and reduction.
\qed

\begin{lemma}[\ContextApplicationOverReductionName]
  \label{lemma:\ContextApplicationOverReductionName}
  \ContextApplicationOverReductionBody
\end{lemma}
\proof

By induction on the reduction derivation.

\begin{itemize}
  \item Case \[\RApp\]
    \begin{longtable}[l]{lll}
      & $\applye \tctx {e_1} \redto \applye \tctx {e_1'}$& By induction
      hypothesis \\
      & $\applye \tctx {e_1~ e_2}$ & \\
      & $ = \applye \tctx {e_1} ~ \applye \tctx {e_2}$ & By property of
      substitution  \\
      & $\redto \applye \tctx {e_1'} ~ \applye \tctx {e_2}$& By \rul{R-App} \\
      & $= \applye \tctx {e_1' ~ e_2}$& By property of substitution
    \end{longtable}
  \item Case \[\RBeta\]
    \begin{longtable}[l]{lll}
      & $\applye \tctx {(\blam x \sigma {e_1}) ~ e_2}$ & \\
      & $= {(\blam x {\applye \tctx \sigma} {\applye \tctx {e_1}}) ~
        \applye \tctx {e_2}}$ & By
      definition of substitution \\
      & $\redto (\applye \tctx {e_1}) \subst x {\applye \tctx {e_2}}$& By
      \rul{R-Beta} \\
      & $= \applye \tctx {e_1 \subst x {e_2}}$& By property of substitution
    \end{longtable}
  \item Case \[\RCastDown\]
    Follows directly from induction hypothesis.
  \item Case \[\RCastDownUp\]
    Follows directly from induction hypothesis.
\end{itemize}

\qed

\begin{lemma}[\OutputIsFullySubstitutedName]
  \label{lemma:OutputIsFullySubstituted}
  \OutputIsFullySubstitutedBody
\end{lemma}
\proof
By induction on typing derivation.
\begin{itemize}
  \item Case \rul{A-Ax}, \rul{A-EVar}, \rul{A-SolvedEVar}, \rul{A-Pi} follows
    directly from $\applye \tctx \star = \star$.
  \item Case \[\AVar\]
    \begin{longtable}[l]{lll}
    & $\applye \tctx {\applye \tctx \sigma} = \applye \tctx \sigma$ & By
    Lemma~\ref{lemma:\ContextApplicationIsIdempotentName}\\
    \end{longtable}
  \item Case \[\ALamAnn\]
    \begin{longtable}[l]{lll}
      & $\applye \tctx {\bpi x {\applye \tctx {\sigma_1}} {\sigma_2}} =
      \bpi x {\applye \tctx {\applye \tctx {\sigma_1}}} {\applye \tctx
        {\sigma_2}}$
      & Follows from definition of context substitution \\
      &$=\bpi x {\applye \tctx {\sigma_1}} {\applye \tctx {\sigma_2}}$& By Lemma~\ref{lemma:\ContextApplicationIsIdempotentName}\\
      &$\applye {\tctx, x: \sigma_1} {\sigma_2} = \sigma_2$ & By hypothesis\\
      &$\applye {\tctx} {\sigma_2} = \sigma_2$& Follows from definition of
      context substitution\\
      &$\applye \tctx {\bpi x {\applye \tctx {\sigma_1}} {\sigma_2}} =
      {\bpi x {\applye \tctx {\sigma_1}} {\sigma_2}}
      $ & By above equalities
    \end{longtable}
  \item Case \[\AApp\]
    \begin{longtable}[l]{lll}
      &$\applye \tctx {\bpi x {\sigma_1} {\sigma_2}} = \bpi x {\sigma_1}
      {\sigma_2}$ & By hypothesis\\
      &$\applye \tctx {\sigma_2} = \sigma_2$& Follows directly from above\\
      &$\applye \tctx {\applye \tctx e} = \applye \tctx e$& By
      Lemma~\ref{lemma:\ContextApplicationIsIdempotentName} \\
      &$\applye \tctx {\sigma_2 \subst x {\applye \tctx {e_1}}}$ & \\
      &$=\applye \tctx {\sigma_2} \subst x {\applye \tctx {\applye \tctx {e_1}}}$&
      Context application is distributed over substitution \\
      &$={\sigma_2} \subst x {\applye \tctx{e_1}}$& By above equalities
    \end{longtable}
  \item Case \[\ACastDn\]
    \begin{longtable}[l]{lll}
      & $\applye \tctx {\sigma_1} = {\sigma_1}$ & By hypothesis \\
      & $\applye \tctx {\sigma_2} = {\sigma_2}$ & By Lemma~\ref{lemma:\ReductionPreservesFullySubstitutionName} \\
    \end{longtable}
  \item Case \[\ACastUp\]
    Follows directly from Lemma~\ref{lemma:\ContextApplicationIsIdempotentName}
\end{itemize}
\qed

\begin{lemma}[\ContextApplicationInContextName]
  \label{lemma:\ContextApplicationInContextName}
  \ContextApplicationInContextBody
\end{lemma}
\proof
By induction on the typing relation.
\begin{itemize}
\item Case \rul{A-AX}, \rul{A-EVar} and \rul{A-SolvedEVar} follows directly.
\item Case \[\AVar\]
  If $x = y$, then result type is $\applye {\tctx} {\applye \tctx \tau} =
  \applye \tctx \tau$ by Lemma~\ref{lemma:\ContextApplicationIsIdempotentName}.
  Otherwise the result is the same.
\item The rest cases follows directly from the hypothesis.
\end{itemize}
\qed

\begin{definition}[Typing Size], The size of typing derivation
  $\tctx \byinf \sigma \infto \tau$,
  defined based on typing process in Figure~\ref{},
  written as $\sizet {\tctx \byinf \sigma \infto \tau}$,
  is defined as:
  \begin{center}
  \begin{tabular}{rll}
    $\sizet {\tctx \byinf \star \infto \star}$ & = & 1 \\
    $\sizet {\tctx \byinf x \infto \applye \tctx \sigma}$ & = & 1 \\
    $\sizet {\tctx \byinf \genA \infto \star}$ & = & 1 if $\genA \in \tctx$\\
    $\sizet {\tctx \byinf \genA \infto \star}$ & = & $1 + \sizet{\tctx_1 \byinf \tau \infto \star}$ if $\tctx = \tctx_1, \genA = \tau, \tctx2$\\
    $\sizet {\tctx \byinf \blam x {\sigma_1} e \infto \bpi x {\applye {\tctx} {\sigma_1}} {\sigma_2}}$
                                               & = &
                                                     $1 + \sizet{\tctx \byinf \sigma_1 \infto \star}
                                                     + \sizet{\tctx, x : \sigma_1 \byinf e \infto \sigma_2}$  \\
    $\sizet{\tctx \byinf \bpi x {\sigma_1} {\sigma_2} \infto \star}$
                                               & = &
                                                     $1 + \sizet{\tctx \byinf \sigma_1 \infto \star}
                                                     + \sizet{\tctx, x : \sigma_1 \byinf \sigma_2 \infto \star}
                                                     $\\
    $\sizet{\tctx \byinf \castdn e \infto \sigma_2}$
                                               & = &
                                                     $1 + \sizet{\tctx \byinf e \infto \sigma_1}$ \\
    $\sizet{\tctx \byinf \castup e \infto \applye {\tctx} {\sigma_1}}$
                                               & = &
                                                     $1 + \sizet {\tctx \byinf e \infto \sigma_2} $
  \end{tabular}
  \end{center}
\end{definition}

\begin{lemma}[\ContextApplicationPreservesTypingName]
  \label{lemma:\ContextApplicationPreservesTypingName}
  \ContextApplicationPreservesTypingBody
\end{lemma}
\proof
By induction on the typing size.

\begin{itemize}
  \item Case \[\AAx\]
    \begin{longtable}[l]{lll}
      & $ \applye \tctx \star = \star$ & Directly from the definition of context
      application\\
      & $ \tctx \byinf \star \infto \star$ & By \rul{A-Ax}
    \end{longtable}
  \item Case \rul{AVar}, \rul{AEVar} are similar as \rul{AAx}, which follows
    from definition directly.
  \item Case \[\ASolvedEVar\]
    \begin{longtable}[l]{lll}
      & let $\tctx = \tctx_1, \genA = \tau, \tctx_2$& \\
      & $\applye \tctx \genA = \applye {\tctx_1} \tau$& Follows from definition of context
      application\\
      & $\tctx_1 \byinf \applye {\tctx_1} \tau \infto \star$& By hypothesis\\
      & $\tctx_1, \genA = \tau, \tctx_2 \byinf \applye {\tctx_1} \tau \infto
      \star$ & By Lemma~\ref{lemma:\TypingWeakeningName} \\
    \end{longtable}
  \item Case \[\ALamAnn\]
    \begin{longtable}[l]{lll}
      & $\applye \tctx {\blam x {\sigma_1} e} = \blam x {\applye \tctx
        {\sigma_1}} {\applye \tctx e} $ & Follows from definition of context application \\
      & $\tctx \byinf \applye \tctx {\sigma_1} \infto \star$ & By hypothesis \\
      & $\tctx, x : \sigma_1 \byinf \applye {\tctx, x: \sigma_1} e \infto \sigma_2$ & By hypothesis \\
      & $\tctx, x : \sigma_1 \byinf \applye {\tctx} e \infto \sigma_2$ & By
      definition of context application \\
      & $\tctx, x : \applye \tctx {\sigma_1} \byinf \applye {\tctx} e \infto
      \sigma_2$ & By Lemma~\ref{lemma:\ContextApplicationInContextName}\\
    \end{longtable}

  \item Case \[\APi\]
    \begin{longtable}[l]{lll}
      & $\applye \tctx {\bpi x {\sigma_1} {\sigma_2}} = \bpi x {\applye \tctx
        {\sigma_1}} {\applye \tctx {\sigma_2}} $ & Follows from definition of context application \\
      & $\tctx \byinf \applye \tctx {\sigma_1} \infto \star$ & By hypothesis \\
      & $\tctx, x : \sigma_1 \byinf \applye {\tctx, x: \sigma_1} {\sigma_2} \infto \star$ & By hypothesis \\
      & $\tctx, x : \sigma_1 \byinf \applye {\tctx} {\sigma_2} \infto \star$ & By
      definition of context application \\
      & $\tctx, x : \applye \tctx {\sigma_1} \byinf \applye {\tctx} {\sigma_2}
      \infto \star$ & By Lemma~\ref{lemma:\ContextApplicationInContextName}\\
    \end{longtable}
  \item Case \rul{A-App}, \rul{A-CastDn}, \rul{A-CastUp} follows directly from
    the hypothesis.
\end{itemize}
\qed

\subsection{Properties of Declarative Typing}

\begin{lemma}[\TypingContextWellFormednessName]
  \label{lemma:\TypingContextWellFormednessName}
  \TypingContextWellFormednessBody
\end{lemma}
\proof

By a straightforward induction on the typing derivation.

\qed

\begin{lemma}[\TypingWeakeningName]
  \label{lemma:\TypingWeakeningName}
  \TypingWeakeningBody
\end{lemma}
\proof

By induction on the typing derivation.
\begin{itemize}
  \item Case \rul{A-Ax}, \rul{A-Var}, \rul{A-EVar} and \rul{A-SolvedEVar}
    follows directly.
  \item Case \[\ALamAnn\]
    \begin{longtable}[l]{lll}
      &$\tctx_1, \ctxl, \tctx_2 \byinf \sigma_1 \infto \star$ & By hypothesis\\
      &$\tctx_1, \ctxl, \tctx_2, x:\sigma_1 \wc$& Follows from the definition of
      context well formedness\\
      &$\tctx_1, \ctxl, \tctx_2, x: \sigma_1 \byinf e \infto \sigma_2$& By
      hypothesis
    \end{longtable}
  \item Case \rul{A-Pi} is similar as \rul{A-LamAnn}.
    \item Case \rul{A-App}, \rul{A-CastDn} and \rul{A-CastUp} follows directly
      from hypothesis.
\end{itemize}
\qed

\begin{lemma}[\TypingStrengtheningName]
  \label{lemma:\TypingStrengtheningName}
  \TypingStrengtheningBody
\end{lemma}
\proof
By induction on the typing derivation.

\begin{itemize}
  \item Case \rul{A-Ax}, \rul{A-EVar} and \rul{A-SolvedEVar} follows directly
    from typing.
  \item Case \[\AVar\]
    Because we know $x \notin FV(\tau)$, so $\tau = y$, where $y \neq x$ and $y
    :\sigma \in \tctx_1, \tctx_2$. Follows by \rul{A-Var}.
  \item All rest cases follow from induction hypothesis directly.
\end{itemize}
\qed

\begin{lemma}[\TypingVariableExchangeName]
  \label{lemma:\TypingVariableExchangeName}
  \TypingVariableExchangeBody
\end{lemma}
\proof

By straightforward induction on the typing derivation.

\qed

\begin{lemma}[\TypingSubstitutionName]
  \label{lemma:\TypingSubstitutionName}
  \TypingSubstitutionBody
\end{lemma}
\proof

By induction on the size of second typing derivation.

\begin{itemize}
  \item Case \rul{A-Ax}, \rul{A-EVar} and \rul{A-SolvedEVar} follows from typing
    rules directly.
  \item Case \[\AVar\]
    If two $x$'s are the same, then $\sigma_2 = \applye \tctx {\sigma_1}$. The
    goal follows directly from given conditions. If two variables are not the
    same, then it follows directly from \rul{A-Var}.
  \item Case \[\ALamAnn\]
    Let the expression be $\blam y {\sigma'} e$ to avoid name conflicts.
    \begin{longtable}[l]{lll}
      & $\tctx \byinf \sigma' \subst x \tau \infto \star$ & By induction hypothesis \\
      & $\tctx, x:\sigma_1, y: \sigma' \subst x \tau \byinf e \infto \sigma_3$ & Given, notice
      we use $\sigma_3$ to avoid name conflicts \\
      & $\tctx \byinf \sigma_1 \infto \star$ & Given \\
      & $\tctx, y: \sigma' \subst x \tau \byinf \sigma_1 \infto \star$ & By
      Lemma~\ref{lemma:\TypingWeakeningName}\\
      & $\tctx, y: \sigma' \subst x \tau, x:\sigma_1 \wc$ & By above hypothesis \\
      & $\tctx, y: \sigma' \subst x \tau, x:\sigma_1 \byinf e \infto \sigma_3$ &
      By Lemma~\ref{lemma:\TypingVariableExchangeName} \\
      & $\tctx \byinf \tau \infto \applye \tctx {\sigma_1}$ & Given \\
      & $\tctx, y: \sigma_1 \subst x \tau \byinf \tau \infto \applye \tctx
      {\sigma_1}$ & By Lemma~\ref{lemma:\TypingWeakeningName}  \\
      & $\tctx, y: \sigma_1 \subst x \tau \byinf e \subst x \tau \infto
      \sigma_3$ & By induction hypothesis \\
    \end{longtable}
  \item Case \rul{A-Pi} is similar as \rul{A-LamAnn}.
  \item Rest cases follow directly from hypothesis.
\end{itemize}
\qed

Type Safety

\subsection{Properties of Context Extension}

\begin{lemma}[\DeclarationPreservationName]
  \label{lemma:\DeclarationPreservationName}
  \DeclarationPreservationBody
\end{lemma}
\proof

By a straightforward induction on the context application.

\qed

\begin{lemma}[\ReverseDeclarationPreservationName]
  \label{lemma:\ReverseDeclarationPreservationName}
  \ReverseDeclarationPreservationBody
\end{lemma}
\proof

By a straightforward induction on the context application.

\qed


\begin{lemma}[\DeclarationOrderPreservationName]
  \label{lemma:\DeclarationOrderPreservationName}
  \DeclarationOrderPreservationBody
\end{lemma}
\proof

By induction on the context application.
\begin{itemize}
  \item Case \[\CEEmpty\]
    Impossible case.
  \item Case \[\CEVar\]
    \begin{itemize}
    \item SubCase $v = x$.
      \begin{longtable}[l]{lll}
        & $u \in \tctx$ & Given \\
        & $u \in \ctxr$ & By Lemma~\ref{lemma:\DeclarationPreservationName} \\
        & So $u$ is to the left of $x$ in $\tctx, x$ &
      \end{longtable}
    \item SubCase $v \neq x$.
      \begin{longtable}[l]{lll}
        & $u$ is declared to the left of $v$ in $\tctx$ & Given \\
        & $u$ is declared to the left of $v$ in $\ctxr$ & By induction hypothesis \\
        & $u$ is declared to the left of $v$ in $\ctxr, x: \sigma$ &
      \end{longtable}
    \end{itemize}
  \item Case \rul{CE-EVar}, \rul{CE-SolvedEVar}, \rul{CE-Solve} are
    similar as Case \rul{CE-Var}.
  \item Case \[\CEAdd\]
      \begin{longtable}[l]{lll}
        & $u$ is declared to the left of $v$ in $\ctxr$ & By induction
        hypothesis \\
        & So $u$ is declared to the left of $v$ in $\ctxr, \genA$ &
      \end{longtable}
  \item Case \[\CEAddSolved\]
      \begin{longtable}[l]{lll}
        & $u$ is declared to the left of $v$ in $\ctxr$ & By induction
        hypothesis \\
        & So $u$ is declared to the left of $v$ in $\ctxr, \genA = \tau$ &
      \end{longtable}
\end{itemize}
\qed

\begin{lemma}[\ReverseDeclarationOrderPreservationName]
  \label{lemma:\ReverseDeclarationOrderPreservationName}
  \ReverseDeclarationOrderPreservationBody
\end{lemma}
\proof

We can prove this lemma by contradiction. Suppose $u$ is declared to the right
of $v$ in $\tctx$, then by Lemma~\ref{lemma:\DeclarationOrderPreservationName},
we have that $u$ is declared to the right of $v$ in $\tctx$. Because we already
have $u$ is declared to the left of $v$, we have a contradiction. Therefore, $u$
is declared to the left of $v$ in $\tctx$.

\qed

\begin{lemma}[\SubstitutionExtensionInvarianceName]
  \label{lemma:\SubstitutionExtensionInvarianceName}
  \SubstitutionExtensionInvarianceBody
\end{lemma}

\proof

We first prove $\applye \ctxr \sigma = \applye \tctx {\applye \ctxr \sigma}$,
then prove $\applye \ctxr \sigma = \applye \ctxr {\applye \tctx \sigma}$.

\begin{description}
\item [Part 1]
  We do induction on the context extension.
  \begin{itemize}
    \item Case \[\CEEmpty\]
      Trivial case.
    \item Case \[\CEVar\]
      We use $\sigma_1$ to replace the $\sigma$ in \rul{CE-Var}.
      \begin{longtable}[l]{lll}
        & $\applye {\ctxr, x: \sigma_1} \sigma $ & \\
        & $= \applye \ctxr \sigma$ & Follows by definition of context application
        \\
        & $= \applye \tctx {\applye \ctxr \sigma}$ & By induction hypothesis \\
        & $= \applye {\tctx, x: \sigma_1} {\applye {\ctxr, x: \sigma_1} \sigma}$& By definition of context application
      \end{longtable}
    \item Case \[\CEEVar\]
      Similar as Case \rul{CE-Var}.
    \item Case \[\CESolvedEVar\]
      \begin{longtable}[l]{lll}
        & $\applye {\ctxr, \genA = \tau} \sigma $ & \\
        & $ = \applye {\ctxr} {\sigma \subst \genA \tau} $ & By definition of
        context application \\
        & $ = \applye \tctx {\applye {\ctxr} {\sigma \subst \genA \tau}} $ & By
        induction hypothesis \\
        & $ = \applye {\tctx, \genA = \tau} {\applye {\ctxr} {\sigma \subst \genA \tau}}$ & $\genA \notin FV(\applye \ctxr {\sigma \subst \genA \tau})$ \\
        & $ = \applye {\tctx, \genA = \tau} {\applye {\ctxr, \genA = \tau}
          \sigma}$ & By definition of context application
      \end{longtable}
    \item Case \[\CESolve\]
      \begin{longtable}[l]{lll}
        & $\applye {\ctxr, \genA = \tau} \sigma $ & \\
        & $ = \applye {\ctxr} {\sigma \subst \genA \tau} $ & By definition of
        context application \\
        & $ = \applye \tctx {\applye {\ctxr} {\sigma \subst \genA \tau}} $ & By
        induction hypothesis \\
        & $ = \applye {\tctx, \genA} {\applye {\ctxr} {\sigma \subst \genA
            \tau}} $ & By definition of context application\\
        & $ = \applye {\tctx, \genA} {\applye {\ctxr, \genA = \tau} {\sigma}} $
        & By definition of context application
      \end{longtable}
    \item Case \[\CEAdd\]
      \begin{longtable}[l]{lll}
        & $\applye {\ctxr, \genA} \sigma $ & \\
        & $ = \applye {\ctxr} \sigma $ & By definition of
        context application \\
        & $ = \applye \tctx {\applye {\ctxr} \sigma} $ & By
        induction hypothesis \\
        & $ = \applye \tctx {\applye {\ctxr, \genA} \sigma} $ & By definition of
        context application
      \end{longtable}
    \item Case \[\CEAddSolved\]
      \begin{longtable}[l]{lll}
        & $\applye {\ctxr, \genA = \tau} \sigma $ & \\
        & $ = \applye {\ctxr} {\sigma \subst \genA \tau} $ & By definition of
        context application \\
        & $ = \applye \tctx {\applye {\ctxr} {\sigma \subst \genA \tau}} $ & By
        induction hypothesis \\
        & $ = \applye \tctx {\applye {\ctxr, \genA = \tau} \sigma} $ & By
        definition of context application
      \end{longtable}
  \end{itemize}


\item [Part 2]
  By induction on the size of the typing derivation.

  \begin{itemize}
  \item Case \[\AAx\]
    Follows directly by for all context $\tctx$, $\applye \tctx \star = \star$.
  \item Case \[\AVar\]
    Follows directly by for all context $\tctx$, $\applye \tctx x = x$.
  \item Case \[\AEVar\]
    \begin{longtable}[l]{lll}
      & $\applye \tctx \genA = \genA$& \\
      & $\applye \ctxr {\applye \tctx \genA} = \applye \ctxr \genA$ & Follows
      directly \\
    \end{longtable}
  \item Case \[\ASolvedEVar\]
    \begin{longtable}[l]{lll}
      & $\tctx \exto \ctxr$ & Given \\
      & $\genA = \tau \in \tctx$ & Given \\
      & $\ctxr = \ctxr_1, \genA = \tau, \ctxr_2$ & By inversion \\
      & $\applye \ctxr \genA$ & \\
      & $= \applye \ctxr \tau$ & By definition of context application \\
      & $= \applye \ctxr {\applye \tctx \tau}$ & By induction hypothesis \\
      & $= \applye \ctxr {\applye \tctx \genA}$ & By definition of context application
    \end{longtable}
  \item Case \[\ALamAnn\]
    \begin{longtable}[l]{lll}
      & $\applye \ctxr {\sigma} = \applye \ctxr {\applye \tctx {\sigma_1}}$& By
      induction hypothesis \\
      & $\tctx \exto \ctxr$ & Given \\
      & $\tctx, x: \sigma_1 \exto \ctxr, x : \sigma_1 $ & By \rul{CE-Var}\\
      & $\applye {\ctxr, x: \sigma_1} e = \applye {\ctxr, x: \sigma_1} {\applye
        {\tctx, x: \sigma_1} e}$ & By induction hypothesis \\
      & $\applye \ctxr e = \applye \ctxr {\applye \tctx e}$ & By definition of
      context application \\
    \end{longtable}
  \item Case \[\APi\]
    Similar as Case \rul{A-LamAnn}.
  \item Case \[\AApp\]
    Follows directly from induction hypothesis.
  \item Case \[\ACastDn\]
    Follows directly from induction hypothesis.
  \item Case \[\ACastUp\]
    Follows directly from induction hypothesis.
\end{itemize}
\end{description}

\qed

\begin{lemma}[\ExtensionEqualityPreservationName]
  \label{lemma:\ExtensionEqualityPreservationName}
  \ExtensionEqualityPreservationBody
\end{lemma}

\proof

\mbox{} % an empty line to make sure long table appear after proof
\begin{longtable}[l]{lll}
  & $\applye \ctxr {\sigma_1}$ & \\
  & $= \applye \ctxr {\applye \tctx {\sigma_1}}$& By
  Lemma~\ref{lemma:\SubstitutionExtensionInvarianceName}\\
  & $= \applye \ctxr {\applye \tctx {\sigma_2}}$ & Given \\
  & $= \applye \ctxr {\sigma_2}$ & By
  Lemma~\ref{lemma:\SubstitutionExtensionInvarianceName}
\end{longtable}

\qed

\begin{lemma}[\ContextExtensionReflexivityName]
  \label{lemma:\ContextExtensionReflexivityName}
  \ContextExtensionReflexivityBody
\end{lemma}

\proof

By induction on the well-formedness of context.

\begin{itemize}
  \item Case \[\ACEmpty\]
    Follows directly from \rul{CE-Empty}.
  \item Case \[\ACVar\]
    \begin{longtable}[l]{lll}
      & $\tctx \exto \tctx$ & By induction hypothesis \\
      & $\tctx, x : \sigma \exto \tctx, x : \sigma$ & By \rul{CE-Var}
    \end{longtable}
  \item Case \[\ACEVar\]
    \begin{longtable}[l]{lll}
      & $\tctx \exto \tctx$ & By induction hypothesis \\
      & $\genA \notin \tctx$ & Given \\
      & $\tctx, \genA \exto \tctx, \genA$ & By \rul{CE-EVar}
    \end{longtable}
  \item Case \[\ACSolvedEVar\]
    \begin{longtable}[l]{lll}
      & $\tctx \exto \tctx$ & By induction hypothesis \\
      & $\genA \notin \tctx$ & Given \\
      & $\tctx, \genA = \tau \exto \tctx, \genA = \tau$ & By \rul{CE-SolvedEVar}
    \end{longtable}
\end{itemize}

\qed

\begin{lemma}[\ContextExtensionTransitivityName]
  \label{lemma:\ContextExtensionTransitivityName}
  \ContextExtensionTransitivityBody
\end{lemma}

\proof

By induction on the derivation $\tctx \exto \ctxr$.

\begin{itemize}
  \item Case \[\CEEmpty\]
    Our goal $\ctxl \exto \ctxinit $ is given.
  \item Case \[\CEVar\]
    \begin{longtable}[l]{lll}
      & $\ctxl \exto \tctx, x: \sigma$ & Given \\
      & $\ctxl = \ctxl', x: \sigma$ & By inversion \\
      & $\ctxl' \exto \tctx$ & By inversion \\
      & $\ctxl' \exto \ctxr$ & By induction hypothesis \\
      & $\ctxl', x: \sigma \exto \ctxr, x: \sigma$ & By \rul{CE-Var} \\
      & $\ctxl \exto \ctxr, x: \sigma$ & Namely
    \end{longtable}
  \item Case \[\CEEVar\]
    We are given $\ctxl \exto \tctx, \genA$.
    By inversion, we have two subcases.
    \begin{itemize}
    \item SubCase  \[ \inferrule{
            \ctxl \exto \tctx
         \\ \genA \notin \tctx
            }{
            \ctxl, \genA \exto \tctx, \genA
            }\rname{CE-EVar}\]
      \begin{longtable}[l]{lll}
        & $\ctxl \exto \tctx$ & Given\\
        & $\ctxl \exto \ctxr$ & By induction hypothesis \\
        & $\ctxl, \genA \exto \ctxr, \genA$ & By \rul{CE-EVar} \\
      \end{longtable}
    \item SubCase \[\inferrule{
            \ctxl \exto \tctx
         \\ \genA \notin \tctx
            }{
            \ctxl \exto \tctx, \genA
            }\rname{CE-Add}\]
      \begin{longtable}[l]{lll}
        & $\ctxl \exto \tctx$ & Given\\
        & $\ctxl \exto \ctxr$ & By induction hypothesis \\
        & $\ctxl \exto \ctxr, \genA$ & By \rul{CE-Add} \\
      \end{longtable}
    \end{itemize}
  \item Case \[\CESolvedEVar\]
    We are given $\ctxl \exto \tctx, \genA= \tau$.
    By inversion, we have three subcases.
    \begin{itemize}
    \item SubCase \[\inferrule{
            \ctxl \exto \tctx
         \\ \genA \notin \tctx
            }{
            \ctxl, \genA = \tau \exto \tctx, \genA = \tau
            }\rname{CE-SolvedEVar} \]
      \begin{longtable}[l]{lll}
        & $\ctxl \exto \ctxr$ & By induction hypothesis \\
        & $\ctxl, \genA = \tau \exto \ctxr, \genA = \tau$ & By \rul{CE-SolvedEVar} \\
      \end{longtable}
    \item SubCase \[\inferrule{
            \ctxl \exto \tctx
         \\ \genA \notin \tctx
         \\ \tctx \bywf \tau
            }{
            \ctxl, \genA \exto \tctx, \genA = \tau
            }\rname{CE-Solve} \]
          \begin{longtable}[l]{lll}
            & $\ctxl \exto \tctx$ & Given \\
            & $\ctxl \exto \ctxr$ & By induction hypothesis \\
            & $\tctx \bywf \tau$ & Given \\
            & $\tctx \wc$ & By
            Lemma~\ref{lemma:\TypingContextWellFormednessName} \\
            & $\ctxr \wc$ & By
            Lemma~\ref{lemma:\ContextExtensionPreservesContextWellFormednessName}\\
            & $\ctxr \bywf \tau$ & By
            Corollary~\ref{lemma:\ExtensionWeakningWellFormednessName} \\
            & $\ctxl \exto \ctxr, \genA = \tau$ & By \rul{CE-AddSolved} \\
          \end{longtable}
    \end{itemize}
  \item Case \[\CESolve\]
    We are given $\ctxl \exto \tctx, \genA$.
    By induction, we have two subcases.
    \begin{itemize}
    \item SubCase \[\inferrule{
            \ctxl \exto \tctx
         \\ \genA \notin \tctx
            }{
            \ctxl, \genA \exto \tctx, \genA
            }\rname{CE-EVar} \]
      \begin{longtable}[l]{lll}
        & $\ctxl \exto \tctx$ & Given \\
        & $\ctxl \exto \ctxr$ & By induction hypothesis \\
        & $\ctxl, \genA \exto \ctxr, \genA = \tau$ & By \rul{CE-Solve} \\
      \end{longtable}
    \item SubCase \[\inferrule{
            \ctxl \exto \tctx
         \\ \genA \notin \tctx
            }{
            \ctxl \exto \tctx, \genA
            }\rname{CE-Add} \]
      \begin{longtable}[l]{lll}
        & $\ctxl \exto \tctx$ & Given \\
        & $\ctxl \exto \ctxr$ & By induction hypothesis \\
        & $\ctxl \exto \ctxr, \genA = \tau$ & By \rul{CE-AddSolved} \\
      \end{longtable}
    \end{itemize}
  \item Case \[\CEAdd\]
    \begin{longtable}[l]{lll}
      & $\ctxl \exto \tctx$ & Given \\
      & $\ctxl \exto \ctxr$ & By induction hypothesis \\
      & $\ctxl \exto \ctxr, \genA$ & By \rul{CE-Add} \\
    \end{longtable}
  \item Case \[\CEAddSolved\]
    \begin{longtable}[l]{lll}
      & $\ctxl \exto \tctx$ & Given \\
      & $\ctxl \exto \ctxr$ & By induction hypothesis \\
      & $\ctxl \exto \ctxr, \genA = \tau$ & By \rul{CE-Add} \\
    \end{longtable}
\end{itemize}

\qed

\begin{definition}[Softness]
  A Context $\tctx$ is soft if and only if it consists only of $\genA$ and
  $\genA = \tau$ declarations.
\end{definition}

\begin{lemma}[\RightSoftnessName]
  \label{lemma:\RightSoftnessName}
  \RightSoftnessBody
\end{lemma}

\proof

By induction on $\ctxl$, and apply rule \rul{CE-Add} and \rul{CE-AddSolved} as needed.

\qed

\begin{lemma}[\ExtensionOrderName]\leavevmode
  \label{lemma:\ExtensionOrderName}
  \ExtensionOrderBody
\end{lemma}

\proof

\begin{description}
\item [Part 1]
  By induction on the context extension $\tctx_L, y : \sigma, \tctx_R \exto
  \ctxr$.
  \begin{itemize}
    \item Case \[\CEEmpty\]
      Impossible case.
    \item Case \[\CEVar\]
      Depending on whether $x = y$, we have two subcases.
      \begin{itemize}
        \item SubCase $x = y$.
          Therefore $\tctx_R = \ctxr_R = \ctxinit$,
          and $\tctx_L = \tctx, \ctxr_L = \ctxr$.
          All goal follows directly.
        \item SubCase $x \neq y$.
          Then we have
          \[\inferrule{
            \tctx, y:\sigma, \tctx_R' \exto \ctxr
            }{
            \tctx, y:\sigma, \tctx_R', x:\sigma \exto \ctxr, x:\sigma
            }\rname{CE-Var}
          \]
          \begin{longtable}[l]{lll}
            & $\ctxr = \ctxr_L, y:\sigma, \ctxr_R'$ & By induction hypothesis \\
            & $\tctx_L \exto \ctxr_L$ & By induction hypothesis\\
            & $\tctx_R', x: \sigma$ & is not soft\\
            & $\ctxr_R', x: \sigma$ & is not soft
          \end{longtable}
        \end{itemize}
      \item Case \[\inferrule{
            \tctx_L, y: \sigma, \tctx_R' \exto \ctxr
            \\ \genA \notin \ctxr
          }{
            \tctx_L, y:\sigma, \tctx_R', \genA \exto \ctxr, \genA
          }\rname{CE-EVar} \]
        \begin{longtable}[l]{lll}
          & $\ctxr = \ctxr_L, y:\sigma, \ctxr_R'$ & By induction hypothesis \\
          & $\tctx_L \exto \ctxr_L$ & By induction hypothesis\\
          & $\tctx_R'$ is soft  iff $\ctxr_R'$ is soft & By induction
          hypothesis \\
          & $\tctx_R', \genA $ is soft  iff $\ctxr_R', \genA$ is soft &
          Follows directly
        \end{longtable}
      \item Case \[\inferrule{
            \tctx_L, y: \sigma, \tctx_R' \exto \ctxr
         \\ \genA \notin \ctxr
            }{
            \tctx_L, y:\sigma, \tctx_R', \genA = \tau \exto \ctxr, \genA = \tau
            }\rname{CE-SolvedEVar}\]
        \begin{longtable}[l]{lll}
          & $\ctxr = \ctxr_L, y:\sigma, \ctxr_R'$ & By induction hypothesis \\
          & $\tctx_L \exto \ctxr_L$ & By induction hypothesis\\
          & $\tctx_R'$ is soft  iff $\ctxr_R'$ is soft & By induction
          hypothesis \\
          & $\tctx_R', \genA = \tau$ is soft  iff $\ctxr_R', \genA = \tau$ is soft &
          Follows directly
        \end{longtable}
      \item Case \[\inferrule{
            \tctx_L, y: \sigma, \tctx_R' \exto \ctxr
         \\ \genA \notin \ctxr
         \\ \ctxr \bywf \tau
            }{
            \tctx_L, y: \sigma, \tctx_R', \genA \exto \ctxr, \genA = \tau
            }\rname{CE-Solve} \]
        \begin{longtable}[l]{lll}
          & $\ctxr = \ctxr_L, y:\sigma, \ctxr_R'$ & By induction hypothesis \\
          & $\tctx_L \exto \ctxr_L$ & By induction hypothesis\\
          & $\tctx_R'$ is soft  iff $\ctxr_R'$ is soft & By induction
          hypothesis \\
          & $\tctx_R', \genA $ is soft  iff $\ctxr_R', \genA = \tau$ is soft &
          Follows directly
        \end{longtable}
      \item Case \[\inferrule{
            \tctx_L, y:\sigma, \tctx_R \exto \ctxr
         \\ \genA \notin \ctxr
            }{
            \tctx_L, y:\sigma, \tctx_R \exto \ctxr, \genA
            }\rname{CE-Add}
        \]
        \begin{longtable}[l]{lll}
          & $\ctxr = \ctxr_L, y:\sigma, \ctxr_R'$ & By induction hypothesis \\
          & $\tctx_L \exto \ctxr_L$ & By induction hypothesis\\
          & $\tctx_R$ is soft  iff $\ctxr_R'$ is soft & By induction
          hypothesis \\
          & $\tctx_R $ is soft  iff $\ctxr_R', \genA$ is soft &
          Follows directly
        \end{longtable}
      \item Case \[
          \inferrule{
            \tctx_L, y :\sigma, \tctx_R \exto \ctxr
         \\ \genA \notin \ctxr
         \\ \ctxr \bywf \tau
            }{
            \tctx_L, y:\sigma, \tctx_R \exto \ctxr, \genA = \tau
            }\rname{CE-AddSolved} \]
        \begin{longtable}[l]{lll}
          & $\ctxr = \ctxr_L, y:\sigma, \ctxr_R'$ & By induction hypothesis \\
          & $\tctx_L \exto \ctxr_L$ & By induction hypothesis\\
          & $\tctx_R$ is soft  iff $\ctxr_R'$ is soft & By induction
          hypothesis \\
          & $\tctx_R $ is soft  iff $\ctxr_R', \genA = \tau$ is soft &
          Follows directly
        \end{longtable}
  \end{itemize}
  \item [Part 2] Similar to Part 1.
  \item [Part 3] Similar to Part 1.
\end{description}

\qed

\begin{lemma}[\ExtensionWeakningName]
  \label{lemma:\ExtensionWeakningName}
  \ExtensionWeakningBody
\end{lemma}

\proof

By induction on the typing derivation.

\begin{itemize}
\item Case \[\AAx\]
  Follows directly by \rul{A-Ax}.
\item Case \[\AVar\]
  \begin{longtable}[l]{lll}
    & $x : \sigma \in \tctx$ & Given \\
    & $x : \sigma \in \ctxr$ & By Lemma~\ref{lemma:\ExtensionOrderName} \\
    & $\ctxr \byinf x \infto \applye \ctxr \sigma$ & By \rul{A-Var} \\
    & $\tau_2 = \applye \tctx \sigma$ & Given \\
    & $\applye \ctxr \sigma $ & \\
    & $= \applye \ctxr {\applye \tctx \sigma} $ & By
    Lemma~\ref{lemma:\SubstitutionExtensionInvarianceName}\\
    & $= \applye \ctxr {\tau_2} $ & Substitute above equality
  \end{longtable}
\item Case \[\AEVar\]
  By Lemma~\ref{lemma:\ExtensionOrderName},
  we have either $\genA$ or $\genA = \tau$ in $\ctxr$.
  Then by rule \rul{A-EVar} or \rul{A-SolvedEVar}, we have
  $\ctxr \byinf \genA \infto \star$ directly.
\item Case \[\ASolvedEVar\]
  By Lemma~\ref{lemma:\ExtensionOrderName},
  and rule \rul{A-SolveEVar}, we have
  $\ctxr \byinf \genA \infto \star$ directly.
\item Case \[\ALamAnn\]
  \begin{longtable}[l]{lll}
    & $\ctxr \byinf \sigma_1 \infto \star$ & By induction hypothesis \\
    & $\tctx, x: \sigma_1 \exto \ctxr, x: \sigma_1$ & By \rul{CE-Var} \\
    & $\ctxr, x: \sigma_1 \byinf e \infto \applye {\ctxr, x: \sigma_1}
    {\sigma_2}$ & By induction hypothesis \\
    & $\ctxr, x: \sigma_1 \byinf e \infto \applye \ctxr
    {\sigma_2}$ & By definition of context substitution \\
    & $\ctxr \byinf \blam x {\sigma_1} e \infto \bpi x {\applye \ctxr {\sigma_1}}
    {\applye \ctxr {\sigma_2}}$& By \rul{A-LamAnn} \\
    & $\applye \ctxr {\sigma_1} = \applye \ctxr {\applye \tctx {\sigma_1}}$ & By
    Lemma~\ref{lemma:\SubstitutionExtensionInvarianceName} \\
    & $\ctxr \byinf \blam x {\sigma_1} e \infto \bpi x {\applye \ctxr {\applye
        \tctx {\sigma_1}}}
    {\applye \ctxr {\sigma_2}}$& Substitute above equality \\
    & $\ctxr \byinf \blam x {\sigma_1} e \infto \applye \ctxr {\bpi x {\applye
        \tctx {\sigma_1}} {\sigma_2}}$& Follows directly
  \end{longtable}
\item Case \[\APi\]
  \begin{longtable}[l]{lll}
    & $\ctxr \byinf \sigma_1 \infto \star$ & By induction hypothesis \\
    & $\tctx, x: \sigma_1 \exto \ctxr, x: \sigma_1$ & By \rul{CE-Var} \\
    & $\ctxr, x: \sigma_1 \byinf \sigma_2 \infto \star$ & By induction
    hypothesis \\
    & $\ctxr \byinf \bpi x {\sigma_1} {\sigma_2} \infto \star$ & By \rul{A-Pi}
  \end{longtable}
\item Case \[\AApp\]
  \begin{longtable}[l]{lll}
    & $\ctxr \byinf e_1 \infto \applye \ctxr {\bpi x {\sigma_1} {\sigma_2}}$ &
    By induction hypothesis \\
    & $\ctxr \byinf e_1 \infto {\bpi x {\applye \ctxr {\sigma_1}} {\applye \ctxr
        {\sigma_2}}}$ &
    Follows directly \\
    & $\ctxr \byinf e_2 \infto \applye \ctxr {\sigma_1}$ &
    By induction hypothesis \\
    & $\ctxr \byinf e_1 ~ e_2 \infto
    (\applye \ctxr {\sigma_2}) \subst x {\applye \ctxr {e_1}}$ &
    By \rul{A-App} \\
    & $(\applye \ctxr {\sigma_2}) \subst x {\applye \ctxr {e_1}}$ & \\
    & $ = (\applye \ctxr {\sigma_2}) \subst x {\applye \ctxr {\applye \tctx
        {e_1}}}$ & By Lemma~\ref{lemma:\SubstitutionExtensionInvarianceName}\\
    & $ = \applye \ctxr {\sigma_2 \subst x {\applye \tctx {e_1}}}$ & Property of
    substitution \\
  \end{longtable}
\item Case \[\ACastDn\]
  \begin{longtable}[l]{lll}
    & $\ctxr \byinf e \infto \applye \ctxr {\sigma_1}$& By induction
    hypothesis \\
    & $\applye \ctxr {\sigma_1} \redto \applye \ctxr {\sigma_2}$& By
    Lemma~\ref{lemma:\ContextApplicationOverReductionName} \\
    & $\ctxr \byinf \castdn e \infto \applye \ctxr {\sigma_2}$& By \rul{A-CastDn}
  \end{longtable}

\item Case \[\ACastUp\]
  \begin{longtable}[l]{lll}
    & $\ctxr \byinf e \infto \applye \ctxr {\sigma_2}$& By induction
    hypothesis \\
    & $\applye \ctxr {\sigma_1} $ & \\
    & $ = \applye \ctxr {\applye \tctx {\sigma_1}} $ & By
    Lemma~\ref{lemma:\SubstitutionExtensionInvarianceName} \\
    & $ \redto \applye \ctxr {\sigma_2}$& By
    Lemma~\ref{lemma:\ContextApplicationOverReductionName} \\
    & $\ctxr \byinf \castup e \infto \applye \ctxr {\sigma_1}$& By \rul{A-CastUp}
  \end{longtable}
\end{itemize}

\qed

\begin{corollary}[\ExtensionWeakningWellFormednessName]
  \label{lemma:\ExtensionWeakningWellFormednessName}
  \ExtensionWeakningWellFormednessBody
\end{corollary}

\proof

Follows directly by Lemma~\ref{lemma:\ExtensionWeakningName} since $\applye
\ctxr \star = \star$.

\qed

\begin{lemma}[\ContextExtensionPreservesContextWellFormednessName]
  \label{lemma:\ContextExtensionPreservesContextWellFormednessName}
  \ContextExtensionPreservesContextWellFormednessBody
\end{lemma}

\proof

By induction on the context extension.

\begin{itemize}
\item Case \[\CEEmpty\]
  Here we have $\tctx = \ctxr$, so our goal is given.
\item Case \[\CEVar\]
  \begin{longtable}[l]{lll}
    & $\tctx \wc$ & By hypothesis \\
    & $\tctx \bywf \sigma$ & By hypothesis \\
    & $x \notin \tctx$ & By hypothesis \\
    & $\ctxr \wc$& By induction hypothesis \\
    & $\ctxr \bywf \sigma$& By
    Corollary~\ref{lemma:\ExtensionWeakningWellFormednessName} \\
    & $x \notin \ctxr$ & Context extension add no variables\\
    & $\ctxr, x :\sigma \wc$ & By \rul{AC-Var}
  \end{longtable}

\item Case \[\CEEVar\]
  \begin{longtable}[l]{lll}
    & $\tctx \wc$ & By hypothesis \\
    & $\ctxr \wc$& By induction hypothesis \\
    & $\genA \notin \ctxr$ & Given \\
    & $\ctxr, \genA \wc$ & By \rul{AC-EVar}
  \end{longtable}

\item Case \[\CESolvedEVar\]
  \begin{longtable}[l]{lll}
    & $\tctx \wc$ & By hypothesis \\
    & $\tctx \bywf \tau$ & By hypothesis \\
    & $\ctxr \wc$& By induction hypothesis \\
    & $\ctxr \bywf \tau$ & By
    Corollary~\ref{lemma:\ExtensionWeakningWellFormednessName} \\
    & $\ctxr, \genA = \tau$ & By \rul{AC-SolvedEVar}
  \end{longtable}
\item Case \[\CESolve\]
  \begin{longtable}[l]{lll}
    & $\tctx \wc$ & By hypothesis \\
    & $\ctxr \wc$& By induction hypothesis \\
    & $\ctxr, \genA = \tau$ & By \rul{AC-SolvedEVar}
  \end{longtable}
\item Case \[\CEAdd\]
  \begin{longtable}[l]{lll}
    & $\tctx \wc$ & By hypothesis \\
    & $\ctxr \wc$& By induction hypothesis \\
    & $\ctxr, \genA$ & By \rul{AC-EVar}
  \end{longtable}
\item Case \[\CEAddSolved\]
  \begin{longtable}[l]{lll}
    & $\tctx \wc$ & By hypothesis \\
    & $\ctxr \wc$& By induction hypothesis \\
    & $\ctxr, \genA = \tau$ & By \rul{AC-EVar}
  \end{longtable}
\end{itemize}

\qed

\begin{lemma}[\SolutionAdmissibilityForExtensionName]
  \label{lemma:\SolutionAdmissibilityForExtensionName}
  \SolutionAdmissibilityForExtensionBody
\end{lemma}

\proof

By induction on $\tctx_R$.

\begin{itemize}
\item Case $\ctxinit$.
  \begin{longtable}[l]{lll}
    & $\tctx_L \exto \tctx_L $ & By
    Lemma~\ref{lemma:\ContextExtensionReflexivityName} \\
    & $\tctx_L, \genA \exto \tctx_L, \genA = \tau$ & By \rul{CE-Solve}
  \end{longtable}
\item Case $\tctx_R = \tctx_R', x: \sigma$
  \begin{longtable}[l]{lll}
    & $\tctx_L, \genA, \tctx_R' \exto \tctx_L, \genA = \tau, \tctx_R' $ & By
    induction hypothesis \\
    & $\tctx_L, \genA, \tctx_R', x: \sigma \exto \tctx_L, \genA = \tau,
    \tctx_R', x: \sigma $ & By \rul{CE-Var}
  \end{longtable}
\item Case $\tctx_R = \tctx_R', \genB$
  \begin{longtable}[l]{lll}
    & $\tctx_L, \genA, \tctx_R' \exto \tctx_L, \genA = \tau, \tctx_R' $ & By
    induction hypothesis \\
    & $\tctx_L, \genA, \tctx_R', \genB \exto \tctx_L, \genA = \tau,
    \tctx_R', \genB $ & By \rul{CE-EVar}
  \end{longtable}
\item Case $\tctx_R = \tctx_R', \genB = \sigma$
  \begin{longtable}[l]{lll}
    & $\tctx_L, \genA, \tctx_R' \exto \tctx_L, \genA = \tau, \tctx_R' $ & By
    induction hypothesis \\
    & $\tctx_L, \genA, \tctx_R' \byinf \sigma \infto \star$ & Given\\
    & $\tctx_L, \genA = \tau, \tctx_R' \byinf \sigma \infto \star$ & By
    Lemma~\ref{lemma:\ExtensionWeakningName}\\
    & $\tctx_L, \genA, \tctx_R', \genB = \sigma \exto \tctx_L, \genA = \tau,
    \tctx_R', \genB=\sigma $ & By \rul{CE-SolvedEVar}
  \end{longtable}
\end{itemize}
\qed

\begin{lemma}[\UnsolvedVariableAdditionForExtensionName]
  \label{lemma:\UnsolvedVariableAdditionForExtensionName}
  \UnsolvedVariableAdditionForExtensionBody
\end{lemma}

\proof

By induction on $\tctx_R$.

\begin{itemize}
\item Case $\ctxinit$.
  \begin{longtable}[l]{lll}
    & $\tctx_L \exto \tctx_L $ & By
    Lemma~\ref{lemma:\ContextExtensionReflexivityName} \\
    & $\tctx_L \exto \tctx_L, \genA$ & By \rul{CE-Add}
  \end{longtable}
\item Case $\tctx_R = \tctx_R', x: \sigma$
  \begin{longtable}[l]{lll}
    & $\tctx_L, \tctx_R' \exto \tctx_L, \genA, \tctx_R' $ & By
    induction hypothesis \\
    & $\tctx_L, \tctx_R', x: \sigma \exto \tctx_L, \genA,
    \tctx_R', x: \sigma $ & By \rul{CE-Var}
  \end{longtable}
\item Case $\tctx_R = \tctx_R', \genB$
  \begin{longtable}[l]{lll}
    & $\tctx_L, \tctx_R' \exto \tctx_L, \genA, \tctx_R' $ & By
    induction hypothesis \\
    & $\tctx_L, \tctx_R', \genB \exto \tctx_L, \genA,
    \tctx_R', \genB $ & By \rul{CE-EVar}
  \end{longtable}
\item Case $\tctx_R = \tctx_R', \genB = \sigma$
  \begin{longtable}[l]{lll}
    & $\tctx_L, \tctx_R' \exto \tctx_L, \genA, \tctx_R' $ & By
    induction hypothesis \\
    & $\tctx_L, \tctx_R' \byinf \sigma \infto \star$ & Given\\
    & $\tctx_L, \genA , \tctx_R' \byinf \sigma \infto \star$ & By
    Lemma~\ref{lemma:\ExtensionWeakningName}\\
    & $\tctx_L, \tctx_R', \genB = \sigma \exto \tctx_L, \genA,
    \tctx_R', \genB=\sigma $ & By \rul{CE-SolvedEVar}
  \end{longtable}
\end{itemize}
\qed

\begin{lemma}[\SolvedVariableAdditionForExtensionName]
  \label{lemma:\SolvedVariableAdditionForExtensionName}
  \SolvedVariableAdditionForExtensionBody
\end{lemma}

\proof

\mbox{} % an empty line to make sure long table appear after proof
\begin{longtable}[l]{lll}
  & $\tctx_L, \tctx_R \exto \tctx_L, \genA, \tctx_R $ & By
  Lemma~\ref{lemma:\UnsolvedVariableAdditionForExtensionName} \\
  & $\tctx_L, \genA, \tctx_R \exto \tctx_L, \genA = \tau, \tctx_R $ & By
  Lemma~\ref{lemma:\SolutionAdmissibilityForExtensionName} \\
  & $\tctx_L, \tctx_R \exto \tctx_L, \genA = \tau, \tctx_R$ &
  By Lemma~\ref{lemma:\ContextExtensionTransitivityName}
\end{longtable}

\qed

\subsection{Properties of Type Sanitization}


\subsection{Properties of Unification}

equivalence

well typed


\section{Implicit Polymorphic Type System}