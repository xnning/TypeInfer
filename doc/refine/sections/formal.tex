% Setup spaces between column
\setlength{\tabcolsep}{2pt}

% ------------------------------------------------------
% DECLARATIVE WELL FORMEDNESS
% ------------------------------------------------------

\newcommand*{\DCEmpty}{\inferrule{
            }{
            \ctxinit \wc
            }\rname{DC-Empty}}

\newcommand*{\DCVar}{\inferrule{
            \dctx \wc
         \\ x \notin \dctx
         \\ \dctx \bychk \sigma \infto \star
            }{
            \dctx, x: \sigma \wc
            }\rname{DC-Var}}

% ------------------------------------------------------
% DECLARATIVE TYPING
% ------------------------------------------------------

\newcommand*{\DAx}{\inferrule{
            \dctx \wc
            }{
            \dctx \byinf \star \infto \star
            }\rname{D-Ax}}

\newcommand*{\DVar}{\inferrule{
            \dctx \wc
         \\ x: \sigma \in \dctx
            }{
            \dctx \byinf x \infto \sigma
            }\rname{D-Var}}

\newcommand*{\DLam}{\inferrule{
            \dctx \byinf \tau \infto \star
         \\ \dctx, x : \tau \byinf e \infto \sigma
            }{
            \dctx \byinf \erlam x e \infto \bpi x \tau \sigma
            }\rname{D-Lam}}

\newcommand*{\DLamAnn}{\inferrule{
            \dctx \byinf \sigma_1 \infto \star
         \\ \dctx, x : \sigma_1 \byinf e \infto \sigma_2
            }{
            \dctx \byinf \blam x {\sigma_1} e \infto \bpi x {\sigma_1} {\sigma_2}
            }\rname{D-LamAnn}}

\newcommand*{\DPi}{\inferrule{
            \dctx \byinf \sigma_1 \infto \star
         \\ \dctx, x : \sigma_1 \byinf \sigma_2 \infto \star
            }{
            \dctx \byinf \bpi x {\sigma_1} {\sigma_2} \infto \star
            }\rname{D-Pi}}

\newcommand*{\DAll}{\inferrule{
            \dctx \byinf \tau \infto \star
         \\ \dctx, x : \tau \byinf \sigma \infto \star
            }{
            \dctx \byinf \forall x : {\tau} . {\sigma} \infto \star
            }\rname{D-All}}

\newcommand*{\DApp}{\inferrule{
            \dctx \byinf e_1 \infto \bpi x {\sigma_1}  {\sigma_2}
         \\ \dctx \byinf e_2 \infto \sigma_1
            }{
            \dctx \byinf e_1 ~ e_2 \infto \sigma_2 \subst x {e_2}
            }\rname{D-App}}

\newcommand*{\DAppPi}{\inferrule{
            \dctx \byinf e \infto \sigma_3
         \\ \dctx \bysub \sigma_3 \dsub \sigma_1
            }{
            \dctx \byinf (\bpi x {\sigma_1} {\sigma_2}) \app e \infto \sigma_2
            \subst x e
            }\rname{D-App-Pi}}

\newcommand*{\DAppPoly}{\inferrule{
            \dctx \byinf \tau_2 \infto \tau_1
         \\ \dctx \byinf \sigma_1 \subst x {\tau_2} \app e \infto \sigma_2
            }{
            \dctx \byinf (\forall x:\tau_1 . \sigma_1) \app e \infto \sigma_2
            }\rname{D-App-Poly}}

\newcommand*{\DSub}{\inferrule{
            \dctx \byinf e \infto \sigma_1
         \\ \dctx \bysub \sigma_1 \dsub \sigma_2
            }{
            \dctx \byinf e \infto \sigma_2
            }\rname{D-Sub}}


% ------------------------------------------------------
% DECLARATIVE SUBTYPING
% ------------------------------------------------------

\newcommand*{\DSAx}{\inferrule{
            \dctx \wc
            }{
            \dctx \bysub \star \dsub \star
            }\rname{DS-Ax}}

\newcommand*{\DSVar}{\inferrule{
            \dctx \wc
         \\ x : \sigma \in \dctx
            }{
            \dctx \bysub x \dsub x
            }\rname{DS-Var}}

\newcommand*{\DSAllL}{\inferrule{
            \dctx \byinf \tau \infto \star
         \\ \dctx \bysub \sigma_1 \subst x {\tau} \dsub \sigma_2
            }{
            \dctx \bysub \forall x. \sigma_1 \dsub \sigma_2
            }\rname{DS-AllL}}

\newcommand*{\DSAllR}{\inferrule{
            \dctx, x:\star \bysub \sigma_1 \dsub \sigma_2
            }{
            \dctx \bysub \sigma_1 \dsub \forall x. \sigma_2
            }\rname{DS-AllR}}

\newcommand*{\DSPi}{\inferrule{
            \dctx \bysub \sigma_3 \dsub \sigma_1
         \\ \dctx, x : \sigma_1 \bysub \sigma_2 \dsub \sigma_4
            }{
              \dctx \bysub \bpi x {\sigma_1} {\sigma_2} \dsub
              \bpi x {\sigma_3} {\sigma_4}
            }\rname{DS-Pi}}

\newcommand*{\DSLamAnn}{\inferrule{
            \dctx, x : \sigma \bysub e_1 \dsub e_2
            }{
              \dctx \bysub \blam x {\sigma} {e_1} \dsub
              \blam x {\sigma} {e_2}
            }\rname{DS-LamAnn}}

\newcommand*{\DSApp}{\inferrule{
            \dctx \bysub e_1 \dsub e_2
            }{
            \dctx \bysub e_1 ~ e \dsub e_2 ~ e
            }\rname{DS-App}}

% ------------------------------------------------------
% EXTENSION WELL FORMEDNESS
% ------------------------------------------------------

\newcommand*{\ACEmpty}{\inferrule{
            }{
            \ctxinit \wc
            }\rname{AC-Empty}}

\newcommand*{\ACVar}{\inferrule{
            \tctx \wc
         \\ x \notin \dctx
         \\ \tctx \byinf \sigma \infto \star
            }{
            \tctx, x: \sigma \wc
            }\rname{AC-Var}}

\newcommand*{\ACEVar}{\inferrule{
            \tctx \wc
         \\ \genA \notin \dctx
            }{
            \tctx, \genA \wc
            }\rname{AC-EVar}}

\newcommand*{\ACSolvedEVar}{\inferrule{
            \tctx \wc
         \\ \genA \notin \tctx
         \\ \tctx \byinf \tau \infto \star
            }{
            \tctx, \genA = \tau \wc
            }\rname{AC-SolvedEVar}}

% ------------------------------------------------------
% EXTENSION TYPING
% ------------------------------------------------------

\newcommand*{\AAx}{\inferrule{
            \tctx \wc
            }{
            \tctx \byinf \star \infto \star
            }\rname{A-Ax}}

\newcommand*{\AVar}{\inferrule{
            \tctx \wc
         \\ x: \sigma \in \tctx
            }{
            \tctx \byinf x \infto \applye \tctx \sigma
            }\rname{A-Var}}

\newcommand*{\AEVar}{\inferrule{
            \tctx \wc
         \\ \genA \in \tctx
            }{
            \tctx \byinf \genA \infto \star
            }\rname{A-EVar}}

\newcommand*{\ASolvedEVar}{\inferrule{
            \tctx \wc
         \\ \genA = \tau \in \tctx
            }{
            \tctx \byinf \genA \infto \star
            }\rname{A-SolvedEVar}}

\newcommand*{\ALamAnn}{\inferrule{
            \tctx \byinf \sigma_1 \infto \star
         \\ \tctx, x : \sigma_1 \byinf e \infto \sigma_2
            }{
            \tctx \byinf \blam x {\sigma_1} e \infto \bpi x {\applye {\tctx} {\sigma_1}} {\sigma_2}
            }\rname{A-LamAnn}}

\newcommand*{\APi}{\inferrule{
            \tctx \byinf \sigma_1 \infto \star
         \\ \tctx, x : \sigma_1 \byinf \sigma_2 \infto \star
            }{
            \tctx \byinf \bpi x {\sigma_1} {\sigma_2} \infto \star
            }\rname{A-Pi}}

\newcommand*{\AAll}{\inferrule{
            \tctx \byinf \tau \infto \star
         \\ \tctx, x : \tau \byinf \sigma \infto \star
            }{
            \tctx \byinf \forall x : {\tau} . {\sigma} \infto \star
            }\rname{A-All}}

\newcommand*{\AApp}{\inferrule{
            \tctx \byinf e_1 \infto \bpi x {\sigma_1} {\sigma_2}
         \\ \tctx \byinf e_2 \infto \sigma_1
            }{
            \tctx \byinf e_1 ~ e_2 \infto \sigma_2 \subst x {\applye \tctx {e_1}}
            }\rname{A-App}}

\newcommand*{\AAppPi}{\inferrule{
            \tctx \byinf e \infto \sigma_3
         \\ \tctx \bysub \applye \tctx {\sigma_3} \dsub \applye \tctx {\sigma_1}
            }{
            \tctx \byinf (\bpi x {\sigma_1} {\sigma_2}) \app e \infto \sigma_2
            \subst x e
            }\rname{A-App-Pi}}

\newcommand*{\AAppPoly}{\inferrule{
            \tctx \byinf \tau_2 \infto \tau_1
         \\ \tctx \byinf \sigma_1 \subst x {\tau_2} \app e \infto \sigma_2
            }{
            \tctx \byinf (\forall x:\tau_1 . \sigma_1) \app e \infto \sigma_2
            }\rname{A-App-Poly}}

\newcommand*{\ASub}{\inferrule{
            \tctx \byinf e \infto \sigma_1
         \\ \tctx \bysub \applye \tctx {\sigma_1} \dsub \applye \tctx {\sigma_2}
            }{
            \tctx \byinf e \infto \sigma_2
            }\rname{A-Sub}}

\newcommand*{\ACastDn}{\inferrule{
            \tctx \byinf e \infto \sigma_1
         \\ \sigma_1 \redto \sigma_2
            }{
            \tctx \byinf \castdn e \infto \sigma_2
            }\rname{A-CastDn}}

\newcommand*{\ACastUp}{\inferrule{
            \tctx \byinf e \infto \sigma_2
         \\ \applye {\tctx} {\sigma_1} \redto \sigma_2
            }{
            \tctx \byinf \castup e \infto \applye {\tctx} {\sigma_1}
            }\rname{A-CastUp}}

% ------------------------------------------------------
% EXTENSION SUBTYPING
% ------------------------------------------------------

\newcommand*{\ASAx}{\inferrule{
            \tctx \wc
            }{
            \tctx \bysub \star \tsub \star
            }\rname{AS-Ax}}

\newcommand*{\ASVar}{\inferrule{
            \tctx \wc
         \\ x : \sigma \in \tctx
            }{
            \tctx \bysub x \tsub x
            }\rname{AS-Var}}

\newcommand*{\ASEVar}{\inferrule{
            \tctx \wc
         \\ \genA \in \tctx
            }{
            \tctx \bysub \genA \tsub \genA
            }\rname{DS-EVar}}

\newcommand*{\ASAllL}{\inferrule{
            \tctx \byinf \tau \infto \star
         \\ \tctx \bysub \sigma_1 \subst x {\tau} \tsub \sigma_2
            }{
            \tctx \bysub \forall x. \sigma_1 \tsub \sigma_2
            }\rname{AS-AllL}}

\newcommand*{\ASAllR}{\inferrule{
            \tctx, x:\star \bysub \sigma_1 \tsub \sigma_2
            }{
            \tctx \bysub \sigma_1 \tsub \forall x. \sigma_2
            }\rname{AS-AllR}}

\newcommand*{\ASPi}{\inferrule{
            \tctx \bysub \sigma_3 \tsub \sigma_1
         \\ \tctx, x : \sigma_1 \bysub \sigma_2 \tsub \sigma_4
            }{
              \tctx \bysub \bpi x {\sigma_1} {\sigma_2} \tsub
              \bpi x {\sigma_3} {\sigma_4}
            }\rname{AS-Pi}}

\newcommand*{\ASLamAnn}{\inferrule{
            \tctx, x : \sigma \bysub e_1 \tsub e_2
            }{
              \tctx \bysub \blam x {\sigma} {e_1} \tsub
              \blam x {\sigma} {e_2}
            }\rname{AS-LamAnn}}

\newcommand*{\ASApp}{\inferrule{
            \tctx \bysub e_1 \tsub e_2
            }{
            \tctx \bysub e_1 ~ e \tsub e_2 ~ e
            }\rname{AS-App}}

% ------------------------------------------------------
% EXTENSION WELL SCOPEDNESS
% ------------------------------------------------------

\newcommand*{\WSVar}{\inferrule{
            x : \sigma \in \tctx
            }{
            \tctx \bywt x
            }\rname{WS-Var}}

\newcommand*{\WSEVar}{\inferrule{
            \genA \in \tctx
            }{
            \tctx \bywt \genA
            }\rname{WS-EVar}}

\newcommand*{\WSSolvedEVar}{\inferrule{
            \genA = \tau \in \tctx
            }{
            \tctx \bywt \genA
            }\rname{WS-SolvedEVar}}

\newcommand*{\WSLamAnn}{\inferrule{
            \tctx \bywt \sigma
         \\ \tctx, x: \sigma \bywt e
            }{
            \tctx \bywt \blam x {\sigma} e
            }\rname{WS-LamAnn}}

\newcommand*{\WSPi}{\inferrule{
            \tctx \bywt \sigma_1
         \\ \tctx, x: \sigma_1 \bywt \sigma_2
            }{
            \tctx \bywt \bpi x {\sigma_1} {\sigma_2}
            }\rname{WS-Pi}}

\newcommand*{\WSApp}{\inferrule{
            \tctx \bywt e_1
         \\ \tctx \bywt e_2
            }{
            \tctx \bywt e_1 ~ e_2
            }\rname{WS-App}}

\newcommand*{\WSCastUp}{\inferrule{
            \tctx \bywf e
            }{
            \tctx \bywt \castup e
            }\rname{WS-CastUp}}

\newcommand*{\WSCastDn}{\inferrule{
            \tctx \bywf e
            }{
            \tctx \bywt \castdn e
            }\rname{WS-CastDn}}

% ------------------------------------------------------
% UNIFICATION
% ------------------------------------------------------

\newcommand*{\UAEq}{\inferrule{
            }{
            \tctx \bybuni \sigma \uni \sigma \toctxo
            }\rname{U-AEq}}

\newcommand*{\UEVarTy}{\inferrule{
            \genA \notin FV(\tau_1)
         \\ \tctx[\genA] \bysa \tau_1 \sa \tau_2 \toctx_1, \genA, \ctxl_2
         \\ \ctxl_1 \bywt \tau_2
            }{
            \tctx[\genA] \bysuni \genA \uni \tau_1 \toctx_1, \genA = \tau_2, \ctxl_2
            }\rname{U-EVarTy}}

\newcommand*{\UTyEVar}{\inferrule{
            \genA \notin FV(\tau_1)
         \\ \tctx[\genA] \bysa \tau_1 \sa \tau_2 \toctx_1, \genA, \ctxl_2
         \\ \ctxl_1 \bywt \tau_2
            }{
            \tctx[\genA] \bysuni \tau_1 \uni \genA \toctx_1, \genA = \tau_2, \ctxl_2
            }\rname{U-TyEVar}}

\newcommand*{\UApp}{\inferrule{
            \tctx \byeuni e_1 \uni e_3 \toctx_1
         \\ \ctxl_1 \byeuni \applye {\ctxl_1} {e_2} \uni \applye {\ctxl_1} {e_4} \toctx
            }{
            \tctx \bybuni e_1 ~ e_2 \uni e_3 ~ e_4 \toctx
            }\rname{U-App}}

\newcommand*{\ULamAnn}{\inferrule{
            \tctx \bysuni \sigma_1 \uni \sigma_2 \toctx_1
         \\ \ctxl_1, x:\sigma_1 \byeuni \applye {\ctxl_1} {e_1} \uni \applye {\ctxl_1} {e_2} \toctx, x:\sigma_1
            }{
            \tctx \byeuni \blam x {\sigma_1} {e_1} \uni \blam x {\sigma_2} {e_2} \toctx
            }\rname{U-LamAnn}}

\newcommand*{\UPi}{\inferrule{
            \tctx \bysuni \sigma_1 \uni \sigma_3 \toctx_1
         \\ \ctxl_1, x:\sigma_1 \bysuni \applye {\ctxl_1} {\sigma_2} \uni
                \applye {\ctxl_1} {\sigma_4} \toctx, x:\sigma_1
            }{
            \tctx \bybuni \bpi x {\sigma_1} {\sigma_2} \uni \bpi x {\sigma_3} {\sigma_4} \toctx
            }\rname{U-Pi}}

\newcommand*{\UAll}{\inferrule{
            \tctx, x:\star \bysuni \sigma_1 \uni \sigma_2 \toctx, x: \star
            }{
            \tctx \bybuni \forall x. \sigma_1 \uni \forall x. \sigma_2 \toctx
            }\rname{U-All}}

\newcommand*{\UCastDn}{\inferrule{
            \tctx \byeuni e_1 \uni e_2 \toctx
            }{
            \tctx \bybuni \castdn {e_1} \uni \castdn {e_2} \toctx
            }\rname{U-CastDn}}

\newcommand*{\UCastUp}{\inferrule{
            \tctx \byeuni {e_1} \uni {e_2} \toctx
            }{
            \tctx \byeuni \castup {e_1} \uni \castup {e_2} \toctx
            }\rname{U-CastUp}}

% ------------------------------------------------------
% TYPE SANITIZATION
% ------------------------------------------------------

\newcommand*{\IEVarAfter}{\inferrule{
            }{
            \tctx[\genA][\genB] \bysa \genB \sa \genA_1 \toctxo[\genA_1, \genA][\genB=\genA_1]
            }\rname{I-EVarAfter}}

\newcommand*{\IEVarBefore}{\inferrule{
            }{
            \tctx[\genB][\genA] \bysa \genB \sa \genB \toctxo[\genB][\genA]
            }\rname{I-EVarBefore}}

\newcommand*{\IStar}{\inferrule{
            }{
            \tctx \bysa \star \sa \star \toctxo
            }\rname{I-Star}}

\newcommand*{\IVar}{\inferrule{
            }{
            \tctx \bysa x \sa x \toctxo
            }\rname{I-Var}}

\newcommand*{\IApp}{\inferrule{
            \tctx \bysa e_1 \sa e_3 \toctx_1
         \\ \ctxl_1 \bysa \applye {\ctxl_1} {e_2} \sa e_4 \toctx
            }{
            \tctx \bysa e_1 ~ e_2 \sa e_3 ~ e_4 \toctx
            }\rname{I-App}}

\newcommand*{\ILamAnn}{\inferrule{
            \tctx \bysa \tau_1 \sa \tau_2 \toctx_1
         \\ \ctxl_1, x: \tau_1 \bysa \applye {\ctxl_1} {e_1} \sa e_2 \toctx, x:\tau_1
            }{
            \tctx \bysa \blam x {\tau_1} {e_1} \sa \blam x {\tau_2} {e_2} \toctx
            }\rname{I-LamAnn}}

\newcommand*{\IPi}{\inferrule{
            \tctx \bysa \tau_1 \sa \tau_3 \toctx_1
         \\ \ctxl_1, x: \tau_1 \bysa \applye {\ctxl_1} {\tau_2} \sa \tau_4 \toctx, x: \tau_1
            }{
            \tctx \bysa \bpi x {\tau_1} {\tau_2} \sa \bpi x {\tau_3} {\tau_4} \toctx
            }\rname{I-Pi}}

\newcommand*{\ICastDn}{\inferrule{
            \tctx \bysa e_1 \sa e_2  \toctx
            }{
            \tctx \bysa \castdn e_1 \uni \castdn e_2 \toctx
            }\rname{I-CastDn}}

\newcommand*{\ICastUp}{\inferrule{
            \tctx \bysa e_1 \sa e_2  \toctx
            }{
            \tctx \bysa \castup e_1 \uni \castup e_2 \toctx
            }\rname{I-CastUp}}

% ------------------------------------------------------
% SUBTYPING FOR SIGMA
% ------------------------------------------------------

\newcommand*{\SAllR}{\inferrule{
            \tctx, x : \star \byssub \sigma_1 \tsub \sigma_2 \toctx, x : \star, \ctxr
            }{
            \tctx \bybsub \sigma_1 \tsub \forall x. \sigma_2 \toctx
            }\rname{S-AllR}}

\newcommand*{\SAllL}{\inferrule{
            \tctx, \genA \byssub \sigma_1 \subst x \genA \tsub \sigma_2 \toctx
            }{
            \tctx \bybsub \forall x. \sigma_1 \tsub \sigma_2 \toctx
            }\rname{S-AllL}}

\newcommand*{\SPi}{\inferrule{
            \tctx \byssub \sigma_3 \tsub \sigma_1 \toctx_1
         \\ \ctxl_1, x: \sigma_1 \byssub \applye {\ctxl_1} {\sigma_2} \tsub \applye
                {\ctxl_1} {\sigma_4} \toctx, x:\sigma_1, \ctxr
            }{
            \tctx \bybsub \bpi x {\sigma_1} {\sigma_2} \tsub \bpi x {\sigma_3}
            {\sigma_4 } \toctx
            }\rname{S-Pi}}

\newcommand*{\SApp}{\inferrule{
            \tctx \byesub \sigma_1 \esub \sigma_3 \toctx_1
         \\ \ctxl_1 \byeuni \applye {\ctxl_1} {\sigma_2} \uni \applye {\ctxl_1} {\sigma_4} \toctx
            }{
            \tctx \bybsub \sigma_1 ~ \sigma_2 \tsub \sigma_3 ~ \sigma_4 \toctx
            }\rname{S-App}}

\newcommand*{\SLamAnn}{\inferrule{
            \tctx \bysuni \sigma_1 \uni \sigma_2 \toctx_1
         \\ \ctxl_1, x: \sigma_1 \byesub \applye {\ctxl_1} {e_1} \esub \applye
                {\ctxl_1} {e_2} \toctx, x:\sigma_1, \ctxr
            }{
            \tctx \byesub \blam x {\sigma_1} ~ {e_1} \tsub \blam x {\sigma_2} ~ {e_2} \toctx
            }\rname{S-LamAnn}}

\newcommand*{\SEVarTy}{\inferrule{
            \tctx[\genA] \bypsa \sigma \sa \tau \toctx_1
         \\ \ctxl_1 \bysuni \genA \uni \tau \toctx
            }{
            \tctx[\genA] \bybsub \genA \tsub \sigma \toctx
            }\rname{S-EVarTy}}

\newcommand*{\STyEVar}{\inferrule{
            \tctx[\genA] \bymsa \sigma \sa \tau \toctx_1
         \\ \ctxl_1 \bysuni \genA \uni \tau \toctx
            }{
            \tctx[\genA] \bybsub \sigma \tsub \genA \toctx
            }\rname{S-TyEVar}}

\newcommand*{\SUnify}{\inferrule{
            \tctx \bybuni \sigma_1 \uni \sigma_2 \toctx
            }{
            \tctx \bybsub \sigma_1 \tsub \sigma_2 \toctx
            }\rname{S-Unify}}

% ------------------------------------------------------
% POLYMORPHIC TYPE SANITIZATION
% ------------------------------------------------------

\newcommand*{\IAllPlus}{\inferrule{
            \tctx[\genB, \genA] \bypsa A \subst \varA {\genB} \sa \sigma \toctx
            }{
            \tctx[\genA] \bypsa \forall \varA. A \sa \sigma \toctx
            }\rname{I-All-Plus}}

\newcommand*{\IAllMinus}{\inferrule{
            \tctx, \varA \bymsa A \sa \sigma \toctx, \varA
            }{
            \tctx \bymsa \forall \varA. A \sa \sigma \toctx
            }\rname{I-All-Minus}}

\newcommand*{\IPiPoly}{\inferrule{
            \tctx \bysa^{-s} A_1 \sa \sigma_1 \toctx_1
         \\ \ctxl_1 \bybsa \applye {\ctxl_1} {A_2} \sa \sigma_2 \toctx
            }{
            \tctx \bybsa A_1 \to A_2 \sa \sigma_1 \to \sigma_2 \toctx
            }\rname{I-Pi-Poly}}

\newcommand*{\ILamAnnPoly}{\inferrule{
            \tctx, x:\tau_1 \bybsa e \sa \tau_2 \toctx, x: \tau_1
            }{
            \tctx \bybsa \blam x {\tau_1} e \sa \blam x {\tau_1} {\tau_2} \toctx
            }\rname{I-LamAnn-Poly}}

\newcommand*{\IAppPoly}{\inferrule{
            \tctx \bybsa e \sa {\tau_1} \toctx
            }{
            \tctx \bybsa e ~ \tau_2 \sa \tau_1 ~ {\tau_2} \toctx
            }\rname{I-App-Poly}}

\newcommand*{\IMono}{\inferrule{
            }{
            \tctx \bybsa \tau \sa \tau \toctxo
            }\rname{I-Mono}}

\newcommand*{\IUnit}{\inferrule{
            }{
            \tctx \bybsa 1 \sa 1 \toctxo
            }\rname{I-Unit}}

\newcommand*{\ITVar}{\inferrule{
            }{
            \tctx[\varA] \bybsa \varA \sa \varA \toctxo[\varA]
            }\rname{I-TVar}}

\newcommand*{\IEVarAfterPoly}{\inferrule{
            }{
            \tctx[\genA][\genB] \bybsa \genB \sa \genA_1 \toctxo[\genA_1, \genA][\genB=\genA_1]
            }\rname{I-EVarAfter-Poly}}

\newcommand*{\IEVarBeforePoly}{\inferrule{
            }{
            \tctx[\genB][\genA] \bybsa \genB \sa \genB \toctxo[\genB][\genA]
            }\rname{I-EVarBefore-Poly}}

% ------------------------------------------------------------------------
% CONTEXT EXTENSION
% ------------------------------------------------------------------------

\newcommand*{\CEEmpty}{\inferrule{
            }{
            \ctxinit \exto \ctxinit
            }\rname{CE-Empty}}

\newcommand*{\CEVar}{\inferrule{
            \tctx \exto \ctxr
            }{
            \tctx, x:\sigma \exto \ctxr, x:\sigma
            }\rname{CE-Var}}

\newcommand*{\CEEVar}{\inferrule{
            \tctx \exto \ctxr
         \\ \genA \notin \ctxr
            }{
            \tctx, \genA \exto \ctxr, \genA
            }\rname{CE-EVar}}

\newcommand*{\CESolvedEVar}{\inferrule{
            \tctx \exto \ctxr
         \\ \genA \notin \ctxr
         \\ \applye \ctxr {\tau_1} = \applye \ctxr {\tau_2}
             }{
            \tctx, \genA = \tau_1 \exto \ctxr, \genA = \tau_2
            }\rname{CE-SolvedEVar}}

\newcommand*{\CESolve}{\inferrule{
            \tctx \exto \ctxr
         \\ \genA \notin \ctxr
         \\ \ctxr \bywf \tau
            }{
            \tctx, \genA \exto \ctxr, \genA = \tau
            }\rname{CE-Solve}}

\newcommand*{\CEAdd}{\inferrule{
            \tctx \exto \ctxr
         \\ \genA \notin \ctxr
            }{
            \tctx \exto \ctxr, \genA
            }\rname{CE-Add}}

\newcommand*{\CEAddSolved}{\inferrule{
            \tctx \exto \ctxr
         \\ \genA \notin \ctxr
         \\ \ctxr \bywf \tau
            }{
            \tctx \exto \ctxr, \genA = \tau
            }\rname{CE-AddSolved}}

% ------------------------------------------------------
% REDUCTION RULE
% ------------------------------------------------------

\newcommand*{\RApp}{\inferrule{
            e_1 \redto e_1'
            }{
            e_1 ~ e_2 \redto e_1' ~ e_2
            }\rname{R-App}}

\newcommand*{\RBeta}{\inferrule{
            }{
            (\blam x \sigma {e_1}) ~ e_2 \redto e_1 \subst x {e_2}
            }\rname{R-Beta}}

\newcommand*{\RCastDown}{\inferrule{
            e \redto e'
            }{
            \castdn e \redto \castdn e'
            }\rname{R-CastDown}}

\newcommand*{\RCastDownUp}{\inferrule{
            }{
            \castdn (\castup e) \redto e
            }\rname{R-CastDownUp}}

% ------------------------------------------------------
% DECLARATIVE SUBTYPING IN DUNFIELD
% ------------------------------------------------------

\newcommand*{\DDunVar}{\inferrule{
            \varA \in \dctx
            }{
            \dctx \bysub \varA \dsub \varA
            }\rname{$\dsub$Var}}

\newcommand*{\DDunUnit}{\inferrule{
            }{
            \dctx \bysub \Unit \dsub \Unit
            }\rname{$\dsub$Unit}}

\newcommand*{\DDunArrow}{\inferrule{
            \dctx \bysub B_1 \dsub A_1
         \\ \dctx \bysub A_2 \dsub B_2
            }{
            \dctx \bysub A_1 \to A_2 \tsub B_1 \to B_2
            }\rname{$\dsub \to$}}

\newcommand*{\DDunForallL}{\inferrule{
            \dctx \bywf \tau
         \\ \dctx \bysub A \subst \varA \tau \tsub B
            }{
            \dctx \bysub \forall \varA. A \tsub B
            }\rname{$\dsub \forall$L}}

\newcommand*{\DDunForallR}{\inferrule{
            \dctx, \varA \bysub A \tsub B
            }{
            \dctx \bysub A \tsub \forall \varA. B
            }\rname{$\dsub \forall$R}}

% ------------------------------------------------------
% ALGORITHMIC SUBTYPING IN DUNFIELD
% ------------------------------------------------------

\newcommand*{\ADunVar}{\inferrule{
            }{
            \tctx[\varA] \bysub \varA \tsub \varA \toctxo[\varA]
            }\rname{Var}}

\newcommand*{\ADunUnit}{\inferrule{
            }{
            \tctx \bysub \Unit \tsub \Unit \toctxo
            }\rname{Unit}}

\newcommand*{\ADunExvar}{\inferrule{
            }{
            \tctx[\genA] \bysub \genA \tsub \genA \toctxo[\genA]
            }\rname{Exvar}}

\newcommand*{\ADunArrow}{\inferrule{
            \tctx \bysub B_1 \tsub A_1 \toctx
         \\ \ctxl \bysub \applye \ctxl {A_2} \tsub \applye \ctxl {B_2} \toctxr
            }{
            \tctx \bysub A_1 \to A_2 \tsub B_1 \to B_2 \toctxr
            }\rname{$\to$}}

\newcommand*{\ADunForallL}{\inferrule{
            \tctx, \marker \genA, \genA \bysub A \subst \genA \varA \tsub B \toctxr, \marker \genA, \ctxl
            }{
            \tctx \bysub \forall \varA. A \tsub B \toctxr
            }\rname{$\forall$L}}

\newcommand*{\ADunForallR}{\inferrule{
            \tctx, \varA \bysub A \tsub B \toctxr, \varA, \ctxl
            }{
            \tctx \bysub A \tsub \forall \varA. B \toctxr
            }\rname{$\forall$R}}

\newcommand*{\ADunInstL}{\inferrule{
            \genA \notin FV(A)
         \\ \tctx[\genA] \byinst \genA \ist A \toctx
            }{
            \tctx[\genA] \byinst \genA \tsub A \toctx
            }\rname{InstL}}

\newcommand*{\ADunInstR}{\inferrule{
            \genA \notin FV(A)
         \\ \tctx[\genA] \byinst A \ist \genA \toctx
            }{
            \tctx[\genA] \byinst A \tsub \genA \toctx
            }\rname{InstR}}

\newcommand*{\ADunSaL}{\inferrule{
            \genA \notin FV(A)
         \\ \tctx[\genA] \bymsa A \sa \sigma \toctx, \genA, \ctxr
         \\ \ctxl \bywf \sigma
            }{
            \tctx[\genA] \bysub \genA \tsub A \toctx, \genA = \sigma, \ctxr
            }\rname{SaL}}

\newcommand*{\ADunSaR}{\inferrule{
            \genA \notin FV(A)
         \\ \tctx[\genA] \bypsa A \sa \sigma \toctx, \genA, \ctxr
         \\ \ctxl \bywf \sigma
            }{
            \tctx[\genA] \bysub A \tsub \genA \toctx, \genA = \sigma, \ctxr
            }\rname{SaR}}

% ------------------------------------------------------------------------
% DUNFIELD WELL FORMEDNESS
% ------------------------------------------------------------------------

\newcommand*{\UvarWF}{\inferrule{
            }{
            \tctx[\varA] \bywf \varA
            }\rname{UvarWF}}

\newcommand*{\UnitWF}{\inferrule{
            }{
            \tctx \bywf \Unit
            }\rname{UnitWF}}

\newcommand*{\ArrowWF}{\inferrule{
            \tctx \bywf A
         \\ \tctx \bywf B
            }{
            \tctx \bywf A \to B
            }\rname{ArrowWF}}

\newcommand*{\ForallWF}{\inferrule{
            \tctx, \varA \bywf A
            }{
            \tctx \bywf \forall \varA. A
            }\rname{ForallWF}}

\newcommand*{\EvarWF}{\inferrule{
            }{
            \tctx[\genA] \bywf \genA
            }\rname{EvarWF}}

\newcommand*{\SolvedEvarWF}{\inferrule{
            }{
            \tctx[\genA = \tau] \bywf \genA
            }\rname{SolvedEvarWF}}

% ------------------------------------------------------------------------
% EXAMPLES
% ------------------------------------------------------------------------

\newcommand*{\ExUni}{\inferrule{
                                       \inferrule{\inferrule{ }
                                                            {\tctx,\genA,\genB \bycg \genB \cgto \genA_1 \toctx}\rname{I-Evar2}
                                                  \quad
                                                  \inferrule{ }
                                                            {\ctxl \bycg x \cgto x \toctx}\rname{I-Var}}
                                                 {\tctx,\genA,\genB \bycg \bpi x \genB x \cgto \bpi x {\genA_1} x \toctx}\rname{I-Pi}
                                      \quad
                                      \inferrule{ %\inferrule{ }{\tctx,\genA_1 \bywf \genA_1}\rname{WF-EVar} \quad \inferrule{ }{\tctx,\genA_1,x \bywf x}\rname{WF-Var}
                                                 }
                                                {\tctx,\genA_1 \bywt \bpi x {\genA_1} x}}
                               {\tctx,\genA,\genB \bysuni \genA \uni \bpi x \genB x \toctxo, \genA_1, \genA=\bpi x {\genA_1} x, \genB=\genA_1} \rname{U-EvarTy}}

\newcommand*{\ExSub}{\inferrule{
                                       \inferrule{\inferrule{
                                       \ctxl \bypsa \genB \to \genB \sa \genB \to \genB \toctx
                                       }
                                                            {\genA \bypsa \forall \varA. \varA \to \varA \sa \genB \to \genB \toctx}\rname{I-All-Plus}
                                                  \quad
                                                  \inferrule{ }
                                                            {\ctxl \bymsa \Unit \sa \Unit \toctx}\rname{I-Unit}}
                                                 {\genA \bymsa (\forall \varA. \varA \to \varA) \to \Unit \sa (\genB \to \genB) \to \Unit \toctx}\rname{I-Pi-Poly}
                                      % \quad
                                      % \inferrule{
                                      %            }
                                      %           {\genB \bywf (\genB \to \genB) \to \Unit}
                                                }
                               {\genA \bysub \genA \tsub (\forall \varA. \varA \to \varA) \to \Unit \dashv \genB, \genA = (\genB \to \genB) \to \Unit} \rname{SaL}}

% ------------------------------------------------------------------------
% TYPE INFERENCE IN CONTEXT
% ------------------------------------------------------------------------

\newcommand*{\Ignore}{\inferrule{
            \tctx \ctxsplit \Xi \byinf \genA \equiv \tau \toctx
         \\ v \notin FTV(\genA, \tau, \Xi)
            }{
            \tctx, v \ctxsplit \Xi \byinf \genA \equiv \tau \toctx
            }\rname{Ignore}}

\newcommand*{\Depend}{\inferrule{
            \tctx \ctxsplit \genB, \Xi \byinf \genA \equiv \tau \toctx
         \\ \genB \in FVT(\tau, \Xi)
            }{
            \tctx, \genB \ctxsplit \Xi \byinf \genA \equiv \tau \toctx
            }\rname{Depend}}

\newcommand*{\Define}{\inferrule{
            }{
            \tctx, \genA \ctxsplit \Xi \byinf \genA \equiv \tau \toctxo, \Xi, \genA = \tau
            }\rname{Define}}

% ------------------------------------------------------------------------
% INSTANTIATION
% ------------------------------------------------------------------------


\newcommand*{\InstLSolve}{\inferrule{
            \tctx_1 \bywf \tau
            }{
            \tctx_1, \genA, \tctx_2 \byuni \genA \ist \tau \toctxo_1, \genA = \tau, \tctx_2
            }\rname{InstLSolve}}

\newcommand*{\InstLReach}{\inferrule{
            }{
            \tctx[\genA][\genB] \byinf \genA \ist \genB \toctxo[\genA][\genB = \genA]
            }\rname{InstLReach}}

\newcommand*{\InstLArr}{\inferrule{
            \tctx[\genA_2, \genA_1, \genA = \genA_1 \to \genA_2] \bywf A_1 \ist \genA_1 \toctx
         \\ \ctxl \byinf \genA_2 \ist \applye \ctxl {A_2} \toctxr
            }{
            \tctx[\genA] \byinf \genA \ist A_1 \to A_2 \toctxr
            }\rname{InstLArr}}

\newcommand*{\InstLAllR}{\inferrule{
            \ctxl[\genA], \varB \byinf \genA \ist B \toctx, \varB, \ctxr
            }{
            \tctx[\genA] \byinf \genA \ist \forall \varB. B \toctx
            }\rname{InstLAllR}}

\newcommand*{\InstRSolve}{\inferrule{
            \tctx_1 \bywf \tau
            }{
            \tctx_1, \genA, \tctx_2 \byuni \tau \ist \genA \toctxo_1, \genA = \tau, \tctx_2
            }\rname{InstRSolve}}

\newcommand*{\InstRReach}{\inferrule{
            }{
            \tctx[\genA][\genB] \byinf \genB \ist \genA \toctxo[\genA][\genB = \genA]
            }\rname{InstRReach}}

\newcommand*{\InstRArr}{\inferrule{
            \tctx[\genA_2, \genA_1, \genA = \genA_1 \to \genA_2] \bywf \genA_1 \ist A_1 \toctx
         \\ \ctxl \byinf  \applye \ctxl {A_2} \ist {\genA_2} \toctxr
            }{
            \tctx[\genA] \byinf A_1 \to A_2 \ist \genA \toctxr
            }\rname{InstLArr}}

\newcommand*{\InstRAllL}{\inferrule{
            \ctxl[\genA], \marker \genB, \genB \byinf B \subst \varB \genB \ist \genA \toctx, \marker \genB, \ctxr
            }{
            \tctx[\genA] \byinf \forall \varB. B \ist \genA \toctx
            }\rname{InstRAllL}}

% ------------------------------------------------------------------------
% CONTEXT EXTENSION IN DUNFIELD
% ------------------------------------------------------------------------

\newcommand*{\DCEEmpty}{\inferrule{
            }{
            \ctxinit \exto \ctxinit
            }\rname{Empty}}

\newcommand*{\DCEVar}{\inferrule{
            \tctx \exto \ctxr
         \\ \applye \ctxr A = \applye \ctxr {A'}
            }{
            \tctx, x: A \exto \ctxr, x: A'
            }\rname{Var}}

\newcommand*{\DCETVar}{\inferrule{
            \tctx \exto \ctxr
            }{
            \tctx, \varA \exto \ctxr, \varA
            }\rname{TVar}}

\newcommand*{\DCEEVar}{\inferrule{
            \tctx \exto \ctxr
            }{
            \tctx, \genA \exto \ctxr, \genA
            }\rname{EVar}}

\newcommand*{\DCESolvedEVar}{\inferrule{
            \tctx \exto \ctxr
         \\ \applye \ctxr {\tau_1} = \applye \ctxr {\tau_2}
             }{
            \tctx, \genA = \tau_1 \exto \ctxr, \genA = \tau_2
            }\rname{SolvedEVar}}

\newcommand*{\DCESolve}{\inferrule{
            \tctx \exto \ctxr
            }{
            \tctx, \genA \exto \ctxr, \genA = \tau
            }\rname{CE-Solve}}

\newcommand*{\DCEAdd}{\inferrule{
            \tctx \exto \ctxr
            }{
            \tctx \exto \ctxr, \genA
            }\rname{CE-Add}}

\newcommand*{\DCEAddSolved}{\inferrule{
            \tctx \exto \ctxr
            }{
            \tctx \exto \ctxr, \genA = \tau
            }\rname{CE-AddSolved}}

\newcommand*{\DCEMarker}{\inferrule{
            \tctx \exto \ctxr
            }{
            \tctx, \marker \genA \exto \ctxr, \marker \genA
            }\rname{CE-Marker}}