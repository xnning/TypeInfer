% SIGPLAN format
\documentclass[oribibl]{llncs}

\usepackage[dvipsnames]{xcolor}

% AMS packages
\usepackage{amsmath}
\usepackage{amssymb}
\usepackage{mathtools}
\usepackage{mdwlist}

% Hyper links
\usepackage{url}
\usepackage{hyperref}
\hypersetup{
   colorlinks,
   citecolor=black,
   filecolor=black,
   linkcolor=black,
   urlcolor=black
}

% Miscellaneous
\usepackage{paralist}
\usepackage{graphicx}
\usepackage{float}
% \usepackage{balance} % Equalize column in the last page

% Revision tools
\usepackage{xspace}
\usepackage{comment}

\usepackage{algpseudocode}
\usepackage{longtable}

\newcommand\authornote[3]{\textcolor{#2}{#1: #3}}
\newcommand\bruno[1]{\authornote{Bruno}{red}{#1}}
\newcommand\ningning[1]{\authornote{Ningning}{blue}{#1}}
\renewcommand\arraystretch{1.1}
%\renewcommand\bruno[1]{}
%\renewcommand\ningning[1]{}

% Literal programming support

\usepackage{listings}
\lstdefinestyle{fun}{
  aboveskip=0.5\baselineskip,
  belowskip=0.5\baselineskip,
  keepspaces=true,
  mathescape=true,
  columns=fullflexible,
  basicstyle=\fontfamily{cmr}\selectfont\small,
  xleftmargin=10pt,
  keywordstyle=\bfseries,
  identifierstyle=\itshape,
  emphstyle=\fontfamily{cmss}\selectfont,
  commentstyle=\fontfamily{cmtt}\selectfont,
  emph={},
  keywords={where, let, in, case, of, data, newtype, ret, if, then,
    else, defrec, def},
  morecomment=[l]{--},
  literate={
    {->}{{$\to$}}1
    {=>}{{$\To$}}1
    {~}{{$\eq$}}1
    {\\}{{$\lambda$}}1
    {/}{{$\mu$}}1
    {\\~}{{$\lambda_\eq$}}1
    {|->}{{$\mapsto$}}1
    {~>}{{$\leadsto$}}1
    {*}{{$\star$}}1
    {**}{{$\times$}}1
    {++}{{+\kern-1.0ex+\kern1.1ex}}1
    {<}{{$\langle$}}1
    {>}{{$\rangle$}}1
    {|=}{{$\vDash$}}1
    {|-}{{$\vdash$}}1
    {||-}{{$\Vdash$}}1
    {G}{{$\Gamma$}}1
    {G0}{{$\cdot$}}1
    {|>}{{$\vartriangleright$}}1
    {forall}{{$\forall$}}1
    {'}{{$^\prime$}}1
    {_1}{{$_1$}}1
    {_2}{{$_2$}}1
    {^2}{{$^2$}}1
    {@}{{$\bullet$}}1
    {@t}{{$\tau$}}1
    {@s}{{$\sigma$}}1
    {castu2}{{$\fun{cast}_\uparrow^2$}}1
    {castd2}{{$\fun{cast}_\downarrow^2$}}1
    {castuf}{{$\fun{cast}_\uparrow^{\fun{f}}$}}1
    {castup}{{$\fun{cast}_\uparrow$}}1
    {castdn}{{$\fun{cast}_\downarrow$}}1
  }
}
\lstset{style=fun}
\newcommand{\lst}[1]{\text{\lstinline$#1$}}

% Table
\usepackage{multirow}
\usepackage{tabularx}
\newcolumntype{Y}{>{\centering\arraybackslash}X}
\newcolumntype{Z}{>{\raggedleft\arraybackslash}X}

% Infer rules
\usepackage{mathpartir}
\newcommand{\rname}[1]{{\,\text{\scriptsize \textsc{#1}}}}
\newcommand{\rul}[1]{\textsc{#1}}

% Extra symbols
\usepackage{stmaryrd}

% Macros for math typesetting

%% Names
% \newcommand{\name}{{\bf $\lambda_{\mu}^{\eq}$}\xspace}

%% Symbols
\newcommand{\syndef}{$::=$}
\newcommand{\synor}{$\mid$}
\newcommand{\syneq}{$\triangleq$}
\newcommand{\header}[1]{\multicolumn{1}{l}{$\boxed{#1}$}}
\newcommand{\headercap}[2]{\multicolumn{1}{l}{$\boxed{#1}$\quad{#2}}}
\newcommand{\headercapm}[2]{\vspace{1pt}\raggedright \framebox{\mbox{$#1$}} \quad
  #2}
\newcommand{\headercapt}[2]{\framebox{\mbox{$#1$}} \quad #2}

%% Arrows
\newcommand{\To}{\Rightarrow}
\newcommand{\Chk}{\Downarrow}
\newcommand{\Inf}{\Uparrow}
\newcommand{\Inst}{{inst}}
\newcommand{\Gen}{{gen}}
\newcommand{\redto}{\hookrightarrow}
\newcommand{\redton}{\hookrightarrow^*}
\newcommand{\eq}{\sim}
\newcommand{\lt}{\sqsubseteq}
\newcommand{\sugar}{\triangleq}
%\newcommand{\trto}[1]{\hl{\rightsquigarrow #1}}
\newcommand{\trto}[1]{}
\newcommand{\opt}[1]{}
\newcommand{\trtop}{\rightsquigarrow}

%% Styles
\newcommand{\kw}[1]{\operatorname{\mathbf{#1}}}
\newcommand{\var}{\mathit}
\newcommand{\fun}{\mathsf}

%% Constructs
\newcommand{\bind}[3]{#1 #2:#3.~}
\newcommand{\blam}{\bind \lambda}
\newcommand{\bmu}{\bind \mu}
\newcommand{\barr}[2]{(#1:#2) \to}

\newcommand{\bindv}[4][]{#2\,\overline{#3:#4}^{#1}.~}
\newcommand{\blamv}[3][]{\bindv[#1] \lambda {#2} {#3}}
\newcommand{\bmuv}[3][]{\bindv[#1] \mu {#2} {#3}}
\newcommand{\barrv}[3][]{\overline{#2:#3}^{#1} \to}

\newcommand{\eqlam}[2]{\lambda_{\eq}({#1} \eq {#2}).~}
\newcommand{\eqty}[2]{({#1} \eq {#2})\Rightarrow}
\newcommand{\eqapp}[2][]{\langle {#2} \rangle^{#1}}
\newcommand{\eqlamv}[3][]{\lambda_{\eq}\overline{{#2} \eq {#3}}^{#1}.~}
\newcommand{\eqtyv}[3][]{\overline{({#2} \eq {#3})}^{#1}\Rightarrow}

\newcommand{\bpi}{\bind \Pi}
\newcommand{\bpiv}[3][]{\bindv[#1] \Pi {#2} {#3}}

\newcommand{\fold}{\fun{fold}}
\newcommand{\unfold}{\fun{unfold}}

\newcommand{\castupz}{\fun{cast}_\uparrow}
\newcommand{\castup}[2][]{\fun{cast}_\uparrow^{#1}~[#2]~}
\newcommand{\castdnz}{\fun{cast}_\downarrow}
\newcommand{\castdn}[1][]{\fun{cast}_\downarrow^{#1}~}
\newcommand{\castdnt}[2][]{\fun{cast}_\downarrow^{#1}~[#2]~}
\newcommand{\subst}[2]{[#1 \mapsto #2]}
\newcommand{\Subst}[2]{[#1 \Mapsto #2]}
\newcommand{\substv}[3][]{\overline{[#2 \mapsto #3]}^{#1}}
\newcommand{\cast}[2][]{\fun{cast}^{#1}~[#2]~}

\newcommand{\triv}{\_}
\newcommand{\trivtm}{\bullet}
\newcommand{\er}[1]{|{#1}|}
\newcommand{\erf}[1]{\|{#1}\|}
\newcommand{\erlam}[1]{\lambda {#1}.~}
\newcommand{\ermu}[1]{\mu {#1}.~}
\newcommand{\ercastup}{\castupz~}
\newcommand{\ereqlam}{\lambda_\eq.~}

\newcommand{\genvar}{\widehat}
\newcommand{\genA}{\genvar{\alpha}}
\newcommand{\genB}{\genvar{\beta}}
\newcommand{\varA}{\alpha}
\newcommand{\varB}{\beta}

%% Context
\newcommand{\dctx}{\Psi}
\newcommand{\tctx}{\Gamma}
\newcommand{\ctxinit}{\varnothing}
\newcommand{\ctxl}{\Theta}
\newcommand{\ctxr}{\Delta}
\newcommand{\cctx}{\Omega}
\newcommand{\byuni}{\vdash}
\newcommand{\byinf}{\vdash_\Inf}
\newcommand{\bychk}{\vdash_\Chk}
\newcommand{\byapp}{\vdash_\bullet}
\newcommand{\byall}{\vdash_\delta}
\newcommand{\bytar}{\vdash}
\newcommand{\byinst}{\vdash_\Inst}
\newcommand{\bygen}{\vdash_\Gen}
\newcommand{\bycg}{\vdash}
\newcommand{\bywf}{\vdash}
\newcommand{\bywt}{\vDash}
\newcommand{\toctx}{\dashv \ctxl}
\newcommand{\toctxo}{\dashv \tctx}
\newcommand{\toctxr}{\dashv \ctxr}
\newcommand{\dpreinf}[1][]{\dctx {#1} \byinf}
\newcommand{\dprechk}[1][]{\dctx {#1} \bychk}
\newcommand{\dpreall}[1][]{\dctx {#1} \byall}
\newcommand{\dpreapp}[1][]{\dctx {#1} \byapp}
\newcommand{\dpreuni}[1][]{\dctx {#1} \byuni}
\newcommand{\dpretar}[1][]{\dtctx {#1} \bytar}
\newcommand{\dpreinst}[1][]{\dctx {#1} \byinst}
\newcommand{\dpregen}[1][]{\dctx {#1} \bygen}
\newcommand{\dprecg}[1][]{\dctx {#1} \bycg}
\newcommand{\dprewf}[1][]{\dctx {#1} \bywf}
\newcommand{\dprewt}[1][]{\dctx {#1} \bywt}
\newcommand{\tpreinf}[1][]{\tctx {#1} \byinf}
\newcommand{\tprechk}[1][]{\tctx {#1} \bychk}
\newcommand{\tpreall}[1][]{\tctx {#1} \byall}
\newcommand{\tpreapp}[1][]{\tctx {#1} \byapp}
\newcommand{\tpreuni}[1][]{\tctx {#1} \byuni}
\newcommand{\tpretar}[1][]{\ttctx {#1} \bytar}
\newcommand{\tpreinst}[1][]{\tctx {#1} \byinst}
\newcommand{\tpregen}[1][]{\tctx {#1} \bygen}
\newcommand{\tprecg}[1][]{\tctx {#1} \bycg}
\newcommand{\tprewf}[1][]{\tctx {#1} \bywf}
\newcommand{\tprewt}[1][]{\tctx {#1} \bywt}
\newcommand{\wc}{\ \var{ctx}\ }
\newcommand{\exto}{\longrightarrow}
\newcommand{\cgto}{\longmapsto}

% Primitives
\newcommand{\Int}{\var{Int}}
\newcommand{\String}{\var{String}}

\newcommand{\overbar}[1]{\mkern 1.5mu\overline{\mkern-1.5mu#1\mkern-1.5mu}\mkern 1.5mu}

%%% Local Variables:
%%% mode: latex
%%% TeX-master: "main"
%%% End:


\usepackage[round, sort]{natbib}
\bibliographystyle {plainnat}

\begin{document}
\title{Sanitized Type Inference in Context}

\author{Ningning Xie \and Bruno C. d. S. Oliveira}
\institute{The University of Hong Kong}

\maketitle

\begin{abstract}
  Gundry et al. proposed type inference in context as a general foundation for
  unification/type inference algorithms. The key idea is based on the notion of
  information increase. Following this work, a more syntactic foundation for
  information increase is also used to deal with higher rank polymorphism.
  However, none of the existing work shows how it could be extended naturally to deal
  with dependency between existential variables in the presence of dependent types.
  In this paper, we propose a strategy called \textit{type
    sanitization} that helps resolve this problem
  in the framework of type inference in context.\bruno{Why do we need
    help; what is the problem? Say this first} \ningning{how about now?}We show
  that type sanitization works on a unification algorithm for a dependent type
  system with alpha-equality and first-order constraints. We then further extend
  this strategy to deal with polymorphic subtyping in a higher ranked
  polymorphic type system.
\end{abstract}

% Setup spaces between column
\setlength{\tabcolsep}{2pt}

%Complete Contexts &
%$\cctx$ & \syndef & $\ctxinit \mid \ctx,x \mid \ctx,x:\tau \mid \ctx,x:\tau=\tau_2$ \\
%&& \synor & $\ctx,\genA=\tau$ \\

% ------------------------------------------------------------------------
% TYPING RULES
% ------------------------------------------------------------------------

\newcommand*{\TAx}{\inferrule{ }{\preinf \star:\star \toctxo}\rname{T-Ax}}
\newcommand*{\TVar}{\inferrule{x:\tau \in \ctx}{\preinf x:\tau \toctxo
  }\rname{T-Var}}
\newcommand*{\TLetVar}{\inferrule{x:\sigma = \tau \in \ctx \\ \preinst \sigma \lt \tau_2 \toctx}{\preinf x:\tau_2 \toctx
  }\rname{T-LetVar}}
\newcommand*{\TSub}{\inferrule{\preinf e : \tau_1 \toctx_1 \\ \ctxl_1 \byuni [\ctxl_1]\tau_1
    \lt [\ctxl_1]\tau_2 \toctx}{\prechk e:\tau_2 \toctx }\rname{T-Sub}}
\newcommand*{\TAnn}{\inferrule{\prechk \tau:\star \toctx_1 \\
    \ctxl_1 \bychk e:\tau \toctx }{\preinf (e:\tau):\tau \toctx
  }\rname{T-Ann}}
\newcommand*{\TLamInf}{\inferrule{\preinf[,\genA,x:\genA]
    e:\tau_2 \toctx, x:\genA, \ctxr }{\preinf \erlam x e : (\bpi x \genA
    [\ctxr]\tau_2) \toctx, UV(\ctxr) }\rname{T-Lam$\Inf$}}
\newcommand*{\TLamChk}{\inferrule{\prechk[,x:\tau_1]
    e:\tau_2 \toctx,x:\tau_1,\ctxr \\
    \opt{\prechk {\tau_1 : \star \toctx_1 }}
    }{\prechk \erlam x e : \bpi x {\tau_1}
    \tau_2 \toctx }\rname{T-Lam$\Chk$}}
\newcommand*{\TLamAnnInf}{\inferrule{\prechk \tau_1 : \star \toctx_1\\
    \ctxl_1,x:\tau_1 \byinf
    e:\tau_2 \toctx, x:\tau_1, \ctxr }{\preinf \blam x {\tau_1} e : (\bpi x {\tau_1}
    [\ctxr]\tau_2) \toctx, UV(\ctxr) }\rname{T-LamAnn$\Inf$}}
\newcommand*{\TLamAnnChk}{\inferrule{\prechk \tau_1 : \star \toctx_1\\
    \ctxl_1 \byuni [\ctxl_1] \tau_1 \lt [\ctxl_1] \tau_3 \toctx_2 \\
    \ctxl_2,x:\tau_1 \bychk
    e:\tau_2 \toctx, x:\tau_1, \ctxr }
    {\prechk \blam x {\tau_1} e : (\bpi x {\tau_3} \tau_2) \toctx }\rname{T-LamAnn$\Chk$}}
\newcommand*{\TApp}{\inferrule{
    \preinf e_1 : \tau_1 \toctx_1 \\
    \ctxl_1 \byapp [\ctxl_1]\tau_1~e_2 : \tau_2 \toctx \\
}{\preinf e_1~e_2:\tau_2 \toctx}\rname{T-App}}
\newcommand*{\TAppPi}{\inferrule{
    \preinf e_1 : \bpi x {\tau_1} \tau_2 \toctx_1 \\
    \ctxl_1 \bychk e_2 : [\ctxl_1]\tau_1 \toctx \\
}{\preinf e_1~e_2:\tau_2 \subst x
    {e_2} \toctx}\rname{T-AppPi}}
\newcommand*{\TAppVar}{\inferrule{
    \preinf e_1 : \genA \toctx_1[\genA] \\
    \ctxl_1[\genA_1,\genA_2,\genA=\bpi x {\genA_1} \genA_2] \bychk e_2 : \genA_1 \toctx \\
}{\preinf e_1~e_2:\genA_2 \toctx}\rname{T-AppVar}}
\newcommand*{\TPi}{\inferrule{\prechk \tau_1 : \star \toctx_1 \\
\ctxl_1,x:\tau_1 \bychk \tau_2 : \star \toctx,x:\tau_1,\ctxr}{\preinf \bpi x {\tau_1} {\tau_2} :
    \star \toctx }\rname{T-Pi}}
\newcommand*{\TLet}{\inferrule{\preinf e_1 : \tau_1 \toctx_1  \\
\ctxl_1 \bygen {\tau_1} \lt \sigma \\
\ctxl_1, x:\sigma = e_1 \byall e_2 : \tau_2 \toctx, x:\sigma = e_1, \ctxr }{\preall \kw{let} x=e_1
\kw{in} e_2 : [x:\sigma=e_1, \ctxr]\tau_2 \toctx, UV(\ctxr) }\rname{T-Let}}
\newcommand*{\TCastUp}{\inferrule{[\ctx]\tau_2
    \redto \tau_1 \\
    \prechk e : \tau_1 \toctx \\
    \opt{\prechk \tau_1 : \star \toctx_1}
    }
  {\prechk \ercastup e : \tau_2 \toctx
    }\rname{T-CastUp}}
\newcommand*{\TCastDn}{\inferrule{\preinf e : \tau_1 \toctx \\
    [\ctxl]\tau_1 \redto \tau_2}{\preinf \castdn e : \tau_2
    \toctx }\rname{T-CastDn}}

% DECLARATIVE

\newcommand*{\DAx}{\inferrule{ }{\preinf \star:\star \trto \star}\rname{D-Ax}}
\newcommand*{\DVar}{\inferrule{x:\tau \in \ctx}{\preinf x:\tau
  \trto x}\rname{D-Var}}
\newcommand*{\DLetVar}{\inferrule{x:\sigma = \tau \in \ctx \\ \preinst \sigma \lt \tau_2 \trto f}{\preinf x:\tau_2
  \trto {f~x}}\rname{D-LetVar}}
\newcommand*{\DSub}{\inferrule{\preinf e : \tau \trto{t}}
  {\prechk e:\tau  \trto{t}}\rname{D-Sub}}
\newcommand*{\DAnn}{\inferrule{\prechk \tau:\star \\
    \ctx \bychk e:\tau  \trto t}{\preinf (e:\tau):\tau
  \trto t}\rname{D-Ann}}
\newcommand*{\DLamInf}{\inferrule{ \prechk \tau_1 : \star \trto {t_1} \\ \preinf[,x:\tau_1]
    e:\tau_2 \trto {t_2}}{\preinf \erlam x e : (\bpi x {\tau_1} {\tau_2})
    \trto {\blam x {t_1} {t_2}}}\rname{D-Lam$\Inf$}}
\newcommand*{\DLamChk}{\inferrule{\prechk[,x:\tau_1]
    e:\tau_2 \trto {t_2} \\
    \opt{\prechk {\tau_1 : \star  \trto {t_1}}}
    }{\prechk \erlam x e : (\bpi x {\tau_1} \tau_2)  \trto {\blam x {t_1} t_2}}\rname{D-Lam$\Chk$}}
\newcommand*{\DLamAnnInf}{\inferrule{\prechk \tau_1 : \star
    \trto {t_1} \\
    \ctx,x:\tau_1 \byinf
    e:\tau_2\trto {t_2}}{\preinf \blam x {\tau_1} e : (\bpi x {\tau_1}
    \tau_2) \trto {\blam x {t_1} t_2}}\rname{D-LamAnn$\Inf$}}
\newcommand*{\DLamAnnChk}{\inferrule{
    \ctx,x:\tau_1 \bychk
    e:\tau_2\trto {t_2}}{\prechk \blam x {\tau_1} e : (\bpi x {\tau_1}
    \tau_2) \trto {\blam x {t_1} t_2}}\rname{D-LamAnn$\Chk$}}
\newcommand*{\DApp}{\inferrule{
    \preinf e_1 : \bpi  x {\tau_1} {\tau_2} \trto {t_1} \\
    \prechk e_2 : \tau_1 \trto {t_2}
}{\preinf e_1~e_2:\tau_2 \subst x {e_2}  \trto {t_1~t_2}}\rname{D-App}}
\newcommand*{\DPi}{\inferrule{\prechk \tau_1 : \star \trto {t_1} \\
\ctx,x:\tau_1 \bychk \tau_2 : \star \trto {t_2}}{\preinf \bpi x {\tau_1} {\tau_2} :
    \star \trto {\bpi x {t_1} t_2}}\rname{D-Pi}}
\newcommand*{\DLet}{\inferrule{\pregen e_1 : \sigma \trto {t_1} \\
\ctx, x:\sigma = e_1 \byall e_2 : \tau_2 \trto {t_2}}{\preall \kw{let} x=e_1
\kw{in} e_2 : \tau_2 \subst x {e_1} \trto {\kw{let} x = t_1 \kw{in} t_2}}\rname{D-Let}}
\newcommand*{\DCastUp}{\inferrule{\tau_2 \redto \tau_1 \\
    \prechk e : \tau_1 \trto {t_2} \\
    \opt{\prechk \tau_1 : \star \trto {t_1}}
    }
  {\prechk \ercastup e : \tau_2
    \trto {\castup {t_1} t_2}}\rname{D-CastUp}}
\newcommand*{\DCastDn}{\inferrule{\preinf e : \tau_1\trto t \\
    \tau_1 \redto \tau_2}{\preinf \castdn e : \tau_2
    \trto {\castdn t}}\rname{D-CastDn}}
\newcommand*{\DConv}{\inferrule{\preinf e_1 : \tau_1 \trto t \\ [\ctx]\tau_1 = [\ctx]\tau_2}
    {\preinf e_1 : \tau_2 \trto t}\rname{D-Conv}}

\newcommand*{\DPoly}{\inferrule{\prechk[x:\star] \sigma : \star \trto {t}}
{\preinf \forall x:\star. \sigma : \star \trto {\bpi x \star t}}\rname{D-Poly}}

\newcommand*{\DInstantiation}{\inferrule{\prechk \overbar{\tau} : \star \trto {\overbar t} \\
\sigma = \forall{\overbar{x:\star}}. \tau_1 \\
\opt{\prechk \sigma : \star \trto{t_1}}
}
{\preinst \sigma \lt \tau_1[\overbar{x} \mapsto \overbar{\tau}] \trto {\blam x {t_1} x ~ \overbar{t}}
} \rname{D-Inst}}

\newcommand*{\DGeneralization}{\inferrule{ \preinf[, \overbar{x:\star}] e : \tau \trto {t_1}
\\  \overbar x \notin FV(e)}
{\pregen e : \forall \overbar{x:\star}. \tau
\trto {(\blam {x_1} \star {\blam {x_2} \star {... \blam {x_n} \star {t_1}}})}} \rname{D-Gen}}

% ------------------------------------------------------------------------
% UNIFICATION RULES
% ------------------------------------------------------------------------

\newcommand*{\UVar}{\inferrule{ }{\preuni[{[x]}] x \lt x \toctxo[x]}\rname{U-Var}}
\newcommand*{\UEVarId}{\inferrule{ }{\preuni[{[\genA]}] \genA \lt \genA \toctxo[\genA]}\rname{U-EVarId}}
\newcommand*{\UEVarTy}{\inferrule{\genA \not \in \fun{FV}(\tau_1) \\ \ctx[\genA] \bycg \tau_1 \cgto \tau_2 \toctx_1, \genA, \ctxl_2 \\ \ctxl_1 \bywt \tau_2}
{\ctx[\genA] \byuni \genA \lt \tau_1 \toctx_1, \genA=\tau_2, \ctxl_2}\rname{U-EvarTy}}
\newcommand*{\UTyEVar}{\inferrule{\genA \not \in \fun{FV}(\tau_1) \\ \ctx[\genA] \bycg \tau_1 \cgto \tau_2 \toctx_1, \genA, \ctxl_2 \\ \ctxl_1 \bywt \tau_2}
{\ctx[\genA] \byuni \tau_1 \lt \genA \toctx_1, \genA=\tau_2, \ctxl_2}\rname{U-TyEVar}}
\newcommand*{\UStar}{\inferrule{ }{\preuni \star \lt \star \toctxo}\rname{U-Star}}
\newcommand*{\UApp}{\inferrule{\preuni \tau_2 \lt \tau_2' \toctx_1 \\
    \ctxl_1 \byuni [\ctxl_1]\tau_1 \lt [\ctxl_1]\tau_1'
    \toctx}{\preuni \tau_1~\tau_2 \lt \tau_1'~\tau_2'
    \toctx}\rname{U-App}}
\newcommand*{\ULam}{\inferrule{\preuni[,x] \tau \lt \tau'
    \toctx,x,\ctxr}{\preuni \erlam x \tau \lt \erlam x \tau' \toctx}\rname{U-Lam}}
\newcommand*{\ULamAnn}{\inferrule{\preuni \tau_1 \lt \tau_3 \toctx_1 \\
    \ctxl_1, x:\tau_1 \byuni [\ctxl_1]\tau_2 \lt [\ctxl_1]\tau_4
    \toctx,x:\tau_1,\ctxr}{\preuni \blam x {\tau_1} \tau_2 \lt \blam x
    {\tau_3} \tau_4 \toctx}\rname{U-LamAnn}}
\newcommand*{\UPi}{\inferrule{\preuni \tau_1' \lt \tau_1 \toctx_1
    \\ \ctxl_1,x:\tau_1 \byuni [\ctxl_1]\tau_2 \lt [\ctxl_1]\tau_2'
    \toctx,x:\tau_1,\ctxr}{\preuni \bpi x {\tau_1} \tau_2 \lt \bpi x
    {\tau_1'} \tau_2' \toctx}\rname{U-Pi}}
\newcommand*{\ULet}{\inferrule{\preuni \tau_1 \lt \tau_1' \toctx_1
    \\ \ctxl_1, x \byuni {[\ctxl_1]}\tau_2 \lt [\ctxl_1]\tau_2'
    \toctx, x, \ctxr}{\preuni \kw{let} x ={\tau_1} \kw{in} \tau_2 \lt \kw{let} x=
    {\tau_1'} \kw{in} \tau_2' \toctx}\rname{U-Let}}
\newcommand*{\UCastUp}{\inferrule{\preuni \tau \lt \tau'
    \toctx}{\preuni \ercastup \tau \lt \ercastup \tau' \toctx}\rname{U-CastUp}}
\newcommand*{\UCastDn}{\inferrule{\preuni \tau \lt \tau'
    \toctx}{\preuni \castdn \tau \lt \castdn \tau' \toctx}\rname{U-CastDn}}
\newcommand*{\UAnn}{\inferrule{\preuni \tau \lt \tau' \toctx_1 \\
    \ctxl_1 \byuni [\ctxl_1]e \lt [\ctxl_1]e'
    \toctx}{\preuni e:\tau \lt e':\tau'
    \toctx}\rname{U-Ann}}

% ------------------------------------------------------------------------
% APPLICATION RULES
% ------------------------------------------------------------------------

\newcommand*{\APi}{\inferrule{\prechk e:\tau_1 \toctx \trto t}{\preapp (\bpi x
    {\tau_1} \tau_2)~e : \tau_2[x \mapsto e] \toctx \trto t}\rname{A-Pi}}
\newcommand*{\AEVar}{\inferrule{\prechk[{[\genA_2,\genA_1,\genA=\bpi x
    {\genA_1} \genA_2]}] e : \genA_1 \toctx \trto t}{\preapp[{[\genA]}]
  \genA~e : \genA_2 \toctx \trto t}\rname{A-EVar}}

% ------------------------------------------------------------------------
% TARGET TYPING RULES
% ------------------------------------------------------------------------

\newcommand*{\EAx}{\inferrule{ }{\pretar \star:\star }\rname{E-Ax}}
\newcommand*{\EVar}{\inferrule{x:s \in \tctx}{\pretar x:s}\rname{E-Var}}
\newcommand*{\EApp}{\inferrule{\pretar t_1:\bpi x {s_1} {s_2} \\ \pretar
    t_2:s_1}{\pretar t_1~t_2:s_2 \subst
  x {t_2}}\rname{E-App}}
\newcommand*{\ELam}{\inferrule{\pretar t_1:\star \\ \pretar[,x:t_1] t_2:s_1
    }{\pretar \blam
    x {t_1} t_2 : \bpi x {t_1} {s_1}}\rname{E-Lam}}
\newcommand*{\EPi}{\inferrule{\pretar t_1:\star \\ \pretar[,x:t_1] t_2:\star }{\pretar \bpi x {t_1} t_2 :
    \star}\rname{E-Pi}}
\newcommand*{\ECastUp}{\inferrule{\pretar t_1 : \star \\ \pretar t_2: s_1 \\ t_1 \redto s_1 }{\pretar \castup {t_1} {t_2} :t_1}\rname{E-CastUp}}
\newcommand*{\ECastDown}{\inferrule{\pretar t_1 : s_1 \\ s_1 \redto s_2 }{\pretar \castdn t_1 :s_2}\rname{E-CastDown}}
\newcommand*{\ELet}{\inferrule{\pretar t_1 : s_1 \\ \pretar[,x:s_1=t_1] t_2:s_2}{\pretar \kw{let} x = t_1 \kw{in} t_2: s_2}\rname{E-Let}}
\newcommand*{\EConv}{\inferrule{\pretar t_1 : s_1 \\ [\tctx]s_1 = [\tctx]s_2}{\pretar t_1 : s_2}\rname{E-Conv}}

% ------------------------------------------------------------------------
% POLYMORPHISM
% ------------------------------------------------------------------------

\newcommand*{\Instantiation}{\inferrule{\sigma = \forall{\overbar{x:\star}}. \tau}{\preinst \sigma \lt \tau[\overbar{x} \mapsto \overbar{\genA}] \toctx, \overbar{\genA}} \rname{Inst}}

\newcommand*{\Generalization}{\inferrule{\tau_2 = [\ctx]\tau \\ \overbar{\genA} = FV(\tau_2) - FV(\ctx)}{\pregen \tau \lt \forall \overbar{x:\star}. \tau_2[\overbar{\genA} \mapsto \overbar{x}]} \rname{Gen}}

% ------------------------------------------------------------------------
% UNIFY TVAR
% ------------------------------------------------------------------------

\newcommand*{\IVar}{\inferrule{ }{\ctx \bycg x \cgto x \toctxo}\rname{I-Var}}
\newcommand*{\IStar}{\inferrule{ }{\ctx \bycg \star \cgto \star \toctxo}\rname{I-Star}}
\newcommand*{\IEVarA}{\inferrule{ }{\ctx[\genB][\genA] \bycg \genB \cgto \genB \toctxo[\genB][\genA]}\rname{I-EVar1}}
\newcommand*{\IEVarB}{\inferrule{ }{\ctx[\genA][\genB] \bycg \genB \cgto \genA_1 \toctxo[\genA_1, \genA][\genB=\genA_1]}\rname{I-EVar2}}
\newcommand*{\IOthers}{\inferrule{\ctx \bycg \tau_0 \cgto \tau_0' \toctx_1 \\ \ctxl_i \bycg [\ctxl_i]\tau_i \cgto \tau_i' \toctx_{i+1}}{\ctx \bycg T\ \overbar{\tau_n} \cgto T\ \overbar{\tau_n'}}\rname{I-Other}}

% ------------------------------------------------------------------------
% WELL FORM
% ------------------------------------------------------------------------

\newcommand*{\WFPoly}{\inferrule{ \prewt[,x:\star] \sigma}{ \prewt \forall x: \star. \sigma}\rname{WF-Poly}}
\newcommand*{\WFOther}{\inferrule{ \prechk \tau : \star}{ \prewt \tau}\rname{WF-Other}}

\newcommand*{\TWFEVar}{\inferrule{\genA \in \ctx }{\prewt \genA}\rname{WF-EVar}}
\newcommand*{\TWFPi}{\inferrule{\prewt \tau_1 \\ \prewt \tau_2 }{\prewt \bpi x {\tau_1} {\tau_2}}\rname{WF-Pi}}
\newcommand*{\TWFPoly}{\WFPoly}
\newcommand*{\TWFOther}{\inferrule{ \prechk \tau : \star \toctx}{ \prewt \tau}\rname{WF-Other}}

\newcommand*{\WCEmpty}{\inferrule{ }{\ctxinit \wc}\rname{WC-Empty}}
\newcommand*{\WCVar}{\inferrule{\ctx \wc \\ x \notin dom(\ctx)}{\ctx, x \wc}\rname{WC-Var}}
\newcommand*{\WCTypedVar}{\inferrule{\ctx \wc \\ x \notin dom(\ctx) \\ \ctx \bywt \tau}{\ctx, x: \tau \wc}\rname{WC-TypedVar}}
\newcommand*{\WCLetVar}{\inferrule{\ctx \wc \\ x \notin dom(\ctx) \\ \ctx \bywt \sigma \\ \sigma = \forall {\overbar {y:\star}}.\tau \\ \prechk[,\overbar{y:\star}] e:\tau}
{\ctx, x:\sigma = e}\rname{WC-LetVar}}

\newcommand*{\TWCEmpty}{\WCEmpty}
\newcommand*{\TWCVar}{\WCVar}
\newcommand*{\TWCTypedVar}{\WCTypedVar}
\newcommand*{\TWCLetVar}{\inferrule{\ctx \wc \\ x \notin dom(\ctx) \\ \ctx \bywt \sigma \\ \sigma = \forall {\overbar {y:\star}}.\tau \\ \prechk[,\overbar{y:\star}] e:\tau\toctx}
{\ctx, x:\sigma = e}\rname{WC-LetVar}}
\newcommand*{\TWCEVar}{\inferrule{\ctx \wc \\ \genA \notin dom(\ctx)}{\ctx, \genA \wc}\rname{WC-EVar}}
\newcommand*{\TWCSolvedEVar}{\inferrule{\ctx \wc \\ \genA \notin dom(\ctx) \\ \ctx \bywt \tau}{\ctx, \genA = \tau \wc}\rname{WC-SolvedEVar}}

% ------------------------------------------------------------------------
% TRANSLATION CONTEXT
% ------------------------------------------------------------------------

\newcommand*{\TCEmpty}{\inferrule{ } {\ctxinit \trtop \ctxinit}\rname{TC-Empty}}
\newcommand*{\TCTypedVar}{\inferrule{\ctx \trtop \tctx \\ \prechk \tau : \star \trtop t} {\ctx, x:\tau \trtop \tctx, x:t}\rname{TC-TypedVar}}
\newcommand*{\TCLetVar}{\inferrule{\ctx \trtop \tctx \\ \prechk \sigma : \star \trtop t_1 \\ \pregen \tau : \sigma \trtop t_2
} {\ctx, x:\sigma = \tau \trtop \tctx, x:t_1 = t_2}\rname{TC-LetVar}}

% ------------------------------------------------------------------------
% CONTEXT EXTENSION
% ------------------------------------------------------------------------

\newcommand*{\CEEmtpy}{\inferrule{  }{\ctxinit \exto \ctxinit}\rname{CE-Empty}}
\newcommand*{\CEVar}{\inferrule{\ctx \exto \ctxr}{\ctx, x \exto \ctxr, x}\rname{CE-Var}}
\newcommand*{\CETypedVar}{\inferrule{\ctx \exto \ctxr \\ [\ctxr]\tau_1 = [\ctxr]\tau_2}{\ctx, x:\tau_1 \exto \ctxr, x:\tau_2}\rname{CE-TypedVar}}
\newcommand*{\CELetVar}{\inferrule{\ctx \exto \ctxr \\ [\ctxr]\tau_1 = [\ctxr]\tau_3 \\ [\ctxr]\tau_2 = [\ctxr]\tau_4}{\ctx, x:\tau_1=\tau_2 \exto \ctxr, x:\tau_3 = \tau_4}\rname{CE-LetVar}}
\newcommand*{\CEEVar}{\inferrule{ }{\ctx, \genA \exto \ctxr, \genA}\rname{CE-EVar}}
\newcommand*{\CESolvedEVar}{\inferrule{\ctx \exto \ctxr \\ [\ctxr]\tau_1 = [\ctxr]\tau_2}{\ctx, \genA = \tau_1 \exto \ctxr, \genA = \tau_2}\rname{CE-SolvedEVar}}
\newcommand*{\CESolve}{\inferrule{\ctx \exto \ctxr \\ \prechk \tau : \star}{\ctx, \genA \exto \ctxr, \genA = \tau}\rname{CE-Solve}}
\newcommand*{\CEAdd}{\inferrule{\ctx \exto \ctxr}{\ctx, \genA \exto \ctxr, \genA}\rname{CE-Add}}
\newcommand*{\CEAddSolved}{\inferrule{\ctx \exto \ctxr \\ \prechk \tau : \star}{\ctx \exto \ctxr, \genA:\tau}\rname{CE-AddSolved}}

% ------------------------------------------------------------------------
% REFERENCE OF ORIGINAL SYSTEM
% ------------------------------------------------------------------------

\newcommand*{\OLamInf}{\inferrule{\prechk[,\genA,\genB,x:\genA]
    e:\genB \toctx, x:\genA, \ctxr}{\preinf \erlam x e : (\bpi x \genA
    \genB) \toctx}\rname{$\rightarrow$ I $\Rightarrow$}}

\newcommand*{\OInstLArr}{\inferrule{\preuni[{[\genA_2, \genA_1, \genA = \genA_1 \to \genA_2]}] \genA_1 \lt A_1 \toctx_1 \\
    \ctxl_1 \byuni [\ctxl_1]A_2 \lt \genA_2 \toctx} {\preuni[{[\genA]}] \genA \lt A_1 \to A_2 \toctx}\rname{InstLArr}}

\newcommand*{\OInstLSolve}{\inferrule{\ctx \bywt \tau}{\ctx, \genA, \ctx' \byuni \genA \lt \tau \toctxo, \genA = \tau, \ctx'}\rname{InstLSolve}}

\newcommand*{\OInstLReach}{\inferrule{ }{\preuni[{[\genA][\genB]}] \genA \lt \genB \toctxo[\genA][\genB=\genA]}\rname{InstLReach}}

% ------------------------------------------------------------------------
% OPERATIONAL SEMANTICS
% ------------------------------------------------------------------------

\newcommand*{\SBetaA}{\inferrule{ }{(\blam x \tau {e_1}) e_2 \redto e_1 \subst x {e_2} }\rname{S-Beta1}}
\newcommand*{\SBetaB}{\inferrule{ }{(\erlam x {e_1}) e_2 \redto e_1 \subst x {e_2}}\rname{S-Beta2}}
\newcommand*{\SApp}{\inferrule{ e_1 \redto e_1' }{e_1~e_2 \redto e_1'~e_2}\rname{S-App}}
\newcommand*{\SCastDownUp}{\inferrule{  }{\castdn (\ercastup e) \redto e}\rname{S-CastDownUp}}
\newcommand*{\SCastDown}{\inferrule{e \redto e'}{\castdn e \redto \castdn e'}\rname{S-CastDown}}
\newcommand*{\SLet}{\inferrule{ }{\kw{let} x = e_1 \kw{in} e_2 \redto e_2 \subst x {e_1}}\rname{S-Let}}
\newcommand*{\SAnn}{\inferrule{e \redto e'}{ e:\tau \redto e':\tau}\rname{S-Ann}}

% ------------------------------------------------------------------------
% EXAMPLES
% ------------------------------------------------------------------------

\newcommand*{\ExUni}{\inferrule{\genA \notin FV(\bpi x \genB x) \quad
                                       \inferrule{\inferrule{ }
                                                            {\ctx,\genA,\genB,\ctxr \bycg \genB \cgto \genA_1 \toctx}\rname{I-Evar2}
                                                  \quad
                                                  \inferrule{ }
                                                            {\ctxl \bycg x \cgto x \toctx}\rname{I-Var}}
                                                 {\ctx,\genA,\genB,\ctxr \bycg \bpi x \genB x \cgto \bpi x {\genA_1} x \toctx}\rname{I-Other}
                                      \quad
                                      \inferrule{ %\inferrule{ }{\ctx,\genA_1 \bywt \genA_1}\rname{WF-EVar} \quad \inferrule{ }{\ctx,\genA_1,x \bywt x}\rname{WF-Var}
                                                 }
                                                {\ctx,\genA_1 \bywt \bpi x {\genA_1} x}\rname{WF-Pi}}
                               {\ctx,\genA,\genB,\ctxr \byuni \genA \lt \bpi x \genB x \toctxo, \genA_1, \genA=\bpi x {\genA_1} x, \genB=\genA_1, \ctxr} \rname{U-EvarTy}}

%%% Local Variables:
%%% mode: latex
%%% TeX-master: "../main"
%%% End:


\newcommand*{\ContextApplicationIsIdempotentName}{Context Application is Idempotent}
\newcommand*{\ContextApplicationIsIdempotentBody}{
  If $\tctx \wc$,
  then $\applye \tctx {\applye \tctx \sigma} = \applye \tctx \sigma$.
}

\newcommand*{\ContextApplicationPreservesTypingName}{Context Application Preserves Typing}
\newcommand*{\ContextApplicationPreservesTypingBody}{
  If $\tctx \byinf \sigma_1 \infto \sigma_2$,
  then $\tctx \byinf \applye \tctx {\sigma_1} \infto \sigma_2$.
}

\newcommand*{\OutputIsFullySubstitutedName}{Output is Fully Substituted}
\newcommand*{\OutputIsFullySubstitutedBody}{
  If $\tctx \byinf \sigma_1 \infto \sigma_2$,
  then $\applye \tctx {\sigma_2} = \sigma_2$.
}

\newcommand*{\ReductionPreservesFullySubstitutionName}{Reduction Preserves Fully Substitution}
\newcommand*{\ReductionPreservesFullySubstitutionBody}{
  If $\tctx \wc$,
  and $\applye \tctx {\sigma} = \sigma$,
  and $\sigma \redto \tau$,
  then $\applye \tctx \tau = \tau$.
}

\newcommand*{\ContextApplicationOverReductionName}{Context Application Over Reduction}
\newcommand*{\ContextApplicationOverReductionBody}{
  If $\sigma_1 \redto \sigma_2$,
  and $\applye \tctx {\sigma_1} \redto \applye \tctx {\sigma_2}$.
}

\newcommand*{\ContextApplicationInContextName}{Context Application In Context}
\newcommand*{\ContextApplicationInContextBody}{
  If $\tctx_1, y: \tau, \tctx_2 \byinf \sigma_1 \infto \sigma_2$,
  and $\tctx_1, y : \applye {\tctx_1} \tau, \tctx_2 \wc $,
  then $\tctx_1, y : \applye {\tctx_1} \tau, \tctx_2 \byinf \sigma_1 \infto \sigma_2$.
}

\newcommand*{\ReverseContextApplicationInContextName}{Reverse Context
  Application In Context}
\newcommand*{\ReverseContextApplicationInContextBody}{
  If $\tctx_1, y: \applye {\tctx_1} \tau, \tctx_2 \byinf \sigma_1 \infto \sigma_2$,
  and $\tctx_1, y : \tau, \tctx_2 \wc $,
  then $\tctx_1, y : \tau, \tctx_2 \byinf \sigma_1 \infto \sigma_2$.
}

\newcommand*{\TypingWeakeningName}{Typing Weakening}
\newcommand*{\TypingWeakeningBody}{
  If $\tctx_1, \tctx_2 \byinf \sigma_1 \infto \sigma_2$,
  and $\tctx_1, \ctxl, \tctx_2 \wc$,
  then $\tctx_1, \ctxl, \tctx_2 \byinf \sigma_1 \infto \sigma_2$.
}

\newcommand*{\TypingSubstitutionName}{Typing Substitution}
\newcommand*{\TypingSubstitutionBody}{
  If $\tctx \byinf \tau \infto \applye \tctx {\sigma_1}$,
  and $\tctx, x : \sigma_1 \byinf \tau' \infto \sigma_2 $,
  then $\tctx \byinf \tau' \subst x \tau \infto \sigma_2 $.
}

\newcommand*{\TypingContextWellFormednessName}{Typing Context Well Formedness}
\newcommand*{\TypingContextWellFormednessBody}{
  If $\tctx \byinf \tau \infto \sigma$,
  then $\tctx \wc$.
}

\newcommand*{\TypingStrengtheningName}{Typing Strengthening}
\newcommand*{\TypingStrengtheningBody}{
  If $\tctx_1, \tctx_2, \tctx_3 \byinf \tau \infto \sigma$,
  and $\tctx_1, \tctx_3 \wc $,
  and $\tctx_1, \tctx_3 \bywt \tau $,
  then $\tctx_1, \tctx_3 \byinf \tau \infto \sigma$.
}

\newcommand*{\TypingVariableExchangeName}{Typing Variable Exchange}
\newcommand*{\TypingVariableExchangeBody}{
  If $\tctx, x: \sigma_1, y :\sigma_2, \ctxl \byinf \tau_1 \infto \tau_2$,
  and $\tctx, y: \sigma_2, x :\sigma_1, \ctxl \wc$,
  then $\tctx, y: \sigma_2, x :\sigma_1, \ctxl \byinf \tau_1 \infto \tau_2$
  with the same size of typing derivation.
}

\newcommand*{\DeclarationPreservationName}{Declaration Preservation}
\newcommand*{\DeclarationPreservationBody}{
  If $\tctx \exto \ctxr$,
  and $u$ is a variable declared in $\tctx$,
  then $u$ is declared in $\ctxr$.
}

\newcommand*{\DeclarationOrderPreservationName}{Declaration Order Preservation}
\newcommand*{\DeclarationOrderPreservationBody}{
  If $\tctx \exto \ctxr$,
  and $u$ is declared to the left of $v$ in $\tctx$,
  then $u$ is declared to the left of $v$ in the $\ctxr$.
}

\newcommand*{\ReverseDeclarationOrderPreservationName}{Reverse Declaration Order Preservation}
\newcommand*{\ReverseDeclarationOrderPreservationBody}{
  If $\tctx \exto \ctxr$,
  and $u$ and $v$ are both declared in $\tctx$,
  and $u$ is declared to the left of $v$ in $\ctxr$,
  then $u$ is declared to the left of $v$ in the $\tctx$.
}

\newcommand*{\SubstitutionExtensionInvarianceName}{Substitution Extension Invariance}
\newcommand*{\SubstitutionExtensionInvarianceBody}{
  If $\tctx \byinf \sigma \infto \sigma'$,
  and $\tctx \exto \ctxr$,
  then $\applye \ctxr \sigma = \applye \ctxr {\applye \tctx \sigma}$,
  and $\applye \ctxr \sigma = \applye \tctx {\applye \ctxr \sigma}$.
}

\newcommand*{\ExtensionEqualityPreservationName}{Extension Equality Preservation}
\newcommand*{\ExtensionEqualityPreservationBody}{
  If $\tctx \byinf \sigma_1 \infto \tau_1$,
  and $\tctx \byinf \sigma_2 \infto \tau_2$,
  and $\applye \tctx {\sigma_1} = \applye \tctx {\sigma_2}$,
  and $\tctx \exto \ctxr$,
  then $\applye \ctxr {\sigma_1} = \applye \ctxr {\sigma_2}$.
}

\newcommand*{\ContextExtensionReflexivityName}{Reflexivity of Context Extension}
\newcommand*{\ContextExtensionReflexivityBody}{
  If $\tctx \wc$,
  then $\tctx \exto \tctx$.
}

\newcommand*{\ContextExtensionTransitivityName}{Transitivity of Context Extension}
\newcommand*{\ContextExtensionTransitivityBody}{
  If $\ctxl \exto \tctx$,
  and $\tctx \exto \ctxr$,
  then $\ctxl \exto \ctxr$.
}

\newcommand*{\ContextExtensionPreservesContextWellFormednessName}{Context Extension Preserves Context Well Formedness}
\newcommand*{\ContextExtensionPreservesContextWellFormednessBody}{
  If $\tctx \wc$,
  and $\tctx \exto \ctxr$,
  then $\ctxr \wc$.
}

\newcommand*{\ReverseDeclarationPreservationName}{Reverse Declaration Preservation}
\newcommand*{\ReverseDeclarationPreservationBody}{
  If $\tctx \exto \ctxl$,
  and $u$ is a variable that $u \notin \ctxl$,
  then $u \notin \tctx$.
}

\newcommand*{\RightSoftnessName}{Right Softness}
\newcommand*{\RightSoftnessBody}{
  If $\tctx \exto \ctxr$,
  and $\ctxl$ is soft,
  and $\ctxr, \ctxl$ is well formed,
  then $\tctx \exto \ctxr, \ctxl$.
}

\newcommand*{\ExtensionOrderName}{Extension Order}
\newcommand*{\ExtensionOrderBody}{
  \begin{itemize}
  \item If $\tctx_L, y : \sigma, \tctx_R \exto \ctxr$,
    then $\ctxr = \ctxr_L, y : \sigma, \ctxr_R$,
    and $\tctx_L \exto \ctxr_L$,
    and $\ctxr_R$ is soft if and only if $\tctx_R$ is soft.
  \item If $\tctx_L, \genA , \tctx_R \exto \ctxr$,
    then $\ctxr = \ctxr_L, \ctxl, \ctxr_R$,
    and $\tctx_L \exto \ctxr_L$,
    and $\ctxl$ is either $\genA$ or $\genA = \tau$ for some $\tau$,
    and $\ctxr_R$ is soft if and only if $\tctx_R$ is soft.
  \item If $\tctx_L, \genA = \sigma, \tctx_R \exto \ctxr$,
    then $\ctxr = \ctxr_L, \genA = \sigma, \ctxr_R$,
    and $\tctx_L \exto \ctxr_L$,
    and $\ctxr_R$ is soft if and only if $\tctx_R$ is soft.
  \end{itemize}
}

\newcommand*{\ExtensionWeakningName}{Extension Weakening}
\newcommand*{\ExtensionWeakningBody}{
  If $\tctx \byinf \tau_1 \infto \tau_2$,
  and $\tctx \exto \ctxr$,
  and $\ctxr \wc$,
  then $\ctxr \byinf \tau_1 \infto \applye \ctxr {\tau_2}$.
}

\newcommand*{\ExtensionWeakningWellFormednessName}{Extension Weakening Well Formedness}
\newcommand*{\ExtensionWeakningWellFormednessBody}{
  If $\tctx \bywf \tau$,
  and $\tctx \exto \ctxr$,
  and $\ctxr \wc$,
  then $\ctxr \bywf \tau$.
}

\newcommand*{\ExtensionWeakeningWellScopednessName}
{Extension Weakening Well Scopedness}
\newcommand*{\ExtensionWeakeningWellScopednessBody}{
  If $\tctx \bywt \tau$,
  and $\tctx \exto \ctxr$,
  then $\ctxr \bywt \tau$.
}


\newcommand*{\SolutionAdmissibilityForExtensionName}{Solution Admissibility for Extension}
\newcommand*{\SolutionAdmissibilityForExtensionBody}{
  If $\tctx_L, \genA, \tctx_R \wc$,
  and $\tctx_L \bywf \tau$,
  then $\tctx_L, \genA, \tctx_R \exto \tctx_L, \genA = \tau, \tctx_R$.
}

\newcommand*{\UnsolvedVariableAdditionForExtensionName}
{Unsolved Variable Addition for Extension}
\newcommand*{\UnsolvedVariableAdditionForExtensionBody}{
  If $\tctx_L, \tctx_R \wc$,
  and $\genA \notin \tctx_L, \tctx_R$,
  then $\tctx_L, \tctx_R \exto \tctx_L, \genA, \tctx_R$.
}

\newcommand*{\SolvedVariableAdditionForExtensionName}
{Solved Variable Addition for Extension}
\newcommand*{\SolvedVariableAdditionForExtensionBody}{
  If $\tctx_L, \tctx_R \wc$,
  and $\genA \notin \tctx_L, \tctx_R$,
  and $\tctx_L \bywf \tau$,
  then $\tctx_L, \tctx_R \exto \tctx_L, \genA = \tau, \tctx_R$.
}

\newcommand*{\ParallelAdmissibilityName}
{Parallel Admissibility}
\newcommand*{\ParallelAdmissibilityBody}{
  If $\tctx_L \exto \ctxr_L $,
  and $\tctx_L, \tctx_R \exto \ctxr_L, \ctxr_R $,
  and $\tctx_R$ has no variable overlapped with $\ctxr_L$,
  and $\genA \notin \tctx_L, \tctx_R, \ctxr_L, \ctxr_R $,
  then:
  \begin{itemize}
  \item $\tctx_L, \genA, \tctx_R \exto \ctxr_L, \genA, \ctxr_R $.
  \item If $\ctxr_L \bywf \tau$,
    then $\tctx_L, \genA, \tctx_R \exto \ctxr_L, \genA = \tau, \ctxr_R $.
  \item If $\tctx_L \bywf \tau_1$,
    and $\ctxr_L \bywf \tau_2$,
    and $\applye {\ctxr_L} {\tau_1} = \applye {\ctxr_L} {\tau_2}$,
    then $\tctx_L, \genA = \tau_1, \tctx_R \exto \ctxr_L, \genA = \tau_2, \ctxr_R $.
  \item If $\tctx_L \bywf \tau$,
    then $\tctx_L, x : \tau, \tctx_R \exto \ctxr_L, x: \tau, \ctxr_R $.
  \end{itemize}
}

\newcommand*{\ParallelExtensionSolutionName}
{Parallel Extension Solution}
\newcommand*{\ParallelExtensionSolutionBody}{
  If $\tctx_L, \genA, \tctx_R \exto \ctxr_L, \genA = \tau_2, \ctxr_R $,
  and $\tctx_L \bywf \tau_1$,
  and $\applye {\ctxr_L} {\tau_1} = \applye {\ctxr_L} {\tau_2}$,
  then $\tctx_L, \genA = \tau_1, \tctx_R \exto \ctxr_L, \genA = \tau_2, \ctxr_R $.
}

\newcommand*{\CompleteContextApplicationExtensionName}
{Complete Context Application Extension}
\newcommand*{\CompleteContextApplicationExtensionBody}{
  If $\tctx \exto \cctx$,
  then $\applye \cctx \tctx \exto \cctx$.
}

% ------------------------------------------------------
% COMPLETE CONTEXTS
% ------------------------------------------------------

\newcommand*{\StabilityOfCompleteContextsName}
{Stability of Complete Contexts}
\newcommand*{\StabilityOfCompleteContextsBody}{
  If $\tctx \exto \cctx$,
  then $\applye \cctx \cctx = \applye \cctx \tctx$.
}

\newcommand*{\FinishingTypesName}
{Finishing Types}
\newcommand*{\FinishingTypesBody}{
  If $\cctx \bywt \sigma$,
  and $\cctx \exto \cctx'$,
  then $\applye \cctx \sigma = \applye {\cctx'} \sigma$.
}

\newcommand*{\FinishingCompletionsName}
{Finishing Completions}
\newcommand*{\FinishingCompletionsBody}{
  If $\cctx \exto \cctx'$,
  then $\applye \cctx \cctx = \applye {\cctx'} {\cctx'}$.
}

\newcommand*{\ConfluenceOfCompletenessName}
{Confluence of Completeness}
\newcommand*{\ConfluenceOfCompletenessBody}{
  If $\ctxr_1 \exto \cctx$,
  and $\ctxr_2 \exto \cctx$,
  then $\applye \cctx {\ctxr_1} = \applye \cctx {\ctxr_2}$.
}

% ------------------------------------------------------
% TYPE SANITIZATION
% ------------------------------------------------------

\newcommand*{\TypeSanitizationExtensionName}
{Type Sanitization Extension}
\newcommand*{\TypeSanitizationExtensionBody}{
  If $\tctx \bysa \tau_1 \sa \tau_2 \toctx$,
  and $\tctx \byinf \tau_1 \infto \sigma$,
  then $\tctx \exto \ctxl$.
}

\newcommand*{\TypeSanitizationEquivalenceName}
{Type Sanitization Equivalence}
\newcommand*{\TypeSanitizationEquivalenceBody}{
  If $\tctx \bysa \tau_1 \sa \tau_2 \toctx$,
  and $\tctx \byinf \tau_1 \infto \sigma$,
  then $\applye \ctxl {\tau_1} = \applye \ctxl {\tau_2}$.
}

\newcommand*{\TypeSanitizationWellFormednessName}
{Type Sanitization Well Formedness}
\newcommand*{\TypeSanitizationWellFormednessBody}{
  If $\tctx \bysa \tau_1 \sa \tau_2 \toctx$,
  and $\tctx \byinf \tau_1 \infto \sigma$,
  then $\ctxl \byinf \tau_2 \infto \applye \ctxl \sigma$.
}

\newcommand*{\TypeSanitizationTailUnchangedName}
{Type Sanitization Tail Unchanged}
\newcommand*{\TypeSanitizationTailUnchangedBody}{
  If $v$ is a binding,
  and $\tctx, v \bysa \tau_1 \sa \tau_2 \toctx$,
  then $\ctxl = \ctxl', v$.
}

\newcommand*{\TypeSanitizationCompletenessName}
{Type Sanitization Completeness}
\newcommand*{\TypeSanitizationCompletenessBody}{
  Given $\tctx \exto \cctx$,
  where $\tctx = \tctx_1, \genA, \tctx_2, \tctx_0$,
  and $\tctx_0$ only contains variables,
  and $\applye \tctx {\sigma_1} = \sigma_1$,
  and $\applye \cctx \tau = \tau$,
  and $\genA \notin FV(\sigma_1)$,
  and $\tctx_1, \tctx_0 \bywt \tau $.
  If $\applye \cctx \tctx \bywf \tau = \applye \cctx {\sigma_1} $,
  then there are $\ctxr, \cctx'$
  such that $\cctx \exto \cctx'$,
  and $\ctxr \exto \cctx'$,
  and $\tctx[\genA] \bysa \sigma_1 \sa \sigma_2 \toctxr$,
  where $\ctxr = \ctxr_1, \genA ,\ctxr_2, \tctx_0$,
  and $\ctxr_1, \tctx_0 \bywt \sigma_2$.
}

\newcommand*{\TypeSanitizationCompletenessPrettyName}
{Type Sanitization Completeness Pretty}
\newcommand*{\TypeSanitizationCompletenessPrettyBody}{
  Given $\tctx \exto \cctx$,
  where $\tctx = \tctx_1, \genA, \tctx_2$,
  and $\applye \tctx {\sigma_1} = \sigma_1$,
  and $\applye \cctx \tau = \tau$,
  and $\genA \notin FV(\sigma_1)$,
  and $\tctx_1 \bywt \tau $.
  If $\applye \cctx \tctx \bywf \tau = \applye \cctx {\sigma_1} $,
  then there are $\ctxr, \cctx'$
  such that $\cctx \exto \cctx'$,
  and $\ctxr \exto \cctx'$,
  and $\tctx[\genA] \bysa \sigma_1 \sa \sigma_2 \toctxr$,
  where $\ctxr = \ctxr_1, \genA ,\ctxr_2$,
  and $\ctxr_1 \bywt \sigma_2$.
}

\newcommand*{\TypeSanitizationCompletenessUnificationName}
{Type Sanitization Completeness w.r.t. Unification}
\newcommand*{\TypeSanitizationCompletenessUnificationBody}{
  Given $\tctx \exto \cctx$,
  where $\tctx = \tctx_1, \genA, \tctx_2$,
  and $\applye \tctx {\sigma_1} = \sigma_1$,
  and $\genA \notin FV(\sigma_1)$.
  \begin{itemize}
  \item If $\applye \cctx \tctx \bywf \applye \cctx \genA
    = \applye \cctx {\sigma_1} $,
    then there are $\ctxr, \cctx'$
    such that $\cctx \exto \cctx'$,
    and $\ctxr \exto \cctx'$,
    and $\tctx[\genA] \byuni \genA \uni \sigma_1 \toctxr$.
  \item If $\applye \cctx \tctx \bywf \applye \cctx {\sigma_1}
    = \applye \cctx {\genA} $,
    then there are $\ctxr, \cctx'$
    such that $\cctx \exto \cctx'$,
    and $\ctxr \exto \cctx'$,
    and $\tctx[\genA] \byuni \sigma_1 \uni \genA \toctxr$.
  \end{itemize}
}

% ------------------------------------------------------
% UNIFICATION
% ------------------------------------------------------

\newcommand*{\UnificationExtensionName}
{Unification Extension}
\newcommand*{\UnificationExtensionBody}{
  \begin{itemize}
  \item If $\tctx \byeuni e_1 \uni e_2 \toctx$,
    and $\tctx \byinf e_1 \infto \sigma_1'$,
    and $\tctx \byinf e_2 \infto \sigma_2'$,
    then $\tctx \exto \ctxl$.
  \item If $\tctx \bysuni \tau_1 \uni \tau_2 \toctx$,
    and $\tctx \byinf \tau_1 \infto \star $,
    and $\tctx \byinf \tau_2 \infto \star $,
    then $\tctx \exto \ctxl$.
  \end{itemize}
}


\newcommand*{\UnificationEquivalenceName}
{Unification Equivalence}
\newcommand*{\UnificationEquivalenceBody}{
  If $\tctx \bybuni \tau_1 \uni \tau_2 \toctx$,
  and $\tctx \byinf \tau_1 \infto \sigma_1'$,
  and $\tctx \byinf \tau_2 \infto \sigma_2'$,
  then $\applye \ctxl {\tau_1} = \applye \ctxl {\tau_2}$.
}

\newcommand*{\UnificationCompletenessName}
{Unification Completeness}
\newcommand*{\UnificationCompletenessBody}{
  Given $\tctx \exto \cctx$,
  and $\applye \tctx {\sigma_1} = \sigma_1$,
  and $\applye \tctx {\sigma_2} = \sigma_2$.
  If $\applye \cctx \tctx \byinf
  \applye \cctx {\sigma_1} = \applye \cctx {\sigma_2} \infto \tau $,
  then there are $\ctxr, \cctx'$ such that
  $\cctx \exto \cctx'$,
  and $\ctxr \exto \cctx'$,
  and $\tctx \bybuni \sigma_1 \uni \sigma_2 \toctxr$
  for any $\delta$ suitable.
}

% ------------------------------------------------------
% POLYMORPHIC TYPE SANITIZATION
% ------------------------------------------------------

\newcommand*{\PolymorphicTypeSanitizationExtensionName}
{Polymorphic Type Sanitization Extension}
\newcommand*{\PolymorphicTypeSanitizationExtensionBody}{
  If $\tctx \bybsa A \sa \sigma \toctx$,
  then $\tctx \exto \ctxl$.
}

\newcommand*{\PolymorphicTypeSanitizationSoundnessName}
{Polymorphic Type Sanitization Soundness}
\newcommand*{\PolymorphicTypeSanitizationSoundnessBody}{
  Given $\ctxl \exto \cctx$,
  and $\applye \tctx A = A$:
  \begin{itemize}
  \item
    If $\tctx[\genA] \bypsa A \sa \sigma \toctx$,
    then $\applye \cctx \ctxl \bysub \applye \cctx A \dsub \applye \cctx \sigma$.
  \item
    If $\tctx[\genA] \bymsa A \sa \sigma \toctx$,
    then $\applye \cctx \ctxl \bysub \applye \cctx \sigma \dsub \applye \cctx A$.
  \end{itemize}
}

\newcommand*{\PolymorphicTypeSanitizationCompletenessName}
{Polymorphic Type Sanitization Completeness}
\newcommand*{\PolymorphicTypeSanitizationCompletenessBody}{
  Given $\tctx \exto \cctx $,
  where $\tctx = \tctx_1, \genA, \tctx_2$,
  and $\applye \tctx A = A $,
  and $\applye \cctx {\tau} = \tau $,
  and $\genA \notin FV(A)$,
  and ${\tctx_1} \bywf \tau $
  :
  \begin{itemize}
  \item
    If $\applye \cctx \tctx \bysub \tau \dsub \applye \cctx A$,
    then there are $\ctxr, \cctx'$
    such that $\cctx \exto \cctx'$,
    and $\ctxr \exto \cctx'$,
    and $\tctx[\genA] \bymsa A \sa \sigma \toctxr$,
    where $\ctxr = \ctxr_1, \genA, \ctxr_2$ ,
    and $\ctxr_1 \bywf \sigma$,
    and $\applye {\cctx'} \sigma = \tau$.
  \item
    If $\applye \cctx \tctx \bysub \applye \cctx A \dsub \tau$,
    then there are $\ctxr, \cctx'$
    such that $\cctx \exto \cctx'$,
    and $\ctxr \exto \cctx'$,
    and $\tctx[\genA] \bypsa A \sa \sigma \toctxr$,
    where $\ctxr = \ctxr_1, \genA, \ctxr_2$ ,
    and $\ctxr_1 \bywf \sigma$,
    and $\applye {\cctx'} \sigma = \tau$.
  \end{itemize}
}

\newcommand*{\PolymorphicTypeSanitizationCompletenessSubtypingName}
{Polymorphic Type Sanitization Completeness w.r.t Subtyping}
\newcommand*{\PolymorphicTypeSanitizationCompletenessSubtypingBody}{
  Given $\tctx \exto \cctx $,
  and $\applye \tctx A = A $,
  and $\genA \notin FV(A)$,
  and $\genA \in unsolved(\tctx)$
  :
  \begin{itemize}
  \item
    If $\applye \cctx \tctx \bysub \applye \cctx \genA \dsub \applye \cctx A$,
    then there are $\ctxr, \cctx'$
    such that $\cctx \exto \cctx'$,
    and $\ctxr \exto \cctx'$,
    and $\tctx[\genA] \bysub \genA \tsub A \toctxr$.
  \item
    If $\applye \cctx \tctx \bysub \applye \cctx A \dsub \applye \cctx \genA$,
    then there are $\ctxr, \cctx'$
    such that $\cctx \exto \cctx'$,
    and $\ctxr \exto \cctx'$,
    and $\tctx[\genA] \bysub A \tsub \genA \toctxr$.
  \end{itemize}
}


\section{Introduction}

Dependent types are currently increasingly adopted in many language
designs due to its expressiveness \cite{xi1999dependent, licata2005formulation,
  pasalic2006concoqtion, mckinna2006dependent,
  norell2009dependently, brady2013idris}.
However, type inference or unification on those language is not easy.
This is because
more power a type system has, more sophisticated the type system
becomes. The dependency between expressions and types bring lots of complexities.

Existing literature \cite{ziliani2015unification, abel2011higher, elliott1989higher}
that tries to give specification for type inference
or unification of
a dependent language is quite complicated,
and even becomes non-intuitive or unpredictable once it involves
so many constructs or features.

In this paper, we presents an easy strategy to do unification based on
alpha-equality for first-order dependent
types. This algorithm is based on alpha-equality for following reasons. Firstly,
for unification algorithms based on beta-equality, if the system has strong
normalization, the algorithm usually reduces all types into normal forms and
then compares the normal forms using
alpha-equality.
Secondly, there are proposals for type system without strong
normalization, for example, doing type-level computations using casts
\cite{van2013explicit, kimmell2012equational, sjoberg2012irrelevance}
. For
those systems, the unification is naturally based on alpha-equality.

Our notations to do formalization are inspired by \cite{dunfield2013complete}.
We come up with a new
process called \textit{type sanitization} that helps resolve the dependency
problem. Later on, the type sanitization process is
extended to deal with restricted polymorphic types.
Based on type sanitization, our algorithm
are remarkably simple and well-behaved.
Though there are no formal proofs for the meta-theory of the system yet
(which is still in progress),
we give many conjectures that we believe are intuitive.

We expect our algorithm serves as a footstone towards
a simple and predictable unification/subtyping algorithm for
dependent types. This is also for
filling the gap between
delicate unification algorithms for simple types
and
sophisticated unification algorithms for dependent types.
A non-goal of our work
is to replace existing matured unification/subtyping algorithm.
More precisely, our main contributions are:

\begin{itemize}
  \item We come up with a strategy called \textit{type sanitization}
    that resolves the
    dependency between types.
  \item Based on type sanitization, we give a specification of an alpha-equality
    based unification algorithm
    for first-order dependent types.
  \item We show how to extend type sanitization for a specification of a
    subtyping algorithm
    in a language including restricted polymorphic types.
\end{itemize}

In Section \ref{sec:language}, we present a overview of a first-order dependently
typed language.
In Section \ref{sec:unification}, we formalize the unification
problem and present the type sanitization and unification. In
Section \ref{sec:extension}, we extend the language with
polymorphic types, and then present the process of extended type sanitization
along with subtyping rules. Finally Section \ref{sec:conclusion} concludes the
paper.
\section{Overview}
\label{sec:overview}

This section provides background and gives an overview of our work.

\subsection{Background and Motivation}

In this section, we discuss the background of type inference in context, also a
variant of this approach in a higher rank polymorphic type system. While
presenting the key ideas of the work, we also talk about the challenges of each
approach, which motivates our work.

\subsubsection{Type Inference In Context.}

\citet{gundry2010type} models unification and type inference from a general
perspective of information increase in the problem context in a ML-style
polymorphic system, based on the invariant that types can only depend on
bindings appearing earlier in the context.

Specifically, information and constraints about variables are stored in the
context, which is a list maintaining information order. For example\footnote{We
  adopt our notations and terminologies for the examples}:

$\tctx_1 = \genA, \genB, x :\genB$

Here $\genA, \genB$ are existential variables waiting to be solved, whose
meaning is given by solutions:

$\tctx_2 = \genA, \genB = \Int, x :\genB$

Then unification problem becomes finding a more informative context that
contains solutions for the existential variables so that two expressions are
equivalent up to substitution of the solutions. For
example, $\tctx_2$ can be the solution context for unifying
$\Int$ and $\genB$ under $\tctx_1$.

Besides contexts being ordered, a key insight of the approach lies in how to
unify existential variables with other types. In this case, unification needs to
resolve the dependency between existential variables. Consider unifying $\genA$
with $\genB \to \genB$ under context $\genA, \genB, x : \genB$. Here $\genB$ is
out of the scope of $\genA$. The way they solve it is to examine variables in
the context from the tail to the head, \textit{moves segments of context to the
  left if necessary}, until the existential variable being unified is found.
This design is implemented by an additional context that records the context
needed to be moved. This can be interpreted from the judgment $\tctx \ctxsplit
\Xi \byuni \genA \equiv \tau \toctx$, which is read as: given input context
$\tctx$, $\Xi$, solve $\genA$ with $\tau$ succeeds and produces an output
context $\ctxl$.

The essential rules involving in this process is given in
Figure~\ref{fig:inference-context}. If a variable is useless for the
unification, it is ignored (\rul{Ignore}). Otherwise, if the variable is needed,
but it is out of the scope of $\genA$, then it is moved to the additional
context (\rul{Depend}). Finally we arrive at the variable we want to unify,
which is $\genA$, we then insert $\Xi$ before $\genA$, and solve $\genA = \tau$
(\rul{Define}). Therefore, the output context for the above unification problem
is $\genB, \genA = \genB \to \genB, x: \genB$. Note that $\genB$ is now placed
in the front of $\genA$.

\begin{figure*}[t]
  \begin{mathpar}
    \Ignore \and \Depend \and \Define
  \end{mathpar}
  \caption{Unification between an existential variable and a type (incomplete).}
  \label{fig:inference-context}
\end{figure*}

\paragraph{Challenges.}

While moving type variables around is a feasible way to resolve the dependency
between existential variables, the unpredictable context movements make
the information increase hard to formalize and reason about. In this sense, the
information increase of contexts is defined in a much \textit{semantic} way:
$\ctxl$ is more informative than $\tctx$, if there exists a substitution $S$
that for every $v \in \tctx$, we have $\ctxl \bywf S(v)$.

This semantic definition makes it hard to prove meta-theory formally, especially
when advanced features are involved. For example, in dependent type systems,
typing and types/contexts well-formedness are usually coupled, which brings even
more complication to the proofs.

\subsubsection{Instantiation in Higher Rank Type System.}

\begin{figure*}[t]
  \begin{mathpar}
    % \framebox{$\tctx \bybuni \sigma_1 \uni \sigma_2 \toctx$} \\
    \InstLSolve \and \InstLReach \and \InstLArr
    \and \InstRReach
  \end{mathpar}
  \caption{Instantiation between an existential variable and a type (incomplete).}
  \label{fig:instantiation}
\end{figure*}

\citet{dunfield2013complete} also use ordered contexts as input and output for
type inference for a higher rank polymorphic type system. However, they do it in
a more \textit{syntactic} way.

Instead of moving variables to the left in the context, in their subtyping
between an existential variable and a type, which they call instantiation, they
choose to decompose type constructs so that unification between existential
variables can only happen between single variables,
 then there are only two cases to discuss: whether the left
existential variable appears first, or the right one appears first.
Specifically, Figure~\ref{fig:instantiation} shows the key idea of instantiation
rules between an existential variable and a type. For space reason, we only
present the rules when the left hand side is an existential variables; but the
other case is quite symmetric. We save the formal explanations for notations
later to Section~\ref{sec:dependent}. But we can still gain the a rough idea
there. Notice in \rul{InstLArr}, the existential variable $\genA$ is solved by a
function type consisting of two fresh existential variables, and then the
function is decomposed to do instantiation successively. Rule \rul{InstLReach}
deals with the case that $\genA$ appears first, and \rul{InstRReach} deals with the
other case.

In this way, the information increase of contexts, which is named context
extension, is formalized in an intuitive and straightforward \textit{syntactic}
way, which enables them to prove the meta-theory thoroughly and formally.
The complete definition of context extension can be found in appendix. Our
definition presented in Section~\ref{sec:context-extension} also mimics their
definition.

\paragraph{Challenges.} While destructing type constructors makes perfect sense
in their setting, it cannot deal with dependent types correctly. For example,
given the context $\genA, \genB$ and we want to unify $\genA$ with a dependent
type $\bpi x \genB x$. Here because $\genB$ appears after $\genA$, we cannot
directly derive $\genA = \bpi x \genB x$ which is ill typed. However, if we try
to decompose this Pi type, then according to rule \rul{InstLArr}, it is obvious
that $\genA_2$ should be solved by $x$. In order to make the solution well
typed, we need to put $x$ before $\genA_2$ into the context. However, this means
$x$ will remain in the context, and it is available for any later existential
variable that should not have access to $x$.

Another drawback of decomposition is that it produces duplication. The rules in
Figure~\ref{fig:instantiation} is repeated for the cases when the existential
variable is on the right. For example, there will be a symmetric \rul{InstRArr}
corresponding to \rul{InstLArr}. Worse, This kind of ``duplication'' would
scales up with the type constructs in the system.

\subsection{Type Sanitization}
\label{subsec:sanitization}

Type sanitization provides another way to resolve dependency between existential
variables. It combines the advantages of previous approaches. First, it is a
simple and quite predictable process, so that information increase can still be
modeled as \textit{syntactic} context extension. Also, it will not decompose
types, nor cause any duplication.

To understand how type sanitization works, we revisit the unification
problem: given context

$\genA, \genB, x: \genB$

\noindent we want to unify $\genA$ with $\genB \to \genB$. The problem here is
that $\genB$ is out of the scope of $\genA$. Therefore, we first ``sanitize''
the type $\genB \to \genB$. The process of type sanitization will sanitize the
existential variables in right hand side that are out of the scope of $\genA$ by
solving them with fresh existential variables that is put in front of $\genA$.
Specifically, we will solve
$\genB$ with a fresh variable $\genA_1$, which results in an output context

$\genA_1, \genA, \genB = \genA_1, x: \genB$

Notice that $\genA_1$ is inserted right before $\genA$. Now the unification
problem becomes unifying $\genA$ with $\genA_1 \to \genA_1$, and $\genA_1 \to
\genA_1$ is a valid solution for $\genA$. Therefore, we get a final solution
context:

$\genA_1, \genA = \genA_1 \to \genA_1, \genB = \genA_1, x : \genB$.


\paragraph{Interpretation of Type Sanitization.}
Moving existential variables around in the approach of type inference in context
\citep{gundry2010type}, the symmetric rules \rul{InstLReach} and
\rul{InstRReach} \citep{dunfield2013complete}, and the approach of type
sanitization all enjoy the same philosophy: \textit{the relative order between
  existential variables does not matter}.

This seems like kind of going against the design principle that the
contexts are ordered lists. However, keeping order is still important for
variables \textit{whose order matter}. For instance, for polymorphic types, the
order between existential variables ($\genA$) and type variables ($\varA$) is
important, so you cannot unify $\genA$ with $\varA$ under the context $(\genA,
\varA)$ since $\varA$ is not in the scope of $\genA$. A similar reason exists in
dependent type systems: you cannot unify $\genA$ with $x$ if $x$ appears behind
$\genA$ in the context.

The task of type sanitization is to ``move'' existential variables to suitable
places through a roundabout way: solving it with a fresh existential variable
and make sure this new fresh variable is in the suitable place.

\subsection{Unification for Dependent Types}

As a first illustration of the utility of the type sanitization, we present a
unification algorithm for dependent types with alpha-equality based
first-order constrains.

\paragraph{Explicit Casts.}

Type systems with explicit controls on type-level computation has been adopted
in several dependent type calculi \cite{???} since it allows type system to have
both general recursion and decidable type checking at the same time. In order to
have type-level computations, explicit casts are forced, which is implemented by
two language constructs: $\castdn e$ that does one-step beta reduction on the
type of $e$, and $\castup e$ that does one-step beta expansion on the type of
$e$. For example, If given

$e: (\blam x \Int \star) ~ 3$

\noindent Then we have

$\castdn e : \star$

$\castup (\castdn e) : (\blam x \Int \star) ~ 3$

In these systems, type comparison is naturally based on alpha-equality. This
simplifies the unification algorithm in the sense that unification can be mostly
structural. However, we still need to deal with the dependency introduced by
dependent types carefully, which is mainly reflected in the unification problems
between existential variables.

\paragraph{Type Sanitization In Dependent Type System.}
Type sanitization is applicable to dependent type system. Consider the previous
example that we unify $\genA$ with $\bpi x \genB x$. By a similar process
described in Section~\ref{subsec:sanitization}, we can sanitize the type to be
$\bpi x {\genA_1} x$ without destructing the Pi type, and solve $\genA$ with
$\bpi x {\genA_1} x$.

% The key rules in the type sanitization are:

% \begin{mathpar}
%   \IEVarAfter \and \IEVarBefore
% \end{mathpar}

% In \rul{I-EVarAfter}, we encounter an existential variable $\genB$ that is out
% of the scope of the existential variable $\genA$, so we sanitize it by a fresh
% variable $\genA_1$. In \rul{I-EVarBefore}, the existential variable $\genB$ is
% contained in the scope of $\genA$, therefore it remains unchanged.

Based on type sanitization, we come up with a unification algorithm, which is
later proved sound and complete.

\subsection{Polymorphic Type Sanitization For Subtyping}

Instead of unification, the instantiation relation in
\citet{dunfield2013complete} is actually aiming at dealing with polymorphic
subtyping relation between existential variables and other types. Here we say
type $A$ is a subtype of $B$ if and only if $A$ is more polymorphic than $B$.
The difficulty of subtyping that it needs to take unification
into account at the same time. For example, given $\genA$ is a subtype of
$\Int$, then the only possible solution is $\genA = \Int$. And if given $\forall
a. a \to a$ is a subtype $\genA$, then a feasible solution can be $\genA = \Int
\to \Int$.

The type sanitization we described above only works for unification. If given
the context

$\genA$

\noindent and that $\forall \varA. \varA \to \varA$ is a subtype of $\genA$, how
can we sanitize the polymorphic type into a valid solution for an existential
variable while solutions can only be monotype? One observation is that, the most
general solution for this subtyping problem is $\genA = \genB \to \genB$ for
fresh $\genB$. Namely, we remove the universal quantifier and replace the
variable $a$ with a fresh existential variable that should be in the scope of
$\genA$, which results in the solution context:

$\genB, \genA = \genB \to \genB$

From this observation, we extend type sanitization to polymorphic type
sanitization, which is able to resolve the polymorphic subtyping relation for
existential variables. Since function types are contra-variant in argument
types, and co-variant in return types, for example

$(\Int \to \Int) \to \Int \tsub (\forall \varA. \varA \to \varA) \to \Int $

$\forall \varA. \varA \to \varA \tsub \Int \to \Int$

\noindent according to the position the universal
quantifier appears, polymorphic type sanitization has two modes, which we call
contra-variant mode and co-variant mode respectively.
In contra-variant mode, the universal quantifier is replaced by a fresh
existential variable, while in co-variant mode, it is put in the context as a
common type variable.

We show that the original instantiation relationship in
\citet{dunfield2013complete} can be replaced by our polymorphic type
sanitization process, while the subtyping remains complete and sound.

%%% Local Variables:
%%% mode: latex
%%% TeX-master: "../main"
%%% org-ref-default-bibliography: "citation.bib"
%%% End:

\section{Unification for Dependent Type System}
\label{sec:dependent}

\subsection{Language Overview}
\label{subsec:language}

The syntax of the system is shown below:\\

\begin{tabular}{lrcl}
  Expressions & $e$ & \syndef & $x \mid \star
                         \mid e_1~e_2 \mid \blam x \sigma e
                         \mid \bpi x {\sigma_1} \sigma_2$ \\
       && \synor & $\castup e \mid \castdn e$ \\
  Types & $\tau, \sigma$ & \syndef & $e \mid \genA$ \\
  Contexts & $\tctx, \ctxl, \ctxr$ & \syndef & $\ctxinit \mid \tctx,x:\sigma
             \mid \tctx, \genA
             \mid \tctx, \genA = \tau $ \\
  Complete Contexts & $\cctx$ & \syndef & $\ctxinit \mid \cctx,x:\sigma
             \mid \cctx, \genA = \tau $ \\
\end{tabular}

\paragraph{Expressions.}
Expressions $e$ include variables x,
a single sort $\star$ to represent the type of
types,
applications $e_1~e_2$,
functions $\blam x \sigma e$,
Pi types
$\bpi x {\sigma_1} {\sigma_2}$,
$\castup e$ that does beta expansion,
and $\castdn e$ that does beta reduction.

\paragraph{Types.}
Types $\tau, \sigma$ are the same as expressions, except types contain
existential variables $\genA$.
The categories of expressions and types are stratified to make sure that
existential variables only appears in positions where types are expected.

\paragraph{Contexts.}
Contexts $\tctx$ are an ordered list of variables and
existential variables, which
can either be unsolved
($\genA$) or solved by a type $\tau$ ($\genA = \tau$).
It is important for a context to be ordered to solve the dependency between
variables, for example, if a variable $x$ is introduced after a
an existential variable $\genA$
in the context, as in
$\genA, x:\star, y :\genA$,
then $\genA$ can never by solved by $x$, namely $y$ can never have type $x$.

Complete contexts $\cctx$ only contain variables and solved existential variables.

\paragraph{Hole notation.}
We use hole notations like $\tctx[x]$ to
denote that the variable $x$ appears in the context, sometimes it is also
written as $\tctx_1, x, \tctx_2$.

Multiple holes also keep the order. For example, $\tctx[x][\genA]$ not only
require the existance of both variable $x$ and $\genA$, but also require that
$x$ appears before $\genA$.

The hole notation is also used for replacement and modification. For example,
$\tctx[\genA = \star]$ means the context keeps unchanged except $\genA$
now is solved by $\star$.

\paragraph{Applying Contexts.} Since the context records all the solutions of
solved existential variables, it can be used as a substitution. Figure
\ref{fig:context-application} defines the substitution process, where all solved
existential variables are substituted by their solutions.

\begin{figure*}[t]
  \centering
  \begin{tabular}{rll}
    $\applye {\emptyset} \sigma$ & = & $\sigma$ \\
    $\applye {\tctx, x: \tau} \sigma$ & = & $\applye \tctx \sigma$ \\
    $\applye {\tctx, \genA} \sigma$ & = & $\applye \tctx \sigma$ \\
    $\applye {\tctx, \genA = \tau} \sigma$ & = & $\applye \tctx {\sigma \subst \genA \tau}$\\
  \end{tabular}
    \caption{Context application.}
    \label{fig:context-application}
\end{figure*}

\subsection{Typing in Detail}

\begin{figure*}[t]
  \begin{mathpar}
    \framebox{$\tctx \byinf \sigma_1 \infto \sigma_2$} \\
    \AAx \and \AVar \and \AEVar \and \ASolvedEVar \and
    \ALamAnn \and \APi \and
    \AApp \and \ACastDn \and \ACastUp
  \end{mathpar}

  \begin{mathpar}
    \framebox{$\sigma_1 \redto \sigma_2$} \\
    \RApp \and \RBeta \and
    \RCastDown \and \RCastDownUp
  \end{mathpar}
    \caption{Typing and semantics.}
    \label{fig:typing}
\end{figure*}

In order to show that our unification algorithm works correctly, we need to make
sure that inputs to the algorithm are well-formed.
In dependent type system, the well-formedness of types and contexts are relying
on typing judgments.
Therefore, to introduce the well-formedness of type and contexts,
we first introduce the typing rules.

Before we give the typing rule, we need to consider: what it
means for an input type to the unification algorithm to be well-formed?
For example, Given a context $(\genA, x : \genA)$,
is type $((\blam y \Int \star) ~ x)$ well formed?
Here, the type requests to solve $\genA = \Int$.
Unfortunately, we do not regard this type well formed, because we keep the
invariant: \textit{the
constrains contained in the input type have already been solved.}
Namely, the unification process only accepts inputs that are already type
checked in current context.
In this example, this type is well-formed under context
$(\genA = \Int, x : \genA)$.
This can also help prevent ill-formed contexts that contains conflicting
constraints, for example, the context:

$\genA, x: \genA, y : ((\blam x \Int \star)~x), z : ((\blam x \Bool \star)~x)$

\noindent contains two constraints that request $\genA$ to be solved by $\Int$
and $\Bool$ respectively, which cannot be satisfied at the same time.

Given this interpretation of well-formedness, the typing rules that servers
specifically for well-formedness is shown at the top of Figure~\ref{fig:typing}.
The judgment $\tctx \byinf \sigma_1 \infto \sigma_2$ is read as: under typing
context $\tctx$, the type $\sigma_1$ has type $\sigma_2$.

Rule \rul{A-Ax} states that $\star$ always has type $\star$.
Rule \rul{A-Var} acquire the type of the variable from the typing context, and
applies the context to the type.
This reveals another invariant we keep:
\textit{the typing output is always fully substituted under current context.}

Rule \rul{A-EVar} and \rul{A-SolvedEVar} ensures that existential variables
always have type $\star$.

Rule \rul{A-LamAnn} first infers the type $\star$ for the annotation, then put $x:
\sigma_1$ into the typing context to infer the body. To make sure output type is
fully substituted, we apply the context to $\sigma_1$ in the output Pi type.

Rule \rul{A-Pi} infers the type $\star$ for the argument type $\sigma_1$, then put
$x: \sigma_1$ into the typing context to infer $\sigma_2$, whose type is also a
$\star$. And the result type for a Pi type is $\star$.

Rule \rul{A-App} first infers a function type for $e_1$, and then infers $e_2$ to
have the argument type. Again, to maintain the invariant, we apply the context
to $e_1$ before substituting $x$ with $e_1$.

Rule \rul{A-CastDn} infers the type $\sigma_1$ of $e$, that reduce the type
$\sigma_1$ to $\sigma_2$, while rule \rul{A-CastUp} finds a fully substituted
type $\applye \tctx
{\sigma_1}$ that reduces to $\sigma_2$ as the output type. The call by name
reduction is defined at the bottom of Figure~\ref{fig:typing}.
Due to the design of the rule \rul{A-CastUp}, the typing rules are
non-deterministic, which does not matter for our purpose: the typing is only
used in propositions (such as lemmas, theories) but it never appears in the
algorithm.

\paragraph{Context well-formedness.}

The first four typing rules have a common precondition $\tctx \wc$,
which requests the context is well-formed.
The judgment is defined at the top of Figure~\ref{fig:type-well}.
Rule \rul{AC-Empty} states that an empty context is always well formed.
Rule \rul{AC-Var} requires $x$ fresh, and the type annotation is typed with
$\star$. Rule \rul{AC-EVar} and \rul{AC-SolvedEVar} are defined in a similar
way.

\paragraph{Type well-formedness.}

We refer that a type $\sigma$ is well formed as $\tctx \byinf \sigma \infto \star$.
We will also sometimes write it as $\tctx \bywf \sigma$.

A weaker version of type well-formedness, which is type well-scopedness, written
as $\tctx \bywt \sigma$, is defined at the bottom of Figure~\ref{fig:type-well}.
Well-scopedness of types only requires all the variables involved in a type are
bound in the typing contexts.

\begin{figure*}[t]
  \begin{mathpar}
    \framebox{$\tctx \wc$} \\
    \ACEmpty \and \ACVar \and
    \ACEVar \and \ACSolvedEVar
  \end{mathpar}

  \begin{mathpar}
    \framebox{$\tctx \bywt \sigma$} \\
    \WSVar \and \WSEVar \and \WSSolvedEVar
    \and \WSPi \and \WSLamAnn \and \WSApp
    \and \WSCastDn \and \WSCastUp
  \end{mathpar}
    \caption{Context well-formedness and type well-scopedness.}
    \label{fig:type-well}
\end{figure*}

\begin{figure*}[t]
  \begin{mathpar}
    \framebox{$\tctx[\genA] \bysa \tau_1 \sa \tau_2 \toctx$} \\
    \IEVarAfter \and \IEVarBefore \and
    \IVar \and \IStar \and
    \IApp \and \ILamAnn \and \IPi
    \and \ICastDn \and \ICastUp
  \end{mathpar}
  \caption{Type sanitization.}
  \label{fig:sanitization}
\end{figure*}

\begin{figure*}[t]
  \begin{mathpar}
    \framebox{$\tctx \bybuni \sigma_1 \uni \sigma_2 \toctx$} \\
    \UAEq \and \UEVarTy \and \UTyEVar \and
    \UApp \and \ULamAnn \and \UPi
    \and \UCastDn \and \UCastUp
  \end{mathpar}
  \caption{Unification.}
  \label{fig:unification}
\end{figure*}


\subsection{Unification}
\label{subsec:unification}

As we mentioned before, our unification is based on alpha-equality. So in most
cases, the unification rules are intuitively structural. The most difficult one
which is also the most essential one, is how to unify an existential variable
with another type. In this section, we first present the judgment of
unification, then we discuss those cases before we present the unification
process.

\paragraph{Judgment of unification.}

Due to our design choice, there are two modes in the
unification: expressions, and types. The expression mode ($\delta = e$) does
unification between expressions, while the type mode ($\delta = \sigma$) does
unification between types.
The judgment of unification problem is formalized as:

\begin{lstlisting}
$\tctx \bybuni \sigma_1 \uni \sigma_2 \toctx$
\end{lstlisting}

The input of the unification is the current context $\tctx$, and two types
(if $\delta = \sigma$) or two expressions (if $\delta = e$).
The output of the unification
is a new context $\ctxl$ which extends the original context with probably more
new existential variables or more existing
existential variables solved.
The formal definition of context extension is discussed in
Section~\ref{sec:context-extension}.
Following is an example of a unification problem:

\begin{lstlisting}
$\genA \bysuni \genA \uni \Int \dashv \genA = \Int$
\end{lstlisting}

\noindent where we want to unify $\genA$ with $\Int$ under the input context
$\genA$, which results in the output context $\genA = \Int$ that solves $\genA$
with $\Int$.

For a valid unification problem, it must have the invariant: $[\tctx] \tau_1 =
\tau_1$, and $[\tctx] \tau_2 = \tau_2$. Namely,
\textit{the input types must be
fully applied under the input context}.
 So the following is not a valid
unification problem input:

\begin{lstlisting}
$\genA = \Bool \bysuni \genA \uni \Int$
\end{lstlisting}

We assume this invariant is given with the inputs at the beginning,
and the unification process would maintain it through the whole
formalization.

\paragraph{Process of type sanitization.}

As we discuss in Section~\ref{sec:overview}, before unifying an existential
variable with a type, we will first sanitize the type so that the existential
variables in the type that are out of the scope are solved by fresh existential
variables within the scope. We call this process \textit{type sanitization},
which is formally defined in Figure \ref{fig:sanitization}. The judgment
$\tctx[\genA] \bysa \tau_1 \sa \tau_2 \toctx$ is interpreted as: under the
context $\tctx$, which contains an existential variable $\genA$, we sanitize all
the existential variables in the type $\tau_1$ that appears before $\genA$,
which results in a sanitized type $\tau_2$. Computationally, there are three
inputs $\tctx$, $\genA$ and $\tau_1$, with one output $\tau_2$.

The most interesting cases are \rul{I-EVarAfter} and \rul{I-EVarBefore}. In
\rul{I-EVarAfter}, because $\genB$ appears after $\genA$, so we create a fresh
existential variable $\genA_1$, which is put before $\genA$, and solve $\genB$
by $\genA_1$. In \rul{I-EVarBefore}, because $\genB$ is in the scope of $\genA$,
so we leave it unchanged.

The rest rules are structural.
And as unification, we always apply intermediate output
contexts to the input types to maintain the invariant that the types are fully
substituted under current contexts.

The sanitization process is remarkably simple, while it solves exactly what we
want: resolve the order of existential variables so that we can focus on the
orders that really matter.

\paragraph{Process of unification.}

Based on type sanitization, Figure \ref{fig:unification} presents the
unification rules.

Rule \rul{U-AEq} corresponds to the case when two types are already
alpha-equivalent. Most of the rest rules are structural. Two most subtle ones
are rule \rul{U-EVarTy} and \rul{U-TyEVar}, which corresponding respectively to
when the existential variable is on the left and on the right. We go through the
first one. There are three preconditions. First is the occurs check, which is to
make sure $\genA$ does not appear in the free variables of $\tau_1$. Then we use
type sanitization to make sure all the existential variables in $\tau_1$ that
are out of scope of $\genA$ are turned into fresh ones that are in the scope of
$\genA$. This process gives us the output type $\tau_2$, and output context
$\ctxl_1, \genA, \ctxl_2$. Finally, $\tau_2$ could also contain variables whose
order matters, so we use $\ctxl_1 \bywt \tau_2$ to make sure $\tau_2$ is well
scoped. Rule \rul{U-TyEVar} is symmetric to \rul{U-EVarTy}. Using
well-scopedness instead of well-formedness gets us rid of the dependency on
typing.

\paragraph{Example.}
Below shows the process for the unification problem
$\tctx, \genA, \genB \bysuni \genA \uni \bpi x \genB x$.
For clarity, we denote $\ctxl = \tctx,\genA_1,\genA,\genB =\genA_1$. And it is
easy to verify  $\genA \notin FV(\bpi x \genB x)$.

\[
   \ExUni
\]

\subsection{Context Extension}
\label{sec:context-extension}

\begin{figure*}[t]
  \begin{mathpar}
    \framebox{$\tctx \exto \ctxl$}
     \\
    \CEEmpty \and \CEVar \and \CEEVar \and
    \CESolvedEVar \and \CESolve \and
    \CEAdd \and \CEAddSolved
  \end{mathpar}
  \caption{Context extension.}
  \label{fig:context-extension}
\end{figure*}

We mentioned that the algorithm output context extends the input context with
new existential variables or more existential variables solved. To accurately
capture this kind of information increase, we present the definition of context
extension in Figure~\ref{fig:context-extension},
as similar in \citet{dunfield2013complete}.

The empty context is an extension of itself (\rul{CE-Empty}). Existences of 
variables or existential variables are preserved during the extension
(\rul{CE-Var}, \rul{CE-EVar}), while the solutions for existential variables can
be different only if they are equivalent under context application
(\rul{CE-SolvedEVar}). The definition in \citet{{dunfield2013complete}} further
allows the type annotation to change (in \rul{CE-Var}), which is not necessary for
our algorithm. The extension can also add solutions to unsolved existential
variables (\rul{CE-Solve}), or add new existential variables (\rul{CE-Add},
\rul{CE-AddSolved}).

\paragraph{Context application on contexts.}

Complete contexts $\cctx$ are contexts with all existential variables solved, as
defined in Section~\ref{subsec:language}. Applying a complete context to a
well-formed type $\sigma$ yields a type without existential variables $\applye
\cctx \sigma$. Similarly, we can also apply a complete context to a context that
can extends to it to yield a context without existential variables. The formal
definition of the operation is defined in
Figure~\ref{fig:context-application-on-context}.

\begin{figure*}[t]
  \centering
  \begin{tabular}{rlll}
    $\applye {\emptyset} \emptyset$ & = & $\emptyset$ \\
    $\applye {\cctx, x: \tau} {\tctx, x: \tau}$ & = & $\applye \cctx \tctx, x : \applye \cctx {\tau} $
    \\
    $\applye {\cctx, \genA = \tau} {\tctx, \genA}$ & = & $\applye \cctx \tctx$ \\
    $\applye {\cctx, \genA = \tau_1} {\tctx, \genA = \tau_2}$ & = & $\applye \cctx \tctx$
                                    & If $\applye \cctx {\tau_1} = \applye \cctx {\tau_2}$ \\
    $\applye {\cctx, \genA = \tau_1} {\tctx}$ & = & $\applye \cctx \tctx$ \\
  \end{tabular}
    \caption{Context application.}
    \label{fig:context-application-on-context}
\end{figure*}

\subsection{Soundness}

Although our overall framework for proofs is quite similar as
\citet{dunfield2013complete}, unlike their work that always implicitly assume
the contexts and types involved are well-formed, we put much extra efforts on
dealing with the well-formedness and the typing explicitly since we have to resolve the dependency carefully.

We proved that our type sanitization strategy and the unification algorithm
are sound.
First, we show that the output context after type
sanitization is indeed an extension of the input context:

\begin{lemma}[\TypeSanitizationExtensionName]
  \TypeSanitizationExtensionBody
\end{lemma}

And except resolving the order problem, type sanitization will not change the
type. Namely, the input type and the output type are equivalent after
substitution by the output context:

\begin{lemma}[\TypeSanitizationEquivalenceName]
  \TypeSanitizationEquivalenceBody
\end{lemma}

Moreover, if the input type is well-formed under the input context, then the
output type is still well-formed under the output context:

\begin{lemma}[\TypeSanitizationWellFormednessName]
  \TypeSanitizationWellFormednessBody
\end{lemma}

Having those lemmas related to type sanitization, we can prove the properties of
the unification algorithm. For example, the output context of unification
extends the input context:

\begin{lemma}[\UnificationExtensionName]\leavevmode
  \UnificationExtensionBody
\end{lemma}

And finally, we can prove that two input types are really unified by the
unification algorithm:

\begin{lemma}[\UnificationEquivalenceName]\leavevmode
  \UnificationEquivalenceBody
\end{lemma}

\subsection{Completeness}

We use the notation $\tctx \bywf \tau_1 = \tau_2$ to mean that
$\tctx \byinf \tau_1 \infto \sigma$, $\tctx \byinf \tau_2 \infto \sigma$ for
some $\sigma$,
and $\tau_1 = \tau_2$.

The completeness of type sanitization is proved by a more general lemma, which
can be found in the appendix. The more readable version of the completeness of
type sanitization is:

\begin{corollary}[\TypeSanitizationCompletenessPrettyName]
  \label{lemma:\TypeSanitizationCompletenessPrettyName}
  \TypeSanitizationCompletenessPrettyBody
\end{corollary}

\noindent which leads directly to the unification between existential variables:

\begin{lemma}[\TypeSanitizationCompletenessUnificationName]\leavevmode
  \label{lemma:\TypeSanitizationCompletenessUnificationName}
  \TypeSanitizationCompletenessUnificationBody
\end{lemma}

Having the completeness of type sanitization, we are ready to prove that our
unification algorithm is complete:

\begin{lemma}[\UnificationCompletenessName]
  \label{lemma:\UnificationCompletenessName}
    \UnificationCompletenessBody
\end{lemma}

\section{Higher Rank Polymorphic Type System}
\label{sec:higherrank}

In this section, we adopt the type sanitization strategy to a higher rank
polymorphic type system from \citet{dunfield2013complete}. We show that type
sanitization can be further extended to polymorphic type sanitization to deal
with subtyping, which can be used to replace the instantiation relation in
original system while preserving the completeness and subtyping.

\subsection{Language}

The syntax of \citet{dunfield2013complete} is given below. Notice that
notations from Section~\ref{sec:dependent} are reused here. We will always
make it clear from the context which system is
being referred. \\

\begin{tabular}{lrcl}
  Type & $A, B$ & \syndef & $\Unit \mid \varA \mid \genA \mid \forall \varA. A \mid A \to B $ \\
  Monotype & $\sigma, \tau$ & \syndef & $\Unit \mid \varA \mid \genA \mid \sigma \to \tau $ \\
  Contexts & $\tctx, \ctxl, \ctxr$ & \syndef & $\ctxinit \mid \tctx, \varA
                                               \mid \tctx, x: A
                                               \mid \tctx, \genA
                                               \mid \tctx, \genA = \tau
                                               \mid \tctx, \marker \genA $\\
  Complete Contexts & $\cctx$ & \syndef & $\ctxinit \mid \cctx, \varA
                                          \mid \cctx, x: A
                                          \mid \cctx, \genA = \tau
                                          \mid \cctx, \marker \genA $\\
\end{tabular}

\paragraph{Types.}
Types $A, B$ include unit type $\Unit$, type variables $\varA$, existential
variables $\genA$, polymorphic types $\forall \varA. A$ and function types $A
\to B$.

Monotypes $\sigma, \tau$ are a special kind of types without universal
quantifiers.

\paragraph{Contexts.}
Contexts $\tctx$ are an ordered list of type variables $\varA$, variables $x:
A$, existential variables $\genA$ and $\genA = \tau$, and special markers
$\marker \genA$ for scoping reasons.

And all existential variables in complete contexts $\cctx$ are solved.


\subsection{Subtyping}

\begin{figure*}[t]
  \begin{mathpar}
    \framebox{$\tctx \bysub A \tsub B \toctx$}
     \\
    \ADunVar \and \ADunUnit \and
    \ADunExvar \and \ADunArrow \and
    \ADunForallL \and \ADunForallR \and
    \ADunInstL \and \ADunInstR
  \end{mathpar}
  \caption{Algorithmic Subtyping (Original).}
  \label{fig:subtyping}
\end{figure*}

The original definition of algorithmic subtyping is given in
Figure~\ref{fig:subtyping}. The judgment $\tctx \bysub A \tsub B \toctx$ is
interpreted as: given the context $\tctx$, $A$ is a subtype of $B$ under the
output context $\ctxl$.
The subtyping relation is reflexive (\rul{Var}, \rul{Exvar} and
\rul{Unit}). The function type is contra-variant on the argument type,
and co-variant on the return type (\rul{$\to$}). Here the intermediate context
$\ctxl$ is applied to $A_2$ and $B_2$ to make sure the input types are fully
substituted under the input context. In \rul{$\forall$L}, new existential
variable $\genA$ is created to represent the universal quantifier. There is a
marker before $\genA$ so that we can throw away all tailing contexts that are
out of scope after the subtyping. Similarly, rule \rul{$\forall$R} puts a
new type variable in the context and throws away all the contexts after the type variable.

When the left hand side (\rul{InstL}) or the right hand side (\rul{InstR}) is an
existential variable, the subtyping leaves all the work to the instantiation
judgment ($\ist$), some of which we have seen in Section~\ref{sec:overview}. Due
to the limitation of space, we put the complete definition of the instantiation
relation in the appendix.

\subsection{Polymorphic Type Sanitization}

\begin{figure*}[t]
  \begin{mathpar}
    \framebox{$\tctx \bysub A \tsub B \toctx$} \\
    \ADunSaL \and \ADunSaR
  \end{mathpar}

  \begin{mathpar}
    \framebox{$\tctx[\genA] \bybsa A \sa \sigma \toctx$}
     \\
    \IAllPlus \and \IAllMinus \and
    \IPiPoly \and \IUnit \and \ITVar \and
    \IEVarAfterPoly \and \IEVarBeforePoly
  \end{mathpar}
  \caption{New subtyping rules, polymorphic type sanitization.}
  \label{fig:polymorphic-sanitization}
\end{figure*}

The goal of polymorphic type sanitization is to replace the instantiation
relationship with even simpler rules.

Therefore, we replace the rules \rul{InstL} and \rul{InstR} with new rules
\rul{SaL} and \rul{SaR} shown at the top of
Figure~\ref{fig:polymorphic-sanitization}. Namely, the subtyping leaves the job
to polymorphic type sanitization to resolve the order problem with existential
variables and also sanitize the input type to a monotype. A final check
ensures that the sanatized type is well-formed under the context before $\genA$.
If the type being sanitized appears on the right (as \rul{SaL}), we say it
appears co-variantly, and we say it appears contra-variantly if it appears on
the left (as \rul{SaR}). This corresponds to the fact that a polymorphic type
can be a subtype of a monotype only if all the universal quantifiers appear
contra-variantly in the type.

The rules for polymorphic type sanitization are shown at the bottom of
Figure~\ref{fig:polymorphic-sanitization}. According to whether the type appears
co-variantly ($s = -$) or contra-variantly ($s = +$), we have two modes. The
judgment $\tctx[\genA] \bybsa A \sa \sigma \toctx$ is interpreted as: under
typing context $\tctx$ which contains $\genA$, sanitize a possibly polymorphic
type $A$ to a monotype
$\sigma$, with output context $\ctxl$. Computationally, there are three inputs
($\tctx, \genA$ and $A$), and two outputs ($\sigma$ and $\ctxl$).

The only difference between these two modes is how to sanitize polymorphic
types. If a polymorphic type appears contra-variantly (\rul{I-All-Plus}), it
means a monotype would make the final type more polymorphic. Therefore, we
replace the universal binder $\varA$ with a fresh existential variable $\genB$
and put it before $\genA$. Otherwise, in rule \rul{I-All-Minus}, we put $\varA$
in the context and sanitize $A$. Notice that the result $\sigma$ might not be
well-formed under the output context $\ctxl$, since $\varA$ is discarded in the
output context. Rule \rul{I-Pi-Poly} is where the mode is flipped.

Polymorphic type sanitization does nothing if it is a unit type (\rul{I-Unit})
or a type variable (\rul{I-TVar}). Rule \rul{I-EVarAfter-Poly} and
\rul{I-EVarBefore-Poly} deals with existential variables, and creates fresh
existential variables if the input existential variable appears after $\genA$,
as we have seen in type sanitization in Section~\ref{subsec:unification}.

\paragraph{Example}

The derivation of the subtyping problem $\genA \bysub \genA \tsub (\forall
\varA. \varA \to \varA) \to \Unit$ is given below. For clarity, we omit some
detailed process, and denote $\ctxl = \genB, \genA$.\bruno{example overflows}

\[
\ExSub
\]

\subsection{Meta-theory}

The soundness and completeness of subtyping relies on the soundness and
completeness of instantiation in \citet{dunfield2013complete}. To prove
polymorphic type sanitization works correctly, there is no need to re-prove the
lemmas about subtyping. Instead, we prove that polymorphic type sanitization
leads to exactly the same result as instantiation.

For soundness, we prove that
under contra-variant mode ($s = +$), the input type is a
declarative subtype of the output type after substituted by a complete context,
while under co-variant mode ($s = -$), the input type is a declarative supertype:

\begin{lemma}[\PolymorphicTypeSanitizationSoundnessName]\leavevmode
  \label{lemma:\PolymorphicTypeSanitizationSoundnessName}
  \PolymorphicTypeSanitizationSoundnessBody
\end{lemma}

For completeness, we first prove that for a possibly polymorphic type $\applye
\cctx A$, if there is a monotype $\tau$ that is more polymorphic than it, there
is a polymorphic type sanitization result $\sigma$ of $A$, and it is the most
general subtype in the sense that we can recover the $\tau$ from applying a
complete context to $\sigma$.

\begin{lemma}[\PolymorphicTypeSanitizationCompletenessName]
  \label{lemma:\PolymorphicTypeSanitizationCompletenessName}
  \PolymorphicTypeSanitizationCompletenessBody
\end{lemma}

Based on the completeness of polymorphic type sanitization, we can prove exactly
the same completeness lemma as the instantiation to show that subtyping for
existential variables is complete:

\begin{corollary}[\PolymorphicTypeSanitizationCompletenessSubtypingName]
  \label{lemma:\PolymorphicTypeSanitizationCompletenessSubtypingName}
  \PolymorphicTypeSanitizationCompletenessSubtypingBody
\end{corollary}

%%% Local Variables:
%%% mode: latex
%%% TeX-master: "../main"
%%% org-ref-default-bibliography: "citation.bib"
%%% End:
\section{Discussion}

\subsection{Type Inference Algorithm For Dependent Types}

In Section~\ref{sec:dependent}, we use unification algorithm for dependent types
to show that type sanitization is applicable to advanced features. However,
notice that the typing rules presented in Section~\ref{subsec:typing} works
specifically for well-formedness of types. In other words, it is \textit{not} an
algorithmic type system containing type inference. Therefore, readers may wonder:
how can we design a type inference algorithm for practical usage based on the
unification algorithm, and what is the relation between the type inference
algorithm and the well-formedness typing rules? In this section, we discuss
these two questions.

\paragraph{Unification and Type Inference.} In a program, programmers may omit
some type annotations, and the type system will try to infer them. A typical
example is unannotated lambdas. For example

$\erlam x {x + 1} $

There is no annotation for the binder $x$, but the type system is able to
recover that $x:\Int$ from the expression $x + 1$. This is done by generating a
fresh existential variable $\genA$ for $x$, and then using the unification
algorithm to unify $\genA$ with $\Int$. In summary, type inference generates
existential variables that are waiting to be solved, while unification is in
charge of finding the solutions according to the type constraints.

Therefore, we can talk about unification separately from type inference. And
type inference for dependent types is also a challenging topic, where the
unification algorithm would definitely help with. Of course, in a type system
with unannotated lambdas, we also need to extend the unification to deal with
the new lambdas, which is not so complicated.

\paragraph{Type Inference and Well-formedness.} Even if we are given an
algorithmic type system containing type inference, the well-formedness system
still works correctly as long as the definition of well-formedness not changed.
Namely, any input type to the unification algorithm is already fully
type-checked, therefore it will introduce no more new constraints. However,
in our unification algorithm, well-formedness is only used for propositions,
so we leave it non-deterministic for simplicity. If it is used in the
algorithm, we would need to change it to an deterministic and decidable
relation.

\subsection{Future Work}

One possible future work is to apply the strategy of type sanitization to type
systems with more advanced features. For example, the polymorphic type
sanitization presented in Section~\ref{sec:higherrank} works specifically for
subtyping between polymorphic types. We can try to extend type sanitization to
other kinds of subtyping, for instance, nominal subtyping and/or intersection
types~\cite{}.

Also, since type sanitization resolves unification problem between dependent
types, another possible future work is to come up with a complete type inference
algorithm for dependent types based on alpha-equality and first-order
constraints. Extending the type sanitization to see whether it could play
any roles in extended setting, such as beta-equality or higher-order constraints
is also a feasible future work.

\section{Related Work}

\subsection{Type Inference in Contexts}

We have discussed the work by \citet{gundry2010type} and
\citet{dunfield2013complete} in Section~\ref{sec:overview}, which in some sense
can both be understood in terms of the proofs systems of
\citet{miller1992unification}. They adopt different strategies to resolve the
order problem of existential variables, with different emphasizes.
\citet{gundry2010type} supports ML-style polymorphism, and as they mention the
longer-term goal is to elaborate high-level dependently typed programs into
fully explicit calculi. However they do not present the algorithm nor prove it.
\citet{dunfield2013complete} use the strategy specifically for higher rank
polymorphism, while as we have seen it cannot be applied to dependent types.

\subsection{Unification for Dependent Types}

Unification for dependent types has been a challenging topic for years. While
our unification only solves first-order constraints based on alpha-equality,
existing literatures take some advanced features into account, with the price
that the algorithm is usually complicated to understand, and also the
meta-theory is hard to prove. \citet{elliott1989higher} develop a higher order
unification algorithm for dependent function types in the spirit of
\citet{huet1975unification}, though the problem is undecidable.
\citet{reed2009higher} comes up with a constraint simplification algorithm that
works on dynamic pattern fragment of higher-order unification in a dependent
Later, \citet{abel2011higher} propose a constraint-based unification algorithm
that solves a richer class of patterns.
\citet{ziliani2015unification} formalize the unification algorithm used in the
language Coq\citep{coqsite}, which is quite sophisticated and the correctness
proof is still lacked.

\subsection{Type Inference for Higher Rank Type System}

Type inference involving higher ranks are well studied in recent years.
\citet{jones2007practical} develop an approach using traditional bi-directional
type checking, which is built upon \citet{odersky1996putting}. In this system,
subtyping and unification are separated, and unification is only between
monotypes. \citet{dunfield2013complete} build a simple and concise algorithm for
higher ranked polymorphism, which is also based on traditional bidirectional
type checking. The instantiation in their system is introduced in
Section~\ref{sec:overview}. These two systems are predictive, in the sense that
universal quantifiers can only be instantiated with monotypes. There are also
impredictive systems, including $ML^F$
\citep{le2014mlf,remy2008graphic,le2009recasting}, the HML system
\citep{leijen2009flexible}, the FPH system \citep{vytiniotis2008fph} and so on.
%%% Local Variables:
%%% mode: latex
%%% TeX-master: "../main"
%%% org-ref-default-bibliography: "citation.bib"
%%% End:

\section{Conclusion}

In this paper, we propose a new strategy \textit{type sanitization}, and present
two applications of this strategy. The first application is type sanitization in
a unification algorithm for a dependent type system with alpha-equality based
first order constrains. And the second one is in a subtyping algorithm for a
higher rank polymorphic type system with the extended \textit{polymorphic type
  sanitization}.

We expect type sanitization is applicable to more type systems with other
advanced features, for example, intersection types and more advanced
forms of dependent types.


% \section{Algorithm}

\begin{tabular}{lrcl}
  Expr & $e, A, B$ & \syndef & $x \mid \star \mid
                               e_1 ~ e_2 $ \\%\mid \castup A e \mid \castdn e \\
       &           & \synor  & $\blam x A {e_2} \mid
                               \bpi x A B$ \\
       &           & \synor  & $\erlam x {e_2} \mid \genA$ \\
  Local Context & $\sctx$ & \syndef & $\ctxinit \mid \sctx, x: A$\\


  Meta Context & $\tctx$ & \syndef & $\ctxinit \mid \tctx, \genA : A[\sctx] \mid \tctx, \genA = e : A[\sctx] $ \\
\end{tabular}

\begin{tabular}{llll}
Syntactic Sugar & $A \to B$    & $\triangleq \bpi {x} A B$& where $x \notin FV(B)$ \\
\end{tabular}

\begin{mathpar}
  \framebox{$\tctx \ctxsplit \sctx \byinf e_1 \tsub e_2 \Longleftrightarrow A \toctx$} \\
  \AAx \and \AVarRefl
  \and \AAbsInfer \and \AAbsCheck
  \and \AAbs
  \and \AApp \and \AProd
  % \and \ACastUp \and \ACastDn
  \and \ASub
  % \and \ASolvedEVarL \and \ASolvedEVarR
  \and \AEVarL \and \AEVarR
\end{mathpar}

Notes:
\begin{itemize}
  \item Missing App-EVar.
  \item In A-App, how to deal with variables in local context?
  \item Context well-formedness
    \begin{itemize}
    \item Mutual dependent with typing.
    \item Weakening.
    \end{itemize}
  \item Invariant: $e_1, e_2$ are fully substituted under context application
    \begin{itemize}
    \item Is it necessary?
    \end{itemize}
  \item Decidability
\end{itemize}

%%% Local Variables:
%%% mode: latex
%%% TeX-master: "../main"
%%% org-ref-default-bibliography: "citation.bib"
%%% End:

\bibliography{sections/citation}

\appendix

% \newpage

\section{Dependent Type System}

\subsection{Properties of Context Application}

\begin{lemma}[\ContextApplicationIsIdempotentName]
  \label{lemma:\ContextApplicationIsIdempotentName}
  \ContextApplicationIsIdempotentBody
\end{lemma}
\proof
By induction on the well formedness of context. Case \rul{AC-Empty},
\rul{AC-Var}, and \rul{AC-EVar} are trivial. We discuss \rul{AC-SolvedEVar} below.

\begin{itemize}
  \item Case \[\ACSolvedEVar\]
    \begin{longtable}[l]{lll}
      & $ \applye {\tctx, \genA = \tau} {\applye {\tctx, \genA = \tau} \sigma} =
      \applye {\tctx, \genA = \tau} {\applye \tctx {\sigma \subst \genA
          \tau}} $ & By definition \\
      & $\tctx \byinf \tau \infto \star$ & Given \\
      & $\genA \notin \tctx$ & Given \\
      & $\genA \notin FV(\tau)$ & By above propositions \\
      & $\genA \notin FV(\sigma \subst \genA \tau)$ & Follows directly \\
      & $\genA \notin FV(\applye \tctx {\sigma \subst \genA \tau})$ & Follows
      directly \\
      & $\applye {\tctx, \genA = \tau} {\applye \tctx {\sigma \subst \genA
          \tau}} = \applye {\tctx} {\applye \tctx {\sigma \subst \genA \tau}}$ &
      Substitute fresh variable \\
      & $ = \applye \tctx {\sigma \subst \genA \tau}$ & By induction hypothesis
      \\
      & $ = \applye {\tctx, \genA = \tau} \sigma$ & By definition of context application
    \end{longtable}
\end{itemize}
\qed

\begin{lemma}[\ReductionPreservesFullySubstitutionName]
  \label{lemma:\ReductionPreservesFullySubstitutionName}
  \ReductionPreservesFullySubstitutionBody
\end{lemma}
\proof
Follows directly from the definition of context substitution and reduction.
\qed

\begin{lemma}[\ContextApplicationOverReductionName]
  \label{lemma:\ContextApplicationOverReductionName}
  \ContextApplicationOverReductionBody
\end{lemma}
\proof

By induction on the reduction derivation.

\begin{itemize}
  \item Case \[\RApp\]
    \begin{longtable}[l]{lll}
      & $\applye \tctx {e_1} \redto \applye \tctx {e_1'}$& By induction
      hypothesis \\
      & $\applye \tctx {e_1~ e_2}$ & \\
      & $ = \applye \tctx {e_1} ~ \applye \tctx {e_2}$ & By property of
      substitution  \\
      & $\redto \applye \tctx {e_1'} ~ \applye \tctx {e_2}$& By \rul{R-App} \\
      & $= \applye \tctx {e_1' ~ e_2}$& By property of substitution
    \end{longtable}
  \item Case \[\RBeta\]
    \begin{longtable}[l]{lll}
      & $\applye \tctx {(\blam x \sigma {e_1}) ~ e_2}$ & \\
      & $= {(\blam x {\applye \tctx \sigma} {\applye \tctx {e_1}}) ~
        \applye \tctx {e_2}}$ & By
      definition of substitution \\
      & $\redto (\applye \tctx {e_1}) \subst x {\applye \tctx {e_2}}$& By
      \rul{R-Beta} \\
      & $= \applye \tctx {e_1 \subst x {e_2}}$& By property of substitution
    \end{longtable}
  \item Case \[\RCastDown\]
    Follows directly from induction hypothesis.
  \item Case \[\RCastDownUp\]
    Follows directly from induction hypothesis.
\end{itemize}

\qed

\begin{lemma}[\OutputIsFullySubstitutedName]
  \label{lemma:OutputIsFullySubstituted}
  \OutputIsFullySubstitutedBody
\end{lemma}
\proof
By induction on typing derivation.
\begin{itemize}
  \item Case \rul{A-Ax}, \rul{A-EVar}, \rul{A-SolvedEVar}, \rul{A-Pi} follows
    directly from $\applye \tctx \star = \star$.
  \item Case \[\AVar\]
    \begin{longtable}[l]{lll}
    & $\applye \tctx {\applye \tctx \sigma} = \applye \tctx \sigma$ & By
    Lemma~\ref{lemma:\ContextApplicationIsIdempotentName}\\
    \end{longtable}
  \item Case \[\ALamAnn\]
    \begin{longtable}[l]{lll}
      & $\applye \tctx {\bpi x {\applye \tctx {\sigma_1}} {\sigma_2}} =
      \bpi x {\applye \tctx {\applye \tctx {\sigma_1}}} {\applye \tctx
        {\sigma_2}}$
      & Follows from definition of context substitution \\
      &$=\bpi x {\applye \tctx {\sigma_1}} {\applye \tctx {\sigma_2}}$& By Lemma~\ref{lemma:\ContextApplicationIsIdempotentName}\\
      &$\applye {\tctx, x: \sigma_1} {\sigma_2} = \sigma_2$ & By hypothesis\\
      &$\applye {\tctx} {\sigma_2} = \sigma_2$& Follows from definition of
      context substitution\\
      &$\applye \tctx {\bpi x {\applye \tctx {\sigma_1}} {\sigma_2}} =
      {\bpi x {\applye \tctx {\sigma_1}} {\sigma_2}}
      $ & By above equalities
    \end{longtable}
  \item Case \[\AApp\]
    \begin{longtable}[l]{lll}
      &$\applye \tctx {\bpi x {\sigma_1} {\sigma_2}} = \bpi x {\sigma_1}
      {\sigma_2}$ & By hypothesis\\
      &$\applye \tctx {\sigma_2} = \sigma_2$& Follows directly from above\\
      &$\applye \tctx {\applye \tctx e} = \applye \tctx e$& By
      Lemma~\ref{lemma:\ContextApplicationIsIdempotentName} \\
      &$\applye \tctx {\sigma_2 \subst x {\applye \tctx {e_1}}}$ & \\
      &$=\applye \tctx {\sigma_2} \subst x {\applye \tctx {\applye \tctx {e_1}}}$&
      Context application is distributed over substitution \\
      &$={\sigma_2} \subst x {\applye \tctx{e_1}}$& By above equalities
    \end{longtable}
  \item Case \[\ACastDn\]
    \begin{longtable}[l]{lll}
      & $\applye \tctx {\sigma_1} = {\sigma_1}$ & By hypothesis \\
      & $\applye \tctx {\sigma_2} = {\sigma_2}$ & By Lemma~\ref{lemma:\ReductionPreservesFullySubstitutionName} \\
    \end{longtable}
  \item Case \[\ACastUp\]
    Follows directly from Lemma~\ref{lemma:\ContextApplicationIsIdempotentName}
\end{itemize}
\qed

\begin{lemma}[\ContextApplicationInContextName]
  \label{lemma:\ContextApplicationInContextName}
  \ContextApplicationInContextBody
\end{lemma}
\proof
By induction on the typing relation.
\begin{itemize}
\item Case \rul{A-AX}, \rul{A-EVar} and \rul{A-SolvedEVar} follows directly.
\item Case \[\AVar\]
  If $x = y$, then result type is $\applye {\tctx} {\applye \tctx \tau} =
  \applye \tctx \tau$ by Lemma~\ref{lemma:\ContextApplicationIsIdempotentName}.
  Otherwise the result is the same.
\item The rest cases follows directly from the hypothesis.
\end{itemize}
\qed

\begin{lemma}[\ReverseContextApplicationInContextName]
  \label{lemma:\ReverseContextApplicationInContextName}
  \ReverseContextApplicationInContextBody
\end{lemma}
\proof
By induction on the typing relation.
\begin{itemize}
\item Case \rul{A-AX}, \rul{A-EVar} and \rul{A-SolvedEVar} follows directly.
\item Case \[\AVar\]
  If $x = y$, then result type is
  $\applye \tctx \tau = \applye {\tctx} {\applye \tctx \tau}$
  by Lemma~\ref{lemma:\ContextApplicationIsIdempotentName}.
  Otherwise the result is the same.
\item The rest cases follows directly from the hypothesis.
\end{itemize}
\qed


\begin{definition}[Typing Size], The size of typing derivation
  $\tctx \byinf \sigma \infto \tau$,
  defined based on typing process in Figure~\ref{},
  written as $\sizet {\tctx \byinf \sigma \infto \tau}$,
  is defined as:
  \begin{center}
  \begin{tabular}{rll}
    $\sizet {\tctx \byinf \star \infto \star}$ & = & 1 \\
    $\sizet {\tctx \byinf x \infto \applye \tctx \sigma}$ & = & 1 \\
    $\sizet {\tctx \byinf \genA \infto \star}$ & = & 1 if $\genA \in \tctx$\\
    $\sizet {\tctx \byinf \genA \infto \star}$ & = & $1 + \sizet{\tctx_1 \byinf \tau \infto \star}$ if $\tctx = \tctx_1, \genA = \tau, \tctx2$\\
    $\sizet {\tctx \byinf \blam x {\sigma_1} e \infto \bpi x {\applye {\tctx} {\sigma_1}} {\sigma_2}}$
                                               & = &
                                                     $1 + \sizet{\tctx \byinf \sigma_1 \infto \star}
                                                     + \sizet{\tctx, x : \sigma_1 \byinf e \infto \sigma_2}$  \\
    $\sizet{\tctx \byinf \bpi x {\sigma_1} {\sigma_2} \infto \star}$
                                               & = &
                                                     $1 + \sizet{\tctx \byinf \sigma_1 \infto \star}
                                                     + \sizet{\tctx, x : \sigma_1 \byinf \sigma_2 \infto \star}
                                                     $\\
    $\sizet{\tctx \byinf \castdn e \infto \sigma_2}$
                                               & = &
                                                     $1 + \sizet{\tctx \byinf e \infto \sigma_1}$ \\
    $\sizet{\tctx \byinf \castup e \infto \applye {\tctx} {\sigma_1}}$
                                               & = &
                                                     $1 + \sizet {\tctx \byinf e \infto \sigma_2} $
  \end{tabular}
  \end{center}
\end{definition}

\begin{lemma}[\ContextApplicationPreservesTypingName]
  \label{lemma:\ContextApplicationPreservesTypingName}
  \ContextApplicationPreservesTypingBody
\end{lemma}
\proof
By induction on the typing size.

\begin{itemize}
  \item Case \[\AAx\]
    \begin{longtable}[l]{lll}
      & $ \applye \tctx \star = \star$ & Directly from the definition of context
      application\\
      & $ \tctx \byinf \star \infto \star$ & By \rul{A-Ax}
    \end{longtable}
  \item Case \rul{AVar}, \rul{AEVar} are similar as \rul{AAx}, which follows
    from definition directly.
  \item Case \[\ASolvedEVar\]
    \begin{longtable}[l]{lll}
      & let $\tctx = \tctx_1, \genA = \tau, \tctx_2$& \\
      & $\applye \tctx \genA = \applye {\tctx_1} \tau$& Follows from definition of context
      application\\
      & $\tctx_1 \byinf \applye {\tctx_1} \tau \infto \star$& By hypothesis\\
      & $\tctx_1, \genA = \tau, \tctx_2 \byinf \applye {\tctx_1} \tau \infto
      \star$ & By Lemma~\ref{lemma:\TypingWeakeningName} \\
    \end{longtable}
  \item Case \[\ALamAnn\]
    \begin{longtable}[l]{lll}
      & $\applye \tctx {\blam x {\sigma_1} e} = \blam x {\applye \tctx
        {\sigma_1}} {\applye \tctx e} $ & Follows from definition of context application \\
      & $\tctx \byinf \applye \tctx {\sigma_1} \infto \star$ & By hypothesis \\
      & $\tctx, x : \sigma_1 \byinf \applye {\tctx, x: \sigma_1} e \infto \sigma_2$ & By hypothesis \\
      & $\tctx, x : \sigma_1 \byinf \applye {\tctx} e \infto \sigma_2$ & By
      definition of context application \\
      & $\tctx, x : \applye \tctx {\sigma_1} \byinf \applye {\tctx} e \infto
      \sigma_2$ & By Lemma~\ref{lemma:\ContextApplicationInContextName}\\
    \end{longtable}

  \item Case \[\APi\]
    \begin{longtable}[l]{lll}
      & $\applye \tctx {\bpi x {\sigma_1} {\sigma_2}} = \bpi x {\applye \tctx
        {\sigma_1}} {\applye \tctx {\sigma_2}} $ & Follows from definition of context application \\
      & $\tctx \byinf \applye \tctx {\sigma_1} \infto \star$ & By hypothesis \\
      & $\tctx, x : \sigma_1 \byinf \applye {\tctx, x: \sigma_1} {\sigma_2} \infto \star$ & By hypothesis \\
      & $\tctx, x : \sigma_1 \byinf \applye {\tctx} {\sigma_2} \infto \star$ & By
      definition of context application \\
      & $\tctx, x : \applye \tctx {\sigma_1} \byinf \applye {\tctx} {\sigma_2}
      \infto \star$ & By Lemma~\ref{lemma:\ContextApplicationInContextName}\\
    \end{longtable}
  \item Case \rul{A-App}, \rul{A-CastDn}, \rul{A-CastUp} follows directly from
    the hypothesis.
\end{itemize}
\qed

\begin{lemma}[\ReverseContextApplicationPreservesTypingName]
  \label{lemma:\ReverseContextApplicationPreservesTypingName}
  \ReverseContextApplicationPreservesTypingBody
\end{lemma}

\proof

By induction on the typing derivation.

Then we analyze the shape of $\sigma_1$.
Because when we do inversion, there is always one case that $\sigma_1 = \genA$
for some $\genA$, we deal with the case first and then ignore all the cases when
$\sigma$ is an existential variable.

\begin{itemize}
  \item Case $\sigma = \genA$.
    According to $\genA$ is solved or not in $\tctx$,
    we have two subcases.
    \begin{itemize}
    \item $\tctx = \tctx_1, \genA, \tctx_2$.
      Then $\applye \tctx \genA = \genA$.
      Therefore the goal follows trivially.
    \item $\tctx = \tctx_1, \genA = \tau, \tctx_2 $.
      \begin{longtable}[l]{lll}
        & $\tctx \wc$
        & By Lemma~\ref{lemma:\TypingContextWellFormednessName} \\
        & $\tctx \byinf \genA \infto \star$
        & By \rul{A-SolvedEVar}
      \end{longtable}
    \end{itemize}
  \item Case $\applye \tctx {\sigma_1} = \star$
      \begin{longtable}[l]{lll}
        & $\sigma_1 = \star$
        & By inversion \\
        & $\tctx \byinf \star \infto \star$
        & By \rul{A-Ax}
      \end{longtable}
  \item Case $\applye \tctx {\sigma_1} = x$
      \begin{longtable}[l]{lll}
        & $\sigma_1 = x$
        & By inversion \\
        & $\tctx \byinf x \infto \sigma_2$
        & By \rul{A-Var}
      \end{longtable}
  \item Case $\applye \tctx {\sigma_1} = \genA$
    Then we have $\sigma_1 = \genB$ for some $\genB$.
    We can prove it as in the first case.
  \item Case $\applye \tctx {\sigma_1} = \blam x {\tau_1} {e_1} $
      \begin{longtable}[l]{lll}
        & $\sigma_1 = \blam x {\tau_2} {e_2}$
        & By inversion \\
        & $\applye  \tctx {\tau_2} = \tau_1 $
        & As above \\
        & $\applye  \tctx {e_2} = e_1 $
        & As above \\
        & $\tctx \byinf \tau_1 \infto \star$
        & By inversion \\
        & $\tctx, x : \tau_1 \byinf e_1 \infto \sigma$
        & As above \\
        & $\sigma_2 = \bpi x {\applye \tctx {\tau_1}} {\sigma} $
        & Given \\
        & $\tctx \byinf \tau_2 \infto \star$
        & By induction hypothesis \\
        & $\tctx, x : \tau_1 \byinf e_2 \infto \sigma$
        & By induction hypothesis \\
        & $\tctx, x : \applye {\tctx} {\tau_2} \byinf {e_2} \infto \sigma$
        & By substituting the equality \\
        & $\tctx, x : {\tau_2} \byinf {e_2} \infto \sigma$
        & By Lemma~\ref{lemma:\ReverseContextApplicationInContextName} \\
        & $\tctx \byinf \sigma_1 \infto \bpi x {\applye \tctx {\tau_2}} \sigma $
        & By \rul{A-LamAnn} \\
        & $\applye \tctx \tau_1 = \tau_1$
        & By \ref{lemma:\ContextApplicationIsIdempotentName} \\
        & $\tctx \byinf \sigma_1 \infto \sigma_2 $
        & By substituting the equalities
      \end{longtable}
  \item Case \rul{A-Pi} and \rul{A-App} are similar as last case.
  \item Case $\sigma_1 = \castdn e_1$
      \begin{longtable}[l]{lll}
        & $\sigma_1 = \castdn e_2$
        & By inversion \\
        & $\applye  \tctx {e_2} = e_1 $
        & As above \\
        & $\tctx \byinf e_1 \infto \sigma$
        & By inversion \\
        & $\sigma \redto \sigma_2$
        & As above \\
        & $\tctx \byinf e_2 \infto \sigma$
        & By induction hypothesis \\
        & $\tctx \byinf \castdn e_2 \infto \sigma_2$
        & By \rul{A-CastDn}
      \end{longtable}
    \item Case $\sigma_1 = \castup e_1$
      Similar as last case.
\end{itemize}

\qed

\subsection{Properties of Declarative Typing}

\begin{lemma}[\TypingContextWellFormednessName]
  \label{lemma:\TypingContextWellFormednessName}
  \TypingContextWellFormednessBody
\end{lemma}
\proof

By a straightforward induction on the typing derivation.

\qed

\begin{lemma}[\TypingWeakeningName]
  \label{lemma:\TypingWeakeningName}
  \TypingWeakeningBody
\end{lemma}
\proof

By induction on the typing derivation.
\begin{itemize}
  \item Case \rul{A-Ax}, \rul{A-Var}, \rul{A-EVar} and \rul{A-SolvedEVar}
    follows directly.
  \item Case \[\ALamAnn\]
    \begin{longtable}[l]{lll}
      &$\tctx_1, \ctxl, \tctx_2 \byinf \sigma_1 \infto \star$ & By hypothesis\\
      &$\tctx_1, \ctxl, \tctx_2, x:\sigma_1 \wc$
      & By \rul{AC-Var} \\
      &$\tctx_1, \ctxl, \tctx_2, x: \sigma_1 \byinf e \infto \sigma_2$& By
      hypothesis
    \end{longtable}
  \item Case \rul{A-Pi} is similar as \rul{A-LamAnn}.
    \item Case \rul{A-App}, \rul{A-CastDn} and \rul{A-CastUp} follows directly
      from hypothesis.
\end{itemize}
\qed

\begin{lemma}[\TypingStrengtheningName]
  \label{lemma:\TypingStrengtheningName}
  \TypingStrengtheningBody
\end{lemma}
\proof

By induction on the typing derivation.

\begin{itemize}
\item Case \rul{A-Ax}
  Follows trivially.
\item Case \[\AVar\]
  \begin{longtable}[l]{lll}
    & $\tctx_1, \tctx_3 \bywt x$ & Given \\
    & $x:\sigma \in \tctx_1, \tctx_3$ & By inversion \\
    & $\tctx_1, \tctx_3 \byinf x \infto \applye {\tctx_1, \tctx_3} \sigma$
    & By \rul{A-Var} \\
    & $\tctx_1, \tctx_3 \wc$ & Given \\
    & $\tctx_1, \tctx_3$ & does not contain any existential variable in $\tctx_2$ \\
    & $\applye {\tctx_1, \tctx_3} \sigma = \applye {\tctx_1, \tctx_2, \tctx_3}
    \sigma $
    & Follows directly
  \end{longtable}
\item \[\AEVar\]
  \begin{longtable}[l]{lll}
    & $\tctx_1, \tctx_3 \bywt \genA$ & Given \\
    & $\genA \in \tctx_1, \tctx_3$ & By inversion \\
    & $\tctx_1, \tctx_3 \byinf \genA \infto \star$
    & By \rul{A-EVar} \\
  \end{longtable}
\item \[\ASolvedEVar\]
  \begin{longtable}[l]{lll}
    & $\tctx_1, \tctx_3 \bywt \genA$ & Given \\
    & $\genA = \tau \in \tctx_1, \tctx_3$ & By inversion \\
    & $\tctx_1, \tctx_3 \byinf \genA \infto \star$
    & By \rul{A-EVar} \\
  \end{longtable}
\item \[\ALamAnn\]
  \begin{longtable}[l]{lll}
    & $\tctx_1, \tctx_3 \bywt \sigma_1$ & Given \\
    & $\tctx_1, \tctx_3 \byinf \sigma_1 \infto \star $
    & By induction hypothesis \\
    & $\tctx_1, \tctx_3, x : \sigma_1 \wc $
    & By \rul{AC-Var} \\
    & $\tctx_1, \tctx_3, x: \sigma_1 \bywt e$ & Given \\
    & $\tctx_1, \tctx_3, x: \sigma_1 \byinf e \infto \sigma_2 $
    & By induction hypothesis \\
    & $\tctx_1, \tctx_3 \byinf \blam x {\sigma_1} e \infto
    \bpi x {\applye {\tctx_1, \tctx_3} {\sigma_1}} \sigma_2 $
    & By \rul{A-LamAnn} \\
    & $\tctx_1, \tctx_3 \wc$ & Given \\
    & $\tctx_1, \tctx_3$ & does not contain any existential variable in $\tctx_2$ \\
    & $\applye {\tctx_1, \tctx_3} {\sigma_1}
    = \applye {\tctx_1, \tctx_2, \tctx_3} {\sigma_1} $
    & Follows directly
  \end{longtable}
\item \[\APi\]
  Similar as Case \rul{A-LamAnn}.
\item \[\AApp\]
  \begin{longtable}[l]{lll}
    & $\tctx_1, \tctx_3 \bywt e_1$ & Given \\
    & $\tctx_1, \tctx_3 \byinf e_1 \infto \bpi x {\sigma_1} {\sigma_2} $
    & By induction hypothesis \\
    & $\tctx_1, \tctx_3 \bywt e_2$ & Given \\
    & $\tctx_1, \tctx_3 \byinf e_2 \infto {\sigma_1} $
    & By induction hypothesis \\
    & $\tctx_1, \tctx_3 \byinf e_1 ~ e_2 \infto
    {\sigma_2 \subst x {\applye {\tctx_1, \tctx_3} {e_1}}} $
    & By \rul{A-App} \\
    & $\tctx_1, \tctx_3 \bywt e_1$ & Given \\
    & $\tctx_1, \tctx_3 \wc$ & Given \\
    & $\tctx_1, \tctx_3, e_1$ & does not contain any existential variable in $\tctx_2$ \\
    & $\applye {\tctx_1, \tctx_3} {e_1}
    = \applye {\tctx_1, \tctx_2, \tctx_3} {\sigma_1} $
    & Follows directly
  \end{longtable}
\item \[\ACastDn\]
  \begin{longtable}[l]{lll}
    & $\tctx_1, \tctx_3 \bywt e$ & Given \\
    & $\tctx_1, \tctx_3 \byinf e \infto \sigma_1 $
    & By induction hypothesis \\
    & $\sigma_1 \redto \sigma_2$
    & Given \\
    & $\tctx_1, \tctx_3 \byinf \castdn e \infto {\sigma_2} $
    & By \rul{A-CastDn}
  \end{longtable}
\item \[\ACastUp\]
  Similar as Case \rul{A-CastDn}.
\end{itemize}
\qed

\begin{lemma}[\TypingVariableExchangeName]
  \label{lemma:\TypingVariableExchangeName}
  \TypingVariableExchangeBody
\end{lemma}
\proof

By straightforward induction on the typing derivation.

\qed

\begin{lemma}[\TypingSubstitutionName]
  \label{lemma:\TypingSubstitutionName}
  \TypingSubstitutionBody
\end{lemma}
\proof

By induction on the size of second typing derivation.

\begin{itemize}
  \item Case \rul{A-Ax}, \rul{A-EVar} and \rul{A-SolvedEVar} follows from typing
    rules directly.
  \item Case \[\AVar\]
    If two $x$'s are the same, then $\sigma_2 = \applye \tctx {\sigma_1}$. The
    goal follows directly from given conditions. If two variables are not the
    same, then it follows directly from \rul{A-Var}.
  \item Case \[\ALamAnn\]
    Let the expression be $\blam y {\sigma'} e$ to avoid name conflicts.
    \begin{longtable}[l]{lll}
      & $\tctx \byinf \sigma' \subst x \tau \infto \star$ & By induction hypothesis \\
      & $\tctx, x:\sigma_1, y: \sigma' \subst x \tau \byinf e \infto \sigma_3$ & Given, notice
      we use $\sigma_3$ to avoid name conflicts \\
      & $\tctx \byinf \sigma_1 \infto \star$ & Given \\
      & $\tctx, y: \sigma' \subst x \tau \byinf \sigma_1 \infto \star$ & By
      Lemma~\ref{lemma:\TypingWeakeningName}\\
      & $\tctx, y: \sigma' \subst x \tau, x:\sigma_1 \wc$ & By \rul{AC-Var} \\
      & $\tctx, y: \sigma' \subst x \tau, x:\sigma_1 \byinf e \infto \sigma_3$ &
      By Lemma~\ref{lemma:\TypingVariableExchangeName} \\
      & $\tctx \byinf \tau \infto \applye \tctx {\sigma_1}$ & Given \\
      & $\tctx, y: \sigma_1 \subst x \tau \byinf \tau \infto \applye \tctx
      {\sigma_1}$ & By Lemma~\ref{lemma:\TypingWeakeningName}  \\
      & $\tctx, y: \sigma_1 \subst x \tau \byinf e \subst x \tau \infto
      \sigma_3$ & By induction hypothesis \\
    \end{longtable}
  \item Case \rul{A-Pi} is similar as \rul{A-LamAnn}.
  \item Rest cases follow directly from hypothesis.
\end{itemize}
\qed

\subsection{Properties of Context Extension}

\begin{lemma}[\DeclarationPreservationName]
  \label{lemma:\DeclarationPreservationName}
  \DeclarationPreservationBody
\end{lemma}
\proof

By a straightforward induction on the context application.

\qed

\begin{lemma}[\ReverseDeclarationPreservationName]
  \label{lemma:\ReverseDeclarationPreservationName}
  \ReverseDeclarationPreservationBody
\end{lemma}
\proof

By a straightforward induction on the context application.

\qed


\begin{lemma}[\DeclarationOrderPreservationName]
  \label{lemma:\DeclarationOrderPreservationName}
  \DeclarationOrderPreservationBody
\end{lemma}
\proof

By induction on the context application.
\begin{itemize}
  \item Case \[\CEEmpty\]
    Impossible case.
  \item Case \[\CEVar\]
    \begin{itemize}
    \item SubCase $v = x$.
      \begin{longtable}[l]{lll}
        & $u \in \tctx$ & Given \\
        & $u \in \ctxr$ & By Lemma~\ref{lemma:\DeclarationPreservationName} \\
        & So $u$ is to the left of $x$ in $\tctx, x$ &
      \end{longtable}
    \item SubCase $v \neq x$.
      \begin{longtable}[l]{lll}
        & $u$ is declared to the left of $v$ in $\tctx$ & Given \\
        & $u$ is declared to the left of $v$ in $\ctxr$ & By induction hypothesis \\
        & $u$ is declared to the left of $v$ in $\ctxr, x: \sigma$ &
      \end{longtable}
    \end{itemize}
  \item Case \rul{CE-EVar}, \rul{CE-SolvedEVar}, \rul{CE-Solve} are
    similar as Case \rul{CE-Var}.
  \item Case \[\CEAdd\]
      \begin{longtable}[l]{lll}
        & $u$ is declared to the left of $v$ in $\ctxr$ & By induction
        hypothesis \\
        & So $u$ is declared to the left of $v$ in $\ctxr, \genA$ &
      \end{longtable}
  \item Case \[\CEAddSolved\]
      \begin{longtable}[l]{lll}
        & $u$ is declared to the left of $v$ in $\ctxr$ & By induction
        hypothesis \\
        & So $u$ is declared to the left of $v$ in $\ctxr, \genA = \tau$ &
      \end{longtable}
\end{itemize}
\qed

\begin{lemma}[\ReverseDeclarationOrderPreservationName]
  \label{lemma:\ReverseDeclarationOrderPreservationName}
  \ReverseDeclarationOrderPreservationBody
\end{lemma}
\proof

We can prove this lemma by contradiction. Suppose $u$ is declared to the right
of $v$ in $\tctx$, then by Lemma~\ref{lemma:\DeclarationOrderPreservationName},
we have that $u$ is declared to the right of $v$ in $\tctx$. Because we already
have $u$ is declared to the left of $v$, we have a contradiction. Therefore, $u$
is declared to the left of $v$ in $\tctx$.

\qed

\begin{lemma}[\SubstitutionExtensionInvarianceName]
  \label{lemma:\SubstitutionExtensionInvarianceName}
  \SubstitutionExtensionInvarianceBody
\end{lemma}

\proof

We first prove $\applye \ctxr \sigma = \applye \tctx {\applye \ctxr \sigma}$,
then prove $\applye \ctxr \sigma = \applye \ctxr {\applye \tctx \sigma}$.

\begin{description}
\item [Part 1]
  We do induction on the context extension.
  \begin{itemize}
    \item Case \[\CEEmpty\]
      Trivial case.
    \item Case \[\CEVar\]
      We use $\sigma_1$ to replace the $\sigma$ in \rul{CE-Var}.
      \begin{longtable}[l]{lll}
        & $\applye {\ctxr, x: \sigma_1} \sigma $ & \\
        & $= \applye \ctxr \sigma$ & Follows by definition of context application
        \\
        & $= \applye \tctx {\applye \ctxr \sigma}$ & By induction hypothesis \\
        & $= \applye {\tctx, x: \sigma_1} {\applye {\ctxr, x: \sigma_1} \sigma}$& By definition of context application
      \end{longtable}
    \item Case \[\CEEVar\]
      Similar as Case \rul{CE-Var}.
    \item Case \[\CESolvedEVar\]
      \begin{longtable}[l]{lll}
        & $\applye {\ctxr, \genA = \tau_2} \sigma $ & \\
        & $ = \applye {\ctxr} {\sigma \subst \genA {\tau_2}} $ & By definition of
        context application \\
        & $ = \applye \tctx {\applye {\ctxr} {\sigma \subst \genA {\tau_2}}} $ & By
        induction hypothesis \\
        & $ = \applye {\tctx, \genA = \tau_1} {\applye {\ctxr} {\sigma \subst
            \genA {\tau_2}}}$
        & $\genA \notin FV(\applye \ctxr {\sigma \subst \genA {\tau_2}})$ \\
        & $ = \applye {\tctx, \genA = \tau_1} {\applye {\ctxr, \genA = \tau_2}
          \sigma}$ & By definition of context application
      \end{longtable}
    \item Case \[\CESolve\]
      \begin{longtable}[l]{lll}
        & $\applye {\ctxr, \genA = \tau} \sigma $ & \\
        & $ = \applye {\ctxr} {\sigma \subst \genA \tau} $ & By definition of
        context application \\
        & $ = \applye \tctx {\applye {\ctxr} {\sigma \subst \genA \tau}} $ & By
        induction hypothesis \\
        & $ = \applye {\tctx, \genA} {\applye {\ctxr} {\sigma \subst \genA
            \tau}} $ & By definition of context application\\
        & $ = \applye {\tctx, \genA} {\applye {\ctxr, \genA = \tau} {\sigma}} $
        & By definition of context application
      \end{longtable}
    \item Case \[\CEAdd\]
      \begin{longtable}[l]{lll}
        & $\applye {\ctxr, \genA} \sigma $ & \\
        & $ = \applye {\ctxr} \sigma $ & By definition of
        context application \\
        & $ = \applye \tctx {\applye {\ctxr} \sigma} $ & By
        induction hypothesis \\
        & $ = \applye \tctx {\applye {\ctxr, \genA} \sigma} $ & By definition of
        context application
      \end{longtable}
    \item Case \[\CEAddSolved\]
      \begin{longtable}[l]{lll}
        & $\applye {\ctxr, \genA = \tau} \sigma $ & \\
        & $ = \applye {\ctxr} {\sigma \subst \genA \tau} $ & By definition of
        context application \\
        & $ = \applye \tctx {\applye {\ctxr} {\sigma \subst \genA \tau}} $ & By
        induction hypothesis \\
        & $ = \applye \tctx {\applye {\ctxr, \genA = \tau} \sigma} $ & By
        definition of context application
      \end{longtable}
  \end{itemize}


\item [Part 2]
  By induction on the size of the typing derivation.

  \begin{itemize}
  \item Case \[\AAx\]
    Follows directly by for all context $\tctx$, $\applye \tctx \star = \star$.
  \item Case \[\AVar\]
    Follows directly by for all context $\tctx$, $\applye \tctx x = x$.
  \item Case \[\AEVar\]
    \begin{longtable}[l]{lll}
      & $\applye \tctx \genA = \genA$& \\
      & $\applye \ctxr {\applye \tctx \genA} = \applye \ctxr \genA$ & Follows
      directly \\
    \end{longtable}
  \item Case \[\ASolvedEVar\]
    \begin{longtable}[l]{lll}
      & $\tctx \exto \ctxr$ & Given \\
      & $\genA = \tau \in \tctx$ & Given \\
      & $\ctxr = \ctxr_1, \genA = \tau_2, \ctxr_2$
      & By Lemma~\ref{lemma:\ExtensionOrderName} \\
      & $\applye {\ctxr_1} {\tau} = \applye {\ctxr_1} {\tau_2}$
      & As above \\
      & $\applye \ctxr {\tau} = \applye {\ctxr} {\tau_2}$
      & Since $\genA, \ctxr_2$ should be fresh contexts for $\tau, \tau_2$\\
      & $\applye \ctxr \genA$ & \\
      & $= \applye {\ctxr} {\tau_2}$ & By definition of context application \\
      & $= \applye {\ctxr} {\tau}$ & Known \\
      & $= \applye \ctxr {\applye \tctx \tau}$ & By induction hypothesis \\
      & $= \applye \ctxr {\applye \tctx \genA}$ & By definition of context application
    \end{longtable}
  \item Case \[\ALamAnn\]
    \begin{longtable}[l]{lll}
      & $\applye \ctxr {\sigma} = \applye \ctxr {\applye \tctx {\sigma_1}}$& By
      induction hypothesis \\
      & $\tctx \exto \ctxr$ & Given \\
      & $\tctx, x: \sigma_1 \exto \ctxr, x : \sigma_1 $ & By \rul{CE-Var}\\
      & $\applye {\ctxr, x: \sigma_1} e = \applye {\ctxr, x: \sigma_1} {\applye
        {\tctx, x: \sigma_1} e}$ & By induction hypothesis \\
      & $\applye \ctxr e = \applye \ctxr {\applye \tctx e}$ & By definition of
      context application \\
    \end{longtable}
  \item Case \[\APi\]
    Similar as Case \rul{A-LamAnn}.
  \item Case \[\AApp\]
    Follows directly from induction hypothesis.
  \item Case \[\ACastDn\]
    Follows directly from induction hypothesis.
  \item Case \[\ACastUp\]
    Follows directly from induction hypothesis.
\end{itemize}
\end{description}

\qed

\begin{lemma}[\ExtensionEqualityPreservationName]
  \label{lemma:\ExtensionEqualityPreservationName}
  \ExtensionEqualityPreservationBody
\end{lemma}

\proof

\mbox{} % an empty line to make sure long table appear after proof
\begin{longtable}[l]{lll}
  & $\applye \ctxr {\sigma_1}$ & \\
  & $= \applye \ctxr {\applye \tctx {\sigma_1}}$& By
  Lemma~\ref{lemma:\SubstitutionExtensionInvarianceName}\\
  & $= \applye \ctxr {\applye \tctx {\sigma_2}}$ & Given \\
  & $= \applye \ctxr {\sigma_2}$ & By
  Lemma~\ref{lemma:\SubstitutionExtensionInvarianceName}
\end{longtable}

\qed

\begin{lemma}[\ContextExtensionReflexivityName]
  \label{lemma:\ContextExtensionReflexivityName}
  \ContextExtensionReflexivityBody
\end{lemma}

\proof

By induction on the well-formedness of context.

\begin{itemize}
  \item Case \[\ACEmpty\]
    Follows directly from \rul{CE-Empty}.
  \item Case \[\ACVar\]
    \begin{longtable}[l]{lll}
      & $\tctx \exto \tctx$ & By induction hypothesis \\
      & $\tctx, x : \sigma \exto \tctx, x : \sigma$ & By \rul{CE-Var}
    \end{longtable}
  \item Case \[\ACEVar\]
    \begin{longtable}[l]{lll}
      & $\tctx \exto \tctx$ & By induction hypothesis \\
      & $\genA \notin \tctx$ & Given \\
      & $\tctx, \genA \exto \tctx, \genA$ & By \rul{CE-EVar}
    \end{longtable}
  \item Case \[\ACSolvedEVar\]
    \begin{longtable}[l]{lll}
      & $\tctx \exto \tctx$ & By induction hypothesis \\
      & $\genA \notin \tctx$ & Given \\
      & $\tctx, \genA = \tau \exto \tctx, \genA = \tau$ & By \rul{CE-SolvedEVar}
    \end{longtable}
\end{itemize}

\qed

\begin{lemma}[\ContextExtensionTransitivityName]
  \label{lemma:\ContextExtensionTransitivityName}
  \ContextExtensionTransitivityBody
\end{lemma}

\proof

By induction on the derivation $\tctx \exto \ctxr$.

\begin{itemize}
  \item Case \[\CEEmpty\]
    Our goal $\ctxl \exto \ctxinit $ is given.
  \item Case \[\CEVar\]
    \begin{longtable}[l]{lll}
      & $\ctxl \exto \tctx, x: \sigma$ & Given \\
      & $\ctxl = \ctxl', x: \sigma$ & By inversion \\
      & $\ctxl' \exto \tctx$ & By inversion \\
      & $\ctxl' \exto \ctxr$ & By induction hypothesis \\
      & $\ctxl', x: \sigma \exto \ctxr, x: \sigma$ & By \rul{CE-Var} \\
      & $\ctxl \exto \ctxr, x: \sigma$ & Namely
    \end{longtable}
  \item Case \[\CEEVar\]
    We are given $\ctxl \exto \tctx, \genA$.
    By inversion, we have two subcases.
    \begin{itemize}
    \item SubCase  \[ \inferrule{
            \ctxl \exto \tctx
         \\ \genA \notin \tctx
            }{
            \ctxl, \genA \exto \tctx, \genA
            }\rname{CE-EVar}\]
      \begin{longtable}[l]{lll}
        & $\ctxl \exto \tctx$ & Given\\
        & $\ctxl \exto \ctxr$ & By induction hypothesis \\
        & $\ctxl, \genA \exto \ctxr, \genA$ & By \rul{CE-EVar} \\
      \end{longtable}
    \item SubCase \[\inferrule{
            \ctxl \exto \tctx
         \\ \genA \notin \tctx
            }{
            \ctxl \exto \tctx, \genA
            }\rname{CE-Add}\]
      \begin{longtable}[l]{lll}
        & $\ctxl \exto \tctx$ & Given\\
        & $\ctxl \exto \ctxr$ & By induction hypothesis \\
        & $\ctxl \exto \ctxr, \genA$ & By \rul{CE-Add} \\
      \end{longtable}
    \end{itemize}
  \item Case \[\CESolvedEVar\]
    We are given $\ctxl \exto \tctx, \genA= \tau_1$.
    By inversion, we have three subcases.
    \begin{itemize}
    \item SubCase \[\inferrule{
            \ctxl \exto \tctx
         \\ \genA \notin \tctx
         \\ \applye \tctx {\tau_1} = \applye \tctx {\tau_3}
            }{
            \ctxl, \genA = \tau_3 \exto \tctx, \genA = \tau_1
            }\rname{CE-SolvedEVar} \]
      \begin{longtable}[l]{lll}
        & $\ctxl \exto \ctxr$ & By induction hypothesis \\
        & $\ctxl, \genA = \tau_3 \wc$
        & Given \\
        & $\ctxl \bywf \tau_3$
        & By inversion \\
        & $\tctx \bywf \tau_3$
        & By Corollary ~\ref{lemma:\ContextExtensionPreservesContextWellFormednessName} \\
        & $\applye \ctxr {\tau_3} $ \\
        & $= \applye \ctxr {\applye \tctx {\tau_3}} $
        & By Lemma~\ref{lemma:\SubstitutionExtensionInvarianceName} \\
        & $= \applye \ctxr {\applye \tctx {\tau_1}} $
        & Given \\
        & $= \applye \ctxr {\tau_1} $
        & By Lemma~\ref{lemma:\SubstitutionExtensionInvarianceName} \\
        & $= \applye \ctxr {\tau_2} $
        & Known\\
        & $\ctxl, \genA = \tau_3 \exto \ctxr, \genA = \tau_2$ & By \rul{CE-SolvedEVar} \\
      \end{longtable}
    \item SubCase \[\inferrule{
            \ctxl \exto \tctx
         \\ \genA \notin \tctx
         \\ \tctx \bywf \tau
            }{
            \ctxl, \genA \exto \tctx, \genA = \tau
            }\rname{CE-Solve} \]
          \begin{longtable}[l]{lll}
            & $\ctxl \exto \tctx$ & Given \\
            & $\ctxl \exto \ctxr$ & By induction hypothesis \\
            & $\tctx \bywf \tau$ & Given \\
            & $\tctx \wc$ & By
            Lemma~\ref{lemma:\TypingContextWellFormednessName} \\
            & $\ctxr \wc$ & By
            Lemma~\ref{lemma:\ContextExtensionPreservesContextWellFormednessName}\\
            & $\ctxr \bywf \tau$ & By
            Corollary~\ref{lemma:\ExtensionWeakningWellFormednessName} \\
            & $\ctxl \exto \ctxr, \genA = \tau$ & By \rul{CE-AddSolved} \\
          \end{longtable}
    \end{itemize}
  \item Case \[\CESolve\]
    We are given $\ctxl \exto \tctx, \genA$.
    By induction, we have two subcases.
    \begin{itemize}
    \item SubCase \[\inferrule{
            \ctxl \exto \tctx
         \\ \genA \notin \tctx
            }{
            \ctxl, \genA \exto \tctx, \genA
            }\rname{CE-EVar} \]
      \begin{longtable}[l]{lll}
        & $\ctxl \exto \tctx$ & Given \\
        & $\ctxl \exto \ctxr$ & By induction hypothesis \\
        & $\ctxl, \genA \exto \ctxr, \genA = \tau$ & By \rul{CE-Solve} \\
      \end{longtable}
    \item SubCase \[\inferrule{
            \ctxl \exto \tctx
         \\ \genA \notin \tctx
            }{
            \ctxl \exto \tctx, \genA
            }\rname{CE-Add} \]
      \begin{longtable}[l]{lll}
        & $\ctxl \exto \tctx$ & Given \\
        & $\ctxl \exto \ctxr$ & By induction hypothesis \\
        & $\ctxl \exto \ctxr, \genA = \tau$ & By \rul{CE-AddSolved} \\
      \end{longtable}
    \end{itemize}
  \item Case \[\CEAdd\]
    \begin{longtable}[l]{lll}
      & $\ctxl \exto \tctx$ & Given \\
      & $\ctxl \exto \ctxr$ & By induction hypothesis \\
      & $\ctxl \exto \ctxr, \genA$ & By \rul{CE-Add} \\
    \end{longtable}
  \item Case \[\CEAddSolved\]
    \begin{longtable}[l]{lll}
      & $\ctxl \exto \tctx$ & Given \\
      & $\ctxl \exto \ctxr$ & By induction hypothesis \\
      & $\ctxl \exto \ctxr, \genA = \tau$ & By \rul{CE-Add} \\
    \end{longtable}
\end{itemize}

\qed

\begin{definition}[Softness]
  A Context $\tctx$ is soft if and only if it consists only of $\genA$ and
  $\genA = \tau$ declarations.
\end{definition}

\begin{lemma}[\RightSoftnessName]
  \label{lemma:\RightSoftnessName}
  \RightSoftnessBody
\end{lemma}

\proof

By induction on $\ctxl$, and apply rule \rul{CE-Add} and \rul{CE-AddSolved} as needed.

\qed

\begin{lemma}[\ExtensionOrderName]\leavevmode
  \label{lemma:\ExtensionOrderName}
  \ExtensionOrderBody
\end{lemma}

\proof

\begin{description}
\item [Part 1]
  By induction on the context extension $\tctx_L, y : \sigma, \tctx_R \exto
  \ctxr$.
  \begin{itemize}
    \item Case \[\CEEmpty\]
      Impossible case.
    \item Case \[\CEVar\]
      Depending on whether $x = y$, we have two subcases.
      \begin{itemize}
        \item SubCase $x = y$.
          Therefore $\tctx_R = \ctxr_R = \ctxinit$,
          and $\tctx_L = \tctx, \ctxr_L = \ctxr$.
          All goal follows directly.
        \item SubCase $x \neq y$.
          Then we have
          \[\inferrule{
            \tctx, y:\sigma, \tctx_R' \exto \ctxr
            }{
            \tctx, y:\sigma, \tctx_R', x:\sigma \exto \ctxr, x:\sigma
            }\rname{CE-Var}
          \]
          \begin{longtable}[l]{lll}
            & $\ctxr = \ctxr_L, y:\sigma, \ctxr_R'$ & By induction hypothesis \\
            & $\tctx_L \exto \ctxr_L$ & By induction hypothesis\\
            & $\tctx_R', x: \sigma$ & is not soft\\
            & $\ctxr_R', x: \sigma$ & is not soft
          \end{longtable}
        \end{itemize}
      \item Case \[\inferrule{
            \tctx_L, y: \sigma, \tctx_R' \exto \ctxr
            \\ \genA \notin \ctxr
          }{
            \tctx_L, y:\sigma, \tctx_R', \genA \exto \ctxr, \genA
          }\rname{CE-EVar} \]
        \begin{longtable}[l]{lll}
          & $\ctxr = \ctxr_L, y:\sigma, \ctxr_R'$ & By induction hypothesis \\
          & $\tctx_L \exto \ctxr_L$ & By induction hypothesis\\
          & $\tctx_R'$ is soft  iff $\ctxr_R'$ is soft & By induction
          hypothesis \\
          & $\tctx_R', \genA $ is soft  iff $\ctxr_R', \genA$ is soft &
          Follows directly
        \end{longtable}
      \item Case \[\inferrule{
            \tctx_L, y: \sigma, \tctx_R' \exto \ctxr
         \\ \genA \notin \ctxr
         \\ \applye {\ctxr} {\tau_1} = \applye \ctxr {\tau_2}
            }{
            \tctx_L, y:\sigma, \tctx_R', \genA = \tau_1 \exto \ctxr, \genA = \tau_2
            }\rname{CE-SolvedEVar}\]
        \begin{longtable}[l]{lll}
          & $\ctxr = \ctxr_L, y:\sigma, \ctxr_R'$ & By induction hypothesis \\
          & $\tctx_L \exto \ctxr_L$ & By induction hypothesis\\
          & $\tctx_R'$ is soft  iff $\ctxr_R'$ is soft & By induction
          hypothesis \\
          & $\tctx_R', \genA = \tau$ is soft  iff $\ctxr_R', \genA = \tau$ is soft &
          Follows directly
        \end{longtable}
      \item Case \[\inferrule{
            \tctx_L, y: \sigma, \tctx_R' \exto \ctxr
         \\ \genA \notin \ctxr
         \\ \ctxr \bywf \tau
            }{
            \tctx_L, y: \sigma, \tctx_R', \genA \exto \ctxr, \genA = \tau
            }\rname{CE-Solve} \]
        \begin{longtable}[l]{lll}
          & $\ctxr = \ctxr_L, y:\sigma, \ctxr_R'$ & By induction hypothesis \\
          & $\tctx_L \exto \ctxr_L$ & By induction hypothesis\\
          & $\tctx_R'$ is soft  iff $\ctxr_R'$ is soft & By induction
          hypothesis \\
          & $\tctx_R', \genA $ is soft  iff $\ctxr_R', \genA = \tau$ is soft &
          Follows directly
        \end{longtable}
      \item Case \[\inferrule{
            \tctx_L, y:\sigma, \tctx_R \exto \ctxr
         \\ \genA \notin \ctxr
            }{
            \tctx_L, y:\sigma, \tctx_R \exto \ctxr, \genA
            }\rname{CE-Add}
        \]
        \begin{longtable}[l]{lll}
          & $\ctxr = \ctxr_L, y:\sigma, \ctxr_R'$ & By induction hypothesis \\
          & $\tctx_L \exto \ctxr_L$ & By induction hypothesis\\
          & $\tctx_R$ is soft  iff $\ctxr_R'$ is soft & By induction
          hypothesis \\
          & $\tctx_R $ is soft  iff $\ctxr_R', \genA$ is soft &
          Follows directly
        \end{longtable}
      \item Case \[
          \inferrule{
            \tctx_L, y :\sigma, \tctx_R \exto \ctxr
         \\ \genA \notin \ctxr
         \\ \ctxr \bywf \tau
            }{
            \tctx_L, y:\sigma, \tctx_R \exto \ctxr, \genA = \tau
            }\rname{CE-AddSolved} \]
        \begin{longtable}[l]{lll}
          & $\ctxr = \ctxr_L, y:\sigma, \ctxr_R'$ & By induction hypothesis \\
          & $\tctx_L \exto \ctxr_L$ & By induction hypothesis\\
          & $\tctx_R$ is soft  iff $\ctxr_R'$ is soft & By induction
          hypothesis \\
          & $\tctx_R $ is soft  iff $\ctxr_R', \genA = \tau$ is soft &
          Follows directly
        \end{longtable}
  \end{itemize}
  \item [Part 2] Similar to Part 1.
  \item [Part 3] Similar to Part 1.
\end{description}

\qed

\begin{lemma}[\ExtensionWeakningName]
  \label{lemma:\ExtensionWeakningName}
  \ExtensionWeakningBody
\end{lemma}

\proof

By induction on the typing derivation.

\begin{itemize}
\item Case \[\AAx\]
  Follows directly by \rul{A-Ax}.
\item Case \[\AVar\]
  \begin{longtable}[l]{lll}
    & $x : \sigma \in \tctx$ & Given \\
    & $x : \sigma \in \ctxr$ & By Lemma~\ref{lemma:\ExtensionOrderName} \\
    & $\ctxr \byinf x \infto \applye \ctxr \sigma$ & By \rul{A-Var} \\
    & $\tau_2 = \applye \tctx \sigma$ & Given \\
    & $\applye \ctxr \sigma $ & \\
    & $= \applye \ctxr {\applye \tctx \sigma} $ & By
    Lemma~\ref{lemma:\SubstitutionExtensionInvarianceName}\\
    & $= \applye \ctxr {\tau_2} $ & Substitute above equality
  \end{longtable}
\item Case \[\AEVar\]
  By Lemma~\ref{lemma:\ExtensionOrderName},
  we have either $\genA$ or $\genA = \tau$ in $\ctxr$.
  Then by rule \rul{A-EVar} or \rul{A-SolvedEVar}, we have
  $\ctxr \byinf \genA \infto \star$ directly.
\item Case \[\ASolvedEVar\]
  By Lemma~\ref{lemma:\ExtensionOrderName},
  and rule \rul{A-SolveEVar}, we have
  $\ctxr \byinf \genA \infto \star$ directly.
\item Case \[\ALamAnn\]
  \begin{longtable}[l]{lll}
    & $\ctxr \byinf \sigma_1 \infto \star$ & By induction hypothesis \\
    & $\tctx, x: \sigma_1 \exto \ctxr, x: \sigma_1$ & By \rul{CE-Var} \\
    & $\ctxr, x: \sigma_1 \byinf e \infto \applye {\ctxr, x: \sigma_1}
    {\sigma_2}$ & By induction hypothesis \\
    & $\ctxr, x: \sigma_1 \byinf e \infto \applye \ctxr
    {\sigma_2}$ & By definition of context substitution \\
    & $\ctxr \byinf \blam x {\sigma_1} e \infto \bpi x {\applye \ctxr {\sigma_1}}
    {\applye \ctxr {\sigma_2}}$& By \rul{A-LamAnn} \\
    & $\applye \ctxr {\sigma_1} = \applye \ctxr {\applye \tctx {\sigma_1}}$ & By
    Lemma~\ref{lemma:\SubstitutionExtensionInvarianceName} \\
    & $\ctxr \byinf \blam x {\sigma_1} e \infto \bpi x {\applye \ctxr {\applye
        \tctx {\sigma_1}}}
    {\applye \ctxr {\sigma_2}}$& Substitute above equality \\
    & $\ctxr \byinf \blam x {\sigma_1} e \infto \applye \ctxr {\bpi x {\applye
        \tctx {\sigma_1}} {\sigma_2}}$& Follows directly
  \end{longtable}
\item Case \[\APi\]
  \begin{longtable}[l]{lll}
    & $\ctxr \byinf \sigma_1 \infto \star$ & By induction hypothesis \\
    & $\tctx, x: \sigma_1 \exto \ctxr, x: \sigma_1$ & By \rul{CE-Var} \\
    & $\ctxr, x: \sigma_1 \byinf \sigma_2 \infto \star$ & By induction
    hypothesis \\
    & $\ctxr \byinf \bpi x {\sigma_1} {\sigma_2} \infto \star$ & By \rul{A-Pi}
  \end{longtable}
\item Case \[\AApp\]
  \begin{longtable}[l]{lll}
    & $\ctxr \byinf e_1 \infto \applye \ctxr {\bpi x {\sigma_1} {\sigma_2}}$ &
    By induction hypothesis \\
    & $\ctxr \byinf e_1 \infto {\bpi x {\applye \ctxr {\sigma_1}} {\applye \ctxr
        {\sigma_2}}}$ &
    Follows directly \\
    & $\ctxr \byinf e_2 \infto \applye \ctxr {\sigma_1}$ &
    By induction hypothesis \\
    & $\ctxr \byinf e_1 ~ e_2 \infto
    (\applye \ctxr {\sigma_2}) \subst x {\applye \ctxr {e_1}}$ &
    By \rul{A-App} \\
    & $(\applye \ctxr {\sigma_2}) \subst x {\applye \ctxr {e_1}}$ & \\
    & $ = (\applye \ctxr {\sigma_2}) \subst x {\applye \ctxr {\applye \tctx
        {e_1}}}$ & By Lemma~\ref{lemma:\SubstitutionExtensionInvarianceName}\\
    & $ = \applye \ctxr {\sigma_2 \subst x {\applye \tctx {e_1}}}$ & Property of
    substitution \\
  \end{longtable}
\item Case \[\ACastDn\]
  \begin{longtable}[l]{lll}
    & $\ctxr \byinf e \infto \applye \ctxr {\sigma_1}$& By induction
    hypothesis \\
    & $\applye \ctxr {\sigma_1} \redto \applye \ctxr {\sigma_2}$& By
    Lemma~\ref{lemma:\ContextApplicationOverReductionName} \\
    & $\ctxr \byinf \castdn e \infto \applye \ctxr {\sigma_2}$& By \rul{A-CastDn}
  \end{longtable}

\item Case \[\ACastUp\]
  \begin{longtable}[l]{lll}
    & $\ctxr \byinf e \infto \applye \ctxr {\sigma_2}$& By induction
    hypothesis \\
    & $\applye \ctxr {\sigma_1} $ & \\
    & $ = \applye \ctxr {\applye \tctx {\sigma_1}} $ & By
    Lemma~\ref{lemma:\SubstitutionExtensionInvarianceName} \\
    & $ \redto \applye \ctxr {\sigma_2}$& By
    Lemma~\ref{lemma:\ContextApplicationOverReductionName} \\
    & $\ctxr \byinf \castup e \infto \applye \ctxr {\sigma_1}$& By \rul{A-CastUp}
  \end{longtable}
\end{itemize}

\qed

\begin{corollary}[\ExtensionWeakningWellFormednessName]
  \label{lemma:\ExtensionWeakningWellFormednessName}
  \ExtensionWeakningWellFormednessBody
\end{corollary}

\proof

Follows directly by Lemma~\ref{lemma:\ExtensionWeakningName} since $\applye
\ctxr \star = \star$.

\qed

\begin{lemma}[\ExtensionWeakeningWellScopednessName]
  \label{lemma:\ExtensionWeakeningWellScopednessName}
  \ExtensionWeakeningWellScopednessBody
\end{lemma}

\proof

By a straightforward induction on the context extension.

\qed

\begin{lemma}[\ContextExtensionPreservesContextWellFormednessName]
  \label{lemma:\ContextExtensionPreservesContextWellFormednessName}
  \ContextExtensionPreservesContextWellFormednessBody
\end{lemma}

\proof

By induction on the context extension.

\begin{itemize}
\item Case \[\CEEmpty\]
  Here we have $\tctx = \ctxr$, so our goal is given.
\item Case \[\CEVar\]
  \begin{longtable}[l]{lll}
    & $\tctx \wc$ & Given \\
    & $\tctx \bywf \sigma$ & Given \\
    & $x \notin \tctx$ & By hypothesis \\
    & $\ctxr \wc$& By induction hypothesis \\
    & $\ctxr \bywf \sigma$& By
    Corollary~\ref{lemma:\ExtensionWeakningWellFormednessName} \\
    & $x \notin \ctxr$ & Context extension add no variables\\
    & $\ctxr, x :\sigma \wc$ & By \rul{AC-Var}
  \end{longtable}

\item Case \[\CEEVar\]
  \begin{longtable}[l]{lll}
    & $\tctx \wc$ & Given \\
    & $\ctxr \wc$& By induction hypothesis \\
    & $\genA \notin \ctxr$ & Given \\
    & $\ctxr, \genA \wc$ & By \rul{AC-EVar}
  \end{longtable}

\item Case \[\CESolvedEVar\]
  \begin{longtable}[l]{lll}
    & $\tctx \wc$ & Given \\
    & $\tctx \bywf \tau_1$ & Given \\
    & $\ctxr \wc$& By induction hypothesis \\
    & $\ctxr \bywf \tau_1$
    & By Corollary~\ref{lemma:\ExtensionWeakningWellFormednessName} \\
    & $\ctxr \bywf \applye \ctxr {\tau_1}$
    & By Lemma~\ref{lemma:\ContextApplicationPreservesTypingName} \\
    & $\ctxr \bywf \applye \ctxr {\tau_2}$
    & Known \\
    & $\ctxr \bywf {\tau_2}$
    & By Lemma~\ref{lemma:\ReverseContextApplicationPreservesTypingName} \\
    & $\ctxr, \genA = \tau$ & By \rul{AC-SolvedEVar}
  \end{longtable}
\item Case \[\CESolve\]
  \begin{longtable}[l]{lll}
    & $\tctx \wc$ & Given \\
    & $\ctxr \wc$& By induction hypothesis \\
    & $\ctxr, \genA = \tau$ & By \rul{AC-SolvedEVar}
  \end{longtable}
\item Case \[\CEAdd\]
  \begin{longtable}[l]{lll}
    & $\tctx \wc$ & Given \\
    & $\ctxr \wc$& By induction hypothesis \\
    & $\ctxr, \genA$ & By \rul{AC-EVar}
  \end{longtable}
\item Case \[\CEAddSolved\]
  \begin{longtable}[l]{lll}
    & $\tctx \wc$ & Given  \\
    & $\ctxr \wc$& By induction hypothesis \\
    & $\ctxr, \genA = \tau$ & By \rul{AC-EVar}
  \end{longtable}
\end{itemize}

\qed

Notice since this lemma holds, many preconditions
in previous lemmas are hold automatically.
For example,
Lemma~\ref{lemma:\ExtensionWeakningName}
requests $\ctxr$ is a well-formed context,
which is automatically satisfied given $\tctx$ is well-formed
and this lemma.
Therefore, while applying those kind of lemmas, we will ignore
this kind of ``self-hold'' preconditions.

Furthermore, from now on, we will sometimes implicitly assume
the contexts we mentioned are well-formed.
Since we have already dealt with the dependency between well-formedness
and typing through previous lemmas.

\begin{lemma}[\SolutionAdmissibilityForExtensionName]
  \label{lemma:\SolutionAdmissibilityForExtensionName}
  \SolutionAdmissibilityForExtensionBody
\end{lemma}

\proof

By induction on $\tctx_R$.

\begin{itemize}
\item Case $\ctxinit$.
  \begin{longtable}[l]{lll}
    & $\tctx_L \exto \tctx_L $ & By
    Lemma~\ref{lemma:\ContextExtensionReflexivityName} \\
    & $\tctx_L, \genA \exto \tctx_L, \genA = \tau$ & By \rul{CE-Solve}
  \end{longtable}
\item Case $\tctx_R = \tctx_R', x: \sigma$
  \begin{longtable}[l]{lll}
    & $\tctx_L, \genA, \tctx_R' \exto \tctx_L, \genA = \tau, \tctx_R' $ & By
    induction hypothesis \\
    & $\tctx_L, \genA, \tctx_R', x: \sigma \exto \tctx_L, \genA = \tau,
    \tctx_R', x: \sigma $ & By \rul{CE-Var}
  \end{longtable}
\item Case $\tctx_R = \tctx_R', \genB$
  \begin{longtable}[l]{lll}
    & $\tctx_L, \genA, \tctx_R' \exto \tctx_L, \genA = \tau, \tctx_R' $ & By
    induction hypothesis \\
    & $\tctx_L, \genA, \tctx_R', \genB \exto \tctx_L, \genA = \tau,
    \tctx_R', \genB $ & By \rul{CE-EVar}
  \end{longtable}
\item Case $\tctx_R = \tctx_R', \genB = \sigma$
  \begin{longtable}[l]{lll}
    & $\tctx_L, \genA, \tctx_R' \exto \tctx_L, \genA = \tau, \tctx_R' $ & By
    induction hypothesis \\
    & $\tctx_L, \genA, \tctx_R' \byinf \sigma \infto \star$ & Given\\
    & $\tctx_L, \genA = \tau, \tctx_R' \byinf \sigma \infto \star$ & By
    Lemma~\ref{lemma:\ExtensionWeakningName}\\
    & $\tctx_L, \genA, \tctx_R', \genB = \sigma \exto \tctx_L, \genA = \tau,
    \tctx_R', \genB=\sigma $ & By \rul{CE-SolvedEVar}
  \end{longtable}
\end{itemize}
\qed

\begin{lemma}[\UnsolvedVariableAdditionForExtensionName]
  \label{lemma:\UnsolvedVariableAdditionForExtensionName}
  \UnsolvedVariableAdditionForExtensionBody
\end{lemma}

\proof

By induction on $\tctx_R$.

\begin{itemize}
\item Case $\ctxinit$.
  \begin{longtable}[l]{lll}
    & $\tctx_L \exto \tctx_L $ & By
    Lemma~\ref{lemma:\ContextExtensionReflexivityName} \\
    & $\tctx_L \exto \tctx_L, \genA$ & By \rul{CE-Add}
  \end{longtable}
\item Case $\tctx_R = \tctx_R', x: \sigma$
  \begin{longtable}[l]{lll}
    & $\tctx_L, \tctx_R' \exto \tctx_L, \genA, \tctx_R' $ & By
    induction hypothesis \\
    & $\tctx_L, \tctx_R', x: \sigma \exto \tctx_L, \genA,
    \tctx_R', x: \sigma $ & By \rul{CE-Var}
  \end{longtable}
\item Case $\tctx_R = \tctx_R', \genB$
  \begin{longtable}[l]{lll}
    & $\tctx_L, \tctx_R' \exto \tctx_L, \genA, \tctx_R' $ & By
    induction hypothesis \\
    & $\tctx_L, \tctx_R', \genB \exto \tctx_L, \genA,
    \tctx_R', \genB $ & By \rul{CE-EVar}
  \end{longtable}
\item Case $\tctx_R = \tctx_R', \genB = \sigma$
  \begin{longtable}[l]{lll}
    & $\tctx_L, \tctx_R' \exto \tctx_L, \genA, \tctx_R' $ & By
    induction hypothesis \\
    & $\tctx_L, \tctx_R' \byinf \sigma \infto \star$ & Given\\
    & $\tctx_L, \genA , \tctx_R' \byinf \sigma \infto \star$ & By
    Lemma~\ref{lemma:\ExtensionWeakningName}\\
    & $\tctx_L, \tctx_R', \genB = \sigma \exto \tctx_L, \genA,
    \tctx_R', \genB=\sigma $ & By \rul{CE-SolvedEVar}
  \end{longtable}
\end{itemize}
\qed

\begin{lemma}[\SolvedVariableAdditionForExtensionName]
  \label{lemma:\SolvedVariableAdditionForExtensionName}
  \SolvedVariableAdditionForExtensionBody
\end{lemma}

\proof

\mbox{} % an empty line to make sure long table appear after proof
\begin{longtable}[l]{lll}
  & $\tctx_L, \tctx_R \exto \tctx_L, \genA, \tctx_R $ & By
  Lemma~\ref{lemma:\UnsolvedVariableAdditionForExtensionName} \\
  & $\tctx_L, \genA, \tctx_R \exto \tctx_L, \genA = \tau, \tctx_R $ & By
  Lemma~\ref{lemma:\SolutionAdmissibilityForExtensionName} \\
  & $\tctx_L, \tctx_R \exto \tctx_L, \genA = \tau, \tctx_R$ &
  By Lemma~\ref{lemma:\ContextExtensionTransitivityName}
\end{longtable}

\qed

\begin{lemma}[\ParallelAdmissibilityName]\leavevmode
  \label{lemma:\ParallelAdmissibilityName}
  \ParallelAdmissibilityBody
\end{lemma}

\proof

\begin{description}
  \item [Part 1] By induction on $\ctxr_R$.
    \begin{itemize}
    \item Case $\ctxr_R = \ctxinit$.
      Since $\tctx_R$ has no variable overlapped with $\ctxr_L$,
      we have $\tctx_R = \ctxinit$.
      Therefore, $\tctx_L \exto \ctxr_L$.
      By applying \rul{CE-EVar} we are done.
    \item Case $\ctxr_R = \ctxr_R', \genB$.
      We have $\tctx_L, \tctx_R \exto \ctxr_L, \ctxr_R', \genB$.
      By inversion, we have two subcases.
      \begin{itemize}
      \item SubCase $\tctx_R = \tctx_R', \genB$,
        and $\tctx_L, \tctx_R' \exto \ctxr_L, \ctxr_R'$.
        Notice here we assume $\tctx_R$ is a non-empty context.
        Since if it is empty, then $\genB \in \tctx_L$.
        Then according to Lemma~\ref{lemma:\DeclarationPreservationName},
        we have $\genB \in \ctxr_L$, which should not be right because we
        implicitly assume $\ctxr_L, \ctxr_R$ is well-formed.
        By induction hypothesis,
        we have $\tctx_L, \genA, \tctx_R' \exto \ctxr_L, \genA, \ctxr_R'$.
        Then by \rul{CE-EVar}
        we have $\tctx_L, \genA, \tctx_R \exto \ctxr_L, \genA, \ctxr_R$.
      \item SubCase $\tctx_L, \tctx_R \exto \ctxr_L, \ctxr_R$.
        Then by induction hypothesis we have
        $\tctx_L, \genA, \tctx_R \exto \ctxr_L, \genA, \ctxr_R$.
        By \rul{CE-Add} we are done.
      \end{itemize}
    \item All rest cases are similar as last case.
    \end{itemize}
  \item [Part 2] All rest parts are similar as Part 1.
\end{description}

\qed

\begin{lemma}[\ParallelExtensionSolutionName]\leavevmode
  \label{lemma:\ParallelExtensionSolutionName}
  \ParallelExtensionSolutionBody
\end{lemma}

\proof

By induction on $\ctxr_R$.

Similar to the proof of
Lemma~\ref{lemma:\ParallelAdmissibilityName}.

\qed

\begin{lemma}[\StabilityOfCompleteContextsName]
  \label{lemma:\StabilityOfCompleteContextsName}
  \StabilityOfCompleteContextsBody
\end{lemma}

\proof

By induction on the derivation $\tctx \exto \cctx$.

\begin{itemize}
  \item Case \[\CEEmpty\]
    Holds trivially.
  \item Case \[
      \inferrule{\tctx' \exto \cctx'
      }{
        \tctx', x: \sigma \exto \cctx', x: \sigma
      }\rname{CE-Var}
    \]
    \begin{longtable}[l]{lll}
      & $\applye \cctx \cctx$
      & \\
      & $= \applye {\cctx'} {\cctx'}, x:\sigma $
      & By definition of context application \\
      & $= \applye {\cctx'} {\tctx'}, x:\sigma $
      & By induction hypothesis \\
      & $= \applye {\cctx} {\tctx}$
      & By definition of context application \\
    \end{longtable}
  \item Case \[
      \inferrule{\tctx' \exto \cctx'
        \\ \genA \notin \cctx'
      }{
        \tctx', \genA \exto \cctx', \genA
      }\rname{CE-EVar}
    \]
    Similar as Case \rul{CE-Var}.
  \item Case \[
      \inferrule{\tctx' \exto \cctx'
        \\ \genA \notin \cctx'
        \\ \applye {\cctx'} {\tau_1} = \applye {\cctx'} {\tau_2}
      }{
        \tctx', \genA = \tau_1 \exto \cctx', \genA = \tau_2
      }\rname{CE-SolvedEVar}
    \]
    Similar as Case \rul{CE-Var}.
  \item Case \[
      \inferrule{\tctx' \exto \cctx'
        \\ \genA \notin \cctx'
        \\ \cctx' \bywf \tau
      }{
        \tctx', \genA \exto \cctx', \genA = \tau
      }\rname{CE-Solve}
    \]
    \begin{longtable}[l]{lll}
      & $\applye \cctx \cctx$
      & \\
      & $= \applye {\cctx'} {\cctx'} $
      & By definition of context application \\
      & $= \applye {\cctx'} {\tctx'} $
      & By induction hypothesis \\
      & $= \applye {\cctx} {\tctx}$
      & By definition of context application \\
    \end{longtable}
  \item Case \[
      \inferrule{\tctx' \exto \cctx'
      \\ \genA \notin \cctx'
      }{
        \tctx' \exto \cctx', \genA
      }\rname{CE-Add}
    \]
    Impossible since $\cctx$ is a complete context.
  \item Case \[
      \inferrule{\tctx' \exto \cctx'
      \\ \genA \notin \cctx'
      \\ \cctx' \bywf \tau
      }{
        \tctx' \exto \cctx', \genA = \tau
      }\rname{CE-AddSolved}
    \]
    \begin{longtable}[l]{lll}
      & $\applye \cctx \cctx$
      & \\
      & $= \applye {\cctx'} {\cctx'} $
      & By definition of context application \\
      & $= \applye {\cctx'} {\tctx'} $
      & By induction hypothesis \\
      & $= \applye {\cctx} {\tctx}$
      & By definition of context application \\
    \end{longtable}
\end{itemize}

\qed

\begin{lemma}[\FinishingTypesName]
  \label{lemma:\FinishingTypesName}
  \FinishingTypesBody
\end{lemma}

\proof
\mbox{} % an empty line to make sure long table appear after proof

\begin{longtable}[l]{lll}
  & $\applye {\cctx'} \sigma $
  & \\
  & $= \applye {\cctx'} {\applye \cctx \sigma} $
  & By Lemma~\ref{lemma:\SubstitutionExtensionInvarianceName} \\
  & $\applye \cctx \sigma$ contains no existential variables
  & \\
  & $\applye {\cctx'} {\applye \cctx \sigma} $
  & \\
  & $= \applye {\cctx} \sigma $
  & Follows directly
\end{longtable}

\qed

\begin{lemma}[\FinishingCompletionsName]
  \label{lemma:\FinishingCompletionsName}
  \FinishingCompletionsBody
\end{lemma}

\proof

By a straightforward induction on the derivation $\cctx \exto \cctx'$.

The case analysis is like
Lemma~\ref{lemma:\StabilityOfCompleteContextsName}.

\qed

\begin{lemma}[\ConfluenceOfCompletenessName]
  \label{lemma:\ConfluenceOfCompletenessName}
  \ConfluenceOfCompletenessBody
\end{lemma}

\proof

\mbox{} % an empty line to make sure long table appear after proof

\begin{longtable}[l]{lll}
  & $\applye \cctx {\ctxr_1} $
  & \\
  & $= \applye \cctx \cctx $
  & By Lemma~\ref{lemma:\StabilityOfCompleteContextsName} \\
  & $= \applye \cctx {\ctxr_2} $
  & By Lemma~\ref{lemma:\StabilityOfCompleteContextsName} \\
\end{longtable}

\qed

\subsection{Properties of Type Sanitization}

\begin{lemma}[\TypeSanitizationExtensionName]
  \label{lemma:\TypeSanitizationExtensionName}
  \TypeSanitizationExtensionBody
\end{lemma}

\proof

By induction on type sanitization.

\begin{itemize}
  \item Case \[\IEVarAfter\]
    \begin{longtable}[l]{lll}
      & $\tctx[\genA][\genB] \exto \tctx[\genA_1, \genA][\genB] $ & By
      Lemma~\ref{lemma:\UnsolvedVariableAdditionForExtensionName} \\
      & $\tctx[\genA_1, \genA][\genB] \exto \tctx[\genA_1, \genA][\genB =
      \genA_1] $ & By
      Lemma~\ref{lemma:\SolutionAdmissibilityForExtensionName} \\
      & $\tctx[\genA][\genB] \exto \tctx[\genA_1, \genA][\genB =
      \genA_1] $ & By
      Lemma~\ref{lemma:\ContextExtensionTransitivityName}
    \end{longtable}
  \item Case \[\IEVarBefore\]
    Follows directly by Lemma~\ref{lemma:\ContextExtensionReflexivityName}.
  \item Case \[\IVar\]
    Follows directly by Lemma~\ref{lemma:\ContextExtensionReflexivityName}.
  \item Case \[\IStar\]
    Follows directly by Lemma~\ref{lemma:\ContextExtensionReflexivityName}.
  \item Case \[\IApp\]
    \begin{longtable}[l]{lll}
      & $\tctx \exto \ctxl_1$ & By induction hypothesis \\
      & $\tctx \byinf e_1 ~ e_2 \infto \sigma $ & Given \\
      & $\tctx \byinf e_2 \infto \sigma' $ & By inversion \\
      & $\ctxl_1 \byinf e_2 \infto \applye {\ctxl_1} {\sigma'} $
      & By Lemma~\ref{lemma:\ExtensionWeakningName} \\
      & $\ctxl_1 \byinf \applye {\ctxl_1} {e_2} \infto
      {\sigma'} $
      & By Lemma~\ref{lemma:\ContextApplicationPreservesTypingName} \\
      & $\ctxl_1 \exto \ctxl$ & By induction hypothesis \\
      & $\tctx \exto \ctxl $ & By
      Lemma~\ref{lemma:\ContextExtensionTransitivityName}
    \end{longtable}
  \item Case \[\ILamAnn\]
    \begin{longtable}[l]{lll}
      & $\tctx \exto \ctxl_1$ & By induction hypothesis \\
      & $\tctx \byinf \blam x {\tau_1} {e_1} \infto \sigma $ & Given \\
      & $\tctx, {x : \tau_1} \byinf {e_1} \infto \sigma' $ & By inversion \\
      & $\tctx, {x : \tau_1} \exto \ctxl_1, x : \tau_1 $ & By \rul{CE-Var} \\
      & $\ctxl_1, {x : \tau_1} \byinf {e_1} \infto \applye {\ctxl_1, x:
        \tau_1} {\sigma'} $
      & By Lemma~\ref{lemma:\ExtensionWeakningName} \\
      & $\ctxl_1, {x : \tau_1} \byinf \applye {\ctxl_1, x: \tau_1} {e_1} \infto
      {\applye {\ctxl_1, x: \tau_1} {\sigma'}} $
      & By Lemma~\ref{lemma:\ContextApplicationPreservesTypingName} \\
      & $\ctxl_1, {x : \tau_1} \byinf \applye {\ctxl_1} {e_1} \infto
      {\applye {\ctxl_1} {\sigma'}} $
      & By definition of context application \\
      & $\ctxl_1, x: \tau_1 \exto \ctxl, x: \tau_1$ & By induction hypothesis \\
      & $\ctxl_1 \exto \ctxl$ & By inversion \\
      & $\tctx \exto \ctxl $ & By
      Lemma~\ref{lemma:\ContextExtensionTransitivityName}
    \end{longtable}
  \item Case \[\IPi\]
    Similar as Case \rul{I-LamAnn}.
  \item Case \[\ICastDn\]
    Follows directly from induction hypothesis.
  \item Case \[\ICastUp\]
    Follows directly from induction hypothesis.
\end{itemize}

\qed

\begin{lemma}[\TypeSanitizationEquivalenceName]
  \label{lemma:\TypeSanitizationEquivalenceName}
  \TypeSanitizationEquivalenceBody
\end{lemma}

\proof

By induction on type sanitization.

\begin{itemize}
  \item Case \[\IEVarAfter\]
    \begin{longtable}[l]{lll}
      & $\applye {\tctx[\genA_1, \genA][\genB = \genA_1]} {\genB} = \genA_1$ &
      By definition of context substitution \\
      & $\applye {\tctx[\genA_1, \genA][\genB = \genA_1]} {\genA_1} = \genA_1$ &
      By definition of context substitution
    \end{longtable}
  \item Case \[\IEVarBefore\]
    Follows trivially.
  \item Case \[\IVar\]
    Follows trivially.
  \item Case \[\IStar\]
    Follows trivially.
  \item Case \[\IApp\]
    \begin{longtable}[l]{lll}
      & $\applye {\ctxl_1} {e_1} = \applye {\ctxl_1} {e_3}$
      & By induction hypothesis \\
      & $\tctx \byinf e_1 ~ e_2 \infto \sigma $ & Given \\
      & $\tctx \byinf e_1 \infto \sigma' $ & By inversion \\
      & $\tctx \byinf e_2 \infto \sigma'' $ & By inversion \\
      & $\tctx \exto \ctxl_1$
      & By Lemma~\ref{lemma:\TypeSanitizationExtensionName} \\
      & $\ctxl_1 \byinf e_2 \infto \applye {\ctxl_1} {\sigma''} $
      & By Lemma~\ref{lemma:\ExtensionWeakningName} \\
      & $\ctxl_1 \byinf \applye {\ctxl_1} {e_2} \infto
      {\applye {\ctxl_1} {\sigma''}} $
      & By Lemma~\ref{lemma:\ContextApplicationPreservesTypingName} \\
      & $\ctxl_1 \exto \ctxl$
      & By Lemma~\ref{lemma:\TypeSanitizationExtensionName} \\
      & $\applye {\ctxl} {\applye {\ctxl_1} {e_2}} = \applye {\ctxl} {e_4}$
      & By induction hypothesis \\
      & $\applye {\ctxl} {e_2} = \applye {\ctxl} {e_4}$
      & By Lemma~\ref{lemma:\SubstitutionExtensionInvarianceName} \\
      & $\applye \ctxl {e_3} $& \\
      & $= \applye \ctxl {\applye {\ctxl_1} {e_3}} $
      & By Lemma~\ref{lemma:\SubstitutionExtensionInvarianceName} \\
      & $= \applye \ctxl {\applye {\ctxl_1} {e_1}} $
      & By above equalities \\
      & $= \applye \ctxl {e_1} $
      & By Lemma~\ref{lemma:\SubstitutionExtensionInvarianceName} \\
    \end{longtable}
  \item Case \[\ILamAnn\]
    \begin{longtable}[l]{lll}
      & $\tctx \byinf \blam x {\tau_1} {e_1} \infto \sigma$
      & Given \\
      & $\tctx \byinf \tau_1 \infto \star$
      & By inversion\\
      & $\tctx \exto \ctxl_1$
      & By Lemma~\ref{lemma:\TypeSanitizationExtensionName} \\
      & $\applye {\ctxl_1} {\tau_1} = \applye {\ctxl_1} {\tau_2}$
      & By induction hypothesis \\
      & $\tctx, x: \tau_1 \exto \ctxl_1, x : \tau_1$
      & By \rul{CE-Var} \\
      & $\tctx, x:\tau_1 \byinf e_1 \infto \sigma'$
      & By inversion\\
      & $\ctxl_1, x: \tau_1 \byinf e_1 \infto
      \applye {\ctxl_1, x: \tau_1} {\sigma'}$
      & By Lemma~\ref{lemma:\ExtensionWeakningName}\\
      & $\ctxl_1, x: \tau_1 \byinf \applye {\ctxl_1, x: \tau_1} {e_1} \infto
      \applye {\ctxl_1, x: \tau_1} {\sigma'}$
      & By Lemma~\ref{lemma:\ContextApplicationPreservesTypingName}\\
      & $\ctxl_1, x: \tau_1 \byinf \applye {\ctxl_1} {e_1} \infto
      \applye {\ctxl_1} {\sigma'}$
      & By definition of context application \\
      & $\ctxl_1, x: \tau_1 \exto \ctxl, x:\tau_1$
      & By Lemma~\ref{lemma:\TypeSanitizationExtensionName} \\
      & $\ctxl_1 \exto \ctxl$
      & By inversion \\
      & $\applye {\ctxl, x: \tau_1} {\applye {\ctxl_1} {e_1}}
      = \applye {\ctxl, x: \tau_1} {e_2}$
      & By induction hypothesis \\
      & $\applye {\ctxl} {\applye {\ctxl_1} {e_1}}
      = \applye {\ctxl} {e_2}$
      & By definition of context application \\
      & $\applye {\ctxl} {e_1}
      = \applye {\ctxl} {e_2}$
      & By Lemma~\ref{lemma:\SubstitutionExtensionInvarianceName} \\
      & $\applye {\ctxl} {\tau_1} = \applye {\ctxl} {\tau_2}$
      & Similarly
    \end{longtable}
  \item Case \[\IPi\]
    Similar as Case \rul{I-LamAnn}.
  \item Case \[\ICastDn\]
    Follows directly from induction hypothesis.
  \item Case \[\ICastUp\]
    Follows directly from induction hypothesis.
\end{itemize}

\qed

\begin{lemma}[\TypeSanitizationWellFormednessName]
  \label{lemma:\TypeSanitizationWellFormednessName}
  \TypeSanitizationWellFormednessBody
\end{lemma}

\proof

By induction on the type sanitization.

\begin{itemize}
  \item Case \[\IEVarAfter\]
    \begin{longtable}[l]{lll}
    & $\tau_1 = \genB$, $\sigma = \star$ & By inversion  \\
    & $\tctx[\genA_1, \genA][\genB =\genA_1] \byinf \genA_1 \infto \star$
    & Follows directly by \rul{A-EVar}.
    \end{longtable}
  \item Case \[\IEVarBefore\]
    Holds trivially.
  \item Case \[\IStar\]
    Holds trivially.
  \item Case \[\IApp\]
    \begin{longtable}[l]{lll}
      & $\tctx \byinf e_1 \infto \bpi x {\sigma_1} {\sigma_2}$ & By inversion \\
      & $\tctx \byinf e_2 \infto \sigma_1$ & By inversion \\
      & $\sigma = \sigma_2 \subst x {\applye \tctx {e_2}}$ & By inversion \\
      & $\ctxl_1 \byinf e_3 \infto \applye {\ctxl_1} {\bpi x {\sigma_1} {\sigma_2}}$
      & By induction hypothesis \\
      & $\tctx \exto \ctxl_1$
      & By Lemma~\ref{lemma:\TypeSanitizationExtensionName} \\
      & $\ctxl_1 \byinf e_2 \infto \applye {\ctxl_1} {\sigma_1}$
      & By Lemma~\ref{lemma:\ExtensionWeakningName} \\
      & $\ctxl_1 \byinf \applye {\ctxl_1} {e_2} \infto
      {\applye {\ctxl_1} {\sigma_1}}$
      & By Lemma~\ref{lemma:\ContextApplicationPreservesTypingName} \\
      & $\ctxl \byinf e_4 \infto \applye {\ctxl} {\applye {\ctxl_1} {\sigma_1}}$
      & By induction hypothesis \\
      & $\ctxl_1 \exto \ctxl$
      & By Lemma~\ref{lemma:\TypeSanitizationExtensionName} \\
      & $\ctxl \byinf e_4 \infto \applye {\ctxl} {\sigma_1}$
      & By Lemma~\ref{lemma:\SubstitutionExtensionInvarianceName} \\
      & $\ctxl \byinf e_3 \infto \applye \ctxl {\applye {\ctxl_1} {\bpi x {\sigma_1} {\sigma_2}}}$
      & By Lemma~\ref{lemma:\ExtensionWeakningName} \\
      & $\ctxl \byinf e_3 \infto \applye \ctxl {\bpi x {\sigma_1} {\sigma_2}}$
      & By Lemma~\ref{lemma:\SubstitutionExtensionInvarianceName} \\
      & $\ctxl \byinf e_3 ~ e_4 \infto (\applye \ctxl {\sigma_2}) \subst x
      {\applye \ctxl {e_4}} $
      & By \rul{A-App} \\
      & $\applye \ctxl {\applye {\ctxl_1} {e_2}} = \applye \ctxl {e_4} $
      & By Lemma~\ref{lemma:\TypeSanitizationEquivalenceName} \\
      & $\applye \ctxl {e_2} = \applye \ctxl {e_4} $
      & By Lemma~\ref{lemma:\SubstitutionExtensionInvarianceName} \\
      & $\ctxl \byinf e_3 ~ e_4 \infto (\applye \ctxl {\sigma_2}) \subst x
      {\applye \ctxl {e_2}} $
      & By substituting above equality \\
      & $\tctx \exto \ctxl$
      & By Lemma~\ref{lemma:\ContextExtensionTransitivityName} \\
      & $\applye \ctxl {e_2} = \applye \ctxl {\applye {\tctx} {e_2}}$
      & By Lemma~\ref{lemma:\SubstitutionExtensionInvarianceName} \\
      & $\ctxl \byinf e_3 ~ e_4 \infto (\applye \ctxl {\sigma_2}) \subst x
      {\applye \ctxl {\applye \tctx {e_2}}} $
      & By substituting above equality \\
      & $\ctxl \byinf e_3 ~ e_4 \infto \applye \ctxl
      {\sigma_2 \subst x {\applye \tctx {e_2}}} $
      & By property of substitution
    \end{longtable}
  \item Case \[\ILamAnn\]
    \begin{longtable}[l]{lll}
      & $\tctx \byinf \tau_1 \infto \star $ & By inversion \\
      & $\tctx, x: \tau_1 \byinf e_1 \infto \sigma_1$ & By inversion \\
      & $\sigma = \bpi x {\applye \tctx {\tau_1}} {\sigma_1}$ & By inversion \\
      & $\ctxl_1 \byinf \tau_2 \infto \star$
      & By induction hypothesis \\
      & $\tctx \exto \ctxl_1$
      & By Lemma~\ref{lemma:\TypeSanitizationExtensionName} \\
      & $\tctx, x: \tau_1 \exto \ctxl_1, x:\tau_1$
      & By \rul{CE-Var} \\
      & $\ctxl_1, x: \tau_1 \byinf e_1 \infto \applye {\ctxl_1, x: \tau_1} {\sigma_1}$
      & By Lemma~\ref{lemma:\ExtensionWeakningName} \\
      & $\ctxl_1, x: \tau_1 \byinf \applye {\ctxl_1, x: \tau_1} {e_1}
      \infto \applye {\ctxl_1, x : \tau_1} {\sigma_1}$
      & By Lemma~\ref{lemma:\ContextApplicationPreservesTypingName} \\
      & $\ctxl_1, x: \tau_1 \byinf \applye {\ctxl_1} {e_1}
      \infto \applye {\ctxl_1} {\sigma_1}$
      & By definition on context application \\
      & $\ctxl, x:\tau_1 \byinf e_2 \infto \applye {\ctxl, x :\tau_1} {\applye {\ctxl_1} {\sigma_1}}$
      & By induction hypothesis \\
      & $\ctxl, x:\tau_1 \byinf e_2 \infto \applye {\ctxl} {\applye {\ctxl_1} {\sigma_1}}$
      & By definition of context substitution \\
      & $\ctxl_1, x:\tau_1 \exto \ctxl, x : \tau_1$
      & By Lemma~\ref{lemma:\TypeSanitizationExtensionName} \\
      & $\ctxl_1 \exto \ctxl$
      & By inversion \\
      & $\ctxl, x: \tau_1 \byinf e_2 \infto \applye {\ctxl} {\sigma_1}$
      & By Lemma~\ref{lemma:\SubstitutionExtensionInvarianceName} \\
      & $\ctxl \byinf \tau_2 \infto \star$
      & By Lemma~\ref{lemma:\SubstitutionExtensionInvarianceName} \\
      & $\tctx \exto \ctxl$
      & By Lemma~\ref{lemma:\ContextExtensionTransitivityName} \\
      & $\ctxl \byinf \tau_1 \infto \star$
      & By Lemma~\ref{lemma:\ExtensionWeakningName} \\
      & $\ctxl \byinf \applye {\ctxl} {\tau_1} \infto \star$
      & By Lemma~\ref{lemma:\ContextApplicationPreservesTypingName} \\
      & $\ctxl, x: \applye \ctxl {\tau_1} \byinf e_2 \infto \applye {\ctxl} {\sigma_1}$
      & By Lemma~\ref{lemma:\ContextApplicationInContextName} \\
      & $\applye {\ctxl_1} {\tau_1} = \applye {\ctxl_1} {\tau_2} $
      & By Lemma~\ref{lemma:\TypeSanitizationEquivalenceName} \\
      & $\applye {\ctxl} {\tau_1} = \applye {\ctxl} {\tau_2} $
      & By Lemma~\ref{lemma:\SubstitutionExtensionInvarianceName} \\
      & $\ctxl, x: \applye \ctxl {\tau_2} \byinf e_2 \infto \applye {\ctxl} {\sigma_1}$
      & By substituting above equality \\
      & $\ctxl, x: {\tau_2} \byinf e_2 \infto \applye {\ctxl} {\sigma_1}$
      & By Lemma~\ref{lemma:\ReverseContextApplicationInContextName}\\
      & $\ctxl \byinf \blam x {\tau_2} {e_2} \infto \bpi x {\applye \ctxl
        {\tau_2}} {\applye \ctxl {\sigma_1}} $
      & By \rul{A-LamAnn} \\
      & $\bpi x {\applye \ctxl {\tau_2}} {\applye \ctxl {\sigma_1}}$ & \\
      & $= \bpi x {\applye \ctxl {\applye {\tctx} {\tau_2}}} {\applye \ctxl
        {\sigma_1}}$
      & By Lemma~\ref{lemma:\SubstitutionExtensionInvarianceName} \\
      & $= \applye \ctxl {\bpi x {\applye {\tctx} {\tau_2}} {\sigma_1}}$
      & By property of substitution
    \end{longtable}
  \item Case \[\IPi\]
    Similar as Case \rul{I-LamAnn}.
  \item Case \[\ICastDn\]
    \begin{longtable}[l]{lll}
      & $\tctx \byinf e_1 \infto \sigma_1 $ & By inversion \\
      & $\sigma_1 \redto \sigma$ & By inversion \\
      & $\ctxl \byinf e_2 \infto \applye \ctxl {\sigma_1} $
      & By induction hypothesis \\
      & $\applye \ctxl {\sigma_1} \redto \applye \ctxl {\sigma}$
      & By Lemma~\ref{lemma:\ContextApplicationOverReductionName} \\
      & $\ctxl \byinf \castdn e_2 \infto \applye \ctxl \sigma$
      & By \rul{A-CastDn}
    \end{longtable}
  \item Case \[\ICastUp\]
    Similar as Case \rul{I-CastDn}
\end{itemize}

\qed

\begin{lemma}[\TypeSanitizationTailUnchangedName]
  \label{lemma:\TypeSanitizationTailUnchangedName}
  \TypeSanitizationTailUnchangedBody
\end{lemma}

\proof

By a straightforward induction on the type sanitization derivation.

\qed

We use the notation $\tctx \bywf \tau_1 = \tau_2$ to mean that
$\tctx \byinf \tau_1 \infto \sigma$, $\tctx \byinf \tau_2 \infto \sigma$,
and $\tau_1 = \tau_2$.

\begin{lemma}[\TypeSanitizationCompletenessName]\leavevmode
  \label{lemma:\TypeSanitizationCompletenessName}
  \TypeSanitizationCompletenessBody
\end{lemma}

By induction on the type $\tau$.

We then case analyze the shape of $\sigma_1$.

\begin{itemize}
  \item $\sigma_1 = \genB$
      \begin{longtable}[l]{lll}
        & $\genA \notin FV(\sigma_1)$
        & Given \\
        & $\genA \neq \genB$
        & Follows directly \\
        & $\applye \cctx \genB = \tau$
        & Given \\
        & $\cctx \byinf \genB \infto \star$
        & By \rul{A-SolvedEVar} \\
        & $\cctx \byinf \tau \infto \star$
        & By Lemma~\ref{lemma:\ContextApplicationPreservesTypingName} \\
        & $\applye \tctx \genB = \genB$
        & Given \\
        & $\genB \in unsolved(\tctx)$
        & Follows directly \\
      \end{longtable}
      According to whether $\genB$ is on the left of $\genA$ or right, we have
      two cases.
      Moreover, $\genB$ cannot be in $\tctx_0$ since it only contains variables.
      \begin{itemize}
      \item SubCase $\tctx = \tctx_1, \genA, \tctx_3, \genB, \tctx_4, \tctx_0 $.
        \begin{longtable}[l]{lll}
          & $\tctx_1, \genA, \tctx_3, \genB, \tctx_4, \tctx_0 \bysa \genB \sa \genA_1
          \toctxo_1, \genA_1, \genA, \tctx_3, \genB = \genA_1, \tctx_4, \tctx_0
          $
          & By \rul{I-EVarAfter} \\
          & $\tctx_1, \genA, \tctx_3, \genB, \tctx_4, \tctx_0 \exto \cctx $
          & Given \\
          & $\cctx = \cctx_1, \genA = \tau', \cctx_2, \genB = \tau'', \cctx_3,
          \tctx_0, \cctx_4 $
          & By Lemma~\ref{lemma:\ExtensionOrderName}
          and Lemma~\ref{lemma:\DeclarationOrderPreservationName}\\
          & $\tctx_1 \exto \cctx_1$
          & As above \\
          & $\applye \cctx {\genB} = \applye {\cctx_1, \genA = \tau', \cctx_2} \genB$
          & By definition of context application \\
          & $\cctx_1, \genA = \tau', \cctx_2 \byinf \genB \infto \star$
          & By \rul{A-SolvedEVar} \\
          & $\cctx_1, \genA = \tau', \cctx_2 \byinf \tau \infto \star$
          & By \ref{lemma:\ContextApplicationPreservesTypingName} \\
          & $\tau$ has no variable in $\tctx_0$
          & Follows directly \\
          & $\tctx_1, \tctx_0 \bywt \tau$
          & Given \\
          & $\tctx_1 \bywt \tau$
          & Follows directly \\
          & $\cctx_1 \bywt \tau$
          & By Lemma~\ref{lemma:\ExtensionWeakeningWellScopednessName} \\
          & $\cctx_1, \genA = \tau', \cctx_2 \byinf \tau \infto \star$
          & Known \\
          & $\cctx_1 \byinf \tau \infto \star$
          & By Lemma~\ref{lemma:\TypingStrengtheningName} \\
          & $\tctx_1 \bywt \tau$
          & Known \\
          & $\tctx_1 \byinf \tau \infto \star$
          & By repeating Lemma~\ref{lemma:\TypingStrengtheningName} \\
          & $\cctx' =
          \cctx_1, \genA_1 = \tau, \genA = \tau', \cctx_2, \genB = \tau'',
          \cctx_3, \tctx_0, \cctx_4
          $
          & Choose \\
          & $\tctx_1, \genA_1 = \tau, \genA, \tctx_3, \genB, \tctx_4, \tctx_0 \exto
          \cctx'$
          & By Lemma~\ref{lemma:\ParallelAdmissibilityName} \\
          & $\applye {\cctx'} \genB = \tau$
          & Known \\
          & $\tctx_1, \genA_1 = \tau, \genA, \tctx_3, \genB = \genA_1, \tctx_4, \tctx_0 \exto
          \cctx'$
          & By Lemma~\ref{lemma:\ParallelExtensionSolutionName} \\
          & $\tctx_1, \genA_1, \genA , \tctx_3, \genB = \genA_1,
          \tctx_4, \tctx_0
          \exto
          \tctx_1, \genA_1 = \tau, \genA, \tctx_3, \genB =
          \genA_1, \tctx_4, \tctx_0 $
          &  By Lemma~\ref{lemma:\SolutionAdmissibilityForExtensionName} \\
          & $\tctx_1, \genA_1, \genA, \tctx_3, \genB = \genA_1,
          \tctx_4
          \exto
          \cctx'
          $
          & By Lemma~\ref{lemma:\ContextExtensionTransitivityName} \\
          & $\cctx \exto \cctx'$
          & By Lemma~\ref{lemma:\SolvedVariableAdditionForExtensionName} \\
        \end{longtable}
      \item SubCase $\tctx = \tctx_1, \genB, \tctx_3, \genA, \tctx_2, \tctx_0 $.
        \begin{longtable}[l]{lll}
          & $\tctx_1, \genB, \tctx_3, \genA, \tctx_2, \tctx_0 \bysa \genB \sa \genB
          \toctxo_1, \genB, \tctx_3, \genA, \tctx_2, \tctx_0
          $
          & By \rul{I-EVarBefore} \\
          & $\tctx \exto \cctx $
          & Given \\
          & $\cctx' = \cctx$
          & Choose \\
          & $\cctx \exto \cctx'$
          & By Lemma~\ref{lemma:\ContextExtensionReflexivityName} \\
        \end{longtable}
      \end{itemize}
    \item Case $\sigma_1 = x$.
      \begin{longtable}[l]{lll}
        & $\applye \cctx x = x$
        & By definition of context application \\
        & $\tau = x$
        & Given \\
        & $\tctx_1, \tctx_0 \bywt x$
        & Given \\
        & $\tctx \bysa x \sa x \toctxo $
        & By \rul{I-Var} \\
        & $\cctx' = \cctx$
        & Choose \\
        & $\cctx \exto \cctx'$
        & By Lemma~\ref{lemma:\ContextExtensionReflexivityName} \\
      \end{longtable}
    \item Case $\sigma_1 = \star$.
      \begin{longtable}[l]{lll}
        & $\applye \cctx \star = \star$
        & By definition of context application \\
        & $\tau = \star$
        & Given \\
        & $\tctx_1, \tctx_0 \bywt \star$
        & Given \\
        & $\tctx \bysa \star \sa \star \toctxo $
        & By \rul{I-Star} \\
        & $\cctx' = \cctx$
        & Choose \\
        & $\cctx \exto \cctx'$
        & By Lemma~\ref{lemma:\ContextExtensionReflexivityName} \\
      \end{longtable}
    \item Case $\sigma_1 = e_1 ~ e_2$.
      \begin{longtable}[l]{lll}
        & $\applye \cctx {e_1 ~ e_2} = {\applye \cctx {e_1}} ~ {\applye \cctx {e_2}}$
        & By definition of context application \\
        & $\tau = \tau_1 ~ \tau_2$
        & By inversion \\
        & $ \applye {\cctx} \tctx \bywf \tau_1 = {\applye \cctx {e_1}}$
        & As above \\
        & $ \applye {\cctx} \tctx \bywf \tau_2 = {\applye \cctx {e_2}}$
        & As above \\
        & $\tctx \bysa e_1 \sa e_3 \toctx $
        & By induction hypothesis \\
        & $\ctxl = \ctxl_1, \genA, \ctxl_2, \tctx_0 $
        & As above \\
        & $\ctxl_1, \tctx_0 \bywt e_3$
        & As above \\
        & $\ctxl \exto \cctx_1$
        & As above \\
        & $\cctx \exto \cctx_1$
        & As above \\
        & $\tctx \exto \ctxl$
        & By Lemma~\ref{lemma:\TypeSanitizationExtensionName} \\
        & $\tctx \exto \cctx_1$
        & By Lemma~\ref{lemma:\ContextExtensionTransitivityName} \\
        & $\applye {\cctx} \cctx \bywf \tau_2 = {\applye \cctx {e_2}}$
        & By Lemma~\ref{lemma:\StabilityOfCompleteContextsName} \\
        & $\applye {\cctx_1} {\cctx_1} \bywf \tau_2 = {\applye \cctx {e_2}}$
        & By Lemma~\ref{lemma:\FinishingCompletionsName} \\
        & $\applye {\cctx_1} {\cctx_1} \bywf \tau_2 = {\applye {\cctx_1} {e_2}}$
        & By Lemma~\ref{lemma:\FinishingTypesName} \\
        & $\applye {\cctx_1} {\ctxl} \bywf \tau_2 = {\applye {\cctx_1} {e_2}}$
        & By Lemma~\ref{lemma:\StabilityOfCompleteContextsName} \\
        & $\applye {\cctx_1} {\ctxl} \bywf \tau_2 =
        {\applye {\cctx_1} {\applye \ctxl {e_2}}}$
        & By Lemma~\ref{lemma:\SubstitutionExtensionInvarianceName} \\
        & $\ctxl \bysa \applye \ctxl {e_2} \sa e_4 \toctxr $
        & By induction hypothesis \\
        & $\ctxr = \ctxr_1, \genA, \ctxr_2, \tctx_0$
        & As above \\
        & $\ctxr \exto \cctx_2$
        & As above \\
        & $\cctx_1 \exto \cctx_2$
        & As above \\
        & $\ctxr_1, \tctx_0 \bywt e_4$
        & As above \\
        & $\ctxl_1 \exto \ctxr_1$
        & By Lemma~\ref{lemma:\ExtensionOrderName} \\
        & $\ctxl_1, \tctx_0 \exto \ctxr_1, \tctx_0$
        & By repeating \rul{CE-Var} \\
        & $\ctxr_1, \tctx_0 \bywt e_3$
        & By Lemma~\ref{lemma:\ExtensionWeakeningWellScopednessName} \\
        & $\ctxr_1, \tctx_0 \bywt e_3 ~ e_4$
        & Follows directly \\
        & $\tctx \bysa e_1~e_2 \sa e_3~e_4 \toctxr$
        & By \rul{I-App} \\
        & $\cctx' = \cctx_2$
        & Choose \\
        & $\cctx \exto \cctx_2$
        & By Lemma~\ref{lemma:\ContextExtensionTransitivityName}
      \end{longtable}
    \item Case $\sigma_1 = \blam x \sigma e$.
      \begin{longtable}[l]{lll}
        & $\applye \cctx {\blam x \sigma e} = \blam x {\applye \cctx {\sigma}} {\applye \cctx {e}}$
        & By definition of context application \\
        & $\tau = \blam x {\tau_1} {\tau_2}$
        & By inversion \\
        & $ \applye {\cctx} \tctx \bywf \tau_1 = {\applye \cctx {\sigma}}$
        & As above \\
        & $ \applye {\cctx} \tctx, x: \tau_1 \bywf \tau_2 = {\applye \cctx {e}}$
        & As above \\
        & $\tctx \bysa \sigma \sa \sigma' \toctx $
        & By induction hypothesis \\
        & $\ctxl = \ctxl_1, \genA, \ctxl_2, \tctx $
        & As above \\
        & $\ctxl_1, \tctx_0 \bywt \sigma'$
        & As above \\
        & $\ctxl \exto \cctx_1$
        & As above \\
        & $\cctx \exto \cctx_1$
        & As above \\
        & $\tctx \exto \ctxl$
        & By Lemma~\ref{lemma:\TypeSanitizationExtensionName} \\
        & $\tctx \exto \cctx_1$
        & By Lemma~\ref{lemma:\ContextExtensionTransitivityName} \\
        & $\tctx, x: \tau_1 \exto \cctx, x : \tau_1 $
        & By rul{CE-Var} \\
        & $\cctx, x: \tau_1 \exto \cctx_1, x : \tau_1 $
        & By rul{CE-Var} \\
        & $\ctxl, x: \tau_1 \exto \cctx_1, x: \tau_1$
        & As above \\
        & $ \applye {\cctx} \tctx, x: \tau_1 \bywf \tau_2 = {\applye \cctx {e}}$
        & Known \\
        & $ \applye {\cctx, x : \tau_1} {\tctx, x: \tau_1}
        \bywf \tau_2 = {\applye {\cctx, x : \tau_1} {e}}$
        & By definition of context application \\
        & $ \applye {\cctx_1, x : \tau_1} {\tctx, x: \tau_1}
        \bywf \tau_2 = {\applye {\cctx, x : \tau_1} {e}}$
        & By Lemma~\ref{lemma:\FinishingCompletionsName} \\
        & $ \applye {\cctx_1, x : \tau_1} {\ctxl, x: \tau_1}
        \bywf \tau_2 = {\applye {\cctx, x : \tau_1} {e}}$
        & By Lemma~\ref{lemma:\ConfluenceOfCompletenessName} \\
        & $ \applye {\cctx_1, x : \tau_1} {\ctxl, x: \tau_1}
        \bywf \tau_2 = {\applye {\cctx, x : \tau_1} {\applye {\ctxl, x: \tau_1}{e}}}$
        & By Lemma~\ref{lemma:\SubstitutionExtensionInvarianceName} \\
        & $ \applye {\cctx_1, x : \tau_1} {\ctxl, x: \tau_1}
        \bywf \tau_2 = {\applye {\cctx, x : \tau_1} {\applye {\ctxl}{e}}}$
        & By definition of context application \\
        & $\ctxl, x: \tau_1 \bysa \applye \ctxl e \sa e' \toctxr, x: \tau_1$
        & By induction hypothesis \\
        & $\ctxr = \ctxr_1, \genA, \ctxr_2, \tctx_0$
        & As above \\
        & $\ctxr_1, \tctx_0, x : \tau_1 \bywt e' $
        & As above \\
        & $\ctxr, x : \tau_1 \exto \cctx_2$
        & As above \\
        & $\cctx, x :\tau_1 \exto \cctx_2$
        & As above \\
        & $\cctx_2 = \cctx_3, x: \tau_1, \cctx_4$
        & By Lemma~\ref{lemma:\ExtensionOrderName} \\
        & $\cctx \exto \cctx_3$
        & As above \\
        & $\cctx' = \cctx_3$
        & Choose \\
        & $\tctx \bysa \blam x \sigma e \sa \blam x {\sigma'} {e'} \toctxr$
        & By \rul{I-App} \\
        & $\ctxl, x: \tau_1 \exto \ctxr, x: \tau_1$
        & By Lemma~\ref{lemma:\TypeSanitizationExtensionName} \\
        & $\ctxl \exto \ctxr$
        & By inversion \\
        & $\ctxl_1 \exto \ctxr_1$
        & By Lemma~\ref{lemma:\ExtensionOrderName} \\
        & $\ctxl_1, \tctx_0 \exto \ctxr_1, \tctx_0$
        & By repeating \rul{CE-Var} \\
        & $\ctxr_1, \tctx_0 \bywt \sigma'$
        & By Lemma~\ref{lemma:\ExtensionWeakeningWellScopednessName} \\
        & $\ctxr_1, \tctx_0 \bywt \blam x {\sigma'} {e'}$
        & By \rul{WS-LamAnn}
      \end{longtable}
    \item Case $\sigma_1 = \bpi x {\sigma_2} {\sigma_3}$.
      Similar as last case.
    \item Case $\sigma_1 = \castdn e_1$.
      \begin{longtable}[l]{lll}
        & $\applye \cctx {\castdn e_1} = \castdn {\applye \cctx {e_1}}$
        & By definition of context application \\
        & $\tau = \castdn {e_2}$
        & By inversion \\
        & $ \applye {\cctx} \tctx \bywf e_1 = {\applye \cctx {e_2}}$
        & As above \\
        & $\tctx \bysa e_2 \sa e_1' \toctxr $
        & By induction hypothesis \\
        & $\tctx \bysa \castdn e_2 \sa \castdn e_1' \toctxr $
        & By \rul{I-CastDn} \\
        & All rest follows directly from induction hypothesis \\
      \end{longtable}
    \item Case $\sigma_1 = \castdn e_1$.
      Similar as last case.
\end{itemize}

\qed

\begin{corollary}[\TypeSanitizationCompletenessPrettyName]
  \label{lemma:\TypeSanitizationCompletenessPrettyName}
  \TypeSanitizationCompletenessPrettyBody
\end{corollary}

\proof

Follows directly from Lemma~\ref{lemma:\TypeSanitizationCompletenessName}
by choosing $\tctx_0 = \ctxinit$.

\qed

\begin{lemma}[\TypeSanitizationCompletenessUnificationName]\leavevmode
  \label{lemma:\TypeSanitizationCompletenessUnificationName}
  \TypeSanitizationCompletenessUnificationBody
\end{lemma}

\proof

\begin{description}
\item [Part 1]
  \mbox{} % an empty line to make sure long table appear after proof
  \begin{longtable}[l]{lll}
    & $\applye \cctx \genA = \tau$
    & Assume \\
    & $\applye \cctx \tau = \tau$
    & Since $\tau$ contains no existential variables\\
    & $\cctx = \cctx_1, \genA = \tau', \cctx_2$
    & By Lemma~\ref{lemma:\ExtensionOrderName} \\
    & $\tctx_1 \exto \cctx_1$
    & As above \\
    & $\applye \cctx \genA = \applye {\cctx_1} \genA = \tau$
    & By definition of context application \\
    & $\cctx_1 \bywt \tau$
    & Follows directly \\
    & Since $\tau$ contains no existential variables\\
    & And $\tctx_1$ contains all type variables in $\cctx_1$\\
    & $\tctx_1 \bywt \tau$
    & Follows directly \\
    & $\applye \cctx \tctx \bysub \tau = \applye \cctx {\sigma_1} $
    & Given \\
    & $\tctx \bysa \sigma_1 \sa \sigma_2 \toctxr$
    & By Corollary~\ref{lemma:\TypeSanitizationCompletenessUnificationName} \\
    & $\ctxr = \ctxr_1, \genA, \ctxr_2 $
    & As above \\
    & $\ctxr_1 \bywt \sigma_2$
    & As above \\
    & $\ctxr \exto \cctx'$
    & As above \\
    & $\cctx \exto \cctx'$
    & As above \\
    & $\tctx \bysuni \genA \uni \sigma_1 \toctxr_1, \genA = \sigma_2, \ctxr_2 $
    & By \rul{U-EVarTy} \\
    & $\applye {\cctx'} \genA = \applye {\cctx} \genA  = \tau = \applye \cctx {\sigma_1} $
    & By Lemma~\ref{lemma:\FinishingTypesName} \\
    & $\applye {\ctxr} {\sigma_1} = \applye {\ctxr} {\sigma_2}$
    & By Lemma~\ref{lemma:\TypeSanitizationEquivalenceName} \\
    & $\applye {\cctx'} {\sigma_1} = \applye {\cctx'} {\sigma_2}$
    & By Lemma~\ref{lemma:\SubstitutionExtensionInvarianceName} \\
    & $\applye {\cctx'} {\sigma_1} = \applye {\cctx} {\sigma_1}$
    & By Lemma~\ref{lemma:\FinishingTypesName} \\
    & $\applye {\cctx'} \genA = \applye {\cctx'} {\sigma_2}$
    & Follows directly \\
    & $\ctxr_1, \genA = \sigma_2, \ctxr_2 \exto \cctx'$
    & By Lemma~\ref{lemma:\ParallelExtensionSolutionName}
  \end{longtable}
\item [Part 2]
  Similar as Part 1.
\end{description}

\qed


\subsection{Properties of Unification}

\begin{lemma}[\UnificationExtensionName]\leavevmode
  \label{lemma:\UnificationExtensionName}
  \UnificationExtensionBody
\end{lemma}

\proof

By induction on the height of unification derivation.
Two parts reply on each
other but the size of derivation is decreasing.
So the proof can terminate.

\begin{description}
\item [Part 1] In Part 1, we regard all $\delta = e$.
  \begin{itemize}
  \item Case \[\UAEq\]
    Follows trivially by Lemma~\ref{lemma:\ContextExtensionReflexivityName}.
  \item Case \[\UApp\]
    \begin{longtable}[l]{lll}
      & $\tctx \byinf e_1 ~ e_2 \infto \sigma_1' $ & Given \\
      & $\tctx \byinf e_3 ~ e_4 \infto \sigma_2' $ & Given \\
      & $\tctx \byinf e_1 \infto \sigma_1'' $ & By inversion \\
      & $\tctx \byinf e_3 \infto \sigma_2'' $ & By inversion \\
      & $\tctx \exto \ctxl_1 $ & By induction hypothesis \\
      & $\tctx \byinf e_2 \infto \sigma_1''' $ & By inversion \\
      & $\tctx \byinf e_4 \infto \sigma_2''' $ & By inversion \\
      & $\ctxl_1 \byinf e_2 \infto \applye {\ctxl_1} {\sigma_1'''} $
      & By Lemma~\ref{lemma:\ExtensionWeakningName} \\
      & $\ctxl_1 \byinf e_4 \infto \applye {\ctxl_1} {\sigma_2'''} $
      & By Lemma~\ref{lemma:\ExtensionWeakningName} \\
      & $\ctxl_1 \byinf \applye {\ctxl_1} {e_2} \infto
      \applye {\ctxl_1} {\sigma_1'''} $
      & By Lemma~\ref{lemma:\ContextApplicationPreservesTypingName} \\
      & $\ctxl_1 \byinf \applye {\ctxl_1} {e_4} \infto
      {\applye {\ctxl_1} {\sigma_2'''}} $
      & By Lemma~\ref{lemma:\ContextApplicationPreservesTypingName} \\
      & $\ctxl_1 \exto \ctxl$
      & By induction hypothesis \\
      & $\tctx \exto \ctxl$
      & By Lemma~\ref{lemma:\ContextExtensionTransitivityName}
    \end{longtable}
  \item Case \[\ULamAnn\]
    \begin{longtable}[l]{lll}
      & $\tctx \byinf \blam x {\sigma_1} {e_1} \infto \sigma_1' $ & Given \\
      & $\tctx \byinf \sigma_1 \infto \star $ & By inversion \\
      & $\tctx \byinf \blam x {\sigma_2} {e_2} \infto \sigma_2' $ & Given \\
      & $\tctx \byinf \sigma_2 \infto \star $ & By inversion \\
      & $\tctx \exto \ctxl_1 $ & By Part 2 \\
      & $\tctx, x:\sigma_1 \exto \ctxl_1, x:\sigma_1 $
      & By \rul{CE-Var} \\
      & $\tctx, x:\sigma_1 \byinf e_1 \infto \sigma_1'' $ & By inversion \\
      & $\ctxl_1, x: \sigma_1 \byinf e_1 \infto \applye {\ctxl_1, x:\sigma_1} {\sigma_1''} $
      & By Lemma~\ref{lemma:\ExtensionWeakningName} \\
      & $\ctxl_1, x: \sigma_1 \byinf \applye {\ctxl_1, x: \sigma_1} {e_1} \infto
       {\applye {\ctxl_1, x:\sigma_1} {\sigma_1''}} $
      & By Lemma~\ref{lemma:\ContextApplicationPreservesTypingName} \\
      & $\ctxl_1, x: \sigma_1 \byinf \applye {\ctxl_1} {e_1} \infto
      {\applye {\ctxl_1} {\sigma_1''}} $
      & By definition of context substitution \\
      & $\ctxl_1, x: \sigma_1 \byinf \applye {\ctxl_1} {e_2} \infto
      {\applye {\ctxl_1} {\sigma_2''}} $
      & Similarly \\
      & $\ctxl_1, x:\sigma_1 \exto \ctxl, x:\sigma_1 $
      & By induction hypothesis \\
      & $\ctxl_1 \exto \ctxl $
      & By inversion \\
      & $\tctx \exto \ctxl$
      & By Lemma~\ref{lemma:\ContextExtensionTransitivityName}
    \end{longtable}
  \item Case \[\UPi\]
    Similar as Case \rul{U-LamAnn}.
    The difference is that instead of using induction hypothesis,
    two subgoals all use Part 2.
  \item Case \[\UCastDn\]
    \begin{longtable}[l]{lll}
      & $\tctx \byinf \castdn {e_1} \infto \sigma_1'$ & Given \\
      & $\tctx \byinf e_1 \infto \sigma_1''$ & By inversion \\
      & $\tctx \byinf e_2 \infto \sigma_2''$ & Similarly \\
      & $\tctx \exto \ctxl$ & By induction hypothesis
    \end{longtable}
  \item Case \[\UCastUp\]
    Similar as Case \rul{U-CastDn}.
  \end{itemize}
\item [Part 2]
  \begin{itemize}
  \item Case \[\UAEq\]
    Follows trivially by Lemma~\ref{lemma:\ContextExtensionReflexivityName}.
  \item Case \[\UEVarTy\]
    \begin{longtable}[l]{lll}
      & $\tctx[\genA] \byinf \tau_1 \infto \star $& Given \\
      & $\tctx[\genA] \exto \ctxl_1, \genA, \ctxl_2 $
      & By Lemma~\ref{lemma:\TypeSanitizationExtensionName} \\
      & $\ctxl_1, \genA, \ctxl_2 \byinf \tau_2 \infto \star $
      & By Lemma~\ref{lemma:\TypeSanitizationWellFormednessName} \\
      & $\ctxl_1 \bywt \tau_2$
      & Given \\
      & $\ctxl_1 \byinf \tau_2 \infto \star $
      & By Lemma~\ref{lemma:\TypingStrengtheningName} \\
      & $\ctxl_1, \genA, \ctxl_2 \exto \ctxl_1, \genA = \tau_2, \ctxl_2$
      & By Lemma~\ref{lemma:\SolvedVariableAdditionForExtensionName} \\
      & $\tctx[\genA] \exto \ctxl_1, \genA = \tau_2, \ctxl_2$
      & By Lemma~\ref{lemma:\ContextExtensionTransitivityName} \\
    \end{longtable}
  \item Case \[\UTyEVar\]
    Similar as Case \rul{U-EVarTy}.
  \item Case \[\UApp\]
    Similar as Case \rul{U-App} in Part 1.
  \item Case \[\UPi\]
    Similar as Case \rul{U-Pi} in Part 1.
  \item Case \[\UCastDn\]
    Similar as Case \rul{U-CastDn} in Part 1.
  \end{itemize}
\end{description}

\qed

\begin{lemma}[\UnificationEquivalenceName]\leavevmode
  \label{lemma:\UnificationEquivalenceName}
  \UnificationEquivalenceBody
\end{lemma}

\proof

By induction on unification derivation.

\begin{itemize}
  \item Case \[\UAEq\]
    The goal holds trivially.
  \item Case \[\UEVarTy\]
    \begin{longtable}[l]{lll}
      & $\tctx[\genA] \byinf \tau_1 \infto \sigma_2' $ & Given \\
      & $\applye {\ctxl_1, \genA, \ctxl_2} {\tau_1}
      = \applye {\ctxl_1, \genA, \ctxl_2} {\tau_2} $
      & By Lemma~\ref{lemma:\TypeSanitizationEquivalenceName} \\
      & $\applye {\ctxl_1, \genA = \tau_2, \ctxl_2} \genA $ & \\
      & $ = \applye {\ctxl_1} {\tau_2} $
      & By definition of context substitution \\
      & $\applye {\ctxl_1, \genA = \tau_2, \ctxl_2} {\tau_1}$ & \\
      & $= \applye {\ctxl_1, \genA = \tau_2} {\applye {\ctxl_2} {\tau_1}} $
      & By definition of context application \\
      & $= \applye {\ctxl_1} {(\applye {\ctxl_2} {\tau_1}) \subst \genA
        {\tau_2}} $
      & By definition of context application \\
      & $= (\applye {\ctxl_1} {\applye {\ctxl_2} {\tau_1}}) \subst \genA
        {\applye {\ctxl_1} {\tau_2}} $
      & By property of context application and substitution \\
      & $= (\applye {\ctxl_1, \ctxl_2} {\tau_1}) \subst \genA
        {\applye {\ctxl_1} {\tau_2}} $
      & By definition of context application \\
      & $= (\applye {\ctxl_1, \genA, \ctxl_2} {\tau_1}) \subst \genA
        {\applye {\ctxl_1} {\tau_2}} $
      & By definition of context application \\
      & $= (\applye {\ctxl_1, \genA, \ctxl_2} {\tau_2}) \subst \genA
        {\applye {\ctxl_1} {\tau_2}} $
      & By substituting the equality \\
      & $\ctxl_1 \bywt \tau_2 $ & Given \\
      & $\tctx[\genA] \wc$
      & By Lemma~\ref{lemma:\TypingContextWellFormednessName} \\
      & $\tctx[\genA] \exto \ctxl_1, \genA, \ctxl_2 $
      & By Lemma~\ref{lemma:\TypeSanitizationExtensionName} \\
      & $\ctxl_1, \genA, \ctxl_2 \wc $
      & By Lemma~\ref{lemma:\ContextExtensionPreservesContextWellFormednessName}\\
      & $\genA \notin \ctxl_1 \cup FV(\tau_2)$
      & Follows directly \\
      & $\tau_2$ contains no existential variable in $\ctxl_2$
      & Follows directly \\
      & $\applye {\ctxl_1, \genA = \tau_2, \ctxl_2} {\tau_1}$ & \\
      & $= (\applye {\ctxl_1, \genA, \ctxl_2} {\tau_2}) \subst \genA
      {\applye {\ctxl_1} {\tau_2}} $
      & Known \\
      & $= (\applye {\ctxl_1} {\tau_2}) \subst \genA {\applye {\ctxl_1} {\tau_2}}$
      & Substitute fresh context \\
      & $= (\applye {\ctxl_1} {\tau_2})$
      & Substitute fresh variable \\
    \end{longtable}
  \item Case \[\UTyEVar\]
    Similar as Case \rul{U-EVarTy}.
  \item Case \[\UApp\]
    \begin{longtable}[l]{lll}
      & $\tctx \byinf e_1 ~ e_2 \infto \sigma_1'$ & Given \\
      & $\tctx \byinf e_1 \infto \sigma_1''$ & By inversion \\
      & $\tctx \byinf e_2 \infto \sigma_1'''$ & By inversion \\
      & $\tctx \byinf e_3 ~ e_4 \infto \sigma_2'$ & Given \\
      & $\tctx \byinf e_3 \infto \sigma_2''$ & By inversion \\
      & $\tctx \byinf e_4 \infto \sigma_2'''$ & By inversion \\
      & $\applye {\ctxl_1} {e_1} = \applye {\ctxl_1} {e_3}$
      & By induction hypothesis \\
      & $\tctx \exto \ctxl_1$
      & By Lemma~\ref{lemma:\UnificationExtensionName} \\
      & $\ctxl_1 \byinf e_2 \infto \applye {\ctxl_1} {\sigma_1'''}$
      & By Lemma~\ref{lemma:\ExtensionWeakningName} \\
      & $\ctxl_1 \byinf \applye {\ctxl_1} {e_2} \infto
      {\applye {\ctxl_1} {\sigma_1'''}}$
      & By Lemma~\ref{lemma:\ContextApplicationPreservesTypingName} \\
      & $\ctxl_1 \byinf \applye {\ctxl_1} {e_4} \infto \applye {\ctxl_1}
      {\sigma_2'''}$
      & Similarly \\
      & $\applye \ctxl {\applye {\ctxl_1} {e_2}}
      = \applye \ctxl {\applye {\ctxl_1} {e_4}} $
      & By induction hypothesis \\
      & $\ctxl_1 \exto \ctxl $
      & By Lemma~\ref{lemma:\UnificationExtensionName} \\
      & $\applye \ctxl {e_2} = \applye \ctxl {e_4} $
      & By Lemma~\ref{lemma:\SubstitutionExtensionInvarianceName} \\
      & $\applye \ctxl {e_1} = \applye \ctxl {e_3} $
      & By Lemma~\ref{lemma:\SubstitutionExtensionInvarianceName} \\
    \end{longtable}
  \item Case \[\ULamAnn\]
    \begin{longtable}[l]{lll}
      & $\tctx \byinf \blam x {\sigma_1} {e_1} \infto \sigma_1' $
      & Given \\
      & $\tctx \byinf {\sigma_1} \infto \star $
      & By inversion \\
      & $\tctx, x: \sigma_1 \byinf {e_1} \infto \sigma_1'' $
      & By inversion \\
      & $\tctx \byinf {\sigma_2} \infto \star $
      & Similarly \\
      & $\tctx, x: \sigma_2 \byinf {e_2} \infto \sigma_2'' $
      & Similarly \\
      & $\applye {\ctxl_1} {\sigma_1} = \applye {\ctxl_1} {\sigma_2} $
      & By induction hypothesis \\
      & $\tctx \exto \ctxl_1 $
      & By Lemma~\ref{lemma:\UnificationExtensionName} \\
      & $\tctx, x: \sigma_1 \exto \ctxl_1, x:\sigma_1 $
      & By \rul{CE-Var} \\
      & $\ctxl_1, x: \sigma_1 \byinf {e_1} \infto \applye {\ctxl_1, x: \sigma_1}
      {\sigma_1''} $
      & By Lemma~\ref{lemma:\ExtensionWeakningName} \\
      & $\ctxl_1, x: \sigma_1 \byinf {\applye {\ctxl_1, x: \sigma_1} {e_1}}
      \infto \applye {\ctxl_1, x: \sigma_1} {\sigma_1''}$
      & By Lemma~\ref{lemma:\ContextApplicationPreservesTypingName} \\
      & $\ctxl_1, x: \sigma_1 \byinf {\applye {\ctxl_1} {e_1}}
      \infto \applye {\ctxl_1} {\sigma_1''}$
      & By definition of context substitution\\
      & $\tctx, x: \sigma_2 \exto \ctxl_1, x:\sigma_2 $
      & By \rul{CE-Var} \\
      & $\ctxl_1, x: \sigma_2 \byinf \applye {\ctxl_1} {e_2} \infto \applye
      {\ctxl_1} {\sigma_2''} $
      & Similarly \\
      & $\ctxl_1 \byinf {\sigma_2} \infto \star $
      & By Lemma~\ref{lemma:\ExtensionWeakningName} \\
      & $\ctxl_1 \byinf \applye {\ctxl_1} {\sigma_2} \infto \star $
      & By Lemma~\ref{lemma:\ContextApplicationPreservesTypingName} \\
      & $\ctxl_1, x: \applye {\ctxl_1} {\sigma_2} \byinf \applye {\ctxl_1} {e_2} \infto \applye
      {\ctxl_1} {\sigma_2''} $
      & By Lemma~\ref{lemma:\ContextApplicationInContextName} \\
      & $\ctxl_1, x: \applye {\ctxl_1} {\sigma_1} \byinf \applye {\ctxl_1} {e_2} \infto \applye
      {\ctxl_1} {\sigma_2''} $
      & By substituting the equality \\
      & $\ctxl_1 \byinf {\sigma_1} \infto \star $
      & By Lemma~\ref{lemma:\ExtensionWeakningName} \\
      & $\ctxl_1, x: \sigma_1 \byinf \applye {\ctxl_1} {e_2} \infto \applye
      {\ctxl_1} {\sigma_2''} $
      & By Lemma~\ref{lemma:\ReverseContextApplicationInContextName} \\
      & $\applye {\ctxl, x:\sigma_1} {\applye {\ctxl_1} {e_1}}
      = \applye {\ctxl, x: \sigma_1} {\applye {\ctxl_1} {e_2}} $
      & By induction hypothesis \\
      & $\applye {\ctxl} {\applye {\ctxl_1} {e_1}}
      = \applye {\ctxl} {\applye {\ctxl_1} {e_2}} $
      & By definition of context application \\
      & $\ctxl_1, x: \sigma_1 \exto \ctxl, x: \sigma_1 $
      & By Lemma~\ref{lemma:\UnificationExtensionName} \\
      & $\ctxl_1 \exto \ctxl $
      & By inversion \\
      & $\applye {\ctxl} {e_1}
      = \applye {\ctxl} {e_2} $
      & By Lemma~\ref{lemma:\SubstitutionExtensionInvarianceName} \\
      & $\applye {\ctxl} {\sigma_1}
      = \applye {\ctxl} {\sigma_2} $
      & By Lemma~\ref{lemma:\SubstitutionExtensionInvarianceName} \\
    \end{longtable}
  \item Case \[\UPi\]
    Similar as Case \rul{U-LamAnn}.
  \item Case \[\UCastDn\]
    \begin{longtable}[l]{lll}
      & $\tctx \byinf \castdn e_1 \infto \sigma_1' $
      & Given \\
      & $\tctx \byinf e_1 \infto \sigma_1'' $
      & By inversion \\
      & $\tctx \byinf e_2 \infto \sigma_2'' $
      & Similarly \\
      & $\applye \ctxl {e_1} = \applye \ctxl {e_2} $
      & By induction hypothesis
    \end{longtable}
  \item Case \[\UCastUp\]
    Similar as Case \rul{U-CastDn}.
\end{itemize}

\qed

We use the notation $\tctx \byinf \sigma_1 = \sigma_2 \infto \tau$ to mean that
$\tctx \byinf \sigma_1 \infto \tau$,
$\tctx \byinf \sigma_2 \infto \tau$,
and $\sigma_1 = \sigma_2$.

\begin{lemma}[\UnificationCompletenessName]
  \label{lemma:\UnificationCompletenessName}
    \UnificationCompletenessBody
\end{lemma}

\proof

By induction on the typing size
$\applye \cctx \tctx \byinf \applye \cctx {\sigma_1} \infto \tau$.
We then do case analysis on the shape of $\sigma_1$.

\begin{itemize}
\item Case $\sigma_1 = \genA$.
  Depending on whether $\genA \in FV(\sigma_2)$, we have two subcases.
  \begin{itemize}
  \item SubCase $\genA \in FV(\sigma_2)$.
    Then there is only one possible case $\sigma_2 = \genA$.
    Otherwise, $\applye \cctx {\sigma_1}$ cannot be equal to $\applye \cctx
    {\sigma_2}$.
    Therefore, we can use rule \rul{U-AEq} to get that
    $\tctx \byuni \genA \uni \genA \toctxo$.
    Choose $\cctx' = \cctx$ and we are done.
  \item SubCase $\genA \notin FV(\sigma_2)$.
    Follows directly by
    Lemma~\ref{lemma:\TypeSanitizationCompletenessUnificationName}
    and rule \rul{U-EVarTy}.
  \end{itemize}
\item In all rest cases, when we do inversion on the equality, we could have
  case $\sigma_2 = \genA$. Basically the proof for when $\sigma_2$ is an
  existential variable is similar as last case, with rule \rul{U-TyEVar} instead
  of \rul{U-EVarTy}. So in the rest cases, we will ignore the cases when
  $\sigma_2$ is an existential variable.
\item Case $\sigma_1 = x$.
  Then by inversion we have $\sigma_2 = x$.
  Therefore, we can use rule \rul{U-AEq} to get that
  $\tctx \byuni x \uni x \toctxo$.
  Choose $\cctx' = \cctx$ and we are done.
\item Case $\sigma_1 = \star$.
  Then by inversion we have $\sigma_2 = \star$.
  Therefore, we can use rule \rul{U-AEq} to get that
  $\tctx \byuni \star \uni \star \toctxo$.
  Choose $\cctx' = \cctx$ and we are done.
\item Case $\sigma_1 = e_1 ~ e_2$.
  \begin{longtable}[l]{lll}
    & $\sigma_2 = e_3 ~ e_4 $
    & By inversion \\
    & $\applye \cctx \tctx \byinf \applye \cctx {e_1}
    = \applye \cctx {e_3} \infto \tau' $
    & As above \\
    & $\applye \cctx \tctx \byinf \applye \cctx {e_2}
    = \applye \cctx {e_4} \infto \tau'' $
    & As above \\
    & $\tctx \byeuni e_1 \uni e_3 \toctx$
    & By induction hypothesis \\
    & $\ctxl \exto \cctx_1$
    & As above \\
    & $\cctx \exto \cctx_1$
    & As above \\
    & $\applye {\cctx} {\cctx} \byinf \applye \cctx {e_2}
    = \applye \cctx {e_4} \infto \tau'' $
    & By Lemma~\ref{lemma:\StabilityOfCompleteContextsName} \\
    & $\applye {\cctx_1} {\cctx_1} \byinf \applye \cctx {e_2}
    = \applye \cctx {e_4} \infto \tau'' $
    & By Lemma~\ref{lemma:\FinishingCompletionsName} \\
    & $\applye {\cctx_1} {\ctxl} \byinf \applye \cctx {e_2}
    = \applye \cctx {e_4} \infto \tau'' $
    & By Lemma~\ref{lemma:\ConfluenceOfCompletenessName} \\
    & $\applye {\cctx_1} {\ctxl} \byinf \applye {\cctx_1} {e_2}
    = \applye {\cctx_1} {e_4} \infto \tau'' $
    & By Lemma~\ref{lemma:\FinishingTypesName} \\
    & $\applye {\cctx_1} {\ctxl} \byinf \applye {\cctx_1} {\applye \ctxl {e_2}}
    = \applye {\cctx_1} {\applye \ctxl {e_4}} \infto \tau'' $
    & By Lemma~\ref{lemma:\SubstitutionExtensionInvarianceName} \\
    & $\ctxl \byeuni \applye {\ctxl} {e_2} \uni \applye {\ctxl} {e_4}
    \toctxr $
    & By induction hypothesis \\
    & $\ctxr \exto \cctx_2$
    & As above \\
    & $\cctx_1 \exto \cctx_2$
    & As above \\
    & $\tctx \bybuni e_1 ~ e_2 \uni e_3~ e_4 \toctxr $
    & By \rul{U-App} \\
    & $\cctx' = \cctx_2$
    & Choose \\
    & $\cctx \exto \cctx_2$
    & By Lemma~\ref{lemma:\ContextExtensionTransitivityName} \\
  \end{longtable}
  \item Case $\sigma_1 = \blam x {\tau_1} {e_1} $
  \begin{longtable}[l]{lll}
    & $\sigma_2 = \blam x {\tau_2} {e_2} $
    & By inversion \\
    & $\applye \cctx \tctx \byinf
    \applye \cctx {\tau_1}
    = \applye \cctx {\tau_2} \infto \star $
    & As above \\
    & $\applye \cctx \tctx, x: \applye \cctx {\tau_1} \byinf
    \applye \cctx {e_1}
    = \applye \cctx {e_2} \infto \tau' $
    & As above \\
    & $\tctx \bysuni \tau_1 \uni \tau_2 \toctx$
    & By induction hypothesis \\
    & $\ctxl \exto \cctx_1$
    & As above \\
    & $\cctx \exto \cctx_1$
    & As above \\
    & $\cctx, x: \tau_1 \exto \cctx_1, x: \tau_1$
    & By \rul{CE-Var} \\
    & $\ctxl, x: \tau_1 \exto \cctx_1, x : \tau_1$
    & By \rul{CE-Var} \\
    & $\tctx, x: \tau_1 \exto \cctx, x: \tau_1$
    & By Lemma~\ref{lemma:\UnificationExtensionName} \\
    & $\applye {\cctx, x : \tau_1} {\tctx, x: \tau_1} \byinf
    \applye {\cctx, x: \tau_1} {e_1}
    = \applye {\cctx, x: \tau_1} {e_2} \infto \tau' $
    & By definition of context application \\
    & $\applye {\cctx, x : \tau_1} {\cctx, x : \tau_1} \byinf
    \applye {\cctx, x: \tau_1} {e_1}
    = \applye {\cctx, x: \tau_1} {e_2} \infto \tau' $
    & Lemma~\ref{lemma:\StabilityOfCompleteContextsName} \\
    & $\applye {\cctx_1, x : \tau_1} {\cctx_1, x : \tau_1} \byinf
    \applye {\cctx_1, x: \tau_1} {e_1}
    = \applye {\cctx_1, x: \tau_1} {e_2} \infto \tau' $
    & Lemma~\ref{lemma:\FinishingCompletionsName}
    and Lemma~\ref{lemma:\FinishingTypesName}\\
    & $\applye {\cctx_1, x : \tau_1} {\ctxl, x : \tau_1} \byinf
    \applye {\cctx_1, x: \tau_1} {e_1}
    = \applye {\cctx_1, x: \tau_1} {e_2} \infto \tau' $
    & Lemma~\ref{lemma:\StabilityOfCompleteContextsName} \\
    & $\applye {\cctx_1, x : \tau_1} {\ctxl, x : \tau_1} \byinf
    \applye {\cctx_1, x: \tau_1} {\applye {\ctxl} {e_1}}$
    & Lemma~\ref{lemma:\SubstitutionExtensionInvarianceName} \\
    & $= \applye {\cctx_1, x: \tau_1} {\applye {\ctxl} {e_2}} \infto \tau' $
    & and definition of context application\\
    & $\ctxl, x: \tau_1 \byeuni \applye \ctxl {e_1} \uni \applye \ctxl {e_2}
    \toctxr, x: \tau_1 $
    & By induction hypothesis
    and Lemma~\ref{lemma:\TypeSanitizationTailUnchangedName} \\
    & $\ctxr, x: \tau_1 \exto \cctx_2$
    & By induction hypothesis \\
    & $\cctx_1, x:\tau_1 \exto \cctx_2$
    & By induction hypothesis \\
    & $\cctx_2 = \cctx_3, x: \tau_1, \cctx_4 $
    & By Lemma~\ref{lemma:\ExtensionOrderName} \\
    & $\ctxr \exto \cctx_3$
    & As above \\
    & $\cctx_1 \exto \cctx_3$
    & As above \\
    & $\cctx' = \cctx_3$
    & Choose \\
    & $\tctx \byeuni \blam x {\tau_1} {e_1} \uni
    \blam x {\tau_2} {e_2} \toctxr$
    & By \rul{U-LamAnn} \\
    & $\cctx \exto \cctx_3$
    & By Lemma~\ref{lemma:\ContextExtensionTransitivityName}
  \end{longtable}
  \item Case $\sigma_1 = \bpi x {\tau_1} {\tau_2} $
    Similar as last case.
  \item Case $\sigma_1 = \castdn \tau_1 $
  \begin{longtable}[l]{lll}
    & $\sigma_2 = \castdn {\tau_2} $
    & By inversion \\
    & $\applye \cctx \tctx \byinf
    \applye \cctx {\tau_1}
    = \applye \cctx {\tau_2} \infto \tau' $
    & As above \\
    & $\tctx \byuni \tau_1 \uni \tau_2 \toctxr$
    & By induction hypothesis \\
    & $\ctxr \exto \cctx_1$
    & As above \\
    & $\cctx \exto \cctx_1$
    & As above \\
    & $\cctx' = \cctx_1$
    & Choose \\
    & $\tctx \byuni \castdn \tau_1 \uni \castdn \tau_2 \toctxr$
    & By \rul{U-CastDn}
  \end{longtable}
  \item Case $\sigma_1 = \castup \tau_1 $
    Similar as last cast.
\end{itemize}

\qed

\section{Implicit Polymorphic Type System}

\subsection{Referred Lemmas}

\begin{lemma}[Unsolved Variable Addition For Extension]
  \label{lemma:dunfield:UnsolvedVariableAdditionForExtension}
  $\tctx_L, \tctx_R \exto \tctx_L, \genA, \tctx_R $.
\end{lemma}

\begin{lemma}[Solution Admissibility for Extension]
  \label{lemma:dunfield:SolutionAdmissibilityForExtension}
  If $\tctx \bywf \tau$,
  then $\tctx_L, \genA, \tctx_R \exto \tctx_L, \genA = \tau, \tctx_R $.
\end{lemma}

\begin{lemma}[Transitivity]
  \label{lemma:dunfield:Transitivity}
  If $\tctx \exto \ctxr$,
  and $\ctxr \exto \ctxl$,
  then $\tctx \exto \ctxl$.
\end{lemma}

\begin{lemma}[Reflexivity]
  \label{lemma:dunfield:Reflexivity}
  If $\tctx$ is well-formed,
  then $\tctx \exto \tctx$.
\end{lemma}

\begin{lemma}[Confluence of Completeness]
  \label{lemma:dunfield:ConfluenceOfCompleteness}
  If $\ctxr_1 \exto \cctx$,
  and $\ctxr_2 \exto \cctx$,
  then $\applye \cctx {\ctxr_1} = \applye \cctx {\ctxr_2} $.
\end{lemma}

\begin{lemma}[Substitution Extension Invariance]
  \label{lemma:dunfield:SubstitutionExtensionInvariance}
  If $\ctxl \bywf A $,
  and $\ctxl \exto \tctx $,
  then $\applye \tctx A = \applye \tctx {\applye \ctxl A} $,
  and $\applye \tctx A = \applye \ctxl {\applye \tctx A} $.
\end{lemma}

\begin{lemma}[Reflexivity of Declarative Subtyping]
  \label{lemma:dunfield:ReflexivityOfDeclarativeSubtyping}
  If $\dctx \bywf A$,
  then $\dctx \bysub A \dsub A $.
\end{lemma}

\begin{lemma}[Monotype Equality]
  \label{lemma:dunfield:MonotypeEquality}
  If $\dctx \bysub \sigma \dsub \tau$,
  then $\sigma = \tau$.
\end{lemma}

\begin{lemma}[Parallel Admissibility]
  \label{lemma:dunfield:ParallelAdmissibility}
  If $\tctx_L \exto \ctxr_L$,
  and $\tctx_L, \tctx_R \exto \ctxr_L, \ctxr_R$, then:
  \begin{itemize}
    \item $\tctx_L, \genA, \tctx_R  \exto \ctxr_L, \genA, \ctxr_R$
    \item If $\ctxr_L \bywf \tau'$,
      then $\tctx_L, \genA, \tctx_R \exto \ctxr_L, \genA = \tau', \ctxr_R $
    \item  If $\tctx_L \bywf \tau$,
      and $\ctxr_L \bywf \tau'$,
      and $\applye {\ctxr_L} \tau = \applye {\ctxr_L} {\tau'} $,
      then $\tctx_L, \genA = \tau, \tctx_R \exto \ctxr_L, \genA = \tau', \ctxr_R$.
  \end{itemize}
\end{lemma}

\begin{lemma}[Parallel Extension Solution]
  \label{lemma:dunfield:ParallelExtensionSolution}
  If $\tctx_L, \genA, \tctx_R \exto \ctxr_L, \genA = \tau', \ctxr_R $,
  and $\tctx_L \bywf \tau  $,
  and  $\applye {\ctxr_L} \tau = \applye {\ctxr_L} {\tau'} $,
  then $\tctx_L, \genA = \tau, \tctx_R \exto \ctxr_L, \genA = \tau', \ctxr_R $.
\end{lemma}

\begin{lemma}[Solved Variable Addition for Extension]
  \label{lemma:dunfield:SolvedVariableAdditionForExtension}
  If $\tctx_L \bywf \tau$,
  and $\tctx_L, \tctx_R \exto \tctx_L, \genA = \tau, \tctx_R $.
\end{lemma}

\begin{lemma}[Extension Order]\leavevmode
  \label{lemma:dunfield:ExtensionOrder}
  \begin{itemize}
  \item $\tctx_L, \varA, \tctx_R \exto \ctxr$,
    then $\ctxr = (\ctxr_L, \varA, \ctxr_R)$
    where $\tctx_L \exto \ctxr_L$.
    Moreover, if $\tctx_R$ is soft then $\ctxr_R$ is soft.
  \item $\tctx_L, \marker \genA, \tctx_R \exto \ctxr$,
    then $\ctxr = (\ctxr_L, \marker \genA, \ctxr_R)$
    where $\tctx_L \exto \ctxr_L$.
    Moreover, if $\tctx_R$ is soft then $\ctxr_R$ is soft.
  \item $\tctx_L, \genA, \tctx_R \exto \ctxr$,
    then $\ctxr = (\ctxr_L, \ctxl, \ctxr_R)$
    where $\tctx_L \exto \ctxr_L$,
    and $\ctxl$ is either $\genA$ or $\genA = \tau$ for some $\tau$.
  \item $\tctx_L, \genA = \tau, \tctx_R \exto \ctxr$,
    then $\ctxr = (\ctxr_L, \genA = \tau', \ctxr_R)$
    where $\tctx_L \exto \ctxr_L$,
    and $\applye {\ctxr_L} \tau = \applye {\ctxr_L} {\tau'}$.
  \item $\tctx_L, x : A, \tctx_R \exto \ctxr$,
    then $\ctxr = (\ctxr_L, x : A', \ctxr_R)$
    where $\tctx_L \exto \ctxr_L$,
    and $\applye {\ctxr_L} A = \applye {\ctxr_L} {A'}$.
    Moreover, if $\tctx_R$ is soft then $\ctxr_R$ is soft.
  \end{itemize}
\end{lemma}

\begin{lemma}[Reverse Declaration Order Preservation]
  \label{lemma:dunfield:ReverseDeclarationOrderPreservation}
  If $\tctx \exto \ctxr$ and $u$ and $v$ are both declared in $\tctx$ $u$ is
  declared to the left of $v$ in $\ctxr$,
  then $u$ is declared to the left of $v$ in $\tctx$.
\end{lemma}

\begin{lemma}[Stability of Complete Contexts]
  \label{lemma:dunfield:StabilityOfCompleteContexts}
  If $\tctx \exto \cctx $,
  then $\applye \cctx \tctx = \applye \cctx \cctx $.
\end{lemma}

\begin{lemma}[Finishing Types]
  \label{lemma:dunfield:FinishingTypes}
  If $\cctx \bywf A $,
  and $\cctx \exto \cctx' $,
  then $\applye \cctx A = \applye {\cctx'} A $.
\end{lemma}

\begin{lemma}[Finishing Completions]
  \label{lemma:dunfield:FinishingCompletions}
  If $\cctx \exto \cctx' $,
  then $\applye \cctx \cctx = \applye {\cctx'} {\cctx'} $.
\end{lemma}

\begin{lemma}[Extension Weakening]
  \label{lemma:dunfield:ExtensionWeakening}
  If $\tctx \bywf A$,
  and $\tctx \exto \ctxr$,
  then  $\ctxr \bywf A$.
\end{lemma}

\begin{lemma}[Substitution Typing]
  \label{lemma:dunfield:SubstitutionTyping}
  If $\tctx \bywf A$,
  then  $\tctx \bywf \applye \tctx A$.
\end{lemma}

\begin{proposition}[Weakening]
  \label{lemma:dunfield:Weakening}
  If $\dctx \bywf A$,
  then  $\dctx, \dctx' \bywf A$ by a derivation of the same size.
\end{proposition}

\subsection{Properties of Polymorphic Type Sanitization}

\begin{lemma}[\PolymorphicTypeSanitizationExtensionName]\leavevmode
  \label{lemma:\PolymorphicTypeSanitizationExtensionName}
  \PolymorphicTypeSanitizationExtensionBody
\end{lemma}

\proof

By induction on polymorphic type sanitization derivation.

\begin{itemize}
  \item Case \[\IAllPlus\]
    \begin{longtable}[l]{lll}
      & $\tctx[\genB, \genA] \exto \ctxl $
      & By induction hypothesis\\
      & $\tctx[\genA] \exto \tctx[\genB, \genA] $
      & By Lemma~\ref{lemma:dunfield:UnsolvedVariableAdditionForExtension} \\
      & $\tctx[\genA] \exto  \ctxl $
      & By Lemma~\ref{lemma:dunfield:Transitivity}
    \end{longtable}
  \item Case \[\IAllMinus\]
    \begin{longtable}[l]{lll}
      & $\tctx, \varA \exto \ctxl, \varA$
      & By induction hypothesis \\
      & $\tctx \exto \ctxl$
      & By inversion
    \end{longtable}
  \item Case \[\IPiPoly\]
    \begin{longtable}[l]{lll}
      & $\tctx \exto \ctxl_1$
      & By induction hypothesis \\
      & $\ctxl_1 \exto \ctxl$
      & By induction hypothesis \\
      & $\tctx \exto \ctxl$
      & By Lemma~\ref{lemma:dunfield:Transitivity}
    \end{longtable}
  \item Case \[\IUnit\]
    \begin{longtable}[l]{lll}
      & $\tctx$ is well-formed
      & By implicit assumption \\
      & $\tctx \exto \tctx$
      & By Lemma~\ref{lemma:dunfield:Reflexivity}
    \end{longtable}
  \item Case \[\ITVar\]
    \begin{longtable}[l]{lll}
      & $\tctx[\varA]$ is well-formed
      & By implicit assumption \\
      & $\tctx[\varA] \exto \tctx[\varA]$
      & By Lemma~\ref{lemma:dunfield:Reflexivity}
    \end{longtable}
  \item Case \[\IEVarAfterPoly\]
    \begin{longtable}[l]{lll}
      & $\tctx[\genA][\genB] \exto \tctx[\genA_1, \genA][\genB] $
      & By Lemma~\ref{lemma:dunfield:UnsolvedVariableAdditionForExtension} \\
      & $\tctx[\genA_1, \genA] \bywf \genA_1$
      & By \rul{EvarWF} \\
      & $\tctx[\genA_1, \genA][\genB] \exto \tctx[\genA_1, \genA][\genB = \genA_1] $
      & By Lemma~\ref{lemma:dunfield:SolutionAdmissibilityForExtension} \\
      & $\tctx[\genA][\genB] \exto \tctx[\genA_1, \genA][\genB = \genA_1] $
      & By Lemma~\ref{lemma:dunfield:Transitivity}
    \end{longtable}
  \item Case \[\IEVarBeforePoly\]
    \begin{longtable}[l]{lll}
      & $\tctx[\genB][\genA]$ is well-formed
      & By implicit assumption \\
      & $\tctx[\genB][\genA] \exto \tctx[\genB][\genA] $
      & By Lemma~\ref{lemma:dunfield:Reflexivity}
    \end{longtable}
\end{itemize}

\qed

\begin{lemma}[\PolymorphicTypeSanitizationSoundnessName]\leavevmode
  \label{lemma:\PolymorphicTypeSanitizationSoundnessName}
  \PolymorphicTypeSanitizationSoundnessBody
\end{lemma}

\proof

By induction on the polymorphic type sanitization derivation. These two parts
rely on each other, but the derivation get smaller so the proof terminates.

\begin{description}
  \item [Part 1] In this part, we regard $s = +$.
    \begin{itemize}
      \item \[\IAllPlus\]
        \begin{longtable}[l]{lll}
          & $\ctxl \exto \cctx$
          & Given \\
          & $\applye {\tctx[\genA]} {\forall \varA. A } = \forall \varA . A $
          & Given \\
          & $\applye {\tctx[\genB, \genA]} {A \subst \varA \genB}
          = A \subst \varA \genB $
          & Follows directly \\
          & $\applye \cctx \ctxl \bysub \applye \cctx {A \subst \varA \genB}
          \dsub \applye \cctx \sigma $
          & By induction hypothesis \\
          & $\applye \cctx \ctxl \bysub {\applye \cctx A}
          \subst \varA {\applye \cctx \genB}
          \dsub \applye \cctx \sigma $
          & By property of substitution \\
          & $\applye \cctx \ctxl \bysub
          {\applye \cctx {\forall \varA. A}}
          \dsub \applye \cctx \sigma $
          & By \rul{$\dsub \forall$ L} with $\tau = \applye \cctx \genB$
        \end{longtable}
      \item \[\IPiPoly\]
        \begin{longtable}[l]{lll}
          & $\applye \tctx {A_1 \to A_2} = A_1 \to A_2 $
          & Given \\
          & $\applye \tctx {A_1} = A_1 $
          & Follows directly \\
          & $\ctxl_1 \exto \ctxl $
          & By Lemma~\ref{lemma:\PolymorphicTypeSanitizationExtensionName} \\
          & $\ctxl \exto \cctx $
          & Given \\
          & $\ctxl_1 \exto \cctx $
          & By Lemma~\ref{lemma:dunfield:Transitivity} \\
          & $\applye \cctx {\ctxl_1} \bysub
          \applye \cctx {\sigma_1}
          \dsub
          \applye \cctx {A_1}
          $
          & By Part 2 \\
          & $\applye \cctx {\ctxl} \bysub
          \applye \cctx {\sigma_1}
          \dsub
          \applye \cctx {A_1}
          $
          & By Lemma~\ref{lemma:dunfield:ConfluenceOfCompleteness} \\
          & $\applye {\ctxl_1} {\applye {\ctxl_1} {A_2}}
          = {\applye {\ctxl_1} {A_2}} $
          & \\
          & $\applye \cctx {\ctxl} \bysub
          \applye \cctx {\applye {\ctxl_1} {A_2}}
          \dsub
          \applye \cctx {\sigma_2}
          $
          & By induction hypothesis \\
          & $\applye \cctx {\ctxl} \bysub
          \applye \cctx {A_2}
          \dsub
          \applye \cctx {\sigma_2}
          $
          & By Lemma~\ref{lemma:dunfield:SubstitutionExtensionInvariance} \\
          & $\applye \cctx {\ctxl} \bysub
          \applye \cctx {A_1 \to A_2}
          \dsub
          \applye \cctx {\sigma_1 \to \sigma_2}
          $
          & By \rul{$\dsub \to$}
        \end{longtable}
      \item \[\IUnit\]
        \begin{longtable}[l]{lll}
          & $\applye \cctx \tctx \bysub \Unit \dsub \Unit $
          & By \rul{$\dsub$Unit}
        \end{longtable}
      \item \[\ITVar\]
        \begin{longtable}[l]{lll}
          & $\applye \cctx {\tctx[\genA]} \bysub
          \applye \cctx \genA \dsub \applye \cctx \genA $
          & By Lemma~\ref{lemma:dunfield:ReflexivityOfDeclarativeSubtyping}
        \end{longtable}
      \item \[\IEVarAfterPoly\]
        \begin{longtable}[l]{lll}
          & $\applye {\tctx[\genA_1, \genA][\genB = \genA_1]} {\genA_1}$ & \\
          & $= \applye {\tctx[\genA_1, \genA][\genB = \genA_1]} \genB$ & \\
          & $= \genA_1 $
          & By definition of context application \\
          & $\tctx[\genA_1, \genA][\genB = \genA_1] \exto \cctx $
          & Given \\
          & $\applye \cctx \genB = \applye \cctx {\genA_1} $
          & By Lemma~\ref{lemma:dunfield:SubstitutionExtensionInvariance} \\
          & $\applye \cctx \tctx \bysub \applye \cctx \genB \dsub
          \applye \cctx {\genA_1} $
          & By Lemma~\ref{lemma:dunfield:ReflexivityOfDeclarativeSubtyping}
        \end{longtable}
      \item \[\IEVarBeforePoly\]
        \begin{longtable}[l]{lll}
          & $\applye \cctx {\tctx[\genA]} \bysub
          \applye \cctx \genB \dsub \applye \cctx \genB $
          & By Lemma~\ref{lemma:dunfield:ReflexivityOfDeclarativeSubtyping}
        \end{longtable}
    \end{itemize}
  \item [Part 2] In this part, we regard $s = -$.
    \begin{itemize}
      \item \[\IAllMinus\]
        \begin{longtable}[l]{lll}
          & $\ctxl \exto \cctx $
          & Given \\
          & $\ctxl, \varA \exto \cctx, \varA $
          & By \rul{$\exto$ Uvar} \\
          & $\applye \tctx {\forall \varA. A} = \forall \varA . A $
          & Given \\
          & $\applye {\tctx, \varA} A = A $
          & Follows directly \\
          & $\applye {\cctx, \varA} {\tctx, \varA} \bysub
          \applye {\cctx, \varA} A \dsub
          \applye {\cctx, \varA} \sigma
          $
          & By induction hypothesis \\
          & $\applye {\cctx} {\tctx}, \varA \bysub
          \applye {\cctx} A \dsub
          \applye {\cctx} \sigma
          $
          & By definition of context application \\
          & $\applye {\cctx} {\tctx} \bysub
          \applye {\cctx} {\forall \varA. A} \dsub
          \applye {\cctx} \sigma
          $
          & By \rul{$\dsub \forall$ L}
        \end{longtable}
      \item \[\IPiPoly\]
        Similar as in Part 1.
      \item \[\IUnit\]
        Similar as in Part 1.
      \item \[\ITVar\]
        Similar as in Part 1.
      \item \[\IEVarAfterPoly\]
        Similar as in Part 1.
      \item \[\IEVarBeforePoly\]
        Similar as in Part 1.
    \end{itemize}
\end{description}

\qed

\begin{lemma}[\PolymorphicTypeSanitizationCompletenessName]
  \label{lemma:\PolymorphicTypeSanitizationCompletenessName}
  \PolymorphicTypeSanitizationCompletenessBody
\end{lemma}

\proof

By induction on the height of subtyping derivation.

\begin{description}
  \item [Part 1]
    We have $\applye \cctx \tctx \bysub \tau \dsub \applye \cctx
    A $. We now case analyze the shape of $A$.
    \begin{itemize}
    \item Case $A = \genB$.
      \begin{longtable}[l]{lll}
        & $\genA \notin FV(A)$
        & Given \\
        & $\genA \neq \genB$
        & Follows directly \\
        & $\applye \cctx \genB = \tau_2$
        & Assume \\
        & $\applye \cctx \tctx \bysub \tau \dsub \tau_2 $
        & Given \\
        & $\tau = \tau_2$
        & By Lemma~\ref{lemma:dunfield:MonotypeEquality} \\
        & $\applye \tctx \genB = \genB$
        & Given \\
        & $\genB \in unsolved(\tctx)$
        & Follows directly \\
      \end{longtable}
      According to whether $\genB$ is on the left of $\genA$ or right, we have
      two cases.
      \begin{itemize}
      \item SubCase $\tctx = \tctx_1, \genA, \tctx_3, \genB, \tctx_4 $.
        \begin{longtable}[l]{lll}
          & $\tctx_1, \genA, \tctx_3, \genB, \tctx_4 \bymsa \genB \sa \genA_1
          \toctxo_1, \genA_1, \genA, \tctx_3, \genB = \genA_1, \tctx_4
          $
          & By \rul{I-EVarAfterPoly} \\
          & $\tctx_1, \genA, \tctx_3, \genB, \tctx_4 \exto \cctx $
          & Given \\
          & $\cctx = \cctx[\genA = \tau']$
          & Assume \\
          & $\tctx_1, \genA_1 = \tau, \genA, \tctx_3, \genB, \tctx_4 \exto
          \cctx[\genA_1 = \tau, \genA = \tau'] $
          & By Lemma~\ref{lemma:dunfield:ParallelAdmissibility} \\
          & $\applye {\cctx[\genA_1 = \tau, \genA = \tau']} \genB = \tau $
          & Known \\
          & $\tctx_1, \genA_1 = \tau, \genA, \tctx_3, \genB = \genA_1, \tctx_4 \exto
          \cctx[\genA_1 = \tau, \genA = \tau'] $
          & By Lemma~\ref{lemma:dunfield:ParallelExtensionSolution} \\
          & $\tctx_1, \genA_1, \genA , \tctx_3, \genB = \genA_1,
          \tctx_4
          \exto \tctx_1, \genA_1 = \tau, \genA, \tctx_3, \genB =
          \genA_1, \tctx_4 $
          &  By Lemma~\ref{lemma:dunfield:SolutionAdmissibilityForExtension} \\
          & $\tctx_1, \genA_1, \genA, \tctx_3, \genB = \genA_1,
          \tctx_4
          \exto
          \cctx[\genA_1 = \tau, \genA = \tau'] $
          & By Lemma~\ref{lemma:dunfield:Transitivity} \\
          & $\cctx' = \cctx[\genA_1 = \tau, \genA = \tau']$
          & Choose \\
          & $\cctx[\genA = \tau'] \exto \cctx[\genA_1 = \tau, \genA = \tau']$
          & By Lemma~\ref{lemma:dunfield:SolvedVariableAdditionForExtension} \\
          & $\applye {\cctx'} {\genA_1} = \tau $
          & Follows directly
        \end{longtable}
      \item SubCase $\tctx = \tctx_1, \genB, \tctx_3, \genA, \tctx_4 $.
        \begin{longtable}[l]{lll}
          & $\tctx_1, \genB, \tctx_3, \genA, \tctx_4 \bymsa \genB \sa \genB
          \toctxo_1, \genB, \tctx_3, \genA, \tctx_4
          $
          & By \rul{I-EVarBeforePoly} \\
          & $\tctx_1, \genB, \tctx_3, \genA, \tctx_4 \exto \cctx $
          & Given \\
          & $\cctx' = \cctx$
          & Choose \\
          & $\cctx \exto \cctx'$
          & By Lemma~\ref{lemma:dunfield:Reflexivity} \\
          & $\applye {\cctx'} \genB = \tau$
          & Known \\
        \end{longtable}
      \end{itemize}
    \item Case $A = \varA$.
      \begin{longtable}[l]{lll}
        & $\applye \cctx \varA = \varA$
        & By definition of context application \\
        & $\applye \cctx \tctx \bysub \tau \dsub \varA$
        & Given \\
        & $\tau = \varA$
        & By inversion \\
        & $\tctx_1 \bywf \varA$
        & Given \\
        & $\varA$ is declared to the left of $\genA$ in $\tctx $ \\
        & $\tctx_1, \genA, \tctx_2 \bymsa \varA \sa \varA  \toctxo_1, \genA, \tctx_2 $
        & By \rul{I-TVar} \\
        & $\cctx' = \cctx$
        & Choose \\
        & $\cctx \exto \cctx'$
        & By Lemma~\ref{lemma:dunfield:Reflexivity} \\
        & $\applye {\cctx'} \varA = \varA$
        & By definition of context application
      \end{longtable}
    \item Case $A = \Unit$.
      \begin{longtable}[l]{lll}
        & $\applye \cctx \Unit = \Unit$
        & By definition of context application \\
        & $\applye \cctx \tctx \bysub \tau \dsub \Unit$
        & Given \\
        & $\tau = \Unit$
        & By inversion \\
        & $\tctx_1, \genA, \tctx_2 \bymsa \Unit \sa \Unit  \toctxo_1, \genA, \tctx_2 $
        & By \rul{I-Unit} \\
        & $\cctx' = \cctx$
        & Choose \\
        & $\cctx \exto \cctx'$
        & By Lemma~\ref{lemma:dunfield:Reflexivity} \\
        & $\applye {\cctx'} \varA = \varA$
        & By definition of context application
      \end{longtable}
    \item Case $A = A_1 \to A_2$
      \begin{longtable}[l]{lll}
        & $\applye \cctx {A_1 \to A_2} = \applye \cctx {A_1} \to \applye \cctx {A_2}$
        & By definition of context application \\
        & $\applye \cctx \tctx \bysub \tau \dsub
        \applye \cctx {A_1} \to \applye \cctx {A_2}$
        & Given \\
        & $\tau = \tau_1 \to \tau_2 $
        & By inversion \\
        & $\applye \cctx \tctx \bysub \applye \cctx {A_1} \dsub \tau_1  $
        & As above \\
        & $\applye \cctx \tctx \bysub \tau_2 \dsub \applye \cctx {A_2}$
        & As above \\
        & $\tctx \bypsa A_1 \sa \sigma_1 \toctx $
        & By Part 2 \\
        & $\ctxl = \ctxl_1, \genA, \ctxl_2$
        & As above \\
        & $\ctxl_1 \bywf \sigma_1$
        & As above \\
        & $\ctxl \exto \cctx_1 $
        & As above \\
        & $\cctx \exto \cctx_1$
        & As above \\
        & $\applye {\cctx_1} {\sigma_1} = \tau_1$
        & As above \\
        & $\tctx \exto \ctxl$
        & By Lemma~\ref{lemma:\PolymorphicTypeSanitizationExtensionName}\\
        & $\tctx_1 \exto \ctxl_1$
        & By Lemma~\ref{lemma:dunfield:ExtensionOrder}\\
        & $\tctx \exto \cctx_1 $
        & By Lemma~\ref{lemma:dunfield:Transitivity} \\
        & $\applye {\cctx} {\cctx} \bysub \tau_2 \dsub \applye {\cctx}
        {A_2}$
        & By Lemma~\ref{lemma:dunfield:StabilityOfCompleteContexts} \\
        & $\applye {\cctx_1} {\cctx_1} \bysub \tau_2 \dsub \applye {\cctx}
        {A_2}$
        & By Lemma~\ref{lemma:dunfield:FinishingCompletions} \\
        & $\applye {\cctx_1} {\tctx} \bysub \tau_2 \dsub \applye {\cctx}
        {A_2}$
        & By Lemma~\ref{lemma:dunfield:StabilityOfCompleteContexts} \\
        & $\applye {\cctx_1} {\ctxl} \bysub \tau_2 \dsub \applye {\cctx}
        {A_2}$
        & By Lemma~\ref{lemma:dunfield:ConfluenceOfCompleteness} \\
        & $\applye {\cctx_1} {\ctxl} \bysub \tau_2 \dsub \applye {\cctx_1}
        {A_2}$
        & By Lemma~\ref{lemma:dunfield:FinishingTypes} \\
        & $\applye {\cctx_1} {\ctxl} \bysub \tau_2 \dsub \applye {\cctx_1}
        {\applye {\ctxl} {A_2}}$
        & By Lemma~\ref{lemma:dunfield:SubstitutionExtensionInvariance} \\
        & $\ctxl \bymsa \applye \ctxl {A_2} \sa \sigma_2 \toctxr $
        & By induction hypothesis \\
        & $\ctxr = \ctxr_1, \genA, \ctxr_2 $
        & As above \\
        & $\ctxr \exto \cctx_2$
        & As above \\
        & $\cctx_1 \exto \cctx_2$
        & As above \\
        & $\ctxr_1 \bywf \sigma_2$
        & As above \\
        & $\applye {\cctx_2} {\sigma_2} = \tau_2$
        & As above \\
        & $\applye {\cctx_2} {\sigma_1} = \tau_1$
        & By Lemma~\ref{lemma:dunfield:FinishingTypes} \\
        & $\ctxl_1 \exto \ctxr_1$
        & By Lemma~\ref{lemma:dunfield:ExtensionOrder}\\
        & $\ctxr_1 \bywf \sigma_1 $
        & By Lemma~\ref{lemma:dunfield:ExtensionWeakening} \\
        & $\ctxr_1 \bywf \sigma_1 \to \sigma_2$
        & Follows directly \\
        & $\tctx \bypsa A_1 \sa \sigma_1 \toctx $
        & Known \\
        & $\ctxl \bymsa \applye \ctxl {A_2} \sa \sigma_2 \toctxr $
        & Known \\
        & $\tctx \bymsa A_1 \to A_2 \sa \sigma_1 \to \sigma_2 \toctxr $
        & By \rul{I-Pi-Poly} \\
        & $\cctx \exto \cctx_2$
        & By Lemma~\ref{lemma:dunfield:Transitivity} \\
        & $\cctx' = \cctx_2$
        & Choose \\
        & $\applye {\cctx'} {\sigma_1 \to \sigma_2} = \tau_1 \to \tau_2$
        & By definition of context application
      \end{longtable}
    \item Case $A = \forall \varA.  A_1$
      \begin{longtable}[l]{lll}
        & $\applye \cctx {\forall \varA . A_1} = \forall \varA. \applye \cctx {A_1} $
        & By definition of context application \\
        & $\applye \cctx \tctx \bysub \tau \dsub
        \forall \varA. \applye \cctx {A_1}$
        & Given \\
        & $\applye \cctx \tctx, \varA \bysub \tau \dsub
        \applye \cctx {A_1}$
        & By inversion \\
        & $\applye {\cctx, \varA} {\tctx, \varA} \bysub \tau \dsub
        \applye {\cctx, \varA} {A_1}$
        & Rewrite the judgment \\
        & $\tctx, \varA \bymsa A_1 \sa \sigma \toctx $
        & By induction hypothesis \\
        & $\sa$ will not add new variables in the end
        & Follows from definition \\
        & $\ctxl = \ctxl_1, \genA, \ctxl_2, \varA $
        & By induction hypothesis and Lemma~\ref{lemma:dunfield:ExtensionOrder}
        and above \\
        & $\ctxl_1 \bywf \sigma $
        & By induction hypothesis \\
        & $\ctxl \exto \cctx_1 $
        & As above \\
        & $\cctx, \varA \exto \cctx_1 $
        & As above \\
        & $\applye {\cctx_1} \sigma = \tau$
        & As above \\
        & $\cctx_1  = \cctx_2, \varA, \cctx_3 $
        & By Lemma~\ref{lemma:dunfield:ExtensionOrder} \\
        & $\ctxl_1, \genA, \ctxl_2  \exto \cctx_2 $
        & As above \\
        & $\cctx \exto \cctx_2$
        & As above \\
        & $\ctxl_1, \genA, \ctxl_2 \bywf \sigma$
        & By Lemma~\ref{lemma:dunfield:Weakening} \\
        & $\cctx_2 \bywf \sigma$
        & By Lemma~\ref{lemma:dunfield:ExtensionWeakening} \\
        & $\applye {\cctx_1} \sigma = \applye {\cctx_2} \sigma = \tau$
        & Follows directly \\
        & $\tctx  \bymsa \forall \varA. A_1 \sa \sigma \toctx_1, \genA, \ctxl_2 $
        & By \rul{I-All-Minus} \\
        & $\cctx' = \cctx_2$
        & Choose
      \end{longtable}
    \end{itemize}
  \item [Part 2]
    We have $\applye \cctx \tctx \bysub \applye \cctx
    A \dsub \tau$. We now case analyze the shape of $A$.
    These cases are mostly symmetric as in Part 1.
    The only exception is when $A$ is a polymorphic type.
    \begin{itemize}
      \item Case $A = \forall \varA. A_1$
      \begin{longtable}[l]{lll}
        & $\applye \cctx {\forall \varA . A_1} = \forall \varA. \applye \cctx {A_1} $
        & By definition of context application \\
        & $\applye \cctx \tctx \bysub
        \forall \varA. \applye \cctx {A_1} \dsub \tau $
        & Given \\
        & $\applye \cctx \tctx \bysub
        (\applye \cctx {A_1}) \subst \varA {\tau_1}
        \dsub \tau $
        & By inversion \\
        & $\applye \cctx \tctx \bywf \tau_1 $
        & Same as above \\
        & $\tau_1$ contains no existential variable
        & Follows directly \\
        & $\applye \cctx {\tau_1} = \tau_1$
        & Follows directly \\
        & $\applye \cctx \tctx \bysub
        \applye \cctx {A_1 \subst \varA {\tau_1}}
        \dsub \tau $
        & Rewrite the judgment \\
        & $\applye \cctx \tctx \bysub
        \applye \cctx {A_1 \subst \varA \genB \subst \genB {\tau_1}}
        \dsub \tau $
        & Rewrite the judgment \\
        & $\tctx = \tctx_1 , \genA, \tctx_2$
        & Given \\
        & $\tctx \exto \cctx$
        & Given \\
        & $\cctx = \cctx[\genA = \tau'] $
        & By Lemma~\ref{lemma:dunfield:ExtensionOrder} \\
        & $\applye {\cctx[\genB = \tau_1, \genA = \tau']} {\tctx_1, \genB,
          \genA, \tctx_2} \bysub
        \applye {\cctx[\genB = \tau_1, \genA = \tau']} {A_1 \subst \varA \genB}
        \dsub \tau $
        & Rewrite the judgment \\
        & $\tctx_1, \genB, \genA, \tctx_2 \bypsa A_1 \subst \varA \genB \sa \sigma \toctxr $
        & By induction hypothesis \\
        & $\ctxr = \ctxr_1, \genA, \ctxr_2$
        & As above \\
        & $\ctxr_1 \bywf \sigma $
        & As above \\
        & $\ctxr \exto \cctx_1 $
        & As above \\
        & $\cctx[\genB = \tau_1, \genA = \tau'] \exto \cctx_1 $
        & As above \\
        & $\applye {\cctx_1} \sigma = \tau$
        & As above \\
        & $\tctx_1, \genA, \tctx_2 \bypsa \forall \varA. A_1 \sa \sigma \toctxr $
        & By \rul{I-All-Plus} \\
        & $\cctx' = \cctx_1$
        & Choose \\
        & $\cctx[\genA = \tau'] \exto \cctx[\genB = \tau_1, \genA = \tau']$
        & By Lemma~\ref{lemma:dunfield:SolvedVariableAdditionForExtension} \\
        & $\cctx \exto \cctx_1$
        & By Lemma~\ref{lemma:dunfield:Transitivity} \\
      \end{longtable}
    \end{itemize}
\end{description}

\qed

\begin{corollary}[\PolymorphicTypeSanitizationCompletenessSubtypingName]
  \label{lemma:\PolymorphicTypeSanitizationCompletenessSubtypingName}
  \PolymorphicTypeSanitizationCompletenessSubtypingBody
\end{corollary}

\begin{description}
  \item [Part 1]
    \mbox{} % an empty line to make sure long table appear after proof
    \begin{longtable}[l]{lll}
      & $\applye \cctx \genA = \tau$
      & Assume \\
      & $\applye \cctx \tau = \tau$
      & Since $\tau$ contains no existential variables\\
      & $\genA \in unsolved(\tctx)$
      & Given \\
      & $\tctx = \tctx_1, \genA, \tctx_2$
      & Assume \\
      & $\cctx = \cctx_1, \genA = \tau', \cctx_2$
      & By Lemma~\ref{lemma:dunfield:ExtensionOrder} \\
      & $\tctx_1 \exto \cctx_1$
      & Same as above \\
      & $\applye \cctx \genA = \applye {\cctx_1} \genA = \tau$
      & By definition of context application \\
      & $\cctx_1 \bywf \tau$
      & By Lemma~\ref{lemma:dunfield:SubstitutionTyping} \\
      & Since $\tau$ contains no existential variables\\
      & And $\tctx_1$ contains all type variables in $\cctx_1$\\
      & $\tctx_1 \bywf \tau$
      & Follows directly \\
      & $\applye \cctx \tctx \bysub \tau \dsub \applye \cctx A $
      & Given \\
      & $\tctx[\genA] \bymsa A \sa \sigma \toctxr $
      & By Lemma~\ref{lemma:\PolymorphicTypeSanitizationCompletenessName} \\
      & $\ctxr = \ctxr_1, \genA, \ctxr_2 $
      & As above \\
      & $\ctxr_1 \bywf \sigma$
      & As above \\
      & $\ctxr \exto \cctx'$
      & As above \\
      & $\cctx \exto \cctx'$
      & As above \\
      & $\applye {\cctx'} \sigma =\tau$
      & As above \\
      & $\applye {\cctx'} \genA =\tau$
      & By Lemma~\ref{lemma:dunfield:FinishingTypes} \\
      & $\tctx[\genA] \bysub \genA \tsub A \toctxr_1, \genA = \sigma,
      \ctxr_2 $
      & By \rul{$\tsub$ InstL} \\
      & $\ctxr_1, \genA = \sigma, \ctxr_2
      \exto \cctx'
      $
      & By Lemma~\ref{lemma:dunfield:ParallelExtensionSolution}
    \end{longtable}
  \item [Part 2]
    Similar as Part 1.
\end{description}

\qed



\end{document}

%%% Local Variables:
%%% mode: latex
%%% TeX-master: t
%%% End:
