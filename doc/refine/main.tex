% SIGPLAN format
\documentclass[oribibl]{llncs}

\usepackage[dvipsnames]{xcolor}

% AMS packages
\usepackage{amsmath}
\usepackage{amssymb}
\usepackage{mathtools}
\usepackage{mdwlist}

% Hyper links
\usepackage{url}
\usepackage{hyperref}
\hypersetup{
   colorlinks,
   citecolor=black,
   filecolor=black,
   linkcolor=black,
   urlcolor=black
}

% Miscellaneous
\usepackage{paralist}
\usepackage{graphicx}
\usepackage{float}
% \usepackage{balance} % Equalize column in the last page

% Revision tools
\usepackage{xspace}
\usepackage{comment}

\usepackage{algpseudocode}
\usepackage{longtable}

\newcommand\authornote[3]{\textcolor{#2}{#1: #3}}
\newcommand\bruno[1]{\authornote{Bruno}{red}{#1}}
\newcommand\ningning[1]{\authornote{Ningning}{blue}{#1}}
\renewcommand\arraystretch{1.1}
%\renewcommand\bruno[1]{}
\renewcommand\ningning[1]{}

% Literal programming support

\usepackage{listings}
\lstdefinestyle{fun}{
  aboveskip=0.5\baselineskip,
  belowskip=0.5\baselineskip,
  keepspaces=true,
  mathescape=true,
  columns=fullflexible,
  basicstyle=\fontfamily{cmr}\selectfont\small,
  xleftmargin=10pt,
  keywordstyle=\bfseries,
  identifierstyle=\itshape,
  emphstyle=\fontfamily{cmss}\selectfont,
  commentstyle=\fontfamily{cmtt}\selectfont,
  emph={},
  keywords={where, let, in, case, of, data, newtype, ret, if, then,
    else, defrec, def},
  morecomment=[l]{--},
  literate={
    {->}{{$\to$}}1
    {=>}{{$\To$}}1
    {~}{{$\eq$}}1
    {\\}{{$\lambda$}}1
    {/}{{$\mu$}}1
    {\\~}{{$\lambda_\eq$}}1
    {|->}{{$\mapsto$}}1
    {~>}{{$\leadsto$}}1
    {*}{{$\star$}}1
    {**}{{$\times$}}1
    {++}{{+\kern-1.0ex+\kern1.1ex}}1
    {<}{{$\langle$}}1
    {>}{{$\rangle$}}1
    {|=}{{$\vDash$}}1
    {|-}{{$\vdash$}}1
    {||-}{{$\Vdash$}}1
    {G}{{$\Gamma$}}1
    {G0}{{$\cdot$}}1
    {|>}{{$\vartriangleright$}}1
    {forall}{{$\forall$}}1
    {'}{{$^\prime$}}1
    {_1}{{$_1$}}1
    {_2}{{$_2$}}1
    {^2}{{$^2$}}1
    {@}{{$\bullet$}}1
    {@t}{{$\tau$}}1
    {@s}{{$\sigma$}}1
    {castu2}{{$\fun{cast}_\uparrow^2$}}1
    {castd2}{{$\fun{cast}_\downarrow^2$}}1
    {castuf}{{$\fun{cast}_\uparrow^{\fun{f}}$}}1
    {castup}{{$\fun{cast}_\uparrow$}}1
    {castdn}{{$\fun{cast}_\downarrow$}}1
  }
}
\lstset{style=fun}
\newcommand{\lst}[1]{\text{\lstinline$#1$}}

% Table
\usepackage{multirow}
\usepackage{tabularx}
\newcolumntype{Y}{>{\centering\arraybackslash}X}
\newcolumntype{Z}{>{\raggedleft\arraybackslash}X}

% Infer rules
\usepackage{mathpartir}
\newcommand{\rname}[1]{{\,\text{\scriptsize \textsc{#1}}}}
\newcommand{\rul}[1]{\textsc{#1}}

% Extra symbols
\usepackage{stmaryrd}

% Macros for math typesetting

%% Names
% \newcommand{\name}{{\bf $\lambda_{\mu}^{\eq}$}\xspace}

%% Symbols
\newcommand{\syndef}{$::=$}
\newcommand{\synor}{$\mid$}
\newcommand{\syneq}{$\triangleq$}
\newcommand{\header}[1]{\multicolumn{1}{l}{$\boxed{#1}$}}
\newcommand{\headercap}[2]{\multicolumn{1}{l}{$\boxed{#1}$\quad{#2}}}
\newcommand{\headercapm}[2]{\vspace{1pt}\raggedright \framebox{\mbox{$#1$}} \quad
  #2}
\newcommand{\headercapt}[2]{\framebox{\mbox{$#1$}} \quad #2}
\newcommand{\marker}[1]{\blacktriangleright_{#1}}

%% Arrows
\newcommand{\To}{\Rightarrow}
\newcommand{\Chk}{\Downarrow}
\newcommand{\Inf}{\Uparrow}
\newcommand{\Inst}{{inst}}
\newcommand{\Gen}{{gen}}
\newcommand{\redto}{\hookrightarrow}
\newcommand{\redton}{\hookrightarrow^*}
\newcommand{\eq}{\sim}
\newcommand{\lt}{\sqsubseteq}
\newcommand{\sugar}{\triangleq}
\newcommand{\trto}[1]{\textcolor{Orchid}{\rightsquigarrow{#1}}}
\newcommand{\opt}[1]{}
\newcommand{\trtop}{\rightsquigarrow}

%% Styles
\newcommand{\kw}[1]{\operatorname{\mathbf{#1}}}
\newcommand{\var}{\mathit}
\newcommand{\fun}{\mathsf}

%% Constructs
\newcommand{\bind}[3]{#1 #2:#3.~}
\newcommand{\blam}{\bind \lambda}
\newcommand{\bmu}{\bind \mu}
\newcommand{\barr}[2]{(#1:#2) \to}

\newcommand{\bindv}[4][]{#2\,\overline{#3:#4}^{#1}.~}
\newcommand{\blamv}[3][]{\bindv[#1] \lambda {#2} {#3}}
\newcommand{\bmuv}[3][]{\bindv[#1] \mu {#2} {#3}}
\newcommand{\barrv}[3][]{\overline{#2:#3}^{#1} \to}

\newcommand{\eqlam}[2]{\lambda_{\eq}({#1} \eq {#2}).~}
\newcommand{\eqty}[2]{({#1} \eq {#2})\Rightarrow}
\newcommand{\eqapp}[2][]{\langle {#2} \rangle^{#1}}
\newcommand{\eqlamv}[3][]{\lambda_{\eq}\overline{{#2} \eq {#3}}^{#1}.~}
\newcommand{\eqtyv}[3][]{\overline{({#2} \eq {#3})}^{#1}\Rightarrow}

\newcommand{\bpi}{\bind \Pi}
\newcommand{\bpiv}[3][]{\bindv[#1] \Pi {#2} {#3}}

\newcommand{\fold}{\fun{fold}}
\newcommand{\unfold}{\fun{unfold}}

\newcommand{\castupz}{\fun{cast}_\uparrow}
\newcommand{\castup}[2][]{\fun{cast}_\uparrow^{#1}~[#2]~}
\newcommand{\castdnz}{\fun{cast}_\downarrow}
\newcommand{\castdn}[1][]{\fun{cast}_\downarrow^{#1}~}
\newcommand{\castdnt}[2][]{\fun{cast}_\downarrow^{#1}~[#2]~}
\newcommand{\subst}[2]{[#1 \mapsto #2]}
\newcommand{\Subst}[2]{\llbracket #1 \Mapsto #2 \rrbracket}
\newcommand{\substv}[3][]{\overline{[#2 \mapsto #3]}^{#1}}
\newcommand{\cast}[2][]{\fun{cast}^{#1}~[#2]~}

\newcommand{\triv}{\_}
\newcommand{\trivtm}{\bullet}
\newcommand{\er}[1]{|{#1}|}
\newcommand{\erf}[1]{\|{#1}\|}
\newcommand{\erlam}[1]{\lambda {#1}.~}
\newcommand{\ermu}[1]{\mu {#1}.~}
\newcommand{\ercastup}{\castupz~}
\newcommand{\ereqlam}{\lambda_\eq.~}

\newcommand{\genvar}{\widehat}
\newcommand{\genA}{\genvar{\alpha}}
\newcommand{\genB}{\genvar{\beta}}
\newcommand{\genC}{\genvar{\gamma}}
\newcommand{\typA} {\alpha}
\newcommand{\typB} {\beta}
\newcommand{\varA}{\alpha}
\newcommand{\varB}{\beta}
\newcommand{\sa}{\longrightarrow}

%% Context
\newcommand{\dctx}{\Psi}
\newcommand{\tctx}{\Gamma}
\newcommand{\sctx}{\Psi}
\newcommand{\lctx}{\Sigma}
\newcommand{\ctxsplit}{\shortmid}
\newcommand{\ctxinit}{\varnothing}
\newcommand{\ctxl}{\Theta}
\newcommand{\ctxr}{\Delta}
\newcommand{\cctx}{\Omega}
\newcommand{\byuni}{\vdash}
\newcommand{\bylessp}{\vdash}
\newcommand{\byinf}{\vdash}
\newcommand{\byfinf}{\vdash^F}
\newcommand{\bysa}{\vdash}
\newcommand{\infto}{{\color{MidnightBlue}~\Rightarrow~}}
\newcommand{\bychk}{\vdash}
\newcommand{\chkby}{{\color{Orange}~\Leftarrow~}}
\newcommand{\app}{\bullet}
\newcommand{\byall}{\vdash_\delta}
\newcommand{\bytar}{\vdash}
\newcommand{\byinst}{\vdash_\Inst}
\newcommand{\bygen}{\vdash_\Gen}
\newcommand{\bysub}{\vdash}
\newcommand{\bydsub}{\vdash}
\newcommand{\bycg}{\vdash}
\newcommand{\byrf}{\vdash}
\newcommand{\bywf}{\vdash}
\newcommand{\byapp}{\vdash}
\newcommand{\bywt}{\vDash}
\newcommand{\toctx}{\dashv \ctxl}
\newcommand{\toctxo}{\dashv \tctx}
\newcommand{\toctxr}{\dashv \ctxr}
\newcommand{\dpreinf}[1][]{\dctx {#1} \byinf}
\newcommand{\dprechk}[1][]{\dctx {#1} \bychk}
\newcommand{\dpreall}[1][]{\dctx {#1} \byall}
\newcommand{\dpreapp}[1][]{\dctx {#1} \byapp}
\newcommand{\dpreuni}[1][]{\dctx {#1} \byuni}
\newcommand{\dpretar}[1][]{\dtctx {#1} \bytar}
\newcommand{\dpreinst}[1][]{\dctx {#1} \byinst}
\newcommand{\dpregen}[1][]{\dctx {#1} \bygen}
\newcommand{\dprecg}[1][]{\dctx {#1} \bycg}
\newcommand{\dprewf}[1][]{\dctx {#1} \bywf}
\newcommand{\dprewt}[1][]{\dctx {#1} \bywt}
\newcommand{\tpreinf}[1][]{\tctx {#1} \byinf}
\newcommand{\fpreinf}[1][]{\tctx {#1} \byfinf}
\newcommand{\tprechk}[1][]{\tctx {#1} \bychk}
\newcommand{\tpreall}[1][]{\tctx {#1} \byall}
\newcommand{\tpreapp}[1][]{\tctx {#1} \byapp}
\newcommand{\tpreuni}[1][]{\tctx {#1} \byuni}
\newcommand{\tpretar}[1][]{\ttctx {#1} \bytar}
\newcommand{\tpreinst}[1][]{\tctx {#1} \byinst}
\newcommand{\tpregen}[1][]{\tctx {#1} \bygen}
\newcommand{\tprecg}[1][]{\tctx {#1} \bycg}
\newcommand{\tprewf}[1][]{\tctx {#1} \bywf}
\newcommand{\tprewt}[1][]{\tctx {#1} \bywt}
\newcommand{\tpresub}[1][]{\tctx {#1} \bysub}
\newcommand{\presub}[1][]{{#1} \bysub}
\newcommand{\dpresub}[1][]{\dctx {#1} \bydsub}
\newcommand{\wc}{\ \var{ctx}\ }
\newcommand{\bywa}{\vdash}
\newcommand{\reduce}{\hookrightarrow}
\newcommand{\cgto}{\longmapsto}
\newcommand{\rfto}{\rightsquigarrow}
\newcommand{\aeq}{\equiv_\alpha}
\newcommand{\uni}{\sim}
\newcommand{\lessp}{\sqsubseteq}
\newcommand{\dsub}{\leq}
\newcommand{\tsub}{{\color{BrickRed}~\cong~}}
\newcommand{\tsuper}{{\color{BrickRed}~:>~}}
\newcommand{\tvarinst}{:\leqq}
\newcommand{\tsubeither}{<:_m}
\newcommand{\elet}[3]{\kw{let} {#1} = {#2} \kw{in} {#3}}
\newcommand{\gen}[2]{\overbar{{#1}({#2})}}
\newcommand{\agen}[2]{{#1}_{agen}({#2})}
\newcommand{\tgen}[2]{{#1}_{gen}({#2})}

\newcommand{\erase}[1]{|{#1}|}
\newcommand{\etaeq}{\rightsquigarrow_{\eta id}}

\newcommand{\applye}[2]{[{#1}] #2}
\newcommand{\glb}{\rightsquigarrow}


% algorithm subsitution
\newcommand{\as}{S}
% algorithm name supply
\newcommand{\an}{N}
\newcommand{\byalgo}{\vdash}
\newcommand{\byarrow}{\vdash^\to}
\newcommand{\bypair}{\vdash^{()}}
\newcommand{\ato}{\hookrightarrow}
\newcommand{\acons}{\cdot}
\newcommand{\ex}[1]{\setminus_{#1}}


% Primitives
\newcommand{\nat}{Int}
\newcommand{\bool}{\var{Bool}}
\newcommand{\String}{\var{String}}
\newcommand{\bflet}{\textbf{let}\xspace}
\newcommand{\bfLet}{\textbf{Let}\xspace}
\newcommand{\fst}{\kw{fst}}
\newcommand{\snd}{\kw{snd}}
\newcommand{\unknown}{\star}
\newcommand{\match}{\triangleright}

\newcommand{\overbar}[1]{\mkern 1.5mu\overline{\mkern-1.5mu#1\mkern-1.5mu}\mkern 1.5mu}

%%% Local Variables:
%%% mode: latex
%%% TeX-master: "main"
%%% End:


\usepackage[round, sort]{natbib}
\bibliographystyle {plainnat}

\begin{document}
\title{Sanitized Type Inference in Context}

\author{Ningning Xie \and Bruno C. d. S. Oliveira}
\institute{The University of Hong Kong}

\maketitle

\begin{abstract}
  Gundry et al. proposed type inference in context as a general foundation for unification/type
  inference algorithms. The key idea is based on the notion of information increase. In this
  paper, we propose a strategy called \textit{type sanitization} that helps
  resolve the dependency between existential variables in the framework of type
  inference in context.\bruno{Why do we need help; what is the
    problem? Say this first} 
  We show that type sanitization works on a unification algorithm
  for a dependent type system with alpha-equality and first-order constraints. We
  then further extend this strategy to deal with polymorphic subtyping in a
  higher ranked polymorphic type system.
\end{abstract}

% Setup spaces between column
\setlength{\tabcolsep}{2pt}

% ------------------------------------------------------------------------
% TYPING RULES
% ------------------------------------------------------------------------

\newcommand*{\TAx}{\inferrule{\tctx \wc }{\tpreinf \star:\star \toctxo}\rname{T-Ax}}
\newcommand*{\TVar}{\inferrule{x:\tau \in \tctx \\ \tctx \wc}{\tpreinf x:\tau \toctxo
  }\rname{T-Var}}
\newcommand*{\TLetVar}{\inferrule{x:\sigma = \tau \in \tctx \\ \tctx \wc\\
    \tctx \byinst \sigma \lt \tau_2 \toctx}{\tpreinf x:\tau_2 \toctx
  }\rname{T-LetVar}}
\newcommand*{\TSub}{\inferrule{\tpreinf e : \tau_1 \toctx_1 \\ \ctxl_1 \byuni [\ctxl_1]\tau_1
    \lt [\ctxl_1]\tau_2 \toctx}{\tprechk e:\tau_2 \toctx }\rname{T-Sub}}
\newcommand*{\TAnn}{\inferrule{\tprechk \tau:\star \toctx_1 \\
    \ctxl_1 \bychk e:\tau \toctx }{\tpreinf (e:\tau):\tau \toctx
  }\rname{T-Ann}}
\newcommand*{\TLamInf}{\inferrule{\tpreinf[,\genA,x:\genA]
    e:\tau_2 \toctx, x:\genA, \ctxr }{\tpreinf \erlam x e : (\bpi x \genA
    [\ctxr]\tau_2) \toctx, UV(\ctxr) }\rname{T-Lam$\Inf$}}
\newcommand*{\TLamChk}{\inferrule{\tprechk[,x:\tau_1]
    e:\tau_2 \toctx,x:\tau_1,\ctxr \\
    \opt{\tprechk {\tau_1 : \star \toctx_1 }}
    }{\tprechk \erlam x e : \bpi x {\tau_1}
    \tau_2 \toctx }\rname{T-Lam$\Chk$}}
\newcommand*{\TLamAnnInf}{\inferrule{\tprechk \tau_1 : \star \toctx_1\\
    \ctxl_1,x:\tau_1 \byinf
    e:\tau_2 \toctx, x:\tau_1, \ctxr }{\tpreinf \blam x {\tau_1} e : (\bpi x {\tau_1}
    [\ctxr]\tau_2) \toctx, UV(\ctxr) }\rname{T-LamAnn$\Inf$}}
\newcommand*{\TLamAnnChk}{\inferrule{\tprechk \tau_1 : \star \toctx_1\\
    \ctxl_1 \byuni [\ctxl_1] \tau_1 \lt [\ctxl_1] \tau_3 \toctx_2 \\
    \ctxl_2,x:\tau_1 \bychk
    e:\tau_2 \toctx, x:\tau_1, \ctxr }
    {\tprechk \blam x {\tau_1} e : (\bpi x {\tau_3} \tau_2) \toctx }\rname{T-LamAnn$\Chk$}}
\newcommand*{\TApp}{\inferrule{
    \tpreinf e_1 : \tau_1 \toctx_1 \\
    \ctxl_1 \byapp [\ctxl_1]\tau_1~e_2 : \tau_2 \toctx \\
}{\tpreinf e_1~e_2:\tau_2 \toctx}\rname{T-App}}
\newcommand*{\TAppPi}{\inferrule{
    \tpreinf e_1 : \bpi x {\tau_1} \tau_2 \toctx_1 \\
    \ctxl_1 \bychk e_2 : [\ctxl_1]\tau_1 \toctx \\
}{\tpreinf e_1~e_2:\tau_2 \subst x
    {e_2} \toctx}\rname{T-AppPi}}
\newcommand*{\TAppVar}{\inferrule{
    \tpreinf e_1 : \genA \toctx_1[\genA] \\
    \ctxl_1[\genA_1,\genA_2,\genA=\bpi x {\genA_1} \genA_2] \bychk e_2 : \genA_1 \toctx \\
}{\tpreinf e_1~e_2:\genA_2 \toctx}\rname{T-AppVar}}
\newcommand*{\TPi}{\inferrule{\tprechk \tau_1 : \star \toctx_1 \\
\ctxl_1,x:\tau_1 \bychk \tau_2 : \star \toctx,x:\tau_1,\ctxr}{\tpreinf \bpi x {\tau_1} {\tau_2} :
    \star \toctx }\rname{T-Pi}}
\newcommand*{\TLet}{\inferrule{\tpreinf e_1 : \tau_1 \toctx_1  \\
\ctxl_1 \bygen {\tau_1} \lt \sigma \\
\ctxl_1, x:\sigma = e_1 \byall e_2 : \tau_2 \toctx, x:\sigma = e_1, \ctxr }{\tpreall \kw{let} x=e_1
\kw{in} e_2 : [x:\sigma=e_1, \ctxr]\tau_2 \toctx, UV(\ctxr) }\rname{T-Let}}
\newcommand*{\TCastUp}{\inferrule{[\tctx]\tau_2
    \redto \tau_1 \\
    \tprechk e : \tau_1 \toctx \\
    \opt{\tprechk \tau_1 : \star \toctx_1}
    }
  {\tprechk \ercastup e : \tau_2 \toctx
    }\rname{T-CastUp}}
\newcommand*{\TCastDn}{\inferrule{\tpreinf e : \tau_1 \toctx \\
    [\ctxl]\tau_1 \redto \tau_2}{\tpreinf \castdn e : \tau_2
    \toctx }\rname{T-CastDn}}

% DECLARATIVE

\newcommand*{\DAx}{\inferrule{ }{\dpreinf \star:\star \trto \star}\rname{D-Ax}}
\newcommand*{\DVar}{\inferrule{x:\tau \in \dctx}{\dpreinf x:\tau
  \trto x}\rname{D-Var}}
\newcommand*{\DLetVar}{\inferrule{x:\sigma = \tau \in \dctx \\ \dpreinst \sigma \lt \tau_2 \trto f}{\dpreinf x:\tau_2
  \trto {f~x}}\rname{D-LetVar}}
\newcommand*{\DSub}{\inferrule{\dpreinf e : \tau \trto{t}}
  {\dprechk e:\tau  \trto{t}}\rname{D-Sub}}
\newcommand*{\DAnn}{\inferrule{\dprechk \tau:\star \\
    \dctx \bychk e:\tau  \trto t}{\dpreinf (e:\tau):\tau
  \trto t}\rname{D-Ann}}
\newcommand*{\DLamInf}{\inferrule{ \dprechk \tau_1 : \star \trto {t_1} \\ \dpreinf[,x:\tau_1]
    e:\tau_2 \trto {t_2}}{\dpreinf \erlam x e : (\bpi x {\tau_1} {\tau_2})
    \trto {\blam x {t_1} {t_2}}}\rname{D-Lam$\Inf$}}
\newcommand*{\DLamChk}{\inferrule{\dprechk[,x:\tau_1]
    e:\tau_2 \trto {t_2} \\
    \opt{\dprechk {\tau_1 : \star  \trto {t_1}}}
    }{\dprechk \erlam x e : (\bpi x {\tau_1} \tau_2)  \trto {\blam x {t_1} t_2}}\rname{D-Lam$\Chk$}}
\newcommand*{\DLamAnnInf}{\inferrule{\dprechk \tau_1 : \star
    \trto {t_1} \\
    \dctx,x:\tau_1 \byinf
    e:\tau_2\trto {t_2}}{\dpreinf \blam x {\tau_1} e : (\bpi x {\tau_1}
    \tau_2) \trto {\blam x {t_1} t_2}}\rname{D-LamAnn$\Inf$}}
\newcommand*{\DLamAnnChk}{\inferrule{
    \dctx,x:\tau_1 \bychk
    e:\tau_2\trto {t_2}}{\dprechk \blam x {\tau_1} e : (\bpi x {\tau_1}
    \tau_2) \trto {\blam x {t_1} t_2}}\rname{D-LamAnn$\Chk$}}
\newcommand*{\DApp}{\inferrule{
    \dpreinf e_1 : \bpi  x {\tau_1} {\tau_2} \trto {t_1} \\
    \dprechk e_2 : \tau_1 \trto {t_2}
}{\dpreinf e_1~e_2:\tau_2 \subst x {e_2}  \trto {t_1~t_2}}\rname{D-App}}
\newcommand*{\DPi}{\inferrule{\dprechk \tau_1 : \star \trto {t_1} \\
\dctx,x:\tau_1 \bychk \tau_2 : \star \trto {t_2}}{\dpreinf \bpi x {\tau_1} {\tau_2} :
    \star \trto {\bpi x {t_1} t_2}}\rname{D-Pi}}
\newcommand*{\DLet}{\inferrule{\dpregen e_1 : \sigma \trto {t_1} \\
\dctx, x:\sigma = e_1 \byall e_2 : \tau_2 \trto {t_2}}{\dpreall \kw{let} x=e_1
\kw{in} e_2 : \tau_2 \subst x {e_1} \trto {\kw{let} x = t_1 \kw{in} t_2}}\rname{D-Let}}
\newcommand*{\DCastUp}{\inferrule{\tau_2 \redto \tau_1 \\
    \dprechk e : \tau_1 \trto {t_2} \\
    \opt{\dprechk \tau_1 : \star \trto {t_1}}
    }
  {\dprechk \ercastup e : \tau_2
    \trto {\castup {t_1} t_2}}\rname{D-CastUp}}
\newcommand*{\DCastDn}{\inferrule{\dpreinf e : \tau_1\trto t \\
    \tau_1 \redto \tau_2}{\dpreinf \castdn e : \tau_2
    \trto {\castdn t}}\rname{D-CastDn}}
\newcommand*{\DConv}{\inferrule{\dpreinf e_1 : \tau_1 \trto t \\ [\dctx]\tau_1 = [\dctx]\tau_2}
    {\dpreinf e_1 : \tau_2 \trto t}\rname{D-Conv}}

\newcommand*{\DPoly}{\inferrule{\dprechk[x:\star] \sigma : \star \trto {t}}
{\dpreinf \forall x:\star. \sigma : \star \trto {\bpi x \star t}}\rname{D-Poly}}

\newcommand*{\DInstantiation}{\inferrule{\dprechk \overbar{\tau} : \star \trto {\overbar t} \\
\sigma = \forall{\overbar{x:\star}}. \tau_1 \\
\opt{\dprechk \sigma : \star \trto{t_1}}
}
{\dpreinst \sigma \lt \tau_1[\overbar{x} \mapsto \overbar{\tau}] \trto {\blam x {t_1} x ~ \overbar{t}}
} \rname{D-Inst}}

\newcommand*{\DGeneralization}{\inferrule{ \dpreinf[, \overbar{x:\star}] e : \tau \trto {t_1}
\\  \overbar x \notin FV(e)}
{\dpregen e : \forall \overbar{x:\star}. \tau
\trto {(\blam {x_1} \star {\blam {x_2} \star {... \blam {x_n} \star {t_1}}})}} \rname{D-Gen}}

% ------------------------------------------------------------------------
% UNIFICATION RULES
% ------------------------------------------------------------------------

\newcommand*{\UVar}{\inferrule{ }{\tpreuni[{[x]}] x \lt x \toctxo[x]}\rname{U-Var}}
\newcommand*{\UEVarId}{\inferrule{ }{\tpreuni[{[\genA]}] \genA \lt \genA \toctxo[\genA]}\rname{U-EVarId}}
\newcommand*{\UEVarTy}{\inferrule{\genA \not \in \fun{FEV}(\tau_1) \\ \tctx[\genA] \bycg \tau_1 \cgto \tau_2 \toctx_1, \genA, \ctxl_2 \\ \ctxl_1 \bywt \tau_2}
{\tctx[\genA] \byuni \genA \lt \tau_1 \toctx_1, \genA=\tau_2, \ctxl_2}\rname{U-EvarTy}}
\newcommand*{\UTyEVar}{\inferrule{\genA \not \in \fun{FEV}(\tau_1) \\ \tctx[\genA] \bycg \tau_1 \cgto \tau_2 \toctx_1, \genA, \ctxl_2 \\ \ctxl_1 \bywt \tau_2}
{\tctx[\genA] \byuni \tau_1 \lt \genA \toctx_1, \genA=\tau_2, \ctxl_2}\rname{U-TyEVar}}
\newcommand*{\UStar}{\inferrule{ }{\tpreuni \star \lt \star \toctxo}\rname{U-Star}}
\newcommand*{\UApp}{\inferrule{\tpreuni \tau_2 \lt \tau_2' \toctx_1 \\
    \ctxl_1 \byuni [\ctxl_1]\tau_1 \lt [\ctxl_1]\tau_1'
    \toctx}{\tpreuni \tau_1~\tau_2 \lt \tau_1'~\tau_2'
    \toctx}\rname{U-App}}
\newcommand*{\ULam}{\inferrule{\tpreuni[,x] \tau \lt \tau'
    \toctx,x}{\tpreuni \erlam x \tau \lt \erlam x \tau' \toctx}\rname{U-Lam}}
\newcommand*{\ULamAnn}{\inferrule{\tpreuni \tau_1 \lt \tau_3 \toctx_1 \\
    \ctxl_1, x:\tau_1 \byuni [\ctxl_1]\tau_2 \lt [\ctxl_1]\tau_4
    \toctx,x:\tau_1}{\tpreuni \blam x {\tau_1} \tau_2 \lt \blam x
    {\tau_3} \tau_4 \toctx}\rname{U-LamAnn}}
\newcommand*{\UPi}{\inferrule{\tpreuni \tau_1' \lt \tau_1 \toctx_1
    \\ \ctxl_1,x:\tau_1 \byuni [\ctxl_1]\tau_2 \lt [\ctxl_1]\tau_2'
    \toctx,x:\tau_1}{\tpreuni \bpi x {\tau_1} \tau_2 \lt \bpi x
    {\tau_1'} \tau_2' \toctx}\rname{U-Pi}}
\newcommand*{\ULet}{\inferrule{\tpreuni \tau_1 \lt \tau_1' \toctx_1
    \\ \ctxl_1, x \byuni {[\ctxl_1]}\tau_2 \lt [\ctxl_1]\tau_2'
    \toctx, x}{\tpreuni \kw{let} x ={\tau_1} \kw{in} \tau_2 \lt \kw{let} x=
    {\tau_1'} \kw{in} \tau_2' \toctx}\rname{U-Let}}
\newcommand*{\UCastUp}{\inferrule{\tpreuni \tau \lt \tau'
    \toctx}{\tpreuni \ercastup \tau \lt \ercastup \tau' \toctx}\rname{U-CastUp}}
\newcommand*{\UCastDn}{\inferrule{\tpreuni \tau \lt \tau'
    \toctx}{\tpreuni \castdn \tau \lt \castdn \tau' \toctx}\rname{U-CastDn}}
\newcommand*{\UAnn}{\inferrule{\tpreuni \tau \lt \tau' \toctx_1 \\
    \ctxl_1 \byuni [\ctxl_1]e \lt [\ctxl_1]e'
    \toctx}{\tpreuni e:\tau \lt e':\tau'
    \toctx}\rname{U-Ann}}

% ------------------------------------------------------------------------
% APPLICATION RULES
% ------------------------------------------------------------------------

\newcommand*{\APi}{\inferrule{\tprechk e:\tau_1 \toctx \trto t}{\tpreapp (\bpi x
    {\tau_1} \tau_2)~e : \tau_2[x \mapsto e] \toctx \trto t}\rname{A-Pi}}
\newcommand*{\AEVar}{\inferrule{\tprechk[{[\genA_2,\genA_1,\genA=\bpi x
    {\genA_1} \genA_2]}] e : \genA_1 \toctx \trto t}{\tpreapp[{[\genA]}]
  \genA~e : \genA_2 \toctx \trto t}\rname{A-EVar}}

% ------------------------------------------------------------------------
% TARGET TYPING RULES
% ------------------------------------------------------------------------

\newcommand*{\EAx}{\inferrule{ }{\pretar \star:\star }\rname{E-Ax}}
\newcommand*{\EVar}{\inferrule{x:s \in \tctx}{\pretar x:s}\rname{E-Var}}
\newcommand*{\EApp}{\inferrule{\pretar t_1:\bpi x {s_1} {s_2} \\ \pretar
    t_2:s_1}{\pretar t_1~t_2:s_2 \subst
  x {t_2}}\rname{E-App}}
\newcommand*{\ELam}{\inferrule{\pretar t_1:\star \\ \pretar[,x:t_1] t_2:s_1
    }{\pretar \blam
    x {t_1} t_2 : \bpi x {t_1} {s_1}}\rname{E-Lam}}
\newcommand*{\EPi}{\inferrule{\pretar t_1:\star \\ \pretar[,x:t_1] t_2:\star }{\pretar \bpi x {t_1} t_2 :
    \star}\rname{E-Pi}}
\newcommand*{\ECastUp}{\inferrule{\pretar t_1 : \star \\ \pretar t_2: s_1 \\ t_1 \redto s_1 }{\pretar \castup {t_1} {t_2} :t_1}\rname{E-CastUp}}
\newcommand*{\ECastDown}{\inferrule{\pretar t_1 : s_1 \\ s_1 \redto s_2 }{\pretar \castdn t_1 :s_2}\rname{E-CastDown}}
\newcommand*{\ELet}{\inferrule{\pretar t_1 : s_1 \\ \pretar[,x:s_1=t_1] t_2:s_2}{\pretar \kw{let} x = t_1 \kw{in} t_2: s_2}\rname{E-Let}}
\newcommand*{\EConv}{\inferrule{\pretar t_1 : s_1 \\ [\tctx]s_1 = [\tctx]s_2}{\pretar t_1 : s_2}\rname{E-Conv}}

% ------------------------------------------------------------------------
% POLYMORPHISM
% ------------------------------------------------------------------------

\newcommand*{\Instantiation}{\inferrule{\sigma = \forall{\overbar{x:\star}}. \tau}{\tpreinst \sigma \lt \tau[\overbar{x} \mapsto \overbar{\genA}] \toctxo, \overbar{\genA}} \rname{A-Inst}}

\newcommand*{\Generalization}{\inferrule{\tau_2 = [\tctx]\tau \\ \overbar{\genA} = FEV(\tau_2) - FEV(\tctx)}{\tpregen \tau \lt \forall \overbar{x:\star}. \tau_2[\overbar{\genA} \mapsto \overbar{x}]} \rname{A-Gen}}

% ------------------------------------------------------------------------
% UNIFY TVAR
% ------------------------------------------------------------------------

\newcommand*{\IEVarA}{\inferrule{ }{\tctx[\genB][\genA] \bycg \genB \cgto \genB \toctxo[\genB][\genA]}\rname{I-EVar1}}
\newcommand*{\IEVarB}{\inferrule{ }{\tctx[\genA][\genB] \bycg \genB \cgto \genA_1 \toctxo[\genA_1, \genA][\genB=\genA_1]}\rname{I-EVar}}
\newcommand*{\IPi}{\inferrule{\tctx \bycg \tau_1 \cgto \tau_3 \toctx_1 \\ \ctxl_1 \bycg [\ctxl_1]\tau_2 \cgto \tau_4 \toctx }
{\tctx \bycg \bpi x {\tau_1} {\tau_2} \cgto \bpi x {\tau_3} {\tau_4} \toctx}\rname{I-Pi}}
\newcommand*{\IOthers}{\inferrule{  } {\tctx \bycg \tau \cgto \tau \toctxo}\rname{I-Other}}

% ------------------------------------------------------------------------
% WELL FORM
% ------------------------------------------------------------------------

\newcommand*{\WFPoly}{\inferrule{ \dprewf[,x:\star] \sigma}{ \dprewf \forall x: \star. \sigma}\rname{WF-Poly}}
\newcommand*{\WFOther}{\inferrule{ \dprechk \tau : \star}{ \dprewf \tau}\rname{WF-Other}}

\newcommand*{\TWFEVar}{\inferrule{\genA \in \tctx }{\tprewf \genA}\rname{WF-EVar}}
\newcommand*{\TWFPi}{\inferrule{\tprewf \tau_1\\ \tctx, x:\tau_1
    \bywf \tau_2}{\tprewf \bpi x {\tau_1} {\tau_2}}\rname{WF-Pi}}
\newcommand*{\TWFPoly}{\inferrule{\tprewf[, x: \star] \sigma
    }{ \tprewf \forall x: \star. \sigma}\rname{WF-Poly}}
\newcommand*{\TWFOther}{\inferrule{ \tprechk \tau : \star \toctxo, \ctxr}{ \tprewf \tau}\rname{WF-Other}}

\newcommand*{\WCEmpty}{\inferrule{ }{\ctxinit \wc}\rname{WC-Empty}}
\newcommand*{\WCVar}{\inferrule{\dctx \wc \\ x \notin dom(\dctx)}{\dctx, x \wc}\rname{WC-Var}}
\newcommand*{\WCTypedVar}{\inferrule{\dctx \wc \\ x \notin dom(\dctx) \\ \dctx \bywf \tau}{\dctx, x: \tau \wc}\rname{WC-TypedVar}}
\newcommand*{\WCLetVar}{\inferrule{\dctx \wc \\ x \notin dom(\dctx) \\ \dctx \bywf \sigma \\ \sigma = \forall {\overbar {y:\star}}.\tau \\ \dprechk[,\overbar{y:\star}] e:\tau}
{\dctx, x:\sigma = e}\rname{WC-LetVar}}

\newcommand*{\TWCEmpty}{\inferrule{ }{\ctxinit \wc}\rname{WC-Empty}}
\newcommand*{\TWCVar}{\inferrule{\tctx \wc \\ x \notin dom(\tctx)}{\tctx, x \wc}\rname{WC-Var}}
\newcommand*{\TWCTypedVar}{\inferrule{\tctx \wc\\ x \notin dom(\tctx) \\
    \tctx \bywf \tau}{\tctx, x: \tau \wc}\rname{WC-TypedVar}}
\newcommand*{\TWCLetVar}{\inferrule{\tctx \wc \\ x \notin dom(\tctx) \\
    \tctx\bywf \sigma\\
    \sigma = \forall {\overbar {y:\star}}.\tau \\
    \tctx ,\overbar{y:\star} \bychk e:[\tctx]\tau \toctxo, \overbar{y:\star}, \ctxr}
{\tctx, x:\sigma = e \wc}\rname{WC-LetVar}}
\newcommand*{\TWCEVar}{\inferrule{\tctx \wc\\ \genA \notin
    dom(\tctx)}{\tctx, \genA \wc }\rname{WC-EVar}}
\newcommand*{\TWCSolvedEVar}{\inferrule{\tctx \\ \genA \notin
    dom(\tctx) \\ \tctx\bywf \tau }{\tctx, \genA = \tau \wc
  }\rname{WC-SolvedEVar}}

\newcommand*{\TWTVar}{\inferrule{x \in \tctx}{\tprewt x}\rname{WT-Var}}
\newcommand*{\TWTEVar}{\inferrule{\genA \in \tctx}{\tprewt \genA}\rname{WT-EVar}}
\newcommand*{\TWTStar}{\inferrule{ }{\tprewt \star}\rname{WT-Star}}
\newcommand*{\TWTApp}{\inferrule{\tprewt e_1 \\ \tprewt e_2}{\tprewt e_1 ~ e_2}\rname{WT-App}}
\newcommand*{\TWTLam}{\inferrule{\tprewt[,x] e}{\tprewt \erlam x e}\rname{WT-Lam}}
\newcommand*{\TWTPi}{\inferrule{\tprewt \tau_1 \\ \tprewt[,x] \tau_2}{\tprewt \bpi x {\tau_1} {\tau_2}}\rname{WT-Pi}}
\newcommand*{\TWTLet}{\inferrule{\tprewt e_1 \\ \tprewt[,x] e_2}{\tprewt \kw{let} x = e_1 \kw{in} e_2}\rname{WT-Let}}
\newcommand*{\TWTCastUp}{\inferrule{\tprewt e}{\tprewt \castupz e}\rname{WT-CastUp}}
\newcommand*{\TWTCastDn}{\inferrule{\tprewt e}{\tprewt \castdn e}\rname{WT-CastDn}}
\newcommand*{\TWTAnn}{\inferrule{\tprewt e \\ \tprewt \tau}{\tprewt e:\tau}\rname{WT-Ann}}
\newcommand*{\TWTPoly}{\inferrule{\tprewt[,x] \sigma }{\tprewt \forall x:\star. \sigma}\rname{WT-Poly}}

% ------------------------------------------------------------------------
% TRANSLATION CONTEXT
% ------------------------------------------------------------------------

\newcommand*{\TCEmpty}{\inferrule{ } {\ctxinit \trtop \ctxinit}\rname{TC-Empty}}
\newcommand*{\TCTypedVar}{\inferrule{\tctx \trtop \tctx \\ \tprechk \tau : \star \trtop t} {\tctx, x:\tau \trtop \tctx, x:t}\rname{TC-TypedVar}}
\newcommand*{\TCLetVar}{\inferrule{\tctx \trtop \tctx \\ \tprechk \sigma : \star \trtop t_1 \\ \tpregen \tau : \sigma \trtop t_2
} {\tctx, x:\sigma = \tau \trtop \tctx, x:t_1 = t_2}\rname{TC-LetVar}}

% ------------------------------------------------------------------------
% CONTEXT EXTENSION
% ------------------------------------------------------------------------

\newcommand*{\CEEmtpy}{\inferrule{  }{\ctxinit \exto \ctxinit}\rname{CE-Empty}}
\newcommand*{\CEVar}{\inferrule{\tctx \exto \ctxr}{\tctx, x \exto \ctxr, x}\rname{CE-Var}}
\newcommand*{\CETypedVar}{\inferrule{\tctx \exto \ctxr \\ [\ctxr]\tau_1 = [\ctxr]\tau_2}{\tctx, x:\tau_1 \exto \ctxr, x:\tau_2}\rname{CE-TypedVar}}
\newcommand*{\CELetVar}{\inferrule{\tctx \exto \ctxr \\ [\ctxr]\sigma_1 = [\ctxr]\sigma_2 \\ [\ctxr]\tau_1 = [\ctxr]\tau_2}{\tctx, x:\sigma_1=\tau_1 \exto \ctxr, x:\sigma_2 = \tau_2}\rname{CE-LetVar}}
\newcommand*{\CEEVar}{\inferrule{\tctx \exto \ctxr}{\tctx, \genA \exto \ctxr, \genA}\rname{CE-EVar}}
\newcommand*{\CESolvedEVar}{\inferrule{\tctx \exto \ctxr \\ [\ctxr]\tau_1 = [\ctxr]\tau_2}{\tctx, \genA = \tau_1 \exto \ctxr, \genA = \tau_2}\rname{CE-SolvedEVar}}
\newcommand*{\CESolve}{\inferrule{\tctx \exto \ctxr \\ \ctxr \bywf \tau}{\tctx, \genA \exto \ctxr, \genA = \tau}\rname{CE-Solve}}
\newcommand*{\CEAdd}{\inferrule{\tctx \exto \ctxr}{\tctx \exto \ctxr, \genA}\rname{CE-Add}}
\newcommand*{\CEAddSolved}{\inferrule{\tctx \exto \ctxr \\ \ctxr \bywf \tau}{\tctx \exto \ctxr, \genA = \tau}\rname{CE-AddSolved}}

% ------------------------------------------------------------------------
% REFERENCE OF ORIGINAL SYSTEM
% ------------------------------------------------------------------------

\newcommand*{\OLamInf}{\inferrule{\tprechk[,\genA,\genB,x:\genA]
    e:\genB \toctx, x:\genA, \ctxr}{\tpreinf \erlam x e : (\bpi x \genA
    \genB) \toctx}\rname{$\rightarrow$ I $\Rightarrow$}}

\newcommand*{\OInstLArr}{\inferrule{\tpreuni[{[\genA_2, \genA_1, \genA = \genA_1 \to \genA_2]}] \genA_1 \lt A_1 \toctx_1 \\
    \ctxl_1 \byuni [\ctxl_1]A_2 \lt \genA_2 \toctx} {\tpreuni[{[\genA]}] \genA \lt A_1 \to A_2 \toctx}\rname{InstLArr}}

\newcommand*{\OInstLSolve}{\inferrule{\tctx \bywf \tau}{\tctx, \genA, \tctx' \byuni \genA \lt \tau \toctxo, \genA = \tau, \tctx'}\rname{InstLSolve}}

\newcommand*{\OInstLReach}{\inferrule{ }{\tpreuni[{[\genA][\genB]}] \genA \lt \genB \toctxo[\genA][\genB=\genA]}\rname{InstLReach}}

% ------------------------------------------------------------------------
% OPERATIONAL SEMANTICS
% ------------------------------------------------------------------------

\newcommand*{\SBetaA}{\inferrule{ }{(\blam x \tau {e_1}) e_2 \redto e_1 \subst x {e_2} }\rname{S-Beta}}
\newcommand*{\SBetaB}{\inferrule{ }{(\erlam x {e_1}) e_2 \redto e_1 \subst x {e_2}}\rname{S-Beta2}}
\newcommand*{\SApp}{\inferrule{ e_1 \redto e_1' }{e_1~e_2 \redto e_1'~e_2}\rname{S-App}}
\newcommand*{\SCastDownUp}{\inferrule{  }{\castdn (\ercastup e) \redto e}\rname{S-CastDownUp}}
\newcommand*{\SCastDown}{\inferrule{e \redto e'}{\castdn e \redto \castdn e'}\rname{S-CastDown}}
\newcommand*{\SLet}{\inferrule{ }{\kw{let} x = e_1 \kw{in} e_2 \redto e_2 \subst x {e_1}}\rname{S-Let}}
\newcommand*{\SAnn}{\inferrule{e \redto e'}{ e:\tau \redto e':\tau}\rname{S-Ann}}

% ------------------------------------------------------------------------
% EXAMPLES
% ------------------------------------------------------------------------

\newcommand*{\ExUni}{\inferrule{\genA \notin FEV(\bpi x \genB x) \quad
                                       \inferrule{\inferrule{ }
                                                            {\tctx,\genA,\genB,\ctxr \bycg \genB \cgto \genA_1 \toctx}\rname{I-Evar2}
                                                  \quad
                                                  \inferrule{ }
                                                            {\ctxl \bycg x \cgto x \toctx}\rname{I-Var}}
                                                 {\tctx,\genA,\genB,\ctxr \bycg \bpi x \genB x \cgto \bpi x {\genA_1} x \toctx}\rname{I-Other}
                                      \quad
                                      \inferrule{ %\inferrule{ }{\tctx,\genA_1 \bywf \genA_1}\rname{WF-EVar} \quad \inferrule{ }{\tctx,\genA_1,x \bywf x}\rname{WF-Var}
                                                 }
                                                {\tctx,\genA_1 \bywf \bpi x {\genA_1} x}\rname{WF-Pi}}
                               {\tctx,\genA,\genB,\ctxr \byuni \genA \lt \bpi x \genB x \toctxo, \genA_1, \genA=\bpi x {\genA_1} x, \genB=\genA_1, \ctxr} \rname{U-EvarTy}}

%%% Local Variables:
%%% mode: latex
%%% TeX-master: "../main"
%%% End:


\newcommand*{\ContextApplicationIsIdempotentName}{Context Application is Idempotent}
\newcommand*{\ContextApplicationIsIdempotentBody}{
  If $\tctx \wc$,
  then $\applye \tctx {\applye \tctx \sigma} = \applye \tctx \sigma$.
}

\newcommand*{\ContextApplicationPreservesTypingName}{Context Application Preserves Typing}
\newcommand*{\ContextApplicationPreservesTypingBody}{
  If $\tctx \byinf \sigma_1 \infto \sigma_2$,
  then $\tctx \byinf \applye \tctx {\sigma_1} \infto \sigma_2$.
}

\newcommand*{\ReverseContextApplicationPreservesTypingName}
{Reverse Context Application Preserves Typing}
\newcommand*{\ReverseContextApplicationPreservesTypingBody}{
  If $\tctx \byinf \applye \tctx {\sigma_1} \infto \sigma_2$,
  then $\tctx \byinf \sigma_1 \infto \sigma_2$.
}

\newcommand*{\OutputIsFullySubstitutedName}{Output is Fully Substituted}
\newcommand*{\OutputIsFullySubstitutedBody}{
  If $\tctx \byinf \sigma_1 \infto \sigma_2$,
  then $\applye \tctx {\sigma_2} = \sigma_2$.
}

\newcommand*{\ReductionPreservesFullySubstitutionName}{Reduction Preserves Fully Substitution}
\newcommand*{\ReductionPreservesFullySubstitutionBody}{
  If $\tctx \wc$,
  and $\applye \tctx {\sigma} = \sigma$,
  and $\sigma \redto \tau$,
  then $\applye \tctx \tau = \tau$.
}

\newcommand*{\ContextApplicationOverReductionName}{Context Application Over Reduction}
\newcommand*{\ContextApplicationOverReductionBody}{
  If $\sigma_1 \redto \sigma_2$,
  and $\applye \tctx {\sigma_1} \redto \applye \tctx {\sigma_2}$.
}

\newcommand*{\ContextApplicationInContextName}{Context Application In Context}
\newcommand*{\ContextApplicationInContextBody}{
  If $\tctx_1, y: \tau, \tctx_2 \byinf \sigma_1 \infto \sigma_2$,
  and $\tctx_1, y : \applye {\tctx_1} \tau, \tctx_2 \wc $,
  then $\tctx_1, y : \applye {\tctx_1} \tau, \tctx_2 \byinf \sigma_1 \infto \sigma_2$.
}

\newcommand*{\ReverseContextApplicationInContextName}{Reverse Context
  Application In Context}
\newcommand*{\ReverseContextApplicationInContextBody}{
  If $\tctx_1, y: \applye {\tctx_1} \tau, \tctx_2 \byinf \sigma_1 \infto \sigma_2$,
  and $\tctx_1, y : \tau, \tctx_2 \wc $,
  then $\tctx_1, y : \tau, \tctx_2 \byinf \sigma_1 \infto \sigma_2$.
}

\newcommand*{\TypingWeakeningName}{Typing Weakening}
\newcommand*{\TypingWeakeningBody}{
  If $\tctx_1, \tctx_2 \byinf \sigma_1 \infto \sigma_2$,
  and $\tctx_1, \ctxl, \tctx_2 \wc$,
  then $\tctx_1, \ctxl, \tctx_2 \byinf \sigma_1 \infto \sigma_2$.
}

\newcommand*{\TypingSubstitutionName}{Typing Substitution}
\newcommand*{\TypingSubstitutionBody}{
  If $\tctx \byinf \tau \infto \applye \tctx {\sigma_1}$,
  and $\tctx, x : \sigma_1 \byinf \tau' \infto \sigma_2 $,
  then $\tctx \byinf \tau' \subst x \tau \infto \sigma_2 $.
}

\newcommand*{\TypingContextWellFormednessName}{Typing Context Well Formedness}
\newcommand*{\TypingContextWellFormednessBody}{
  If $\tctx \byinf \tau \infto \sigma$,
  then $\tctx \wc$.
}

\newcommand*{\TypingStrengtheningName}{Typing Strengthening}
\newcommand*{\TypingStrengtheningBody}{
  If $\tctx_1, \tctx_2, \tctx_3 \byinf \tau \infto \sigma$,
  and $\tctx_1, \tctx_3 \wc $,
  and $\tctx_1, \tctx_3 \bywt \tau $,
  then $\tctx_1, \tctx_3 \byinf \tau \infto \sigma$.
}

\newcommand*{\TypingVariableExchangeName}{Typing Variable Exchange}
\newcommand*{\TypingVariableExchangeBody}{
  If $\tctx, x: \sigma_1, y :\sigma_2, \ctxl \byinf \tau_1 \infto \tau_2$,
  and $\tctx, y: \sigma_2, x :\sigma_1, \ctxl \wc$,
  then $\tctx, y: \sigma_2, x :\sigma_1, \ctxl \byinf \tau_1 \infto \tau_2$
  with the same size of typing derivation.
}

\newcommand*{\DeclarationPreservationName}{Declaration Preservation}
\newcommand*{\DeclarationPreservationBody}{
  If $\tctx \exto \ctxr$,
  and $u$ is a variable declared in $\tctx$,
  then $u$ is declared in $\ctxr$.
}

\newcommand*{\DeclarationOrderPreservationName}{Declaration Order Preservation}
\newcommand*{\DeclarationOrderPreservationBody}{
  If $\tctx \exto \ctxr$,
  and $u$ is declared to the left of $v$ in $\tctx$,
  then $u$ is declared to the left of $v$ in the $\ctxr$.
}

\newcommand*{\ReverseDeclarationOrderPreservationName}{Reverse Declaration Order Preservation}
\newcommand*{\ReverseDeclarationOrderPreservationBody}{
  If $\tctx \exto \ctxr$,
  and $u$ and $v$ are both declared in $\tctx$,
  and $u$ is declared to the left of $v$ in $\ctxr$,
  then $u$ is declared to the left of $v$ in the $\tctx$.
}

\newcommand*{\SubstitutionExtensionInvarianceName}{Substitution Extension Invariance}
\newcommand*{\SubstitutionExtensionInvarianceBody}{
  If $\tctx \byinf \sigma \infto \sigma'$,
  and $\tctx \exto \ctxr$,
  then $\applye \ctxr \sigma = \applye \ctxr {\applye \tctx \sigma}$,
  and $\applye \ctxr \sigma = \applye \tctx {\applye \ctxr \sigma}$.
}

\newcommand*{\ExtensionEqualityPreservationName}{Extension Equality Preservation}
\newcommand*{\ExtensionEqualityPreservationBody}{
  If $\tctx \byinf \sigma_1 \infto \tau_1$,
  and $\tctx \byinf \sigma_2 \infto \tau_2$,
  and $\applye \tctx {\sigma_1} = \applye \tctx {\sigma_2}$,
  and $\tctx \exto \ctxr$,
  then $\applye \ctxr {\sigma_1} = \applye \ctxr {\sigma_2}$.
}

\newcommand*{\ContextExtensionReflexivityName}{Reflexivity of Context Extension}
\newcommand*{\ContextExtensionReflexivityBody}{
  If $\tctx \wc$,
  then $\tctx \exto \tctx$.
}

\newcommand*{\ContextExtensionTransitivityName}{Transitivity of Context Extension}
\newcommand*{\ContextExtensionTransitivityBody}{
  Given $\ctxl, \tctx, \ctxr$ are all well-formed contexts,
  if $\ctxl \exto \tctx$,
  and $\tctx \exto \ctxr$,
  then $\ctxl \exto \ctxr$.
}

\newcommand*{\ContextExtensionPreservesContextWellFormednessName}{Context Extension Preserves Context Well Formedness}
\newcommand*{\ContextExtensionPreservesContextWellFormednessBody}{
  If $\tctx \wc$,
  and $\tctx \exto \ctxr$,
  then $\ctxr \wc$.
}

\newcommand*{\ReverseDeclarationPreservationName}{Reverse Declaration Preservation}
\newcommand*{\ReverseDeclarationPreservationBody}{
  If $\tctx \exto \ctxl$,
  and $u$ is a variable that $u \notin \ctxl$,
  then $u \notin \tctx$.
}

\newcommand*{\RightSoftnessName}{Right Softness}
\newcommand*{\RightSoftnessBody}{
  If $\tctx \exto \ctxr$,
  and $\ctxl$ is soft,
  and $\ctxr, \ctxl$ is well formed,
  then $\tctx \exto \ctxr, \ctxl$.
}

\newcommand*{\ExtensionOrderName}{Extension Order}
\newcommand*{\ExtensionOrderBody}{
  \begin{itemize}
  \item If $\tctx_L, y : \sigma, \tctx_R \exto \ctxr$,
    then $\ctxr = \ctxr_L, y : \sigma, \ctxr_R$,
    and $\tctx_L \exto \ctxr_L$,
    and $\ctxr_R$ is soft if and only if $\tctx_R$ is soft.
  \item If $\tctx_L, \genA , \tctx_R \exto \ctxr$,
    then $\ctxr = \ctxr_L, \ctxl, \ctxr_R$,
    and $\tctx_L \exto \ctxr_L$,
    and $\ctxl$ is either $\genA$ or $\genA = \tau$ for some $\tau$,
    and $\ctxr_R$ is soft if and only if $\tctx_R$ is soft.
  \item If $\tctx_L, \genA = \sigma_1, \tctx_R \exto \ctxr$,
    then $\ctxr = \ctxr_L, \genA = \sigma_2, \ctxr_R$,
    and $\tctx_L \exto \ctxr_L$,
    and $\applye {\ctxr_L} {\sigma_1} = \applye {\ctxr_L} {\sigma_2}$,
    and $\ctxr_R$ is soft if and only if $\tctx_R$ is soft.
  \end{itemize}
}

\newcommand*{\ExtensionWeakningName}{Extension Weakening}
\newcommand*{\ExtensionWeakningBody}{
  If $\tctx \byinf \tau_1 \infto \tau_2$,
  and $\tctx \exto \ctxr$,
  and $\ctxr \wc$,
  then $\ctxr \byinf \tau_1 \infto \applye \ctxr {\tau_2}$.
}

\newcommand*{\ExtensionWeakningWellFormednessName}{Extension Weakening Well Formedness}
\newcommand*{\ExtensionWeakningWellFormednessBody}{
  If $\tctx \bywf \tau$,
  and $\tctx \exto \ctxr$,
  and $\ctxr \wc$,
  then $\ctxr \bywf \tau$.
}

\newcommand*{\ExtensionWeakeningWellScopednessName}
{Extension Weakening Well Scopedness}
\newcommand*{\ExtensionWeakeningWellScopednessBody}{
  If $\tctx \bywt \tau$,
  and $\tctx \exto \ctxr$,
  then $\ctxr \bywt \tau$.
}


\newcommand*{\SolutionAdmissibilityForExtensionName}{Solution Admissibility for Extension}
\newcommand*{\SolutionAdmissibilityForExtensionBody}{
  If $\tctx_L, \genA, \tctx_R \wc$,
  and $\tctx_L \bywf \tau$,
  then $\tctx_L, \genA, \tctx_R \exto \tctx_L, \genA = \tau, \tctx_R$.
}

\newcommand*{\UnsolvedVariableAdditionForExtensionName}
{Unsolved Variable Addition for Extension}
\newcommand*{\UnsolvedVariableAdditionForExtensionBody}{
  If $\tctx_L, \tctx_R \wc$,
  and $\genA \notin \tctx_L, \tctx_R$,
  then $\tctx_L, \tctx_R \exto \tctx_L, \genA, \tctx_R$.
}

\newcommand*{\SolvedVariableAdditionForExtensionName}
{Solved Variable Addition for Extension}
\newcommand*{\SolvedVariableAdditionForExtensionBody}{
  If $\tctx_L, \tctx_R \wc$,
  and $\genA \notin \tctx_L, \tctx_R$,
  and $\tctx_L \bywf \tau$,
  then $\tctx_L, \tctx_R \exto \tctx_L, \genA = \tau, \tctx_R$.
}

\newcommand*{\ParallelAdmissibilityName}
{Parallel Admissibility}
\newcommand*{\ParallelAdmissibilityBody}{
  If $\tctx_L \exto \ctxr_L $,
  and $\tctx_L, \tctx_R \exto \ctxr_L, \ctxr_R $,
  and $\tctx_R$ has no variable overlapped with $\ctxr_L$,
  and $\genA \notin \tctx_L, \tctx_R, \ctxr_L, \ctxr_R $,
  then:
  \begin{itemize}
  \item $\tctx_L, \genA, \tctx_R \exto \ctxr_L, \genA, \ctxr_R $.
  \item If $\ctxr_L \bywf \tau$,
    then $\tctx_L, \genA, \tctx_R \exto \ctxr_L, \genA = \tau, \ctxr_R $.
  \item If $\tctx_L \bywf \tau$,
    and $\tctx_L \bywf \tau_1$,
    and $\ctxr_L \bywf \tau_2$,
    and $\applye {\ctxr_L} {\tau_1} = \applye {\ctxr_L} {\tau_2}$,
    then $\tctx_L, \genA = \tau_1, \tctx_R \exto \ctxr_L, \genA = \tau_2, \ctxr_R $.
  \item If $\tctx_L \bywf \tau$,
    then $\tctx_L, x : \tau, \tctx_R \exto \ctxr_L, x: \tau, \ctxr_R $.
  \end{itemize}
}

\newcommand*{\ParallelExtensionSolutionName}
{Parallel Extension Solution}
\newcommand*{\ParallelExtensionSolutionBody}{
  If $\tctx_L, \genA, \tctx_R \exto \ctxr_L, \genA = \tau_2, \ctxr_R $,
  and $\tctx_L \bywf \tau_1$,
  and $\applye {\ctxr_L} {\tau_1} = \applye {\ctxr_L} {\tau_2}$,
  then $\tctx_L, \genA = \tau_1, \tctx_R \exto \ctxr_L, \genA = \tau_2, \ctxr_R $.
}

% ------------------------------------------------------
% COMPLETE CONTEXTS
% ------------------------------------------------------

\newcommand*{\StabilityOfCompleteContextsName}
{Stability of Complete Contexts}
\newcommand*{\StabilityOfCompleteContextsBody}{
  If $\tctx \exto \cctx$,
  then $\applye \cctx \cctx = \applye \cctx \tctx$.
}

\newcommand*{\FinishingTypesName}
{Finishing Types}
\newcommand*{\FinishingTypesBody}{
  If $\cctx \byinf \sigma \infto \tau$,
  and $\cctx \exto \cctx'$,
  then $\applye \cctx \sigma = \applye {\cctx'} \sigma$.
}

\newcommand*{\FinishingCompletionsName}
{Finishing Completions}
\newcommand*{\FinishingCompletionsBody}{
  If $\cctx \exto \cctx'$,
  then $\applye \cctx \cctx = \applye {\cctx'} {\cctx'}$.
}

\newcommand*{\ConfluenceOfCompletenessName}
{Confluence of Completeness}
\newcommand*{\ConfluenceOfCompletenessBody}{
  If $\ctxr_1 \exto \cctx$,
  and $\ctxr_2 \exto \cctx$,
  then $\applye \cctx {\ctxr_1} = \applye \cctx {\ctxr_2}$.
}

% ------------------------------------------------------
% TYPE SANITIZATION
% ------------------------------------------------------

\newcommand*{\TypeSanitizationExtensionName}
{Type Sanitization Extension}
\newcommand*{\TypeSanitizationExtensionBody}{
  If $\tctx \bysa \tau_1 \sa \tau_2 \toctx$,
  and $\tctx \byinf \tau_1 \infto \sigma$,
  then $\tctx \exto \ctxl$.
}

\newcommand*{\TypeSanitizationEquivalenceName}
{Type Sanitization Equivalence}
\newcommand*{\TypeSanitizationEquivalenceBody}{
  If $\tctx \bysa \tau_1 \sa \tau_2 \toctx$,
  and $\tctx \byinf \tau_1 \infto \sigma$,
  then $\applye \ctxl {\tau_1} = \applye \ctxl {\tau_2}$.
}

\newcommand*{\TypeSanitizationWellFormednessName}
{Type Sanitization Well Formedness}
\newcommand*{\TypeSanitizationWellFormednessBody}{
  If $\tctx \bysa \tau_1 \sa \tau_2 \toctx$,
  and $\tctx \byinf \tau_1 \infto \sigma$,
  then $\ctxl \byinf \tau_2 \infto \applye \ctxl \sigma$.
}

\newcommand*{\TypeSanitizationTailUnchangedName}
{Type Sanitization Tail Unchanged}
\newcommand*{\TypeSanitizationTailUnchangedBody}{
  If $v$ is a binding,
  and $\tctx, v \bysa \tau_1 \sa \tau_2 \toctx$,
  then $\ctxl = \ctxl', v$.
}

\newcommand*{\TypeSanitizationCompletenessName}
{Type Sanitization Completeness}
\newcommand*{\TypeSanitizationCompletenessBody}{
  Given $\tctx \exto \cctx$,
  where $\tctx = \tctx_1, \genA, \tctx_2, \tctx_0$,
  and $\tctx_0$ only contains variables,
  and $\applye \tctx {\sigma_1} = \sigma_1$,
  and $\applye \cctx \tau = \tau$,
  and $\genA \notin FV(\sigma_1)$,
  and $\tctx_1, \tctx_0 \bywt \tau $.
  If $\applye \cctx \tctx \bywf \tau = \applye \cctx {\sigma_1} $,
  then there are $\ctxr, \cctx'$
  such that $\cctx \exto \cctx'$,
  and $\ctxr \exto \cctx'$,
  and $\tctx[\genA] \bysa \sigma_1 \sa \sigma_2 \toctxr$,
  where $\ctxr = \ctxr_1, \genA ,\ctxr_2, \tctx_0$,
  and $\ctxr_1, \tctx_0 \bywt \sigma_2$.
}

\newcommand*{\TypeSanitizationCompletenessPrettyName}
{Type Sanitization Completeness Pretty}
\newcommand*{\TypeSanitizationCompletenessPrettyBody}{
  Given $\tctx \exto \cctx$,
  where $\tctx = \tctx_1, \genA, \tctx_2$,
  and $\applye \tctx {\sigma_1} = \sigma_1$,
  and $\applye \cctx \tau = \tau$,
  and $\genA \notin FV(\sigma_1)$,
  and $\tctx_1 \bywt \tau $.
  If $\applye \cctx \tctx \bywf \tau = \applye \cctx {\sigma_1} $,
  then there are $\ctxr, \cctx'$
  such that $\cctx \exto \cctx'$,
  and $\ctxr \exto \cctx'$,
  and $\tctx[\genA] \bysa \sigma_1 \sa \sigma_2 \toctxr$,
  where $\ctxr = \ctxr_1, \genA ,\ctxr_2$,
  and $\ctxr_1 \bywt \sigma_2$.
}

\newcommand*{\TypeSanitizationCompletenessUnificationName}
{Type Sanitization Completeness w.r.t. Unification}
\newcommand*{\TypeSanitizationCompletenessUnificationBody}{
  Given $\tctx \exto \cctx$,
  where $\tctx = \tctx_1, \genA, \tctx_2$,
  and $\applye \tctx {\sigma_1} = \sigma_1$,
  and $\genA \notin FV(\sigma_1)$.
  \begin{itemize}
  \item If $\applye \cctx \tctx \bywf \applye \cctx \genA
    = \applye \cctx {\sigma_1} $,
    then there are $\ctxr, \cctx'$
    such that $\cctx \exto \cctx'$,
    and $\ctxr \exto \cctx'$,
    and $\tctx[\genA] \byuni \genA \uni \sigma_1 \toctxr$.
  \item If $\applye \cctx \tctx \bywf \applye \cctx {\sigma_1}
    = \applye \cctx {\genA} $,
    then there are $\ctxr, \cctx'$
    such that $\cctx \exto \cctx'$,
    and $\ctxr \exto \cctx'$,
    and $\tctx[\genA] \byuni \sigma_1 \uni \genA \toctxr$.
  \end{itemize}
}

% ------------------------------------------------------
% UNIFICATION
% ------------------------------------------------------

\newcommand*{\UnificationExtensionName}
{Unification Extension}
\newcommand*{\UnificationExtensionBody}{
  \begin{itemize}
  \item If $\tctx \byeuni e_1 \uni e_2 \toctx$,
    and $\tctx \byinf e_1 \infto \sigma_1'$,
    and $\tctx \byinf e_2 \infto \sigma_2'$,
    then $\tctx \exto \ctxl$.
  \item If $\tctx \bysuni \tau_1 \uni \tau_2 \toctx$,
    and $\tctx \byinf \tau_1 \infto \star $,
    and $\tctx \byinf \tau_2 \infto \star $,
    then $\tctx \exto \ctxl$.
  \end{itemize}
}


\newcommand*{\UnificationEquivalenceName}
{Unification Equivalence}
\newcommand*{\UnificationEquivalenceBody}{
  If $\tctx \bybuni \tau_1 \uni \tau_2 \toctx$,
  and $\tctx \byinf \tau_1 \infto \sigma_1'$,
  and $\tctx \byinf \tau_2 \infto \sigma_2'$,
  then $\applye \ctxl {\tau_1} = \applye \ctxl {\tau_2}$.
}

\newcommand*{\UnificationCompletenessName}
{Unification Completeness}
\newcommand*{\UnificationCompletenessBody}{
  Given $\tctx \exto \cctx$,
  and $\applye \tctx {\sigma_1} = \sigma_1$,
  and $\applye \tctx {\sigma_2} = \sigma_2$.
  If $\applye \cctx \tctx \byinf
  \applye \cctx {\sigma_1} = \applye \cctx {\sigma_2} \infto \tau $,
  then there are $\ctxr, \cctx'$ such that
  $\cctx \exto \cctx'$,
  and $\ctxr \exto \cctx'$,
  and $\tctx \bybuni \sigma_1 \uni \sigma_2 \toctxr$
  for any $\delta$ suitable.
}

% ------------------------------------------------------
% POLYMORPHIC TYPE SANITIZATION
% ------------------------------------------------------

\newcommand*{\PolymorphicTypeSanitizationExtensionName}
{Polymorphic Type Sanitization Extension}
\newcommand*{\PolymorphicTypeSanitizationExtensionBody}{
  If $\tctx \bybsa A \sa \sigma \toctx$,
  then $\tctx \exto \ctxl$.
}

\newcommand*{\PolymorphicTypeSanitizationSoundnessName}
{Polymorphic Type Sanitization Soundness}
\newcommand*{\PolymorphicTypeSanitizationSoundnessBody}{
  Given $\ctxl \exto \cctx$,
  and $\applye \tctx A = A$:
  \begin{itemize}
  \item
    If $\tctx[\genA] \bypsa A \sa \sigma \toctx$,
    then $\applye \cctx \ctxl \bysub \applye \cctx A \dsub \applye \cctx \sigma$.
  \item
    If $\tctx[\genA] \bymsa A \sa \sigma \toctx$,
    then $\applye \cctx \ctxl \bysub \applye \cctx \sigma \dsub \applye \cctx A$.
  \end{itemize}
}

\newcommand*{\PolymorphicTypeSanitizationCompletenessName}
{Polymorphic Type Sanitization Completeness}
\newcommand*{\PolymorphicTypeSanitizationCompletenessBody}{
  Given $\tctx \exto \cctx $,
  where $\tctx = \tctx_1, \genA, \tctx_2$,
  and $\applye \tctx A = A $,
  and $\applye \cctx {\tau} = \tau $,
  and $\genA \notin FV(A)$,
  and ${\tctx_1} \bywf \tau $
  :
  \begin{itemize}
  \item
    If $\applye \cctx \tctx \bysub \tau \dsub \applye \cctx A$,
    then there are $\ctxr, \cctx'$
    such that $\cctx \exto \cctx'$,
    and $\ctxr \exto \cctx'$,
    and $\tctx[\genA] \bymsa A \sa \sigma \toctxr$,
    where $\ctxr = \ctxr_1, \genA, \ctxr_2$ ,
    and $\ctxr_1 \bywf \sigma$,
    and $\applye {\cctx'} \sigma = \tau$.
  \item
    If $\applye \cctx \tctx \bysub \applye \cctx A \dsub \tau$,
    then there are $\ctxr, \cctx'$
    such that $\cctx \exto \cctx'$,
    and $\ctxr \exto \cctx'$,
    and $\tctx[\genA] \bypsa A \sa \sigma \toctxr$,
    where $\ctxr = \ctxr_1, \genA, \ctxr_2$ ,
    and $\ctxr_1 \bywf \sigma$,
    and $\applye {\cctx'} \sigma = \tau$.
  \end{itemize}
}

\newcommand*{\PolymorphicTypeSanitizationCompletenessSubtypingName}
{Polymorphic Type Sanitization Completeness w.r.t Subtyping}
\newcommand*{\PolymorphicTypeSanitizationCompletenessSubtypingBody}{
  Given $\tctx \exto \cctx $,
  and $\applye \tctx A = A $,
  and $\genA \notin FV(A)$,
  and $\genA \in unsolved(\tctx)$
  :
  \begin{itemize}
  \item
    If $\applye \cctx \tctx \bysub \applye \cctx \genA \dsub \applye \cctx A$,
    then there are $\ctxr, \cctx'$
    such that $\cctx \exto \cctx'$,
    and $\ctxr \exto \cctx'$,
    and $\tctx[\genA] \bysub \genA \tsub A \toctxr$.
  \item
    If $\applye \cctx \tctx \bysub \applye \cctx A \dsub \applye \cctx \genA$,
    then there are $\ctxr, \cctx'$
    such that $\cctx \exto \cctx'$,
    and $\ctxr \exto \cctx'$,
    and $\tctx[\genA] \bysub A \tsub \genA \toctxr$.
  \end{itemize}
}


\section{Introduction}

Considering unification and/or type inference from a very general perspective
helps to establish a common foundation to those kind of problems, which makes it
applicative to various features. One of those perspectives is from information
increase, which is specifically implemented by having contexts recording all the
information including unification variables, with contexts as both the input and
the output of the problem. \citet{gundry2010type} presents an implementation of
Hindley-Milner \citep{damas1982principal,hindley69principal} type inference
based on this idea to present a methodological understanding of this strategy.
Also, \citet{dunfield2009greedy} uses this strategy to implement a greedy
bidirectional typechecking algorithm for inferring polymorphic instances. Later
on, \citet{dunfield2013complete} prove formal propositions of this strategy in a
bi-directional algorithm for higher rank polymorphism.

As mentioned in \citet{gundry2010type}, a longer-term objective of this strategy
is to explain the elaboration of high-level dependently typed programs into
fully explicit calculi. However, this objective is not achieved yet, which is
possibly due to the complication of doing unification and/or type inference
for dependent type systems.

Dependent types are currently increasingly adopted in many language designs due
to its expressiveness \citep{xi1999dependent, licata2005formulation,
  pasalic2006concoqtion, mckinna2006dependent, norell2009dependently,
  brady2013idris}. However, it is known that complete type inference or
unification for dependent type system is generally undecidable. Still, there are
many existing literatures aiming at providing some extent of unification for
dependent type calculi \citep{ziliani2015unification, abel2011higher,
  elliott1989higher}. While supporting many advanced features, those unification
algorithm are complicated and hard to reason about.

Due to the sophistication of type checking for dependent types, many recent
studies \citep{van2013explicit, kimmell2012equational, sjoberg2012irrelevance,
  sjoberg2015programming, stump2009verified, sulzmann2007system,
  yang2016unified} attempt to use explicit casts to manage type-level
computations. Casts are language constructs that control beta-reduction and
beta-expansion of the type of an expression. Therefore, type equality is based
on alpha-equality, and beta reductions are managed explicitly. In those type
systems, we can have general recursion without losing decidable type checking.

In this paper, we investigate how to adopt the approach of information increase
in a unification algorithm for a dependent type system with
explicit casts. We propose a strategy called \textit{type sanitization} that
resolves the dependency problem introduced by dependent types. As we will see,
this strategy simplifies the original approach of type inference in context.
Also, to show that type sanitization is applicable to other features, we also
extend it to \textit{polymorphic type sanitization} that deals with polymorphic
subtyping in a higher rank polymorphic type system \citep{dunfield2013complete}.
Specifically, our main contributions are:

\begin{itemize}
\item \textbf{A strategy called \textit{type sanitization}} that helps resolve
  the dependency between existential variables in the context, which simplifies
  but also strengthens the approach of type inference in context
  \citep{gundry2010type} so that information increase can be reasoned in a more
  syntactic way to ease the meta-theory proving. Also, we show that our strategy
  is applicable to advanced features, such as dependent types.
\item \textbf{A specification of a unification algorithm} for dependent type
  systems with controls over type-level computations and first-order
  constraints. The algorithm is remarkably simple and predictable. We prove that
  the algorithm is sound and complete.
\item \textbf{A replacement of the higher-rank type instantiation} for an
  implicitly polymorphic type system with higher rank types
  \citep{dunfield2013complete}. The design of \textit{polymorphic type
    sanitization} simplifies the subtyping for existential variables, and also
  removes the problem of duplication in original instantiation, while preserves
  the completeness and soundness of subtyping.
\end{itemize}

%%% Local Variables:
%%% mode: latex
%%% TeX-master: "../main"
%%% org-ref-default-bibliography: "citation.bib"
%%% End:
\section{Overview}
\label{sec:overview}

This section provides background and gives an overview of our work.

\subsection{Background and Motivation}

In this section, we discuss the background of type inference in context, also a
variant of this approach in a higher rank polymorphic type system. While
presenting the key ideas of the work, we also talk about the challenges of each
approach, which motivates our work.

\subsubsection{Type Inference In Context.}

\citet{gundry2010type} models unification and type inference from a general
perspective of information increase in the problem context, based on the
invariant that types can only depend on bindings appearing earlier in the
context.

Specifically, information and constraints about variables are stored in the
context, which is a list maintaining information order. For example\footnote{We
  adopt our notations and terminologies for the examples}:

$\tctx_1 = \genA, \genB, x :\genB$

Here $\genA, \genB$ are existential variables waiting to be solved, whose
meaning is given by solutions:

$\tctx_2 = \genA, \genB = \Int, x :\genB$

Then unification problem becomes finding a more informative context that
contains solutions for the existential variables so that two expressions are
equivalent up to substitution of the solutions. For
example, $\tctx_2$ can be the solution context for unifying
$\Int$ and $\genB$ under $\tctx_1$.

Besides contexts being ordered, a key insight of the approach lies in how to
unify existential variables with other types. In this case, unification needs to
resolve the dependency between existential variables. Consider unifying $\genA$
with $\genB \to \genB$ under context $\genA, \genB, x : \genB$. Here $\genB$ is
out of the scope of $\genA$. The way they solve it is to examine variables in
the context from the tail to the head, \textit{moves segments of context to the
  left if necessary}, until the existential variable being unified is found.
This design is implemented by an additional context that records the context
needed to be moved. This can be interpreted from the judgment $\tctx \ctxsplit
\Xi \byuni \genA \equiv \tau \toctx$, which is read as: given input context
$\tctx$, $\Xi$, solve $\genA$ with $\tau$ succeeds and produces an output
context $\ctxl$.

The essential rules involving in this process is given in
Figure~\ref{fig:inference-context}. If a variable is useless for the
unification, it is ignored (\rul{Ignore}). Otherwise, if the variable is needed,
but it is out of the scope of $\genA$, then it is moved to the additional
context (\rul{Depend}). Finally we arrive at the variable we want to unify,
which is $\genA$, we then insert $\Xi$ before $\genA$, and solve $\genA = \tau$
(\rul{Define}). Therefore, the output context for the above unification problem
is $\genB, \genA = \genB \to \genB, x: \genB$. Note that $\genB$ is now placed
in the front of $\genA$.

\begin{figure*}[t]
  \begin{mathpar}
    \Ignore \and \Depend \and \Define
  \end{mathpar}
  \caption{Unification between an existential variable and a type (incomplete).}
  \label{fig:inference-context}
\end{figure*}

\paragraph{Challenges.}

While moving type variables around is a feasible way to resolve the dependency
between existential variables, the unpredictable context movements make
the information increase hard to formalize and reason about. In this sense, the
information increase of contexts is defined in a much \textit{semantic} way:
$\ctxl$ is more informative than $\tctx$, if there exists a substitution $S$
that for every $v \in \tctx$, we have $\ctxl \bywf S(v)$.

This semantic definition makes it hard to prove meta-theory formally, especially
when advanced features are involved. For example, in dependent type systems,
typing and types/contexts well-formedness are usually coupled, which brings even
more complication to the proofs.

\subsubsection{Application in Higher Rank Type System.}

\begin{figure*}[t]
  \begin{mathpar}
    % \framebox{$\tctx \bybuni \sigma_1 \uni \sigma_2 \toctx$} \\
    \InstLSolve \and \InstLReach \and \InstLArr
    \and \InstRReach
  \end{mathpar}
  \caption{Instantiation between an existential variable and a type (incomplete).}
  \label{fig:instantiation}
\end{figure*}

\citet{dunfield2013complete} also use ordered contexts as input and output for
type inference for a higher rank polymorphic type system. However, they do it in
a more \textit{syntactic} way.

Instead of moving variables to the left in the context, in their subtyping
between existential variable and a type, which they call instantiation, they
choose to destruct type constructs so that existential variables can be unified
individually, then there are only two cases to discuss: whether the left
existential variable appears first, or the right one appears first.
Specifically, Figure~\ref{fig:instantiation} shows the key idea of instantiation
rules between an existential variable and a type. For space reason, we only
present the rules when the left hand side is an existential variables; but the
other case is quite symmetric. We save the formal explanations for notations
later to Section~\ref{sec:dependent}. But we can still gain the a rough idea
there. Notice in \rul{InstLArr}, the existential variable $\genA$ is solved by a
function type consisting of two fresh existential variables, and then the
function is destructed to do instantiation successively. Rule \rul{InstLReach}
deals with the case he lhs appears first, and \rul{InstRReach} deals with the
other case.

In this way, the information increase of contexts, which they call context
extension, is formalized in an intuitive and straightforward \textit{syntactic}
way, which enables them to prove the meta-theory thoroughly and formally. Due to
the complete definition of context extension can be found in appendix. Our
definition presented in Section~\ref{sec:context-extension} also mimics their
definition.

\paragraph{Challenges.} While destructing type constructors makes perfect sense
in their setting, it cannot deal with dependent types correctly. For example,
given the context $\genA, \genB$ and we want to unify $\genA$ with a dependent
type $\bpi x \genB x$. Here because $\genB$ appears after $\genA$, we cannot
directly derive $\genA = \bpi x \genB x$ which is ill typed. However, if we try
to destruct this Pi type, then according to rule \rul{InstLArr}, it is obvious
that $\genA_2$ should be solved by $x$. In order to make the solution well
typed, we need to put $x$ before $\genA_2$ into the context. However, this means
$x$ will remain in the context, and it is available for any later existential
variable that should not have access to $x$.

Also, destruction causes symmetric rules. The rules in
Figure~\ref{fig:instantiation} is repeated for the cases when the existential
variable is on the right. For example, there will be a \rul{InstRArr}
corresponding to \rul{InstLArr}. This kind of ``duplication'' would grow with
the increase of type constructs.

\subsection{Type Sanitization}
\label{subsec:sanitization}

Type sanitization provides another way to resolve dependency between existential
variables. It combines two advantages of the previous approaches. First, it only
makes predictable and reasonable context changes, so that information increase
can still be modeled as \textit{syntactic} context extension, which ease the
meta-theory proving. Also, it will not destruct types, so it is suitable for
features like dependent types, and it causes no duplication.

To understand how type sanitization works, we revisit the unification
problem: given context

$\genA, \genB, x: \genB$

\noindent we want to unify $\genA$ with $\genB \to \genB$. The problem here is
that $\genB$ is out of the scope of $\genA$. Therefore, we first ``sanitize''
the type $\genB \to \genB$. The process of type sanitization will sanitize the
existential variables in right hand side that are out of the scope of $\genA$ by
solving them with fresh existential variables that is put in front of $\genA$.
Specifically, we will solve
$\genB$ with a fresh variable $\genA_1$, which results in an output context

$\genA_1, \genA, \genB = \genA_1, x: \genB$

Notice that $\genA_1$ is inserted right before $\genA$. Now the unification
problem becomes unifying $\genA$ with $\genA_1 \to \genA_1$, and $\genA_1 \to
\genA_1$ is a valid solution for $\genA$. Therefore, we get a final solution
context:

$\genA_1, \genA = \genA_1 \to \genA_1, \genB = \genA_1, x : \genB$.

\paragraph{Interpretation of Type Sanitization.}
The philosophy behind moving existential variables around in the approach of
type inference in context \citep{gundry2010type}, the symmetric rules
\rul{InstLReach} and \rul{InstRReach} \citep{dunfield2013complete}, and the
approach of type sanitization is: \textit{the relative order between existential
  variables does not matter}.

This is kind of going against the design principle that the contexts are ordered
lists. However, the fact that contexts are ordered is still important for
variables \textit{whose order matter}. For instance, for polymorphic types, the
order between existential variables ($\genA$) and type variables ($\varA$) is
important, so you cannot unify $\genA$ with $\varA$ under the context $(\genA,
\varA)$ since $\varA$ is not in the scope of $\genA$. A similar reason exists in
dependent type systems: you cannot unify $\genA$ with $x$ if $x$ appears behind
$\genA$ in the context.

\subsection{Application: Unification for Dependent Types}

As a first illustration of the utility of the type sanitization, we present a
unification algorithm for dependent types with alpha-equality based
first-order constrains.

\paragraph{Explicit Casts.}

Type systems with explicit controls on type-level computation has been adopted
in several dependent type calculi \cite{???} since it allows type system to have
both general recursion and decidable type checking at the same time. In order to
have type-level computations, explicit casts are forced, which is implemented by
two language constructs: $\castdn e$ that does one-step beta reduction on the
type of $e$, and $\castup e$ that does one-step beta expansion on the type of
$e$. For example, If given

$e: (\blam x \Int \star) ~ 3$

\noindent Then we have

$\castdn e : \star$

$\castup (\castdn e) : (\blam x \Int \star) ~ 3$

In these systems, type comparison is naturally based on alpha-equality. This
simplifies the unification algorithm in the sense that unification can be mostly
structural. However, we still need to deal with the dependency introduced by
dependent types carefully, which is mainly reflected in the unification problems
between existential variables.

\paragraph{Type Sanitization In Dependent Type System.}
Type sanitization is applicable to dependent type system. Consider the previous
example that we unify $\genA$ with $\bpi x \genB x$. By the same process
described in Section~\ref{subsec:sanitization}, we can sanitize the type to be
$\bpi x {\genA_1} x$ without destructing the Pi type, and solve $\genA$ with
$\bpi x {\genA_1} x$.

\subsection{Polymorphic Type Sanitization}


%%% Local Variables:
%%% mode: latex
%%% TeX-master: "../main"
%%% org-ref-default-bibliography: "citation.bib"
%%% End:

\section{Unification and Type Sanitization for Dependent Types}
\label{sec:dependent}

This section introduces a simple dependently typed calculus with 
alpha-equality and type casts. The main novelties are the unification and type
sanitization mechanisms.

\subsection{Language Overview}
\label{subsec:language}

The syntax of the calculus is shown below:\\

\begin{tabular}{lrcl}
  Expressions & $e$ & \syndef & $x \mid \star
                         \mid e_1~e_2 \mid \blam x \sigma e
                         \mid \bpi x {\sigma_1} \sigma_2$ \\
       && \synor & $\castup e \mid \castdn e$ \\
  Types & $\tau, \sigma$ & \syndef & $e \mid \genA$ \\
  Contexts & $\tctx, \ctxl, \ctxr$ & \syndef & $\ctxinit \mid \tctx,x:\sigma
             \mid \tctx, \genA
             \mid \tctx, \genA = \tau $ \\
  Complete Contexts & $\cctx$ & \syndef & $\ctxinit \mid \cctx,x:\sigma
             \mid \cctx, \genA = \tau $ \\
\end{tabular}

\paragraph{Expressions.}
Expressions $e$ include variables x,
a single sort $\star$ to represent the type of
types,
applications $e_1~e_2$,
functions $\blam x \sigma e$,
Pi types
$\bpi x {\sigma_1} {\sigma_2}$,
$\castup e$ that does beta expansion,
and $\castdn e$ that does beta reduction.

\paragraph{Types.}
Types $\tau, \sigma$ are the same as expressions, except that types contain
existential variables $\genA$.
The categories of expressions and types are stratified to make sure that
existential variables only appears in positions where types are expected.

\paragraph{Contexts.}
Contexts $\tctx$ are an ordered list of variables and
existential variables, which
can either be unsolved
($\genA$) or solved by a type $\tau$ ($\genA = \tau$).
It is important for a context to be ordered to solve the dependency between
variables. For example, if a variable $x$ is introduced after
an existential variable $\genA$
in the context, as in
$\genA, x:\star, y :\genA$,
then $\genA$ can never be solved by $x$ (consequentely $y$ can never have type $x$).
Complete contexts $\cctx$ only contain variables and solved existential variables.

\paragraph{Hole notation.}
We use hole notations like $\tctx[x]$~\cite{} to
denote that the variable $x$ appears in the context. Sometimes such a
hole notation is also written as $\tctx_1, x, \tctx_2$.
Multiple holes also keep the order. For example, $\tctx[x][\genA]$ not only
requires the existence of both variables $x$ and $\genA$, but also requires that
$x$ appears before $\genA$.
The hole notation is also used for replacement and modification. For example,
$\tctx[\genA = \star]$ means the context is mostly unchanged except
that $\genA$ now is solved by $\star$.

\paragraph{Applying Contexts.} Since the context records all the solutions of
solved existential variables, it can be used as a substitution. Figure
\ref{fig:context-application} defines the substitution process, where all solved
existential variables are substituted by their solutions.

\begin{figure*}[t]
  \centering
  \begin{tabular}{rlp{2cm}rll}
    $\applye {\emptyset} \sigma$ & = & $\sigma$ &
    $\applye {\tctx, x: \tau} \sigma$ & = & $\applye \tctx \sigma$ \\
    $\applye {\tctx, \genA} \sigma$ & = & $\applye \tctx \sigma$ &
    $\applye {\tctx, \genA = \tau} \sigma$ & = & $\applye \tctx {\sigma \subst \genA \tau}$\\
  \end{tabular}
    \caption{Context application.}
    \label{fig:context-application}
\end{figure*}

\subsection{Typing in Detail}
\label{subsec:typing}

\begin{figure*}[t]
  \begin{mathpar}
    \framebox{$\tctx \byinf \sigma_1 \infto \sigma_2$} \\
    \AAx \and \AVar \and \AEVar \and \ASolvedEVar \and
    \ALamAnn \and \APi \and
    \AApp \and \ACastDn \and \ACastUp
  \end{mathpar}

  \begin{mathpar}
    \framebox{$\sigma_1 \redto \sigma_2$} \\
    \RApp \and \RBeta \and
    \RCastDown \and \RCastDownUp
  \end{mathpar}
    \caption{Typing and semantics.}
    \label{fig:typing}
\end{figure*}

In order to show that our unification algorithm works correctly, we need to make
sure that inputs to the algorithm are well-formed.
In a dependent type system, the well-formedness of types and contexts are relying
on typing judgments.
Therefore, to introduce the well-formedness of type and contexts,
we first introduce the typing rules.

Before we give the typing rules, we need to consider: what does it
means for an input type to the unification algorithm to be well-formed?
For example, Given a context $(\genA, x : \genA)$,
is the type $((\blam y \Int \star) ~ x)$ well formed?
Here, the type requests solving $\genA = \Int$.
We do not regard this type well formed, because we keep the
invariant: \textit{the
constraints contained in the input type have already been solved.}
Namely, the unification process only accepts inputs that are already type
checked in current context.
In this example, this type is well-formed under context
$(\genA = \Int, x : \genA)$.
This can also help prevent ill-formed contexts that contains conflicting
constraints, for example, the context:

$\genA, x: \genA, y : ((\blam x \Int \star)~x), z : ((\blam x \Bool \star)~x)$

\noindent contains two constraints that request $\genA$ to be solved by $\Int$
and $\Bool$ respectively, which cannot be satisfied at the same time.

\paragraph{Type System}
Given this interpretation of well-formedness, the typing rules that serves
specifically for well-formedness is shown at the top of Figure~\ref{fig:typing}.
The judgment $\tctx \byinf \sigma_1 \infto \sigma_2$ is read as: under typing
context $\tctx$, the type $\sigma_1$ has type $\sigma_2$.

Rule \rul{A-Ax} states that $\star$ always has type $\star$.
Rule \rul{A-Var} acquires the type of the variable from the typing context, and
applies the context to the type.
This reveals another invariant that we keep:
\textit{the typing output is always fully substituted under current context.}
Rules \rul{A-EVar} and \rul{A-SolvedEVar} ensures that existential variables
always have type $\star$.
Rule \rul{A-LamAnn} first infers the type $\star$ for the annotation, then puts $x:
\sigma_1$ into the typing context to infer the body. To make sure output type is
fully substituted, we apply the context to $\sigma_1$ in the output Pi type.
Rule \rul{A-Pi} infers the type $\star$ for the argument type $\sigma_1$, then puts
$x: \sigma_1$ into the typing context to infer $\sigma_2$, whose type is also a
$\star$. And the result type for a Pi type is $\star$.
Rule \rul{A-App} first infers a function type for $e_1$, and then infers $e_2$ to
have the argument type. Again, to maintain the invariant, we apply the context
to $e_1$ before substituting $x$ with $e_1$.
Rule \rul{A-CastDn} infers the type $\sigma_1$ of $e$, that reduce the type
$\sigma_1$ to $\sigma_2$, while rule \rul{A-CastUp} finds a fully substituted
type $\applye \tctx
{\sigma_1}$ that reduces to $\sigma_2$ as the output type. The call by name
reduction is defined at the bottom of Figure~\ref{fig:typing}.
Due to the design of the rule \rul{A-CastUp}, the typing rules are
non-deterministic, which does not matter for our purpose: the typing is only
used in propositions (such as lemmas, theories) but it never appears in the
unification and type sanitization algorithms.

\paragraph{Context well-formedness.}

The first four typing rules have a common precondition $\tctx \wc$,
which requests the context is well-formed.
The judgment is defined at the top of Figure~\ref{fig:type-well}.
Rule \rul{AC-Empty} states that an empty context is always well formed.
Rule \rul{AC-Var} requires $x$ fresh, and the type annotation is typed with
$\star$. Rule \rul{AC-EVar} and \rul{AC-SolvedEVar} are defined in a similar
way.

\paragraph{Type well-formedness.}

We denote a well-formed type $\sigma$ as $\tctx \byinf \sigma \infto \star$.
We will also sometimes write it as $\tctx \bywf \sigma$.
A weaker version of type well-formedness, which is type well-scopedness, written
as $\tctx \bywt \sigma$, is defined at the bottom of Figure~\ref{fig:type-well}.
Well-scopedness of types only requires all the variables involved in a type are
bound in the typing contexts.

\begin{figure*}[t]
  \begin{mathpar}
    \framebox{$\tctx \wc$} \\
    \ACEmpty \and \ACVar \and
    \ACEVar \and \ACSolvedEVar
  \end{mathpar}

  \begin{mathpar}
    \framebox{$\tctx \bywt \sigma$} \\
    \WSVar \and \WSEVar \and \WSSolvedEVar
    \and \WSPi \and \WSLamAnn \and \WSApp
    \and \WSCastDn \and \WSCastUp
  \end{mathpar}
    \caption{Context well-formedness and type well-scopedness.}
    \label{fig:type-well}
\end{figure*}

\begin{figure*}[t]
  \begin{mathpar}
    \framebox{$\tctx[\genA] \bysa \tau_1 \sa \tau_2 \toctx$} \\
    \IEVarAfter \and \IEVarBefore \and
    \IVar \and \IStar \and
    \IApp \and \ILamAnn \and \IPi
    \and \ICastDn \and \ICastUp
  \end{mathpar}
  \caption{Type sanitization.}
  \label{fig:sanitization}
\end{figure*}

\begin{figure*}[t]
  \begin{mathpar}
    \framebox{$\tctx \bybuni \sigma_1 \uni \sigma_2 \toctx$} \\
    \UAEq \and \UEVarTy \and \UTyEVar \and
    \UApp \and \ULamAnn \and \UPi
    \and \UCastDn \and \UCastUp
  \end{mathpar}
  \caption{Unification.}
  \label{fig:unification}
\end{figure*}


\subsection{Unification}
\label{subsec:unification}

As we mentioned before, our unification is based on alpha-equality. So in most
cases, the unification rules are intuitively structural. The most difficult one
which is also the most essential one, is how to unify an existential variable
with another type. In this section, we first present the judgment of
unification, then we discuss those cases before we present the unification
process.

\paragraph{Judgment of unification.}

Due to our design choice, there are two modes in the
unification: expressions, and types. The expression mode ($\delta = e$) does
unification between expressions, while the type mode ($\delta = \sigma$) does
unification between types.
The judgment of unification problem is formalized as:

\begin{lstlisting}
$\tctx \bybuni \sigma_1 \uni \sigma_2 \toctx$
\end{lstlisting}

The input of the unification is the current context $\tctx$, and two types
(if $\delta = \sigma$) or two expressions (if $\delta = e$).
The output of the unification
is a new context $\ctxl$ which extends the original context with probably more
new existential variables or more existing
existential variables solved.
The formal definition of context extension is discussed in
Section~\ref{sec:context-extension}.
Following is an example of a unification problem:

\begin{lstlisting}
$\genA \bysuni \genA \uni \Int \dashv \genA = \Int$
\end{lstlisting}

\noindent where we want to unify $\genA$ with $\Int$ under the input context
$\genA$, which results in the output context $\genA = \Int$ that solves $\genA$
with $\Int$.

For a valid unification problem, it must have the invariant: $[\tctx] \tau_1 =
\tau_1$, and $[\tctx] \tau_2 = \tau_2$. Namely,
\textit{the input types must be
fully applied under the input context}.
 So the following is not a valid
unification problem input:

\begin{lstlisting}
$\genA = \Bool \bysuni \genA \uni \Int$
\end{lstlisting}

We assume this invariant is given with the inputs at the beginning,
and the unification process would maintain it through the whole
formalization.

\paragraph{Process of type sanitization.}

As we discuss in Section~\ref{sec:overview}, before unifying an existential
variable with a type, we will first sanitize the type so that the existential
variables in the type that are out of the scope are solved by fresh existential
variables within the scope. We call this process \textit{type sanitization},
which is formally defined in Figure \ref{fig:sanitization}. The judgment
$\tctx[\genA] \bysa \tau_1 \sa \tau_2 \toctx$ is interpreted as: under the
context $\tctx$, which contains an existential variable $\genA$, we sanitize all
the existential variables in the type $\tau_1$ that appears before $\genA$,
which results in a sanitized type $\tau_2$. Computationally, there are three
inputs $\tctx$, $\genA$ and $\tau_1$, with one output $\tau_2$.

The most interesting cases are \rul{I-EVarAfter} and \rul{I-EVarBefore}. In
\rul{I-EVarAfter}, because $\genB$ appears after $\genA$, we create a fresh
existential variable $\genA_1$, which is put before $\genA$, and solve $\genB$
by $\genA_1$. In \rul{I-EVarBefore}, because $\genB$ is in the scope of $\genA$,
we leave the existential variable unchanged.
The remaining rules are structural.
Similarly to unification, we always apply intermediate output
contexts to the input types to maintain the invariant that the types are fully
substituted under current contexts.

The sanitization process is remarkably simple, while it solves exactly what we
want: resolve the order of existential variables so that we can focus on the
order that really matter.

\paragraph{Process of unification.}

Based on type sanitization, Figure \ref{fig:unification} presents the
unification rules.
Rule \rul{U-AEq} corresponds to the case when two types are already
alpha-equivalent. Most of the rest rules are structural. The two most subtle cases
are rules \rul{U-EVarTy} and \rul{U-TyEVar}, which correspond respectively to
when the existential variable is on the left and on the right. We go through the
first rule. There are three preconditions. First is the occurs check, which is to
make sure $\genA$ does not appear in the free variables of $\tau_1$. Then we use
type sanitization to make sure all the existential variables in $\tau_1$ that
are out of scope of $\genA$ are turned into fresh ones that are in the scope of
$\genA$. This process gives us the output type $\tau_2$, and output context
$\ctxl_1, \genA, \ctxl_2$. Finally, $\tau_2$ could also contain variables whose
order matters, so we use $\ctxl_1 \bywt \tau_2$ to make sure $\tau_2$ is well
scoped. Rule \rul{U-TyEVar} is symmetric to \rul{U-EVarTy}. Using
well-scopedness instead of well-formedness gets us rid of the dependency on
typing.

\paragraph{Example.}

The derivation of the unification problem $\tctx, \genA, \genB \bysuni \genA
\uni \bpi x \genB x$ is given below. For clarity, we denote $\ctxl =
\tctx,\genA_1,\genA,\genB =\genA_1$. And it is easy to verify $\genA \notin
FV(\bpi x \genB x)$.

\[
   \ExUni
\]

\subsection{Context Extension}
\label{sec:context-extension}

\begin{figure*}[t]
  \begin{mathpar}
    \framebox{$\tctx \exto \ctxl$}
     \\
    \CEEmpty \and \CEVar \and \CEEVar \and
    \CESolvedEVar \and \CESolve \and
    \CEAdd \and \CEAddSolved
  \end{mathpar}
  \caption{Context extension.}
  \label{fig:context-extension}
\end{figure*}

We mentioned that the algorithmic output context extends the input context with
new existential variables or more existential variables solved. To accurately
capture this kind of information increase, we present the definition of context
extension in Figure~\ref{fig:context-extension}. This definition is
similar to the one by \citet{dunfield2013complete}.

The empty context is an extension of itself (\rul{CE-Empty}). 
Variables or existential variables are preserved during the extension
(\rul{CE-Var}, \rul{CE-EVar}), while the solutions for existential variables can
be different only if they are equivalent under context application
(\rul{CE-SolvedEVar}). The definition in \citet{{dunfield2013complete}} further
allows the type annotation to change (in \rul{CE-Var}), which is not necessary for
our algorithm. The extension can also add solutions to unsolved existential
variables (\rul{CE-Solve}), or add new existential variables (\rul{CE-Add},
\rul{CE-AddSolved}).

\paragraph{Context application on contexts.}

Complete contexts $\cctx$ are contexts with all existential variables solved, as
defined in Section~\ref{subsec:language}. Applying a complete context to a
well-formed type $\sigma$ yields a type without existential variables $\applye
\cctx \sigma$. Similarly, given $\tctx \to \cctx$, we can get a context without
existential variables $\applye \cctx \tctx$.
The formal
definition of context application is in
Figure~\ref{fig:context-application-on-context}.

\begin{figure*}[t]
  \centering
  \begin{tabular}{rlll}
    $\applye {\emptyset} \emptyset$ & = & $\emptyset$ \\
    $\applye {\cctx, x: \tau} {\tctx, x: \tau}$ & = & $\applye \cctx \tctx, x : \applye \cctx {\tau} $
    \\
    $\applye {\cctx, \genA = \tau} {\tctx, \genA}$ & = & $\applye \cctx \tctx$ \\
    $\applye {\cctx, \genA = \tau_1} {\tctx, \genA = \tau_2}$ & = & $\applye \cctx \tctx$
                                    & If $\applye \cctx {\tau_1} = \applye \cctx {\tau_2}$ \\
    $\applye {\cctx, \genA = \tau_1} {\tctx}$ & = & $\applye \cctx \tctx$ \\
  \end{tabular}
    \caption{Context application.}
    \label{fig:context-application-on-context}
\end{figure*}

\subsection{Soundness}

Our overall framework for proofs is quite similar as
\citet{dunfield2013complete}. However, unlike their work which always implicitly assumes
the contexts and types involved are well-formed, we put extra effort on
dealing with the well-formedness and the typing explicitly since we
have to resolve the dependency carefully.

We proved that our type sanitization strategy and the unification algorithm
are sound.
First, we show that the output context after type
sanitization is indeed an extension of the input context:

\begin{lemma}[\TypeSanitizationExtensionName]
  \TypeSanitizationExtensionBody
\end{lemma}

Except for resolving the order problem, type sanitization will not change the
type. Namely, the input type and the output type are equivalent after
substitution by the output context:

\begin{lemma}[\TypeSanitizationEquivalenceName]
  \TypeSanitizationEquivalenceBody
\end{lemma}

Moreover, if the input type is well-formed under the input context, then the
output type is still well-formed under the output context:

\begin{lemma}[\TypeSanitizationWellFormednessName]
  \TypeSanitizationWellFormednessBody
\end{lemma}

Having those lemmas related to type sanitization, we can prove the properties of
the unification algorithm. For example, the output context of unification
extends the input context:

\begin{lemma}[\UnificationExtensionName]\leavevmode
  \UnificationExtensionBody
\end{lemma}

And finally, we can prove that two input types are really unified by the
unification algorithm:

\begin{lemma}[\UnificationEquivalenceName]\leavevmode
  \UnificationEquivalenceBody
\end{lemma}

\subsection{Completeness}

We use the notation $\tctx \bywf \tau_1 = \tau_2$ to mean that
$\tctx \byinf \tau_1 \infto \sigma$, $\tctx \byinf \tau_2 \infto \sigma$ for
some $\sigma$,
and $\tau_1 = \tau_2$.

The completeness of type sanitization is proved by a more general lemma, which
can be found in the appendix. The more readable version of the completeness of
type sanitization is:

\begin{corollary}[\TypeSanitizationCompletenessPrettyName]
  \label{lemma:\TypeSanitizationCompletenessPrettyName}
  \TypeSanitizationCompletenessPrettyBody
\end{corollary}

\noindent which leads directly to the unification between existential variables:

\begin{lemma}[\TypeSanitizationCompletenessUnificationName]\leavevmode
  \label{lemma:\TypeSanitizationCompletenessUnificationName}
  \TypeSanitizationCompletenessUnificationBody
\end{lemma}

Having the completeness of type sanitization, we are ready to prove that our
unification algorithm is complete:

\begin{lemma}[\UnificationCompletenessName]
  \label{lemma:\UnificationCompletenessName}
    \UnificationCompletenessBody
\end{lemma}

\section{Higher Rank Polymorphic Type System}
\label{sec:higherrank}

In this section, we adopt the type sanitization strategy to a higher rank
polymorphic type system from \citet{dunfield2013complete}. We show that type
sanitization can be further extended to polymorphic type sanitization to deal
with subtyping, which can be used to replace the instantiation relation in
original system while preserving the completeness and subtyping.

\subsection{Language}

The syntax of \citet{dunfield2013complete} is given below. Notice that
notations from Section~\ref{sec:dependent} are reused here. We will always
make it clear from the context which system is
being referred. \\

\begin{tabular}{lrcl}
  Type & $A, B$ & \syndef & $\Unit \mid \varA \mid \genA \mid \forall \varA. A \mid A \to B $ \\
  Monotype & $\sigma, \tau$ & \syndef & $\Unit \mid \varA \mid \genA \mid \sigma \to \tau $ \\
  Contexts & $\tctx, \ctxl, \ctxr$ & \syndef & $\ctxinit \mid \tctx, \varA
                                               \mid \tctx, x: A
                                               \mid \tctx, \genA
                                               \mid \tctx, \genA = \tau
                                               \mid \tctx, \marker \genA $\\
  Complete Contexts & $\cctx$ & \syndef & $\ctxinit \mid \cctx, \varA
                                          \mid \cctx, x: A
                                          \mid \cctx, \genA = \tau
                                          \mid \cctx, \marker \genA $\\
\end{tabular}

\paragraph{Types.}
Types $A, B$ include unit type $\Unit$, type variables $\varA$, existential
variables $\genA$, polymorphic types $\forall \varA. A$ and function types $A
\to B$.

Monotypes $\sigma, \tau$ are a special kind of types without universal
quantifiers.

\paragraph{Contexts.}
Contexts $\tctx$ are an ordered list of type variables $\varA$, variables $x:
A$, existential variables $\genA$ and $\genA = \tau$, and special markers
$\marker \genA$ for scoping reasons.

And all existential variables in complete contexts $\cctx$ are solved.


\subsection{Subtyping}

\begin{figure*}[t]
  \begin{mathpar}
    \framebox{$\tctx \bysub A \tsub B \toctx$}
     \\
    \ADunVar \and \ADunUnit \and
    \ADunExvar \and \ADunArrow \and
    \ADunForallL \and \ADunForallR \and
    \ADunInstL \and \ADunInstR
  \end{mathpar}
  \caption{Algorithmic Subtyping (Original).}
  \label{fig:subtyping}
\end{figure*}

The original definition of algorithmic subtyping is given in
Figure~\ref{fig:subtyping}. The judgment $\tctx \bysub A \tsub B \toctx$ is
interpreted as: given the context $\tctx$, $A$ is a subtype of $B$ under the
output context $\ctxl$.
The subtyping relation is reflexive (\rul{Var}, \rul{Exvar} and
\rul{Unit}). The function type is contra-variant on the argument type,
and co-variant on the return type (\rul{$\to$}). Here the intermediate context
$\ctxl$ is applied to $A_2$ and $B_2$ to make sure the input types are fully
substituted under the input context. In \rul{$\forall$L}, new existential
variable $\genA$ is created to represent the universal quantifier. There is a
marker before $\genA$ so that we can throw away all tailing contexts that are
out of scope after the subtyping. Similarly, rule \rul{$\forall$R} puts a
new type variable in the context and throws away all the contexts after the type variable.

When the left hand side (\rul{InstL}) or the right hand side (\rul{InstR}) is an
existential variable, the subtyping leaves all the work to the instantiation
judgment ($\ist$), some of which we have seen in Section~\ref{sec:overview}. Due
to the limitation of space, we put the complete definition of the instantiation
relation in the appendix.

\subsection{Polymorphic Type Sanitization}

\begin{figure*}[t]
  \begin{mathpar}
    \framebox{$\tctx \bysub A \tsub B \toctx$} \\
    \ADunSaL \and \ADunSaR
  \end{mathpar}

  \begin{mathpar}
    \framebox{$\tctx[\genA] \bybsa A \sa \sigma \toctx$}
     \\
    \IAllPlus \and \IAllMinus \and
    \IPiPoly \and \IUnit \and \ITVar \and
    \IEVarAfterPoly \and \IEVarBeforePoly
  \end{mathpar}
  \caption{New subtyping rules, polymorphic type sanitization.}
  \label{fig:polymorphic-sanitization}
\end{figure*}

The goal of polymorphic type sanitization is to replace the instantiation
relationship with even simpler rules.

Therefore, we replace the rules \rul{InstL} and \rul{InstR} with new rules
\rul{SaL} and \rul{SaR} shown at the top of
Figure~\ref{fig:polymorphic-sanitization}. Namely, the subtyping leaves the job
to polymorphic type sanitization to resolve the order problem with existential
variables and also sanitize the input type to a monotype. A final check
ensures that the sanatized type is well-formed under the context before $\genA$.
If the type being sanitized appears on the right (as \rul{SaL}), we say it
appears co-variantly, and we say it appears contra-variantly if it appears on
the left (as \rul{SaR}). This corresponds to the fact that a polymorphic type
can be a subtype of a monotype only if all the universal quantifiers appear
contra-variantly in the type.

The rules for polymorphic type sanitization are shown at the bottom of
Figure~\ref{fig:polymorphic-sanitization}. According to whether the type appears
co-variantly ($s = -$) or contra-variantly ($s = +$), we have two modes. The
judgment $\tctx[\genA] \bybsa A \sa \sigma \toctx$ is interpreted as: under
typing context $\tctx$ which contains $\genA$, sanitize a possibly polymorphic
type $A$ to a monotype
$\sigma$, with output context $\ctxl$. Computationally, there are three inputs
($\tctx, \genA$ and $A$), and two outputs ($\sigma$ and $\ctxl$).

The only difference between these two modes is how to sanitize polymorphic
types. If a polymorphic type appears contra-variantly (\rul{I-All-Plus}), it
means a monotype would make the final type more polymorphic. Therefore, we
replace the universal binder $\varA$ with a fresh existential variable $\genB$
and put it before $\genA$. Otherwise, in rule \rul{I-All-Minus}, we put $\varA$
in the context and sanitize $A$. Notice that the result $\sigma$ might not be
well-formed under the output context $\ctxl$, since $\varA$ is discarded in the
output context. Rule \rul{I-Pi-Poly} is where the mode is flipped.

Polymorphic type sanitization does nothing if it is a unit type (\rul{I-Unit})
or a type variable (\rul{I-TVar}). Rule \rul{I-EVarAfter-Poly} and
\rul{I-EVarBefore-Poly} deals with existential variables, and creates fresh
existential variables if the input existential variable appears after $\genA$,
as we have seen in type sanitization in Section~\ref{subsec:unification}.

\paragraph{Example}

The derivation of the subtyping problem $\genA \bysub \genA \tsub (\forall
\varA. \varA \to \varA) \to \Unit$ is given below. For clarity, we omit some
detailed process, and denote $\ctxl = \genB, \genA$.\bruno{example overflows}

\[
\ExSub
\]

\subsection{Meta-theory}

The soundness and completeness of subtyping relies on the soundness and
completeness of instantiation in \citet{dunfield2013complete}. To prove
polymorphic type sanitization works correctly, there is no need to re-prove the
lemmas about subtyping. Instead, we prove that polymorphic type sanitization
leads to exactly the same result as instantiation.

For soundness, we prove that
under contra-variant mode ($s = +$), the input type is a
declarative subtype of the output type after substituted by a complete context,
while under co-variant mode ($s = -$), the input type is a declarative supertype:

\begin{lemma}[\PolymorphicTypeSanitizationSoundnessName]\leavevmode
  \label{lemma:\PolymorphicTypeSanitizationSoundnessName}
  \PolymorphicTypeSanitizationSoundnessBody
\end{lemma}

For completeness, we first prove that for a possibly polymorphic type $\applye
\cctx A$, if there is a monotype $\tau$ that is more polymorphic than it, there
is a polymorphic type sanitization result $\sigma$ of $A$, and it is the most
general subtype in the sense that we can recover the $\tau$ from applying a
complete context to $\sigma$.

\begin{lemma}[\PolymorphicTypeSanitizationCompletenessName]
  \label{lemma:\PolymorphicTypeSanitizationCompletenessName}
  \PolymorphicTypeSanitizationCompletenessBody
\end{lemma}

Based on the completeness of polymorphic type sanitization, we can prove exactly
the same completeness lemma as the instantiation to show that subtyping for
existential variables is complete:

\begin{corollary}[\PolymorphicTypeSanitizationCompletenessSubtypingName]
  \label{lemma:\PolymorphicTypeSanitizationCompletenessSubtypingName}
  \PolymorphicTypeSanitizationCompletenessSubtypingBody
\end{corollary}

%%% Local Variables:
%%% mode: latex
%%% TeX-master: "../main"
%%% org-ref-default-bibliography: "citation.bib"
%%% End:
\section{Discussion}

\subsection{Type Inference Algorithm For Dependent Types}

In Section~\ref{sec:dependent}, we use unification algorithm for dependent types
to show that type sanitization is applicable to advanced features. However,
notice that the typing rules presented in Section~\ref{subsec:typing} works
specifically for well-formedness of types. In other words, it is \textit{not} an
algorithmic type system containing type inference. Therefore, readers may wonder:
how can we design a type inference algorithm for practical usage based on the
unification algorithm, and what is the relation between the type inference
algorithm and the well-formedness typing rules? In this section, we discuss
these two questions.

\paragraph{Unification and Type Inference.} In a program, programmers may omit
some type annotations, and the type system will try to infer them. A typical
example is unannotated lambdas. For example

$\erlam x {x + 1} $

There is no annotation for the binder $x$, but the type system is able to
recover that $x:\Int$ from the expression $x + 1$. This is done by generating a
fresh existential variable $\genA$ for $x$, and then using the unification
algorithm to unify $\genA$ with $\Int$. In summary, type inference generates
existential variables that are waiting to be solved, while unification is in
charge of finding the solutions according to the type constraints.

Therefore, we can talk about unification separately from type inference. And
type inference for dependent types is also a challenging topic, where the
unification algorithm would definitely help with. Of course, in a type system
with unannotated lambdas, we also need to extend the unification to deal with
the new lambdas, which is not so complicated.

\paragraph{Type Inference and Well-formedness.} Even if we are given an
algorithmic type system containing type inference, the well-formedness system
still works correctly as long as the definition of well-formedness not changed.
Namely, any input type to the unification algorithm is already fully
type-checked, therefore it will introduce no more new constraints. However,
in our unification algorithm, well-formedness is only used for propositions,
so we leave it non-deterministic for simplicity. If it is used in the
algorithm, we would need to change it to an deterministic and decidable
relation.

\subsection{Feature Work}

One possible future work is to apply the strategy of type sanitization to type
systems with more advanced features. For example, the polymorphic type
sanitization presented in Section~\ref{sec:higherrank} works specifically for
subtyping between polymorphic types. We can try to extend type sanitization to
other kind of subtyping, for instance, nominal subtyping and/or intersection
types.

Also, since type sanitization resolves unification problem between dependent
types, another possible future work is to come up with a complete type inference
algorithm for dependent types based on alpha-equality and first-order
constraints. Also, extending the type sanitization to see whether if could play
any roles in extended setting, such as beta-equality or higher-order constraints
is also a feasible future work.

\section{Related Work}

\subsection{Type Inference in Context}

We have discussed the work by \citet{gundry2010type} and
\citet{dunfield2013complete} in Section~\ref{sec:overview}, which in some sense
can both be understood in terms of the proofs systems of
\citet{miller1992unification}. They adopt different strategies to resolve the
order problem of existential variables, with different emphasizes.
\citet{gundry2010type} supports ML-style polymorphism, and as they mention the
longer-term goal is to elaborate high-level dependently typed programs into
fully explicit calculi. However they do not present the algorithm nor prove it.
\citet{dunfield2013complete} use the strategy specifically for higher rank
polymorphism, while as we have seen their strategy cannot be applied to dependent types.

\subsection{Unification for Dependent Types}

Unification for dependent types has been a challenging topic for years. While
our unification only solves first-order constraints based on alpha-equality,
existing literatures takes some advanced features into account, with the price
that the algorithm is usually complicated to understand, and also the
meta-theory is hard to prove. \citet{elliott1989higher} develop a higher order
unification algorithm for dependent function types in the spirit of
\citet{huet1975unification}, though the problem is undecidable.
\citet{reed2009higher} comes up with a constraint simplification algorithm that
works on dynamic pattern fragment of higher-order unification in a
dependent type system.
Later, \citet{abel2011higher} propose a constraint-based unification algorithm
that solves a richer class of patterns.
\citet{ziliani2015unification} formalize the unification algorithm used in the
language Coq~\citep{coqsite}, which is quite sophisticated and the correctness
proof is lacking.

\subsection{Type Inference for Higher Rank Type Systems}

Type inference involving higher ranks has been well studied in recent years.
\citet{jones2007practical} develop an approach using traditional bi-directional
type checking, which is built upon \citet{odersky1996putting}. In this system,
subtyping and unification are separated, and unification is only between
monotypes. \citet{dunfield2013complete} build a simple and concise algorithm for
higher ranked polymorphism, which is also based on traditional bidirectional
type checking. The instantiation in their system is introduced in
Section~\ref{sec:overview}. These two systems are predicative, in the sense that
universal quantifiers can only be instantiated with monotypes. There are also
impredictive systems, including $ML^F$
\citep{le2014mlf,remy2008graphic,le2009recasting}, the HML system
\citep{leijen2009flexible} and the FPH system \citep{vytiniotis2008fph}.
%%% Local Variables:
%%% mode: latex
%%% TeX-master: "../main"
%%% org-ref-default-bibliography: "citation.bib"
%%% End:

\section{Conclusion}

In this paper, we propose a new strategy \textit{type sanitization}, and present
two applications of this strategy. The first application is type sanitization in
a unification algorithm for a dependent type system with alpha-equality based
first order constrains. And the second one is in a subtyping algorithm for a
higher rank polymorphic type system with the extended \textit{polymorphic type
  sanitization}.

We expect type sanitization is applicable to more type systems with other
advanced features, for example, intersection types and more advanced
forms of dependent types.


% \section{Algorithm}

\begin{tabular}{lrcl}
  Expr & $e, A, B$ & \syndef & $x \mid \star \mid
                               e_1 ~ e_2 $ \\%\mid \castup A e \mid \castdn e \\
       &           & \synor  & $\blam x A {e_2} \mid
                               \bpi x A B$ \\
       &           & \synor  & $\erlam x {e_2} \mid \genA[\sctx]$ \\
  Local Context & $\sctx$ & \syndef & $\ctxinit \mid \sctx, x: A$\\


  Meta Context & $\tctx$ & \syndef & $\ctxinit \mid \tctx, \genA[\sctx] : A \mid \tctx, \genA[\sctx] = e : A $ \\
\end{tabular}

\begin{tabular}{llll}
Syntactic Sugar & $A \to B$    & $\triangleq \bpi {x} A B$& where $x \notin FV(B)$ \\
\end{tabular}

\begin{mathpar}
  \framebox{$\tctx \ctxsplit \sctx \byinf e_1 \tsub e_2 \Longleftrightarrow A \toctx$} \\
  \AAx
  \and \AWeakLocal
  \and \AWeakMeta
  \and \AWeakSolvedMeta
  \and \AVar
  \and \AAbsInfer \and \AAbsCheck
  \and \AAbs
  \and \AApp \and \AProd
  % \and \ACastUp \and \ACastDn
  \and \ASub
  % \and \ASolvedEVarL \and \ASolvedEVarR
  \and \AEVarL \and \AEVarR
\end{mathpar}

Notes:
\begin{itemize}
  \item Sometimes empty local context is omitted.
  \item Well-formedness: output context is composed by original context and a
    trailing context. Is it right?
  \item Missing App-EVar.
  \item In A-App, how to deal with variables in local context?
  \item Context well-formedness: Weakening.
  \item Invariant: $e_1, e_2$ are fully substituted under context application
    \begin{itemize}
    \item Is it necessary?
    \end{itemize}
  \item Decidability
\end{itemize}

%%% Local Variables:
%%% mode: latex
%%% TeX-master: "../main"
%%% org-ref-default-bibliography: "citation.bib"
%%% End:

\bibliography{sections/citation}

\appendix

% \section{Appendix}

\subsection{Typing}

Typing is
not one of the contributions this paper emphasizes. But because the well-formedness
of types under contexts relies on typing, below we give the formalized definition
of algorithmic typing used in this paper. Here $UV$ stands for unsolved
unification variables.

\begin{figure*}[h]
  \headercapm{\tctx \byinf e : \sigma \toctx}{}
\[\TAx \quad \TVar \quad \TEVar\]
\[\TPi \]
\[\TPoly \]
\[\TLamAnnInf \]
\[\TApp \]
\[\APi\]
\[\AEVar\]
\[\APoly\]
    \caption{Typing.}
    \label{fig:typing}
\end{figure*}

\rul{T-Ax}, \rul{T-Var} and \rul{T-EVar} are standard, where the context has no change.

In \rul{T-Pi} and \rul{T-Poly}, the context $\ctxr$ after $x$ is thrown because all those variables are out of scope, and all the existential variables in $\ctxr$ will not be refered again when this rule ends. It is safe to throw existential variables because we could assume all unsolved existential variables are solved by $\star$ which satisfys $\star:\star$.

However, in \rul{T-LamAnn}, because $\ctxr$ may still contains unsolved
unification variables that are referred in $\sigma_2$, we keep $UV(\ctxr)$ in
the output context. But all reference to $x$ is out of scope, so we substitute
all solved variables in $\sigma_2$.

\rul{T-App} first infers the type of $e_1$ to get $\sigma_1$, then substitute the context on $\sigma_1$ and enter the application typing rules, which is defined by the A-rules. The judgement $\tctx \byapp \sigma_1 ~ e : \sigma_2 \toctx$ is interpreted as, in context $\tctx$, the type of the expression being applied is $\sigma_1$, the argument is $e_2$, and the output of the rule is the application result type $\sigma_2$ with context $\ctxl$.

In \rul{A-Pi}, the type of $e_1$ is a Pi type, so we infer the type of $e_2$ and
check the $\sigma_3$ is more polymorphic than $\sigma_1$.

In \rul{A-EVar}, the type of $e_1$ is an unsolved existential variable $\genA$.
It first destructs $\genA$ into a Pi type, and infers $e_2$ to have type
$\sigma_1$, then check $\sigma_1$ is more polymorphic than $\genA_1$.
\rul{A-Poly} is when the expression being applied has polymorphic type, then we
instantiate the bound variable with a fresh unification variable.


\subsection{Context Extension}

We mentioned some context extends some context several times in the paper
informally. Here is the formal definition of the context extension.

\begin{figure*}[h]
    \headercapm{\tctx \exto \ctxr}{}

    \[\CEEmtpy \quad \CETypedVar\]
    \[\CEEVar \quad \CESolvedEVar\]
    \[\CESolve \quad\CEAdd \quad \CEAddSolved\]
    \caption{Context Extension.}
    \label{fig:ctx-extension}
\end{figure*}

All original information will be preserved during context extension, including
the existence of variables, their relative orders, type annotations, and solutions for existential. In the meanwhile, information could be increased. Context extension could add more existential variables, or try to solve unsolved existential variables with proper types.

\rul{CE-Empty} states that empty context could be extended to empty context. \rul{CE-TypedVar} is able to replace the type annotation with a contextually equivalent one. \rul{CE-EVar} and \rul{CE-SolvedEVar} keeps the existential variable. \rul{CE-SolvedEVar} solves existential variables with a type, requesting the type is well formed. \rul{CE-Add} and \rul{CE-AddSolved} adds new existential variable in unsolved and solved form respectively.

It is easy
to verify that the output context extends the input context in each relation.


\end{document}

%%% Local Variables:
%%% mode: latex
%%% TeX-master: t
%%% End:
