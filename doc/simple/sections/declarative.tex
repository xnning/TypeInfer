\section{Declarative System}

Before we talk about the algorithm, we first give a declarative system and later on, we show that our algorithm is complete and sound with respect to it. The typing rules are shown in figure \ref{fig:decl-typing}.

\begin{figure*}[h]
    \headercapm{\dpreall e:\tau \trto t}{Expression Typing ($\delta ::= \Inf\mid \Chk$)}
    \[\DAx \quad \DVar \quad \DLetVar \quad \DAnn \]
    \[\DLamInf \quad \DLamChk \quad \DApp \]
    \[\DPi \quad \DLet \quad \DCastUp\]
    \[\DCastDn \quad \DSub \quad \DConv\]
    \\
    \headercapm{\dpreinst \sigma \lt \tau}{Instantiation} \quad \headercapm{\dpregen e : \sigma}{Generalization}
    \[\DInstantiation \quad \DGeneralization\]
    \caption{Declarative typing rules}
    \label{fig:decl-typing}
\end{figure*}

\begin{figure}[t]
    \headercapm{[\dctx]e_1 = e_2}{Apply context to a type}

    \begin{mathpar}
    \begin{tabular}{r c l}
        $[\ctxinit]e$   & = & $e$       \\
        $[\dctx,x:\tau]e$ & = & $[\dctx]e$ \\
        $[\dctx,x:\sigma = \tau]e$ & = & $[\dctx](e[x\mapsto \tau])$ \\
    \end{tabular}
    \end{mathpar}
    \caption{Applying a context}
    \label{fig:decl-apply-ctx}
\end{figure}

\begin{figure}[t]
    \begin{mathpar}
    \begin{tabular}{r c l}
        $FV(x)$   & = & $\{x\}$       \\
        $FV(\star)$    & = & $\varnothing$            \\
        $FV(e_1 ~ e_2)$    & = & $FV(e_1) \cup FV(e_2)$            \\
        $FV(\erlam x e)$   & = & $FV(e) - \{x\}$            \\
        $FV(\bpi x t e)$   & = & $FV(t) \cup FV(e) - \{x\}$            \\
        $FV(\kw{let} x = e_1 \kw{in} e_2)$  & = & $FV(e_1) \cup FV(e_2) - \{x\}$            \\
        $FV(\castupz e)$   & = & $FV(e)$            \\
        $FV(\castdn e)$    & = & $FV(e)$            \\
        $FV(e:t)$          & = & $FV(e) \cup FV(t)$            \\
    \end{tabular}
    \end{mathpar}
    \caption{Free variables}
    \label{fig:decl-free-variables}
\end{figure}


\paragraph{Bidirectional typing.} The judgement $\dpreall e:\sigma \trto t$ is interpreted as, under context $\dctx$, infer the type of the expression (with $\delta=\Inf$), which gives type $\tau$; or check the expression with type $\tau$ (if $\delta = \Chk$). Most typing rules are standard with respect to $\lambda C$ and bidirectional type checking.

\rul{D-Ax} is our impredicative axiom. \rul{D-Var} fetches type information from context for variables. \rul{D-LetVar} is the case for let binding, where we use \rul{D-Inst} to instantiate the type.

\rul{D-Ann} checks the expression with the provided type. \rul{D-Lam$\Inf$} gives $x$ a arbitrary type $\tau$ with ensuring $\tau$ is of type $\star$. \rul{D-Lam$\Chk$} continues to check the body with $x$ whose type is assigned as provided type. \rul{D-App} first infers the type of the function, and checks the argument satisfying parameter type, and substitutes the argument in the result type. \rul{D-Pi} checks the types in the pi type are all $\star$.

\rul{D-Let} generalizes the type of $e_1$ and store the type and the binding with $x$ in the context. The binding will be substituted in the result type. Which means, \lst{let a = int in \\x:a. x} is a well typed expression with type $\bpi x {int} {int}$ instead of $\bpi x a a $. Let has check mode because it needs to deliver the check type information. For example expression $(\kw{let} y = 3 \kw{in} \erlam f {(f~nat~1) + (f~\star~nat))}: (\bpi f {(\bpi x \star {\bpi y x nat})} nat)$.


\rul{D-CastUp} shows $\ercastup$ can only be checked because there are unlimited possibility of $\tau_2$. \rul{D-CastDn} does one-step reduction on original type.

\rul{D-Sub} states that if the type of a expression could be infered, then it could be checked.

\paragraph{Contextual equivalence.} Instead of normal delta-rule ($\delta$-reduction) used to deal with let binding, here we use contextual equivalence as shown in \rul{D-Conv}.

Using context as a substitution is defined in Figure \ref{fig:decl-apply-ctx}, where let binding is the only form that will be applied. But as we will see soon, this notation extends smoothly in the algorithmic system. If two types are alpha-equal under context substitution of $\dctx$, we regard these two types contextual equivalence under $\dctx$.

Having these notations, we could see in the precondition of \rul{D-Conv}, the type of $e$ could be replaced with another type only if these two types are contextual equivalence under current context.

If you have seen some other type systems involving type-level computation, there might be some equivalence over reduction. Here, things are much easier.

\paragraph{Polymorphism.} \rul{D-Inst} and \rul{D-Gen} are specialized forms of the ones in Implicit Calculus of Constructions (ICC), as types are just stars. One advantage of using this form of generalization instead of Hindley-Milner way is this rule helps get rid of type variables.

\rul{D-Inst} turns a polymorphic type into a monotype by instantiating all the variables bound by forall to expressions with type star. \rul{Gen} adds variables in context to infer the type, and extract them as forall in the result type. The definition of FV stands for free variables and is defined in Figure \ref{fig:decl-free-variables}.
