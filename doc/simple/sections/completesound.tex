\section{Completeness and Soundness}

\subsection{Context Extension}

\begin{figure*}[h]
    \headercapm{\tctx \exto \ctxr}{Context Extension}

    \[\CEEmtpy \quad \CEVar\quad\CETypedVar\]
    \[\CELetVar\quad\CEEVar \quad \CESolvedEVar\]
    \[\CESolve\quad\CEAdd \quad \CEAddSolved\]
    \caption{Context Extension.}
    \label{fig:ctx-extension}
\end{figure*}

\begin{figure*}[h]
    \headercapm{[\cctx]\tctx = \dctx}{Apply a complete context to a context}

    \begin{mathpar}
    \begin{tabular}{r c l l}
        $[\ctxinit]\ctxinit$   & = & $\ctxinit$    \\
        $[\cctx,x](\tctx,x)$ & = & $[\cctx]\tctx$  \\
        $[\cctx,x:\tau_1](\tctx,x:\tau_2)$ & = & $[\cctx]\tctx,x:[\cctx]\tau_1$ & if $[\cctx]\tau_1$=$[\cctx]\tau_2$ \\
        $[\cctx,x:\sigma_1=\tau_1](\tctx,x:\sigma_2=\tau_2)$ & = & $[\cctx]\tctx,x:[\cctx]\sigma_1=[\cctx]\tau_1$ & if $[\cctx]\sigma_1$=$[\cctx]\sigma_2$ and $[\cctx]\tau_1$=$[\cctx]\tau_2$ \\
        $[\cctx,\genA=\tau_1](\tctx,\genA)$ & = & $[\cctx]\tctx$ \\
        $[\cctx,\genA=\tau_1](\tctx,\genA=\tau_2)$ & = & $[\cctx]\tctx$ & if $[\cctx]\tau_1$=$[\cctx]\tau_2$ \\
        $[\cctx,\genA=\tau_1]\tctx$ & = & $[\cctx]\tctx$ & if $\genA \notin dom(\tctx)$ \\
    \end{tabular}
    \end{mathpar}
    \caption{Apply complete context.}
    \label{fig:apply-complete-ctx}
\end{figure*}

In Figure \ref{fig:syntax}, we give the definition of complete context $\cctx$ below algorithmic context. The main difference between complete context and context is that a complete context has all existential variables solved.

A context can be extended to gain a complete context by context extension as shown in Figure \ref{fig:ctx-extension}. \rul{CE-Empty} states that empty context could be extended to empty context. \rul{CE-Var} preserves the variable. \rul{CE-TypedVar} replaces the type with a contextually equivalent one. \rul{CE-LetVar} has two types that could be replaced. \rul{CE-EVar} and \rul{CE-SolvedEVar} keeps the existential variable, in the meanwhile \rul{CE-SolvedEVar} solves existential variables with a type. \rul{CE-Add} and \rul{CE-CEAddSolved} adds one existential variable with unsolved and solved form respectively.

Information could only be preserved or increased during context extension, including the existence of variables, their relative orders, type annotations, solutions for existential variables and all other information.

Since complete context has all existential variables solved, applying a complete context to a type will result in a type without existential variables, which is well-formed under a declarative context. The way to obtain this declarative context is to apply the complete context to itself $[\cctx]\cctx$. Applying a complete context to a context is defined in Figure \ref{fig:apply-complete-ctx}. Specifically, applying a complete context to itself will apply all the solved existential variables to types before throw away those existential variables.

\begin{lemma}[Stability of Complete Contexts]
  \label{lemma:stability-of-complete-context}

  If $\tctx \exto \cctx$, then $[\cctx]\tctx = [\cctx]\cctx$.
\end{lemma}

\subsection{Soundness}

Having the notation of complete context, this section gives the lemma about the soundness of algorithmic system with respect to the declarative system, namely soundness of unification and soundness of typing.

\begin{theorem}[Soundness of Unification]

If $\tctx \byuni \tau_1 \lt \tau_2 \toctx$ where $[\tctx] \tau_1 = \tau_1$ and $[\tctx] \tau_2 = \tau_2$,
and $\ctxl \exto \cctx$,
then $[\cctx]\ctxl \bywf [\cctx]\tau_1$ and $[\cctx]\ctxl \bywf [\cctx]\tau_2$ and $[\cctx]\tau_1 = [\cctx]\tau_2$.
\end{theorem}

This theorem states that, if in algorithmic system, with context $\tctx$, unify $\tau_1$ and $\tau_2$ results in a output context $\ctxl$, which is extended to complete context $\cctx$, then we have the two types $[\cctx]\tau_1$, $[\cctx]\tau_2$ are well formed under declarative context $[\cctx]\ctxl$, and they are alpha equivalent.

\begin{theorem}[Soundness of Instantiation]

If $\tpreinst \sigma \lt \tau \toctx$ where $\tprewf \sigma$ and $\tprewf \tau$,
and $\ctxl \exto \cctx$,
then $[\cctx]\ctxl \byinst [\cctx]\sigma \lt [\cctx]\tau$.
\end{theorem}

\begin{theorem}[Soundness of Generalization]

If $\tpreinf e:\tau \toctx $ and $\ctxl \bygen \tau \lt \sigma$,
and $\ctxl \exto \cctx$,
then $[\cctx]\ctxl \bygen [\cctx]e:[\cctx]\sigma$.
\end{theorem}

\begin{theorem}[Soundness of Typing]

Given $\ctxl \exto \cctx$
\begin{enumerate}
   \item If $\tpreinf e: \tau \toctx$ then $[\cctx]\ctxl \byinf [\cctx]e: [\cctx]\tau$
   \item If $\tprechk e: \tau \toctx$ then $[\cctx]\ctxl \bychk [\cctx]e: [\cctx]\tau$
   \item If $\tpreinf e_1:\tau_1 \toctx_1$ and $\ctxl_1 \byapp [\ctxl_1]\tau_1 ~ e_2 : \tau_2 \toctx$,
       then $[\cctx]\ctxl \byinf [\cctx]e_1 ~ [\cctx]e_2 : [\cctx]\tau_2$
\end{enumerate}
\end{theorem}

Soundness of typing has two forms corresponding to two modes in typing procedure. It states if in algorithmic system, with context $\tctx$, we could infer expression $e$ of type $\tau$, then in declarative context $[\cctx]\ctxl$, we could infer $e$ of type $[\cctx]\tau$. The statement is similar in the check case.

\subsection{Completeness}

To prove the completeness of the algorithmic system with respect to the declarative system, we need to state that given a declarative typing judgement, we could construct an algorithmic one.

\begin{theorem}[Completeness of Unification]

If $\tctx \exto \cctx$ and $\tctx \bywf \tau_1$ and $\tctx \bywf \tau_2$ and $[\cctx]\tau_1 = [\cctx]\tau_2$, then there exist $\ctxl$ and $\cctx'$ such that $\ctxl \exto \cctx'$ and $\cctx \exto \cctx'$ and $\tctx \byuni [\tctx]\tau_1 \lt [\tctx]\tau_2 \toctx$.
\end{theorem}

\begin{theorem}[Completeness of Instantiation]

If $\tctx \exto \cctx$ and $[\cctx]\tctx \byinst [\cctx]\sigma \lt [\cctx]\tau$
then there exists $\ctxl$ and $\cctx'$
such that $\ctxl \exto \cctx'$ and $\cctx \exto \cctx'$ and $\tctx \byinst [\tctx]\sigma \lt [\tctx]\tau \toctx$.
\end{theorem}

\begin{theorem}[Completeness of Generalization]

If $\tctx \exto \cctx$ and $[\cctx] \tctx \bygen [\cctx]e \lt [\cctx]\sigma$,
then there exists $\ctxl$ and $\cctx'$ and $\tau$
such that $\ctxl \exto \cctx'$ and $\cctx \exto \cctx'$ and $\tctx \byinf e : \tau \toctx$ and $\ctxl \bygen \tau \lt \sigma$.
\end{theorem}

\begin{theorem}[Completeness of Typing]

Given $\tctx \exto \cctx$
\begin{enumerate}
    \item If $\tctx \bywf \tau$ and $[\cctx]\tctx \bychk [\cctx]e : [\cctx]\tau$\\
          then there exists $\ctxl$ and $\cctx'$\\
          such that $\ctxl \exto \cctx'$ and $\cctx \exto \cctx'$ and $\tctx \bychk e: [\tctx]\tau \toctx$.
    \item If $[\cctx]\tctx \byinf [\cctx]e : \tau$\\
          then there exists $\ctxl$ and $\cctx'$ and $\tau'$\\
          such that $\ctxl \exto \cctx'$ and $\cctx \exto \cctx'$ and $\tctx \byinf e:\tau' \toctx$ and $\tau = [\cctx']\tau'$.
    \item If $[\cctx]\tctx \byinf [\cctx]e_1 ~ [\cctx]e_2 :\tau_1$\\
          then there exists $\ctxl$ and $\ctxl_1$ and $\cctx'$ and $\tau_1'$ and $\tau_2$\\
          such that $\ctxl \exto \cctx'$ and $\cctx \exto \cctx'$ and $[\cctx']\ctxl_1 \byinf [\cctx]e_1 :[\cctx']\tau_2$ and $\tctx \byinf e_1 : \tau_2 \toctx_1$ and $\ctxl_1 \byapp [\ctxl_1]\tau_2 ~ e_2 :\tau_1' \toctx$ and $\tau_1 = [\cctx']\tau_1'$.
\end{enumerate}
\end{theorem}
