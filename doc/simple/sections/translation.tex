\section{Translation}

In this chapter, we use type-directed translation to translate the calculus into fully typed dependent type calculus with cast construct.

\subsection{Target System Overview}

\begin{figure}[h]
    \begin{tabular}{lrcl}
        Target Expr. & $t, s$ & \syndef & $x \mid \star \mid t_1~t_2$ \\
        && \synor & $\blam x {t_1} t_2 \mid \bpi x {t_1} t_2$ \\
        && \synor & $\castup {t_1} t_2 \mid \castdn t$ \\
        && \synor & $\kw{let} x = t_1 \kw{in} t_2$ \\
        Target Ctx. &
        $\tctx$ & \syndef & $\ctxinit \mid x : t \mid \ctx, x:t_1 = t_2$
    \end{tabular}
    \caption{Target Language.}
    \label{fig:target}
\end{figure}

Figure \ref{fig:target} show the syntax of target language. The main difference is the target language is fully annotated, as shown in lambda abstraction and cast up construct.

Target context includes variable with type contraint, and variable for let binding.

\subsection{Target Declarative Typing}


\begin{figure}[h]
\headercapm{\pretar s_1 : t_1}{Expression $s_1$ has type $t_1$ under $\tctx$}
    \[\EAx \quad \EVar\]
    \[\EApp\]
    \[\ELam\]
    \[\EPi\]
    \[\ECastUp\]
    \[\ECastDown\]
    \[\ELet\]
    \[\EConv\]
    \caption{Target declarative typing rules.}
    \label{fig:targettyping}
\end{figure}

The declarative typing rules is shown in Figure \ref{fig:targettyping}.

\rul {E-AX} is for star case. \rul {E-Var} fetches information from context. \rul {E-App} first infers the type of the abstraction, and wants the argument type to match the parameter type, at last substitutes the argument in the result type. \rul {E-Lam} has an annotation with star type, and adds the type of \lst{x} to get the body type, which returns a Pi type at last. \rul {E-Pi} has the argument of star type, and adds \lst x to the context to has the result of star type. \rul {E-CastUp} has annotation with type star, and the type of the expression can be reduced to the annotated type. \rul {E-CastDown} reduces the type of the expression. \rul {E-Let} first gets the type of the let binding, and adds the let binding to the context to get the result type. \rul {E-Conv} is the conversion rule for let binding.

\subsection{Type-Directed Translation}

\renewcommand{\trto}[1]{\hl{\rightsquigarrow #1}}
\renewcommand{\opt}[1]{\hl{#1}}

\begin{figure*}[h]
    \headercapm{\preall e:\tau \toctx \trto t}{Expression Typing ($\delta ::= \Inf\mid \Chk$)}
    \[\DAx \quad \DVar \quad \DLetVar \]
    \[\DLamInf \quad \DLamChk\]
    \[\DLamAnn \quad \DApp\]
    \[\DLet \quad \DPi\]
    \[\DCastUp \quad \DCastDn\]
    \[\DAnn \quad \DSub \quad \DConv\]
    \[\DPoly\]
    \\
    \headercapm{\preinst \sigma \lt \tau \toctx}{Instantiation} \quad \headercapm{\pregen e : \sigma}{Generalization}
    \[\DInstantiation\]
    \\
    \[\DGeneralization\]
    \\
    \caption{Translation}
    \label{fig:translation}
\end{figure*}

\begin{figure}[h]
\headercapm{\ctx \trtop \tctx}
  \[\TCEmpty\]
  \[\TCTypedVar\]
  \[\TCLetVar\]
  \caption{Translation of context}
  \label{fig:translatecontext}
\end{figure}

We specify the translation in the type rules of Figure \ref {fig:typingrules}, which results to Figure \ref {fig:translation}.

The translation of context is defined in Figure \ref {fig:translatecontext}.


