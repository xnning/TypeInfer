\section{Algorithmic Typing}

Declarative system servers well as a specification. However, in order to make type system practical, we cannot keep guessing types: as an example notice the $\tau_1$ in rule \rul{D-Lam$\Chk$} comes from nowhere.

One of main contributions of our work is a simple and deterministic algorithmic system, which could work as an algorithm for solving unknown types. We adopt the same notation as in Dunfield's work, the existential variable that represents types to be solved later. It brings more subtlety in a dependently typed calculus, but as we will see, things are not getting that complex.

\subsection{Language}

\begin{figure}[h]
    \begin{tabular}{lrcl}
        Type Scheme & $\sigma$ & \syndef & $\typA \mid \genA \mid e$ \\

        Expr & $e$ & \syndef & $x \mid \star \mid e_1~e_2$ \\
        && \synor & $\erlam x e \mid \bpi x {\sigma_1} \sigma_2$ \\
        && \synor & $\ercastup e \mid \castdn e$ \\
        && \synor & $e : \sigma$ \\
        && \synor & $\forall \typA: \star. \sigma$ \\
        Monotype & $\tau$ & \syndef & $ \{ \sigma' \in \sigma, \forall \notin \sigma'\} $ \\
        Contexts &
        $\tctx,\ctxl,\ctxr$ & \syndef & $\ctxinit$ \\
        && \synor & $\tctx, \typA \mid \tctx,x \mid \tctx,x:\sigma$ \\
        && \synor & $\tctx,\genA \mid \tctx,\genA=\tau$ \\
        Complete Ctx &
        $\cctx$ & \syndef & $\ctxinit$ \\
        && \synor & $\cctx, \typA \mid \cctx,x \mid \cctx,x:\sigma$ \\
        && \synor & $\cctx,\genA=\tau$ \\
    \end{tabular}
    \caption{syntax.}
    \label{fig:algo-syntax}
\end{figure}

Figure \ref{fig:algo-syntax} gives the syntax of type and expression, and the definition of context for algorithmic system. The main difference is the existential variable $\genA$, which is quite like unification variables that are waited to be solved. But as context is ordered, unlike unification variables, existential variables also have scopes.

There are two forms of existential variables in context: unsolved form $\genA$ and solved form $\genA=\tau$. Once an existential vairable is solved, its solution will never be changed.

\begin{figure}[h]
    \headercapm{\tctx \bywt \sigma}{Well Term of type $\sigma$ under context $\tctx$}
    \[\TWTVar \quad \TWTEVar \quad \TWTStar\]
    \[\TWTPoly \quad \TWTApp\]
    \[\TWTLam \quad \TWTPi\]
    \[\TWTLet \quad \TWTCastUp\]
    \[\TWTCastDn \quad \TWTAnn\]

    \headercapm{\tctx \bywf \sigma}{Well Form of type $\sigma$ under context $\tctx$}

    \[\TWFOther \]

    \headercapm{\tctx\wc}{Context $\tctx$ is well formed}

    \[\TWCEmpty \quad \TWCVar\]
    \[\TWCTypedVar\]
    \[\TWCTVar\]
    \[\TWCEVar\]
    \[\TWCSolvedEVar\]
    \caption{Well formedness with existential variables.}
    \label{fig:existwellform}
\end{figure}

The change of corresponding well formedness for existential variables and context, with a new notation well-termedness are given in Figure \ref{fig:existwellform}. Well-termedness $\tctx \bywt \sigma$ means under context $\tctx$, $\sigma$ is a syntactically legal term with all variables bounded, but it does not concern about type. The usage of well-termedness will become clear when we get into unification rules. Well formedness of a type now take existential variables into consideration. Existential variable is a well-formed type if it is contained in the context. An expression which is not a Pi type or a type scheme, is well-formed if it can be checked with $\star$.

Context substitution is extended smoothly with existential variables, as shown in Figure \ref{fig:applyctx}. Intuitively, once an existential variables is solved, we could substitute it with its solution. Different from declarative system, context substitution plays a more important role in algorithmic system, since some procedures only accept fully substituted expressions.

In algorithmic system, we will use FEV to stands for free existential variables in a type. The definition is given in Figure \ref{fig:algo-free-variables}. FEV of a context collects all FEV in type annotations, and does not include let binding and existential solutions.

\begin{figure}[t]
    \headercapm{[\tctx]e_1 = e_2}{Apply context to a type}

    \begin{mathpar}
    \begin{tabular}{r c l l}
        $[\ctxinit]e$   & = & $e$       \\
        $[\tctx,x]e$      & = & $[\tctx]e$ \\
        $[\tctx,x:\tau]e$ & = & $[\tctx]e$ \\
        $[\tctx,\typA]e$ & = & $[\tctx]e$ \\
        $[\tctx, \genA]e$ & = & $[\tctx]e$ \\
        $[\tctx, \genA=\tau]e$ & = & $[\tctx](e[\genA \mapsto \tau])$
    \end{tabular}
    \end{mathpar}
    \caption{Applying a context with existential variables.}
    \label{fig:applyctx}
\end{figure}

\begin{figure}[t]
    \headercapm{FEV(e)}{free existential variables of a type}
        \begin{mathpar}
        \begin{tabular}{r c l}
            $FEV(\genA)$        & = & $\{\genA\}$       \\
            $FEV(x)$            & = & $\varnothing$       \\
            $FEV(\star)$        & = & $\varnothing$            \\
            $FEV(e_1 ~ e_2)$    & = & $FEV(e_1) \cup FEV(e_2)$            \\
            $FEV(\erlam x e)$   & = & $FEV(e)$            \\
            $FEV(\bpi x t e)$   & = & $FEV(t) \cup FEV(e)$            \\
            $FEV(\kw{let} x = e_1 \kw{in} e_2)$  & = & $FEV(e_1) \cup FEV(e_2)$            \\
            $FEV(\castupz e)$   & = & $FEV(e)$            \\
            $FEV(\castdn e)$    & = & $FEV(e)$            \\
            $FEV(e:t)$          & = & $FEV(e) \cup FEV(t)$            \\
            $FEV(\forall x:\star. \sigma)$     & = & $FEV(\sigma)$            \\
        \end{tabular}
        \end{mathpar}
    \caption{Free existential variables.}
    \label{fig:algo-free-variables}
\end{figure}
\paragraph{Hole notation.} We keep the same hole notations like $\tctx[x]$ to require the variable $x$ appear in the context, sometimes also written as $\tctx_1, x, \tctx_2$. According to this definition, the precondition of \rul{WF-EVar} could be removed and the conclusion is replaced by $\tctx[\genA] \bywf \genA$.

Multiple holes also keep the order. For example, $\tctx[x][\genA]$ not only require the existance of both variable $x$ and $\genA$, and also $x$ appear before $\genA$.

Hole notation is also used for replacement and modification. Later in the unification rules, we will see situations where $\tctx[\genA]$ could become $\tctx[\genA = \star]$, which means the context keeps unchanged except $\genA$ now is solved.

\subsection{Algorithmic Typing}


\begin{figure*}[h]
    \headercapm{\tpreall e:\sigma \toctx \trto t}{Expression Typing ($\delta ::= \Inf\mid \Chk$)}
    \[\TVar \quad \TAnn \quad \TPi\]
    \[\TAx \quad \TTVar \quad \TLamInf \quad \TLamChk\]
    \[\TCastUp \quad \TCastDn \quad \TPoly\]
    \[\TEVar \quad \TApp \quad \TSub\]
    \[\APi \quad \AEVar \quad \APoly\]
    \\
    \caption{Typing rules}
    \label{fig:algo-typing-rules}
\end{figure*}

\begin{figure*}[h]
    \headercapm{\tpresub \sigma_1 \lt \sigma_2 \toctx}{Subtyping}
    \[\SPolyR \quad \SPolyL \quad \SPi\]
    \[\SEVarL \quad \SEVarR \quad \SUnify\]
    \caption{Subtyping}
    \label{fig:subtyping}
\end{figure*}

\begin{figure*}[h]
    \headercapm{\tctx[\genA] \byrf^s \sigma \rfto \tau \toctx}{sanitize $\sigma$ to $\tau$ so it becomes monotype. $s = + | -$}
    \[\IPoly \quad \IPi \quad \ITau\]

     \headercapm{\tctx[\genA] \bycg \tau_1 \cgto \tau_2 \toctx}{Under context $\tctx$, sanitize $\tau_1$ to $\tau_2$ so it contains no existential variables out of the scope of $\genA$}
    \[\IEVarB \quad \IEVarA\]
    \[\IVar \quad \IStar \quad \IOthersNew\]
    \caption{Type Sanitization}
    \label{fig:algo-resolve}
\end{figure*}

\begin{figure*}[h]
\headercapm{\tpreuni \sigma_1 \aeq \sigma_2 \toctx}{Unification of Types}
    \[\UAEq \quad \UEVarTy\]
    \[\UTyEVar \quad \UOthers\]
    \caption{Unification rules}
    \label{fig:algo-unification}
\end{figure*}

\begin{figure*}[h]
    $\ctxl = \tctx,\genA_1,\genA,\genB =\genA_1,\ctxr$
    \[\ExUni\]
    \caption{Solve unification problem.}
    \label{fig:algo-solve-unify}
\end{figure*}

Algorithmic bidirectional typing rules directly follow declarative rules, with existential variables representing unknown types, and a output context. Figure \ref{fig:algo-typing-rules} gives the algorithmic bidirectional typing rules. The interpretation of judgement $\tpreall e:\tau \toctx \trto t$ is, under context $\tctx$, infer expression $e$ (if $\delta = \Inf$), which gives output type $\tau$ and output context $\ctxl$; or check $e$ with type $\tau$ (if $\delta = \Chk$), which gives output context $\ctxl$.

\paragraph{Output context.}
Given a input context, during typing process, new existential variables could be created, useless existential variables could be removed, unsolved existential variables could get solved. The output context of typing is to keep those updates. For example, in the preconditions of rule \rul{Ann}, the first output context will become the input context for next typing.

Before we dive into typing rules, we discuss instantiation and generalization rules first, since typing rules rely on them and they are just natural extension of declarative rules.

\subsubsection{Instantiation and Generalization}

Instantiation turns a type scheme into a monotype by instantiating forall bounds. Unlike declarative system where we guess the types to instantiate, we use existential variables. As shown in the bottom of Figure \ref{fig:algo-typing-rules}, \rul{Inst} substitute the bounds variables with created new existential variables and put them into output context so further typing could refer to those vairables.

In generalization, \rul{Gen} firstly substitutes the context, and extract the free variables to make a polymorphic type. It is very Hindley-Milner way and does not modify the context.

\subsubsection{Typing rules}

We now go through the typing rules.

\rul{T-Ax} and \rul{T-Var} remain standard, where the context has no change. \rul{T-LetVar} relies on the instantiation, and takes the output context of instantiation as output context. \rul{T-Ann} is a natural extensions of the declarative rules with extra interaction with contexts.

\paragraph{Throw out context.}
In \rul{T-Pi}, the context $\ctxr$ after $x$ is thrown because all those variables are out of scope, and all the existential variables in $\ctxr$ will not be refered again when this rule ends. It is safe to throw existential variables because we could assume all unsolved existential variables are solved by $\star$ which satisfys $\star:\star$.

\rul{T-Lam$\Chk$} could be interpreted in the same way.

\paragraph{Typing Dependent Functions.}
In how to type infer abstraction, Dunfield and Krishnaswami use a clever way to give result type an existential variable, and make sure it appears before $x$, so it will not be thrown away while throwing trailing context:

\begin{mathpar}
\OLamInf
\end{mathpar}

It works well when there is no dependency of $\genB$ on $x$. But it is not the case in dependently typed function. Consider expression \lst{\\x:*. \\y:x. y} of type $\bpi x \star {\bpi y x x}$. In the outer lambda, the $\genB$ (which is solved as $\bpi y x x$) will refer to $x$. So instead, we use the normal way to type infer abstraction as shown in the precondition in rule \rul{T-Lam$\Inf$}

However, this brings another problem: \emph{the context after $x$ cannot be simply thrown away, because the function type may have references to the existential variables in the context after $x$}. For example, infering type of expression $\erlam x {\erlam y y}$ will give $\bpi x {\genA_1} {\bpi y {\genA_2} {\genA_2}}$. If we throw away the context after $x$, we will lose the information about $\genA_2$.

To solve this problem, we notice that the result type has two forms of references to the existential variables after $x$, that is, it can refer to a solved one, or it can refer to an unsolved one. For solved variables, we can substitute the context to the result type. And for unsolved variables, we need to preserve them. So one possible solution is to substitute the context to the result type and preserve the unsolved type variables, as shown in the conclusion in rule \rul{T-Lam$\Inf$}. The formal definition of UV (stands for unsolved existential variables) shows as follows:

\begin{mathpar}
    \begin{tabular}{r c l l}
        $UV(\ctxinit)$    & = & $\cdot$       \\
        $UV(\tctx, \genA)$ & = & $UV(\tctx), \genA$ \\
        $UV(\tctx, \_)$     & = & $UV(\tctx)$
    \end{tabular}
\end{mathpar}

We can minimize the context if replace $UV(\ctxr)$ with $UV(\ctxr) \cap FEV([\ctxr]\tau_2)$. Because only those free existential variables appeaering in result type are what we want to preserve.

\paragraph{Application.}
Consider the application rule in declarative system, which assumes $e_1$ is of a Pi type. It is not the case in algorithmic system because the type of $e_1$ could be either a Pi type or an existential variable. We will take both two cases into consideration.

\rul{T-App} first infers the type of $e_1$ to get $\tau_1$, then substitute the context in $\tau_1$ and enter the application typing rules, which is defined by the A-rules. The judgement $\tctx \byapp \tau_1 ~ e : \tau_2 \toctx$ is interpreted as, in context $\tctx$, the type of the expression being applied is $\tau_1$, the argument is $e_2$, and the output of the rule is the application result type $\tau_2$ with context $\ctxl$.

In \rul{A-Pi}, the type of $e_1$ is a Pi type, so it checks $e_2$ with the argument type provided.

In \rul{A-EVar}, the type of $e_1$ is an unsolved existential variable $\genA$. It first destructs $\genA$ into a Pi type, and checks $e_2$ with the new generated existential variable. One notice is that \rul{A-EVar} actually could only resolve non-dependent function type for $\genA$, which means there is no dependency of $x$ in $\genA_2$. For example, expression $\erlam f {(f~nat~1) + (f~\star~nat)}$ will be rejected, because it requests to infer $f$ as a dependently typed function with type $\bpi x \star {\bpi y x} nat$.

\rul{T-Let} first infers the type of the binding, then use \rul{Gen} to do Damas-Milner style polymorphism. When type checking the body, the let binding is added to the context. The result type will be substituted by the context after $x$ and the binding itself.

\rul{T-CastUp} and \rul{T-CastDn} apply the context before reduction because solutions and let definitions influence reduction, then do one-step reduction according to semantic operations.

\rul{T-Sub} shifts the mode, by requesting the inferred type and checked type to be alpha-equal by using unification, which is discussed in following sections.

\subsubsection{Type Sanitization}

As we mentioned before, our unification is based on alpha-equality, since beta reduction needs to be explicit by using cases. So in most cases, the unification rule will be just recursive calls to components. The most difficult one which is also the most essential one, is how to unify a existential variable with another type. We discuss those cases first.

While unifying existential variable $\genA$ and type $\tau$, We know that existential variables can only be solved as a monotype, and all the types in unification are monotype, so we tends to directly derive that $\genA = \tau$. But the orders in the context are important. Consider a unification example (the notation $\lt$ is discussed in later section):

$\tctx, \genA, \genB, \ctxr \byuni \genA \lt \bpi x \genB x$

Here because $\genB$ appears after $\genA$, so we cannot directly solve $\genA$ by $\bpi x \genB x$ which is ill typed. But this unification do have a solution context if we could somehow sanitize the appearance of $\genB$ by solving it:

$\tctx, \genA_1, \genA, \genB = \genA_1, \ctxr \byuni \genA \lt \bpi x {\genA_1} x$

Now we have a happy solution context: $\tctx, \genA_1, \genA = \bpi x {\genA_1} x, \genB = \genA_1, \ctxr$.

The definition of type sanitization is given in Figure \ref{fig:algo-resolve}.

Rule \rul{I-EVar} does the most interesting one and is one actually doing sanitization: if the existential variable $\genB$ appear behing $\genA$ in context, which means it cannot be assigned as solution of $\genA$, we will generate a new fresh existential variable $\genA_1$ before $\genA$, and replace the appearance of $\genB$ to $\genA_1$.

In rule \rul{I-EVar2}, $\genB$ is reserved safely because it appears before $\genA$. We also have rul{I-Var} and {I-Star} as base cases where the type stays the same.

For the sake of clarity, we represent other expressions by their explicit constructor head $D$ and components $\overbar {e_n}$ so all the cases could be combined into one rule \rul{I-Other}. Here gives some examples:

\begin{mathpar}
    \begin{tabular}{r l l l}
        $e_1 e_2$ & = & $App~e_1~e_2$ & so $D = App, n = 2, \overbar{e_n} = e_1, e_2$\\
        $\kw{let} x = e_1 \kw{in} e_2$ & = & $Let_x~e_1~e_2$ & so $D = Let_x, n = 2, \overbar {e_n} = e_1, e_2$ \\
        $\blam x {\tau_1} {\tau_2} $ & = & $Lam_x~\tau_1~\tau_2$ & so $D = Lam_x, n = 2, \overbar{e_n} = \tau_1, \tau_2$\\
    \end{tabular}
\end{mathpar}

We could easily convert between the syntax and their explicit representations. These notations are also used in unification.

If you are familiar with Dunfield and Krishnaswami's work, you can check that type sanitization here is actually generalizing the work of original rule \rul{InstLArr} and rule \rul{InstLReach}, whose job is to work for cases that some existential variables in the right hand side appear after $\genA$ in context. Our rule could also be used in Dunfield and Krishnaswami's system, with extra treatment of polymorphism types.

\subsubsection{Unification}

Figure \ref{fig:algo-unification} defines the unification procedure used in \rul{T-Sub} rule, which make use of type sanitization. The judgement $\tpreuni \tau_1 \lt \tau_2 \toctx$ is interpreted as, under context $\tctx$, unify two types $\tau_1$ and $\tau_2$, which gives output context $\ctxl$. One feature of the unification is like Hindley-Milner system: there is no forall in unification procedure, which simplifies unification a lot.

Rule \rul{U-Var} states that the same variables are alpha-equal obviously, while \rul{U-EVar} applies on the same existential variables. Rule \rul{U-Star} is for two stars. And like type sanitization, we combine all other data constructors into the rule \rul{U-Others}. The $[,x]$ means in some rules we may add some bound variables into the scope, such as abstraction or Pi, and throw them away before entering successive process.

Two most subtle ones are rule \rul{U-EVarTy} and \rul{U-TyEVar}, which corresponding respectively to when the existential is in the left and in the right. We just go through the first one. First, occurs check, since no recursive types are allowed. After then we use type sanitization to make sure all the existential types in $\tau_1$ that are out of scope of $\genA$ are turned into fresh ones that are in the scope of $\genA$. This process gives us $\tau_2$ that might be different from $\tau_1$, and output context $\ctxl_1, \genA, \ctxl_2$. Even though now we got all the existential variables in $\tau_2$ are in $\ctxl_1$, we still need to make sure $\tau_2$ is well scoped in $\ctxl_1$ because it may refer to some bound variables.

\paragraph{Example.} Now we could solve the unification problem brought up at the beginning of last section, shown in Figure \ref{fig:algo-solve-unify}.

